%模板
\documentclass[10pt,a4paper,twoside]{book}

\usepackage{ctex}
\usepackage{xeCJK}
\usepackage[linesnumbered,boxed]{algorithm2e}
\usepackage{ulem}
\usepackage{yhmath}
\usepackage{amssymb}
\usepackage{makecell}
\usepackage{verbatim}
\usepackage{enumerate}%罗列专用宏包
\usepackage{graphicx}%插入图片的宏包
\usepackage{subfigure} 
\let\widering\relax
\usepackage{newtxtext}
\usepackage{newtxmath}
\usepackage{extarrows}
\usepackage{bm}
\usepackage{esint}
\usepackage{makeidx}%索引专用
\makeindex  %添加索引
\usepackage{fancyhdr}
\usepackage{framed}
\usepackage{amsfonts}
\usepackage{wrapfig}
\usepackage{textcomp}%树叶图案在这个包里
\usepackage{bbding}%很多漂亮的图案
\usepackage[dvipsnames, svgnames, x11names]{xcolor}%导入了所有颜色配置文件的宏包
\usepackage{CJKfntef}
\usepackage{geometry}%页边距调整
\geometry{left=2cm,right=2cm,bottom=2cm,top=2cm}
\usepackage{titletoc}%目录页的宏包
\usepackage{titlesec}%改变章节或标题的样式的宏包
\usepackage[bookmarks=true,colorlinks,linkcolor=black]{hyperref}
\usepackage{enumerate}%使用改宏包优化罗列环境
\usepackage{tcolorbox}%box宏包
\usepackage{colortbl,booktabs}%第二个包定义了几个*rule  
\usepackage{multicol}
\usepackage{multirow}
\usepackage{tikz}
\usetikzlibrary{shapes.geometric}
\usetikzlibrary{arrows,arrows.meta}

%字体设置
\setCJKmainfont[BoldFont={PingFangSC-Medium}]{PingFangSC-Regular}
\setCJKfamilyfont{kai}{STKaitiSC-Regular}%都是用来定义字体的(此处使用电脑自带楷书)
\DeclareMathSizes{10}{10}{5.5}{4}

%章节或标题的样式
\titleformat{\chapter}{\bfseries\Huge\color{titlepurple}}{第\ \thechapter\ 章\ \quad}{0pt}{}
\titleformat{\section}{\Large\color{titlepurpleb}}{\bfseries{\thesection}\quad  }{0pt}{}
\titleformat{\subsection}{\large\color{titlepurplec}}{\bfseries{\thesubsection}\quad  }{0pt}{}
\titlespacing{\section}{0em}{1em}{1em}[1em]
\titlespacing{\subsection}{1.5em}{1em}{0.5em}[1em]
%格式如下:\titlespacing*{章节名称}{左间距}{(前)行间距}{(后)行间距}[右间距(一般都没用,填0.1em即可,但不能不填)]
\titlespacing*{\subsubsection}{2em}{3em}{1em}[1em]


%目录调整
\newcounter{mycontents}
\newcommand{\thecontents}{\refstepcounter{mycontents} \alph{mycontents}.}
%\titlecontents{标题名}[左间距]{标题格式}{标题标志}{无序号标题}{指引线与页码}[下间距]
\titlecontents{chapter}
[0cm]
{\bf \large \vspace{0.8em} }{\contentspush{第 \thecontentslabel\ 章 \hspace*{0.8em}}}{}{\titlerule*[0.5pc]{$\cdot$}\contentspage}
\titlecontents{section}[1.7cm]{\bf  \vspace{0.5em} }{\contentslabel{2.4em}}{\hspace*{-2.5em} \thecontents \hspace*{0.8em}}{\titlerule*[0.5pc]{$\cdot$}\contentspage}
\titlecontents{subsection}[2.5cm]{\small \vspace{0.2em}}{\contentslabel{3em}}{}{\titlerule*[0.5pc]{$\cdot$}\contentspage}

%定义颜色
%定义某个颜色,对应颜色代号查表
\definecolor{titlepurple}{HTML}{5758BB}%一级标题(目前:蓝紫色)
\definecolor{titlepurpleb}{HTML}{3A006F}%二级标题(目前:深紫色)
\definecolor{titlepurplec}{HTML}{006266}%三级标题(目前:墨绿色)
\definecolor{tab1}{HTML}{9698ED}%表格1
\definecolor{tab2}{HTML}{DBDCFF}%表格2
\definecolor{dy0}{HTML}{EA7500}%小标题定义专用(目前:橙黄色)
\definecolor{dl}{HTML}{007500}%小标题定理专用(目前:深绿色)
\definecolor{inference}{HTML}{343300}%小标题推论专用(目前:墨绿色)
\definecolor{ex}{HTML}{7158e2}%小标题例专用(目前:紫色)
\definecolor{dy}{HTML}{BF0060}%夹杂在文本中的定义词的颜色1(目前:深红色)
\definecolor{dy2}{HTML}{6C3365}%夹杂在文本中的定义词的颜色2(目前:红紫色)
\definecolor{超链接}{HTML}{0000C6}%含超链接的文字专用色(目前:蓝紫色)
\definecolor{文字底色}{HTML}{F8FF00}%强调的文字底色(目前:黄色)
\definecolor{eq}{HTML}{F0F0F0}
\definecolor{tl}{HTML}{D94600}
\definecolor{d3}{HTML}{dc4853}


%定义计数器
\newcounter{theorem}[chapter]
\newcounter{defination}[chapter]
\newcounter{example}[chapter]
\newcounter{inference}[chapter]
\newcounter{examples}[chapter]
\newcounter{tl}[chapter]
\newcounter{F}[section]
\newcounter{G}[section]
\newcounter{H}[section]
\renewcommand{\thetheorem}{{ 定理} \textbf{\thechapter.\arabic{theorem}}}
\renewcommand{\thedefination}{{ 定义} \textbf{\thechapter.\arabic{defination}}}
\renewcommand{\theexample}{{ 题型} \textbf{\thechapter.\arabic{example}}}
\renewcommand{\theinference}{{ 方法} \textbf{\thechapter.\arabic{inference}}}
\renewcommand{\theexamples}{{ 例}  \textbf{\thechapter.\arabic{examples}}}
\renewcommand{\thetl}{{ 推论}  \textbf{\thechapter.\arabic{tl}}}
\newcommand{\s}{\hspace*{-2.7em}}

%定义环境
\newcommand{\mybox}[2][]{
	\begin{tcolorbox}[on line,
		arc=0pt,outer arc=0pt,colback=#1!10!white,colframe=#1,
		boxsep=0pt,left=3pt,right=3pt,top=3.5pt,bottom=3.5pt,
		boxrule=0pt,leftrule=1.5pt]#2
\end{tcolorbox}}

%命令格式说明:正常情况的命令就是中文对应的英文名,以下有几个特殊情况进行了微调
%1. 小标题在列表上方,使用enup+英文名;小标题在列表下方,使用开nbelow+英文名
%2. 标题间隔太大,采用t+英文名
%3. 间距太小,用add+英文名
%4. 在列举环境中间距太小用adden+英文名

%定理类
\newcommand{\theorem}[1][]{\vspace{1em}\s \refstepcounter{theorem} \mybox[dl]{\color{dl}\thetheorem\hspace{1em}#1}\vspace{0.5em}  \par}
\newcommand{\enuptheorem}[1][]{\vspace{1em}\s \refstepcounter{theorem}\label{#1} \mybox[dl]{\color{dl}\thetheorem\hspace{1em}#1}\vspace*{-0.8cm}}
\newcommand{\enbelowtheorem}[1][]{\hspace*{-1.5em}\s \refstepcounter{theorem}\label{#1} \mybox[dl]{\color{dl}\thetheorem\hspace{1em}#1}  \par}
\newcommand{\ttheorem}[1][]{\s \refstepcounter{theorem}\label{#1} \mybox[dl]{\color{dl}\thetheorem\hspace{1em}#1}\vspace{0.5em}\par }
\newcommand{\addtheorem}[1][]{\vspace{1.2em}\s \refstepcounter{theorem}\label{#1} \mybox[dl]{\color{dl}\thetheorem\hspace{1em}#1}\vspace*{0.5em}  \par}
\newcommand{\addentheorem}[1][]{\vspace{1.2em}\hspace*{-1.5em}\s \refstepcounter{theorem}\label{#1} \mybox[dl]{\color{dl}\thetheorem\hspace{1em}#1}\vspace{0.5em}  \par}

%推论类
\newcommand{\inference}[1][]{\vspace{1em}\s \refstepcounter{inference} \mybox[inference]{\color{inference}\theinference\hspace{1em}#1}\vspace{0.5em}   \par}
\newcommand{\enupinference}[1][]{\vspace{1em}\s \refstepcounter{inference} \mybox[inference]{\color{inference}\theinference\hspace{1em}#1}\vspace*{-0.8cm}  }
\newcommand{\enbelowinference}[1][]{\hspace*{-1.5em}\s \refstepcounter{inference} \mybox[inference]{\color{inference}\theinference\hspace{1em}#1}  \par}
\newcommand{\tinference}[1][]{\s \refstepcounter{inference}\label{#1} \mybox[inference]{\color{inference}\theinference\hspace{1em}#1}\vspace*{0.5em} \par }
\newcommand{\addinference}[1][]{\vspace{1.2em}\s \refstepcounter{inference}\label{#1} \mybox[inference]{\color{inference}\theinference\hspace{1em}#1}\vspace{0.5em} \par}
\newcommand{\addeninference}[1][]{\vspace{1.2em}\hspace*{-1.5em}\s \refstepcounter{inference}\label{#1} \mybox[inference]{\color{inference}\theinference\hspace{1em}#1}\vspace{0.5em} \par}

%定义类
\newcommand{\defination}[1][]{\vspace{1em}\s \refstepcounter{defination} \mybox[dy0]{\color{dy0}\thedefination\hspace{1em}#1}\vspace{0.5em} \par}
\newcommand{\enupdefination}[1][]{\vspace{1em}\s \refstepcounter{defination}\label{#1} \mybox[dy0]{\color{dy0}\thedefination\hspace{1em}#1}\vspace*{-0.8cm}}
\newcommand{\enbelowdefination}[1][]{\hspace*{-1.5em}\s \refstepcounter{defination}\label{#1} \mybox[dy0]{\color{dy0}\thedefination\hspace{1em}#1} \par }
\newcommand{\tdefination}[1][]{   \s \refstepcounter{defination} \mybox[dy0]{\color{dy0}\thedefination\hspace{1em}#1}\vspace*{0.5em}  \par }
\newcommand{\adddefination}[1][]{\vspace{1.2em}\s \refstepcounter{defination}\label{#1} \mybox[dy0]{\color{dy0}\thedefination\hspace{1em}#1}\vspace{0.5em} \par}
\newcommand{\addendefination}[1][]{\vspace{1.2em}\hspace*{-1.5em}\s \refstepcounter{defination}\label{#1} \mybox[dy0]{\color{dy0}\thedefination\hspace{1em}#1}\vspace{0.5em} \par}

%题型类(无标签)
\newcommand{\example}[1][]{\vspace{1em} \s \refstepcounter{example} \mybox[ex]{\color{ex}\theexample\hspace{1em}#1}\vspace{0.5em} \par }
\newcommand{\enupexample}[1][]{\vspace{1em}\s \refstepcounter{example} \mybox[ex]{\color{ex}\theexample\hspace{1em}#1}\vspace*{-0.8cm}}
\newcommand{\enbelowexample}[1][]{\hspace*{-1.5em}\s \refstepcounter{example}\mybox[ex]{\color{ex}\theexample\hspace{1em}#1} \par }
\newcommand{\texample}[1][]{  \s \refstepcounter{example} \mybox[ex]{\color{ex}\theexample\hspace{1em}#1} \vspace{0.5em} \par }
\newcommand{\addexample}[1][]{\vspace{1.2em}\s \refstepcounter{example} \mybox[ex]{\color{ex}\theexample\hspace{1em}#1}\vspace{0.5em} \par }
\newcommand{\addenexample}[1][]{\vspace{1.2em}\hspace*{-1.5em}\s \refstepcounter{example} \mybox[ex]{\color{ex}\theexample\hspace{1em}#1}\vspace{0.5em} \par }

%\theoremstyle{break}
%\theoremindent0.2cm
%\newtheorem*{theorem}{\hspace{0.2cm}\color{dl}\label{#1} \mybox[dl]{\color{dl}定理\addtocounter{A}{1} \thesection.\arabic{A}}}
%\newtheorem*{defination}{\hspace{0.2cm}\color{dy0}\label{#1} \mybox[dy0]{\color{dy0}定义\addtocounter{B}{1} \thesection.\arabic{B}}}
%\newtheorem*{feature}{\hspace{-0.16cm}\color{ffa725}\label{#1} \mybox[ffa725]{\color{ffa725}性质\addtocounter{C}{1} \thesection.\arabic{C}}}
%\newtheorem*{inference}{\hspace{-0.16cm}\color{1a9850}\label{#1} \mybox[1a9850]{\color{1a9850}推论\addtocounter{D}{1} \thesection.\arabic{D}}}
%\newtheorem*{method}{\hspace{-0.16cm}\color{6a3d9a}\label{#1} \mybox[6a3d9a]{\color{6a3d9a}方法\addtocounter{E}{1} \thesection.\arabic{E}}}
%\newtheorem*{example}{\hspace{-0.16cm}\color{53a9ab}\label{#1} \mybox[53a9ab]{\color{53a9ab}例题\addtocounter{F}{1} \thesection.\arabic{F}}}


%文章标题
\title{
	\Huge{高等数学$\,$\uppercase\expandafter{\romannumeral1}$\,$ 总结}\\
	\quad\\
	\quad\\
	\quad\\
	\quad\\
	\quad\\
	\quad\\
	\quad\\
	\quad\\
	\quad\\
}
\author{
	{\CJKfamily{kai} \large {易鹏\quad 蒋天宇}}\\
	{\CJKfamily{kai} \large 中山大学$\,$航空航天学院}\vspace*{0.5em}\\
	内部版本号:V9.32.68.089$\,\,$(内测版)\\
}



%调整间距(倍数)
\linespread{1.5}

%自定义页眉页脚---------------
\pagestyle{fancy}
\renewcommand{\chaptermark}[1]{\markboth{\;第\ \thechapter\ 章\quad#1\;}{}}
\renewcommand{\sectionmark}[1]{\markright{\;\thesection\ #1\;}}
\fancyhf{}
%\fancyfoot[C]{\bfseries\thepage}
\fancyhead[LO]{\small\CJKfamily{song}\rightmark}
\fancyhead[RE]{\small\CJKfamily{song}\leftmark}
\fancyhead[RO,LE]{\;\thepage\;}
\fancyfoot[RO,LE]{\footnotesize\CJKfamily{heilight}{高等数学}}
\fancyfoot[RE,LO]{\footnotesize\CJKfamily{heilight}Advanced Mathematics}
\renewcommand{\headrulewidth}{0.4pt} % 注意不用\setlength
%\renewcommand{\footrulewidth}{0pt}
\fancyheadoffset[LE,RO]{0cm}
\fancyfootoffset[LE,RO]{0cm}
% 注意不用\setlength
%\renewcommand{\footrulewidth}{0pt}

%自定义命令
%公式命令自定义
\renewcommand{\d}{{\rm{d}}}
\newcommand{\e}{{\rm{e}}}
\newcommand{\eq}[1][]{\colorbox{eq}{$\displaystyle #1$}}
\renewcommand{\oiint}{\varoiint}
\def\degree{{}^{\circ}}
\newcommand{\di}{\displaystyle}
\newcommand{\ch}{\text{ch}}
\newcommand{\sh}{\text{sh}}
\newcommand{\arccot}{\text{arccot}}

%字体格式自定义
\newcommand{\blackbf}[1][]{{\CJKfamily{heiti}#1}}
\newcommand{\link}[1][]{\hyperref[#1] {\color{超链接}#1}}
\newcommand{\sj}{\vspace*{-1em}}
\newcommand{\eqsj}{\vspace*{-2.5em}}
\newcommand{\kg}{\hspace*{2em}}
\newcommand{\jg}{\vspace*{1em}}
\newcommand{\huo}{\quad \mbox{或} \quad}
\newcommand{\dy}[2][]{{\color{dy}#1}\index{#2@#1}}
\newcommand{\ds}[1][]{\colorbox{文字底色}{\hspace*{-0.25em}#1\hspace*{-0.25em}}}
\newcommand*{\circled}[1]{\lower.7ex\hbox{\tikz\draw (0pt, 0pt)%
		circle (.5em) node {\makebox[1em][c]{\small #1}};}}%带圈数字


%--------------------------------------------------------图框定义---------------------------------------------------------%
%证明和解
\newcommand{\proof}{\vspace*{1em} \noindent  \hspace*{0.2em}  \tcbox[colframe =black, colback =black!10!white,boxrule=0.5mm,size=small,on line]{\color{black}{{ 证}}\hspace*{0.25em}}\hspace{1.5em}}
\newcommand{\solve}{\vspace*{1em} \noindent  \hspace*{0.2em}  \tcbox[colframe =black, colback =black!10!white,boxrule=0.5mm,size=small,on line]{\color{black}{{ 解}}\hspace*{0.25em}}\hspace{1.5em}}
\newcommand{\solveother}{\vspace*{1em} \noindent  \hspace*{0.2em}  \tcbox[colframe =black, colback =black!10!white,boxrule=0.5mm,size=small,on line]{\color{black}{{ 另解}}\hspace*{0.25em}}\hspace{1.5em}}
\newcommand{\errsolve}{\vspace*{1em} \noindent  \hspace*{0.2em}  \tcbox[colframe =red, colback =red!10!white,boxrule=0.5mm,size=small,on line]{\color{red}{{ 错解}}\hspace*{0.25em}}\hspace{1.5em}}
\newcommand{\errreason}{\vspace*{1em} \noindent  \hspace*{0.2em}  \tcbox[colframe =red, colback =red!10!white,boxrule=0.5mm,size=small,on line]{\color{red}{{ 错因}}\hspace*{0.25em}}\hspace{1.5em}}
\newcommand{\solvereason}{\vspace*{1em} \noindent  \hspace*{0.2em}  \tcbox[colframe =ForestGreen
	, colback =ForestGreen!15!white,boxrule=0.5mm,size=small,on line]{\color{ForestGreen}{{ 解析}}\hspace*{0.25em}}\hspace{1.5em}}

%例
\newcommand{\examples}{\vspace*{1em}\noindent  \refstepcounter{examples} \tcbox[colframe =ex, colback =ex!10!white,boxrule=0.5mm,size=small,on line]{\color{ex}{\theexamples}\hspace*{0.3em}}\hspace{1.5em}}
\newcommand{\simpleexamples}{ \noindent  \tcbox[colframe =ex, colback =ex!10!white,boxrule=0.5mm,size=small,on line]{  \color{ex}{例}}\hspace{1.5em}}

%推论
\newcommand{\tl}{\vspace*{1em}\noindent \refstepcounter{tl} \tcbox[colframe =tl, colback =tl!10!white,boxrule=0.5mm,size=small,on line]{\color{tl}{\thetl}\hspace*{0.3em}}\hspace{1.5em}}

%注意
\newcommand{\warn}[1][]{
	\vspace*{0.5em}
	\begin{tcolorbox}[colframe=red!75!black, colback=yellow!10!white,title=注意,fonttitle = ]
		#1
\end{tcolorbox}}
\newcommand{\summarize}[1][]{
	\vspace*{0.5em}
	\begin{tcolorbox}[colframe=white!20!black, colback=white!98!black,title=评注,fonttitle = ]
		#1
\end{tcolorbox}}

%罗列
\newcommand{\myboxm}[2][]{
	\begin{tcolorbox}[on line,
		before skip=-0.5em,arc=0pt,outer arc=0pt,colback=#1!8!white,colframe=#1,
		boxsep=0pt,left=5pt,right=3pt,top=5pt,bottom=5pt,
		boxrule=0pt,leftrule=2.5pt]#2
\end{tcolorbox}}
\newcommand{\myitem}[1]{\myboxm[d3]{#1}}

%小结
\newcommand{\summary}[1][]{
	\vspace*{0.5em}
	\begin{tcolorbox}[title=小结]
		#1
\end{tcolorbox}}

%文本高亮
\newcommand{\highlights}[1][]{\tcbox[colframe =Chocolate , colback =Coral,boxrule=0.5mm,size=small,on line]{\color{white}{#1}}}
\newcommand{\highlight}[2]{\colorbox{#1!17}{\hspace*{-0.25em}#2\hspace*{-0.25em}}}
\newcommand{\highlightdark}[2]{\colorbox{#1!47}{\hspace*{-0.25em}#2\hspace*{-0.25em}}}
%------------------------------------------------------------------------------------------------------------------------%

%文档开始
\begin{document}
	%标题及目录
	\pagenumbering{Roman}
	\clearpage {\pagestyle{empty}}
	\maketitle
	\setcounter{page}{1}
	\tableofcontents
	
	%正文部分
	\newpage
	\setcounter{page}{1}
	\pagenumbering{arabic}
	
	%第一章-函数与极限
	\chapter{火箭发动机推进原理}
\thispagestyle{empty}
\section{概述}
\noindent \textbf{1. 火箭发动机工作过程与能量转化}
{
	\begin{center}
		\begin{tikzpicture}[node distance=1.2cm]
			%定义流程图具体形状
			\node(O) [minimum height=0cm,draw, xshift = -9cm,,inner sep=8pt] {燃烧室};
			\node (B) [minimum height=0cm,draw, xshift = -4.25cm,node distance=3.5cm, inner sep=8pt] {喷管};
			\node (C) [minimum height=0cm,draw, xshift = 0cm,node distance=2cm, inner sep=8pt] {推力传递系统};
			
			%连接具体形状
			\draw[arrows={-Stealth}](-12cm,0cm) -- (O)  node[midway,above=0cm]{推进剂} node[midway, below = 0cm]{化学能};
			\draw[arrows={-Stealth}](O) -- (B) node[midway, above = 0cm]{高温高压燃气} node[midway, below = 0cm]{热能};
			\draw[arrows={-Stealth}](B) -- (C) node[midway, above = 0cm]{高速燃气} node[midway, below = 0cm]{动能};
			\draw[arrows={-Stealth}](C) -- +(3.4cm,0) node[midway,above=0cm]{飞行器} node[midway, below = 0cm]{动能};
		\end{tikzpicture}
		\captionof{figure}{火箭发动机工作过程与能量转化}
		\label{火箭工作过程与能量转化}
	\end{center}
}
\vspace*{-0.5em}
火箭发动机的工作过程和能量转化如图\ref{火箭工作过程与能量转化}所示,实际的误差如表\ref*{火箭实际误差}所示。

\begin{table}[!htb]
	\centering
	\setlength{\tabcolsep}{10mm}{
	\begin{tabular}{|c|c|c|c|}
		\hline
		主要过程 & 分析方法 & 实际过程效率 & 产生原因 \\
		\hline
		燃烧放热 & $Q = \dot{m} \Delta t$ & 燃烧不完全 & 液滴,不均匀等\\
		\hline
		燃气加热 & $\Delta T = UI\dot{m}c_\gamma$ & 热损失 & 壁面传热等\\
		\hline
		膨胀加速 & $\Delta E_k  = \Delta H$ &不完全膨胀 & 分离、摩擦等\\
		\hline
		反作用推进 & $F = \dot{m}I_J$ & 分离,非对称等 & \\
		\hline
	\end{tabular}
	}
	\caption{火箭发动机的实际过程与主要误差}
	\label{火箭实际误差}
\end{table}

\noindent 在分析时,做出以下假设以简化模型\vspace*{-0.5em}
\begin{itemize}
	\item 燃烧室内:完全燃烧,化学能全部转化为热能;\vspace*{-0.5em}
	\item 燃烧室内:忽略热损失,热能全部用于燃气升温;\vspace*{-0.5em}
	\item 喷管内:燃气绝热等熵流动,热能转化为动能。
\end{itemize}
\vspace*{-0.5em}

\section{推进系统的推力}
\subsection{牛顿三大定律}
\vspace*{-1em}
\theorem[牛顿三大定律]
{
	第一运动定律
	\begin{equation}
		\sum \bm{F}_i = \dfrac{\d \bm{v}}{\d t} = 0
	\end{equation}
	\hspace*{1.8em} 第二运动定律
	\begin{equation}
		\bm{F} = \dfrac{\d \bm{p}}{\d t} \qquad \bm{F} = \dfrac{\d m}{\d t}\bm{v} + m \dfrac{\d \bm{v}}{\d t}
	\end{equation}
	\hspace*{1.8em} 第三运动定律
	\begin{equation}
		\bm{F}_{12} = \bm{F}_{21}
	\end{equation}
}

其中,$\bm{F}$为力,$\bm{v}$为速度,$m$为质量,$t$为时间,$\bm{p} = m \bm{v}$为动量。
\vspace*{1em}

\subsection{推力公式}
\noindent \textbf{1. 假设条件(理想情况)}\vspace*{-0.5em}
\begin{enumerate}[\hspace*{1.5em} (1) ]
	\item 一维定常流动;\vspace*{-0.5em}
	\item 外界大气压均匀;\vspace*{-0.5em}
	\item 忽略推进剂入口造成的动量。
\end{enumerate}
\vspace*{1em}

\noindent \textbf{2. 推力公式的推导}
\begin{figure}[!htb]
	\centering
	\includegraphics[width=0.8\linewidth]{pic/推力推导.png}
	\vspace*{-1em}
	\caption{火箭喷射示意图}
	\label{推力推导}
\end{figure}

如图\ref{推力推导}所示,火箭发动机在工作时不仅在内表面会受到燃气压强的作用,其外表面还会受到环境气压的作用,即
\begin{equation}
	\bm{F} = \bm{F}_{\text{in}} + \bm{F}_{\text{out}}
\end{equation}

取$x$方向为正方向,由于火箭是沿轴向对称的,垂直于轴线方向的力相互抵消,则只需考虑喷管的出口部分,由气体受到喷管出口向前的力(和速度方向相反),设$\bm{F}_{\text{in}}$方向为$x$轴正向,则由动量定理
\begin{equation}
		-\bm{F}_{\text{in}} - p_{\text{e}} \bm{n} A_{\text{e}} = m (\bm{u}_{\text{e}} - \bm{u}_{\text{in}})
\end{equation}
即
\begin{equation}
	\bm{F}_{\text{in}} = - m (\bm{u}_{\text{e}} - \bm{u}_{\text{in}}) - p_\e \bm{n} A_\e
\end{equation}
由气压外力与$x$轴正向方向一致,则
\begin{equation}
	\bm{F}_{\text{out}} = p_{\e}
\end{equation}
\vspace*{1em}

\noindent \textbf{3. 推力公式}

\theorem[推力公式]
{
	\quad \vspace*{-1em}
	\begin{equation}
		F = \mathop{\underbrace{\dot{m}u_\e}}_{\scriptsize \mbox{\blue[动量推力]}} + \,\,\, \mathop{\underbrace{(p_\e - p_\a)A_\e}}_{\scriptsize \mbox{\blue[压力推力]}}
	\end{equation}
	其中,\vspace*{-0.5em}
	{
		\begin{enumerate}[\hspace*{1.5em}]
			\item $\bm{F}$ \quad 推力,N \vspace*{-0.5em}
			\item $\dot{m}$ \quad 流量,kg/s \vspace*{-0.5em}
			\item $\bm{u}_\e$ \quad 火箭喷气出口速度, m/s \vspace*{-0.5em}
			\item $\bm{p}_\e$ \quad 火箭喷气出口压强,kPa \vspace*{-0.5em}
			\item $\bm{p}_\a$ \quad 当前高度的大气压强,kPa \vspace*{-0.5em}
			\item $A_\e$ \quad 火箭喷气出口的截面积,$\text{m}^2$ 
		\end{enumerate}
	}
}
\vspace*{1em}

\noindent \textbf{4. 推力公式的讨论}

(1) \hspace*{0.5em}$\dot{m}u_\e$为\dy[动量推力]{DLTL},占推力总值的90\%以上,$(p_\e - p_\a)A_\a$为\dy[压力推力]{YLTL}。

(2) \hspace*{0.5em}进入推力室燃料有初始速度?固体发动机和液体发动机。

(3) \hspace*{0.5em}火箭发动机的推力与飞行器的飞行速度无关,与高度(大气压强)相关。

(4) \hspace*{0.5em}实际飞行中,有外阻力? \quad \blue[有]

(5) \hspace*{0.5em}排气速度不均匀?非沿轴向?\quad 

(6) \hspace*{0.5em}是杏需要假设:完全气体假设?绝热等熵?无热损失和摩擦损失?\quad \blue[不需要]

\vspace*{1em}

\noindent \textbf{5. 推力相关概念}

\defination[特征推力]
{
	\dy[特征推力]{TZTL}(\dy[额定推力]{EDDL}、\dy[发动机设计状态推力]{FDJSJZTTL}):当$p_\e = p_\a$时的发动机推力。对应的高度为火箭发动机的设计高度。即$F^0 = \dot{m}u_\e$
}

\defination[真空推力]
{
	发动机在真空环境下工作时的推力,$P_\a = 0$,即$F_\text{V} = \dot{m}u_\e + P_\e A_\e$
}

\defination[地面推力(海平面推力)]
{
	\quad \vspace*{-1em}
	\begin{equation}
		F_0 = \dot{m}u_\e + A_\e (p_\e - p_{\a 0})
	\end{equation}
	 其中,$p_{\a 0}$为地面的大气压强。
}

\section{喷气速度与流量特性}
\subsection{喷气速度}

\noindent \textbf{1. 基本假设}

(1) \hspace*{0.5em}燃烧室内的燃气参数$T,P_\text{c}, \rho$处处相等,忽略燃烧室速度。

(2) \hspace*{0.5em}喷管中的流动是一维定常、等熵流动,且忽略燃气对喷管壁的传热和摩擦。

(3) \hspace*{0.5em}燃气是定压比热为常数的理想气体(量热完全气体)。

(4) \hspace*{0.5em}燃烧室内化学平衡,喷管内组分不变(冻结流)。
\vspace*{1em}

\noindent \textbf{2. 公式推导}

由假设可得燃气流动的能量守恒方程
\begin{equation}
	H + \dfrac{u^2}{2} = H_0
\end{equation}

在截面处,有
\begin{align*}
	&H_\c + \dfrac{u_\c^2}{2} = H_\e + \dfrac{u_\e^2}{2}\\
	\Rightarrow \quad& H_0 = H_\e + \dfrac{u_\e^2}{2}\\
	\Rightarrow \quad& u_\e = \sqrt{2(H_0 - H_\e)}
\end{align*}

将$H_\e = c_pT_\e, \quad H_0 =c_p T_\f$代入,得
\begin{equation}
	u_\e = \sqrt{2c_p(T_\f - T_\e)} = \sqrt{2c_p T_\f \left(1 - \dfrac{T_\e}{T_\f}\right)}
\end{equation}
对于绝热等熵流动,有
\begin{equation*}
	\dfrac{T_\e}{T_\f} = \left(\dfrac{p_\e}{p_\c}\right)^{\textstyle \frac{k-1}{k}}, \qquad c_p = \dfrac{k}{k-1}\dfrac{R_0}{MW}
\end{equation*}
则可以得到喷气速度方程。

\theorem[喷气速度方程]
{
	\vspace*{-1em}
	\begin{equation}
		u_e = \sqrt{\dfrac{2k}{k-1} \dfrac{R_0}{MW}T_\f \left[1 - \left( \dfrac{p_\e}{p_\c} \right)^{\textstyle \frac{k-1}{k}}\right]}
	\end{equation}
	其中,\vspace*{-0.5em}
	{
		\begin{enumerate}[\hspace*{1.5em}]
			\item $R_0$ \quad 通用气体常数,N \vspace*{-0.5em}
			\item $M$ \quad 燃气平均相对分子质量,g / mol \vspace*{-0.5em}
			\item $k$ \quad 比热比 \vspace*{-0.5em}
			\item $T_\f$ \quad 推进剂绝热燃烧温度,K
		\end{enumerate}
	}
}

\noindent \textbf{2. $u_\e$的影响因素}
\begin{enumerate}[\hspace*{1em}(1) \hspace*{0.5em}]
	\item $T_\f$\quad $t_\f \uparrow$,可转换成动能的热能增加$\rightarrow u_\e \uparrow$
	\item $MW$ \quad $MW \downarrow \, \rightarrow \, u_\e \uparrow$
	\item $k$ \quad $k \uparrow \, \rightarrow\, \sqrt{\dfrac{2k}{k-1}}\, \bigg \downarrow$而$\displaystyle \left[1 - \left(\dfrac{p_\e}{p_\c}\right)^{\textstyle \frac{k-1}{k}}\right] \Bigg \uparrow\, \, \rightarrow$综合考虑,$u_\e$随$k$的增大而略有减小
	\item $\dfrac{p_\e}{p_\c}$ \quad 在$MW,k,T_\f$一定的情况下,$u_\e$随$\dfrac{p_\e}{p_\c}$的减小而增大。其物理意义为燃气在喷管中的膨胀程度,膨胀压强比$\dfrac{p_\e}{p_\c}$越小,燃气膨胀得越充分,有更多的热能转换为动能,喷气速度越高。若$p_\e = 0$,则说明此时燃气的全部热能都转换为动能,喷气速度达到极限值,称为\dy[极限喷气速度]{JXPQSD}。
\end{enumerate}

	\defination[极限喷气速度]
	{
		\vspace*{-1em}
		\begin{equation}
			u_{\text{L}} = \sqrt{2h_\c} = \sqrt{\dfrac{2k}{k-1} R T_\f}
		\end{equation}
	}

\section{飞行器飞行性能指标及分析}

\subsection{飞行器加速度特性}
简化飞行过程,垂直升空的起飞加速度为
\begin{equation}
	a = \dfrac{F}{m_0} - g_0 \quad \Rightarrow \quad \dfrac{a}{g_0} = \dfrac{F}{mg_0} - 1
\end{equation}
\dy[起飞推重比]{QFTZB}定义为$\dfrac{F}{m_0g_0}$,大飞行器约为1.2$\, \sim \,$2.2,小导弹约为$50 \, \sim \, 100$.

\subsection{小结}
航天飞行器或火箭飞行器的主要性能参数:最大速度$V_{\max}$(单级火箭),速度增量$\Delta V$(多级火箭),有效载荷,最大射程,最大飞行高度等
\vspace*{0.5em}

\noindent \textbf{1. 性能最佳}
\begin{enumerate}[\hspace*{1.5em} (1)  ]
	\vspace*{-0.5em}
	\item 有效载荷确定时,缩短完成任务的时间\vspace*{-0.5em}
\end{enumerate}

\vspace*{0.5em}
\noindent \textbf{2. 改进推进系统性能提高飞行器性能途径}
\begin{enumerate}[\hspace*{1.5em} (1)  ]
	\vspace*{-0.5em}
	\item 有效出口速度或比冲:直接影响飞行性能。方法:高能推经剂,高室压及大膨胀比喷管(高空上面级)\vspace*{-0.5em}
	\item 质量数增大:对数影响效果。方法:减小结构质量,增大起飞质量,采用多级推进。\vspace*{-0.5em}
	\item 加大起飞推力,缩短动力飞行时间,减少重力损失\vspace*{-0.5em}
	\item 减小阻力\vspace*{-0.5em}
	\item 最优喷管设计\vspace*{-0.5em}
	\item 发射初速度\vspace*{-0.5em}
	\item 有效喷出速度接近飞行器主飞行段的飞行速度
\end{enumerate}

\vspace*{0.5em}
\noindent \textbf{3. 任务速度}

\defination[任务速度]
{
	\dy[任务速度]{RWSU}:
}


\section{本章总结}
\begin{figure}[!htb]
	\centering
	\begin{tikzpicture}
		\node (A) [draw, inner sep = 5pt]{发动机推力};
		\node (A1) [draw, inner sep = 5pt, right of = A, node distance = 9cm]{\makecell[c]{$F=\dot{m}u_e+A_e(P_e - P_a)$:牛顿定律;\\假设条件,推力公式的推导*及其应用}};
		\node (B) [draw, inner sep = 5pt, below of = A, node distance = 2.4cm]{排气速度};
		\node (B1) [draw, inner sep = 5pt, right of = B, node distance = 9cm]{$\displaystyle u_e = \sqrt{\dfrac{2k}{k-1}\rho_e  P_e \left[\left(\dfrac{P}{P_e}\right)^{\textstyle \frac{2}{k}} - \left(\dfrac{P}{P_e}\right)^{\textstyle \frac{k+1}{k}} \right]}$};
		\node (C) [draw, inner sep = 5pt, below of = B, node distance = 2.4cm]{流量特性};
		\node (C1) [draw, inner sep = 5pt, right of = C, node distance = 9cm]{$\dot{m} = A \sqrt{\dfrac{2k}{k-1} \rho_\e P_\e \left[ \left(\dfrac{P}{P_\text{e}}\right)^{\textstyle\frac{2}{k}} - \dfrac{P}{P_\e}^{\textstyle\frac{k+1}{k}}\right]} \Rightarrow \dot{m} = \dfrac{\Gamma}{\sqrt{RT_\f}}P_\e A_\text{t} = C_\text{D}P_\text{c}A_\text{t} = \dfrac{P_\text{c}A_\text{t}}{c^*}$};
		\node (D) [draw, inner sep = 5pt, below of = C, node distance = 2cm]{推力系数};
		\node (D1) [draw, inner sep = 5pt, right of = D, node distance = 9cm]{$F = C_\text{F}P_\text{c}A_\text{t}$,其物理意义及影响因素};
		\node (E) [draw, inner sep = 5pt, below of = D, node distance = 2.4cm]{总冲和比冲};
		\node (E1) [draw, inner sep = 5pt, right of = E, node distance = 9cm]{$\displaystyle I = \int_0^{t_a} F\, \d t, \quad I_s = \dfrac{I}{M_P} = \dfrac{\displaystyle \int_0^{t_a} F \, \d t }{\displaystyle\int_0^{t_a} \dot{m}\, \d t}$,其物理意义、相关概念及影响因素};
		\node (F) [draw, inner sep = 5pt, below of = E, node distance = 2.4cm]{\makecell[c]{推进剂的混合比\\当量比、余氧系数}};
		\node (F1) [draw, inner sep = 5pt, right of = F, node distance = 9cm]{组元确定情况下,氧化剂和燃料之间的比值,决定燃烧室温度。};
		\node (G) [draw, inner sep = 5pt, below of = F, node distance = 2cm]{飞行器运动};
		\node (G1) [draw, inner sep = 5pt, right of = G, node distance = 9cm]{$F = M\dfrac{\d V}{\d t}\quad \red[V_{\max} - V_0 = \ln \mu] \quad M_0 = M_P + M_e + M_s$};
		\node (H) [draw, inner sep = 5pt, below of = G, node distance = 2cm]{\makecell[c]{飞行器性能\\与发动机性能关系}};
		\node (H1) [draw, inner sep = 5pt, right of = H, node distance = 9cm]{$\displaystyle \Delta u_f = \sum_{i = 1}^{n} I_{s_i} \ln \mu_i$};
		
		\draw[arrows={-Stealth[scale=0.8]}] (A) -- (B);
		\draw[arrows={-Stealth[scale=0.8]}] (B) -- (C);
		\draw[arrows={-Stealth[scale=0.8]}] (C) -- (D);
		\draw[arrows={-Stealth[scale=0.8]}] (D) -- (E);
		\draw[arrows={-Stealth[scale=0.8]}] (E) -- (F);
		\draw[arrows={-Stealth[scale=0.8]}] (F) -- (G);
		\draw[arrows={-Stealth[scale=0.8]}] (G) -- (H);
		\draw (A) -- (A1);
		\draw (B) -- (B1);
		\draw (C) -- (C1);
		\draw (D) -- (D1);
		\draw (E) -- (E1);
		\draw (F) -- (F1);
		\draw (G) -- (G1);
		\draw (H) -- (H1);
	\end{tikzpicture}
	\caption{第1章总结图}
	\label{第1章总结图}
\end{figure}

































	
	%第二章-导数与微分
	\chapter{飞行动力学的基础知识}
\thispagestyle{empty}
\section{地球的运动及形状}
\subsection{地球的运动}
\vspace*{-1em}
\defination[地球运动]
{\dy[地球运动]{DQYD}分为\dy[质心运动]{ZXYD}(公转)和\dy[绕心运动]{RXYD}(自转)\vspace*{-0.5em}
{
\begin{itemize}
	\item \dy[公转]{GZ}:以近圆轨道绕太阳公转,周期为1年。\vspace*{-0.5em}
	\item \dy[自转]{ZZ}:地球自转轴成为\dy[地轴]{DZ},地球绕地轴自西向东匀速转动。
\end{itemize}
}
}
\noindent 地球运动的基本参数:\vspace*{-0.5em}
\begin{itemize}
	\item 地球公转周期:$T = 365.25636$个平日\vspace*{-0.5em}
	\item 地球自转周期:$t = 86164.099\,$s $=$ 23$\,$h$\,$56$\,$m$\,$4.099$\,$s\vspace*{-0.5em}
	\item 地球自转角速度:$\omega_e = \dfrac{2 \pi}{86164.1} = 7.292115\times 10^{-5} \text{rad/s}$
\end{itemize}
\vspace*{-0.5em}

\theorem[地球运动的假设]
{在导弹飞行时间内(飞行时间短),可认为\textcolor{red}{地轴在惯性空间指向不变},且\textcolor{red}{地球作匀速直线运动}。但实际上地轴的指向是变化的,存在极移(物质变化)、进动(太阳引力)和章动(月球引力),地球本身也存在加速度。}

其中,进动是指在太阳引力作用下,地轴会绕一个轴作周期约为27500年的圆周运动,类似于不平衡的陀螺,如图所示。而章动是指在月球引力的作用下,地轴并不是做完美的圆周运动,会上下浮动,浮动的周期约为18.6年,如图所示。



\subsection{地球的形状}
\vspace*{-1em}
\defination[地球形状]
{实际应用中采用简单形状描述地球:\vspace*{-0.5em}
{
\begin{itemize}
	\item \red[均质圆球]:$R=6371004$m\vspace*{-0.5em}
	\item \red[总地球椭球体]:$a_e = 6378149\text{m},\,\, b_e = 6356775\text{m}$,地球扁率(离心率)$\alpha_e = \dfrac{(a_e - b_e)}{a_e} = \dfrac{1}{298.257}$
\end{itemize}
}
}


\section{地球大气}
\subsection{地球大气分层}
\vspace*{-1em}
\defination[地球大气分层]
{\dy[地球大气分层]{DQDQFC}是按大气温度分层:\vspace*{-0.5em}
{
\begin{itemize}
	\item \dy[对流层]{DLC}:$0\, \sim \, 18\, \text{km} / 8 \, \text{km}$,75\%大气质量,95\%水汽;\vspace*{-0.5em}
	\item \dy[平流层]{PLC}:$\sim 50 \, \text{km}$,同温层$+$臭氧层,温度升高,大气密度和压强降低,只有地表的0.08\%.\vspace*{-0.5em}
	\item \dy[中间层]{ZJC}:$50\, \text{km} \, \sim \, 90\,\text{km}$,温度降低\vspace*{-0.5em}
	\item \dy[热成层]{RCC}:$90 \, \text{km} \, \sim \, 500\, \text{km}$,温度升高\vspace*{-0.5em}
	\item \dy[外逸层]{WYC}:$>500\, \text{km}$.\vspace*{-0.5em}
\end{itemize}
}
对于运载火箭,一般只考虑90km以下的大气影响。
}

\noindent 大气的物理性质分布如下:

\begin{enumerate}
	\item \textbf{温度分布}\\
	\hspace*{2em}温度随高度的变化曲线在$0 \, \sim \, 80 \, \text{km}$内可以由一系统的折线表示:
	\begin{equation}
		T(h) = T_0 + Gh
	\end{equation}
	
	对于不同的层,相应的参数$G$取值不同。
	
	\item \textbf{压强分布}\\
	\hspace*{2em}大气的实际压强与气温一样变化一场复杂,为了得到一般意义的标准分布,常采用“大气垂直平衡”假设,即认为大气在铅锤方向是静止的,处于力的平衡状态。由$p=Rg_0\rho T$得:
	\begin{equation}
		p(h) = p_0 \e^{\textstyle - \frac{1}{R}\int_0^h \frac{\d h}{T}}
	\end{equation}
	\proof 由“大气垂直平衡”假设,可以得到
	\begin{equation*}
		(p + \d p)\d S + \rho g_0 \d S \d h = p \d S
	\end{equation*}
	化简得到
	\begin{equation*}
		\d p + \rho g \d h = 0
	\end{equation*}
	代入$p=Rg_0\rho T$可得
	\begin{equation*}
		\dfrac{\d p}{p} = - \dfrac{g}{Rg_0T}\, \d h
	\end{equation*}
	积分可得
	\begin{equation}
		\frac{\ln p}{\ln p_0} = \int_{h_0}^h - \frac{g}{Rg_0T}\, \d h \quad \Rightarrow \quad \frac{p}{p_0} = \e^{\textstyle - \frac{1}{R}\int_{h_0}^h \frac{g}{g_0 T}\, \d h}
	\end{equation}
	\item \textbf{密度分布}\\
	\hspace*{2em}由气体状态方程,已知温度$T$和气压$p$,可得
	\begin{equation}
		\dfrac{\rho}{\rho_0} = \frac{pT_0}{p_0 T} = \frac{T_0}{T} \e ^{\textstyle -\frac{1}{R}\int_0^H \frac{\d H}{T}},\, H = \dfrac{1}{g_0}\int_0^h g \, \d h
	\end{equation}
	其中,$H$为\dy[地势高度]{DSGD},相当于具有同等势能的均匀重力场中的高度,其总小于几何高度$h$,但在高度不打时二者差别较小。若认为在某一高度范围内为等温过程,则:
	\begin{equation}
		\dfrac{\rho_2}{\rho_1} = \e^{\textstyle - \frac{H_2 - H_1}{H_{M_1}}}, \, H_{M_1} = RT_1
	\end{equation}
	其中,$H_{M_1}$称为\dy[基准高]{JZG}或\dy[标高]{BG}。
	
	\hspace*{2em} 如果假设在$0 \, \sim \, 80\, \text{km}$内为恒温过程,则有
	\begin{equation}
		\frac{p}{p_0} = \dfrac{\rho}{\rho_0} = \e^{-\beta h}, \quad \beta = \frac{1}{H_{\text{MCP}}}=\frac{1}{7.11\text{km}}
	\end{equation}
	这个模型称为\dy[指数大气模型]{ZSDQMX}。
\end{enumerate}

\subsection{标准大气}
导弹飞行状态随与随高度变化的大气参数有密切关系(压强、密度、温度及音速等)。

\section{坐标系间的方向余弦阵及矢量导数的关系}
由于不同坐标系对同一物理量的描述形式或者坐标投影不同,为了在统一坐标系中描述飞行器的运动,存在不同物理量到基准坐标系的转换需求。

\defination[坐标转换]
{
	同一矢量在不同坐标系下的坐标不同,将矢量$S_a$坐标系中的坐标转换到$S_b$坐标系称为\dy[坐标系转换]{ZBXZH}。
}

\subsection{坐标系之间的方向余弦阵}
\vspace*{-1em}
\defination[坐标系]
{
	\dy[笛卡尔坐标系]{DKEZBX}:由原点及过原点的两(三)条具有方向的坐标轴组成,坐标轴上的度量单位通常相等。\\
	\hspace*{2em} \dy[球坐标系]{QZBX}:球坐标系由原点、方位角、仰角和距离构成。\\
	\hspace*{2em} \dy[极坐标系]{JZBX}:极坐标系由极点、极径及极角构成。
}

考虑两个直角坐标系:$P:O_p-\bm{x}_p\bm{y}_p\bm{z}_p,\quad Q:O_q-\bm{x}_q\bm{y}_q\bm{z}_q$,定义$P_Q$为$Q$系中单位矢量$E_q$变换到$P$系中单位矢量$E_p$的转换矩阵,由
\begin{equation}
	E_p = P_QE_q, \qquad E_p =
	\begin{bmatrix}
		\bm{x}_p^0 & \bm{y}_p^0 & \bm{z}_p^0
	\end{bmatrix}^{\text{T}}
\qquad 
	E_q = 
	\begin{bmatrix}
		\bm{x}_q^0 & \bm{y}_q^0 & \bm{z}_q^0
	\end{bmatrix}^{\text{T}}
\end{equation}
由于
\begin{equation*}
	E_q \cdot E_q^{\text{T}} = 
	\begin{bmatrix}
		\bm{x}_q^0\\
		\bm{y}_q^0\\
		\bm{z}_q^0
	\end{bmatrix}
	\begin{bmatrix}
	\bm{x}_q^0 & \bm{y}_q^0 & \bm{z}_q^0
	\end{bmatrix}
	=
	\begin{bmatrix}
		\bm{x}_q^0 \cdot \bm{x}_q^0 & \bm{x}_q^0 \cdot \bm{y}_q^0 & \bm{x}_q^0 \cdot \bm{z}_q^0 \\ 
		\bm{y}_q^0 \cdot \bm{x}_q^0 & \bm{y}_q^0 \cdot \bm{y}_q^0 & \bm{y}_q^0 \cdot \bm{z}_q^0 \\ 
		\bm{z}_q^0 \cdot \bm{x}_q^0 & \bm{z}_q^0 \cdot \bm{y}_q^0 & \bm{z}_q^0 \cdot \bm{z}_q^0 
	\end{bmatrix}
	=
	\begin{bmatrix}
		1 & 0 & 0 \\
		0 & 1 & 0 \\
		0 & 0 & 1
	\end{bmatrix} = E
\end{equation*}
那么
\begin{equation}
	P_Q = E_p \cdot E_q^{\text{T}} = 
	\begin{bmatrix}
		\bm{x}_p^0 \cdot \bm{x}_q^0 & \bm{x}_p^0 \cdot \bm{y}_q^0 & \bm{x}_p^0 \cdot \bm{z}_q^0 \\ 
		\bm{y}_p^0 \cdot \bm{x}_q^0 & \bm{y}_p^0 \cdot \bm{y}_q^0 & \bm{y}_p^0 \cdot \bm{z}_q^0 \\ 
		\bm{z}_p^0 \cdot \bm{x}_q^0 & \bm{z}_p^0 \cdot \bm{y}_q^0 & \bm{z}_p^0 \cdot \bm{z}_q^0 
	\end{bmatrix}
	=
	\begin{bmatrix}
		\cos(\bm{x}_p, \bm{x}_q) & \cos(\bm{x}_p, \bm{y}_q) & \cos(\bm{x}_p, \bm{z}_q)\\
		\cos(\bm{y}_p, \bm{x}_q) & \cos(\bm{y}_p, \bm{y}_q) & \cos(\bm{y}_p, \bm{z}_q)\\
		\cos(\bm{z}_p, \bm{x}_q) & \cos(\bm{z}_p, \bm{y}_q) & \cos(\bm{z}_p, \bm{z}_q)
	\end{bmatrix}
	\triangleq
	\begin{bmatrix}
		a_{ij}
	\end{bmatrix}
	\quad i,j=1,2,3
	\label{方向余弦阵}
\end{equation}

公式\eqref{方向余弦阵}称为\dy[方向余弦阵]{FXYXZ},同理可以得到
\begin{equation}
	Q_p = E_q \cdot E_p^{\text{T}}
	\begin{bmatrix}
		\bm{x}_q^0 \cdot \bm{x}_p^0 & \bm{x}_q^0 \cdot \bm{y}_p^0 & \bm{x}_q^0 \cdot \bm{z}_p^0 \\ 
		\bm{y}_q^0 \cdot \bm{x}_p^0 & \bm{y}_q^0 \cdot \bm{y}_p^0 & \bm{y}_q^0 \cdot \bm{z}_p^0 \\ 
		\bm{z}_q^0 \cdot \bm{x}_p^0 & \bm{z}_q^0 \cdot \bm{y}_p^0 & \bm{z}_q^0 \cdot \bm{z}_p^0 
	\end{bmatrix}
	=
	\begin{bmatrix}
		\cos(\bm{x}_q, \bm{x}_p) & \cos(\bm{x}_q, \bm{y}_p) & \cos(\bm{x}_q, \bm{z}_p)\\
		\cos(\bm{y}_q, \bm{x}_p) & \cos(\bm{y}_q, \bm{y}_p) & \cos(\bm{y}_q, \bm{z}_p)\\
		\cos(\bm{z}_q, \bm{x}_p) & \cos(\bm{z}_q, \bm{y}_p) & \cos(\bm{z}_q, \bm{z}_p)
	\end{bmatrix}
\end{equation}
又
\begin{equation*}
	P_q^{-1} = Q_p = E_q \cdot E_p^{\text{T}} = \left(E_p \cdot E_q^{\text{T}} \right)^{\text{T}} = P_q^{\text{T}}
\end{equation*}
这说明:\red[方向余弦阵是正交矩阵],那么方向余弦阵只有\blue[三个独立变量]。

\defination[初等转换矩阵]
{
	当两个坐标系之间存在平行轴的时候,此时的方向余弦矩阵称为\dy[初等转换矩阵]{CDZHJZ}。分别绕$x,y,z$轴旋转的初等转换矩阵为
	\begin{equation}
		M_x(\theta) = 
		\begin{bmatrix}
			1 & 0 & 0\\
			0 & \cos \theta & \sin \theta \\
			0 & -\sin \theta & \cos \theta
		\end{bmatrix}
		\qquad
		M_y(\theta) =
		\begin{bmatrix}
			\cos \theta & 0 & -\sin\theta\\
			0 & 1 & 0\\
			\sin \theta & 0 & \cos \theta
		\end{bmatrix}
		\qquad
		M_z (\theta)= 
		\begin{bmatrix}
			\cos \theta & \sin \theta & 0 \\
			-\sin \theta & \cos \theta & 0 \\
			0 & 0 & 1
		\end{bmatrix}
	\end{equation}
}

\theorem[转换矩阵的传递性]
{
	对于直角坐标系$P,Q,S$,它们相互之间的转换矩阵为$P_Q,S_Q,S_P$,则\vspace*{-1em}
	\begin{equation}
		S_Q = S_P \cdot P_Q, \quad P_Q = P_S \cdot S_Q, \quad P_S = P_Q\cdot Q_S
	\end{equation}
}

\subsection{坐标系转换阵的欧拉角表示方法}
\vspace*{-1em}
\defination[转换矩阵的欧拉角表示]
{
	将坐标系视作刚体,则经过三次旋转后可以与另一个坐标系重合,因此可以用这三个旋转角(\dy[欧拉角]{OLJ})作为独立变量,来描述方向余弦阵。
}

\begin{figure}[!htb]
	\centering
	\includegraphics[width=0.3\linewidth]{pic/欧拉角.jpg}
	\caption{坐标系旋转转换}
	\label{欧拉角}
\end{figure}

例如,如图\ref{欧拉角}所示,$o-\bm{x}_q\bm{y}_q\bm{z}_q$分别绕$z,y,x$轴旋转三次,得到
\begin{equation*}
	\begin{split}
		o-\bm{x}_q\bm{y}_q\bm{z}_q \, \xrightarrow{\quad \textstyle M_z[\xi] \quad } o - x_1 y_1 \bm{z}_q \, \xrightarrow{\quad \textstyle M_y[\eta] \quad } o - \bm{x}_p y_1 \bm{z}_1\ \, \xrightarrow{\quad \textstyle M_x[\zeta] \quad } \,o - \bm{x}_p \bm{y}_p \bm{z}_p
	\end{split}
\end{equation*}
即
\begin{equation}
	P_Q = M_x[\xi] \cdot  M_y[\eta]\cdot  M_z[\zeta]
\end{equation}
进一步代入,得
\begin{equation}
	P_Q = 
	\begin{bmatrix}
		\cos \xi \cos \eta & \sin \xi \cos \eta & - \sin \eta \\
		\cos \xi \sin \eta \sin \zeta - \sin \xi \cos \zeta & \sin \xi \sin \eta \sin \zeta + \cos \xi \cos \zeta & \cos \eta \sin \zeta \\
		\cos \xi \sin \eta \cos \zeta + \sin \xi \sin \zeta & \sin \zeta \sin \eta \cos \zeta - \cos \xi \sin \zeta & \cos \eta \cos \zeta
	\end{bmatrix}
\end{equation}

\subsection{坐标间矢量导数的关系}
\vspace*{-1em}
\defination[矢量导数]
{
	\dy[矢量导数]{SLDS}:同一矢量在不同坐标系有不同的投影,其导数的数值不同。
}

设坐标系$O:o-xyz$相对于坐标系$P:o_p-x_py_pz_p$有角速率$\omega.$ 矢量$\bm{a}$在坐标系$O$的投影为
\begin{equation}
	\bm{a} = a_x \bm{x}^0 + a_y \bm{y}^0 + a_z \bm{z}^0
\end{equation}
取微分,得
\begin{equation}
	\dfrac{\d \bm{a}}{\d t} = \dfrac{\d a_x}{\d t}\bm{x}^0 + \dfrac{\d a_y}{\d t} \bm{y}^0 + \dfrac{\d a_z}{\d t} \bm{z}^0 + a_x \dfrac{\bm{x}^0}{\d t} + \dfrac{}{分母}
\end{equation}

\section{常用坐标系及其相互转换}
\subsection{常用坐标系}
常用的坐标系分类
\begin{itemize}
	\item 取地心为原点:地心惯性坐标系,地心固连坐标系\vspace*{-0.5em}
	\item 取发射点为原点:发射坐标系,发射惯性坐标系\vspace*{-0.5em}
	\item 取对象质心为原点:体坐标系,速度坐标系,半速度坐标系
\end{itemize}

\begin{enumerate}
	\item \dy[地心惯性坐标系$I$]{DXGXZBX}:$O_E - X_I Y_I Z_I$(静系)
	\vspace*{1em}
	
	\begin{minipage}{0.6\linewidth}
		\centering
		\setlength{\tabcolsep}{12mm}{
		\begin{tabular}{cl}
			\hline
			原点 & 地心$O_E$\\
			\hline
			$X$轴 & $X_I$:平春分点\\
			\hline
			$Y$轴 & $Y_I$:右手法则\\
			\hline
			$Z$轴 & $Z_I$:地球自转轴\\
			\hline
		\end{tabular}
	}
	\end{minipage}

	\vspace*{1.5em}
	\item \dy[地心坐标系$E$]{DXZBX}:$O_E - X_E Y_E Z_E$(动系)
	\vspace*{1em}
	
	\begin{minipage}{0.6\linewidth}
		\centering
		\setlength{\tabcolsep}{12mm}{
		\begin{tabular}{cl}
			\hline
			原点 & 地心$O_E$\\
			\hline
			$X$轴 & $X_E$:给定子午线\\
			\hline
			$Y$轴 & $Y_E$:右手法则\\
			\hline
			$Z$轴 & $Z_E$:地球自转轴\\
			\hline
		\end{tabular}
	}
	\end{minipage}
	
	\vspace*{1.5em}
	\item \dy[发射坐标系$G$]{FSZBX}:$O - xyz$(动系)
	\vspace*{1em}
	
	\begin{minipage}{0.6\linewidth}
		\centering
		\setlength{\tabcolsep}{8mm}{
			\begin{tabular}{cl}
				\hline
				原点 & 发射点$O$\\
				\hline
				$X$轴 & $x$:发射水平面内指向瞄准方向\\
				\hline
				$Y$轴 & $y$:发射水平面指向上方\\
				\hline
				$Z$轴 & $z$:右手法则\\
				\hline
			\end{tabular}
		}
	\end{minipage}
	\vspace*{0.5em}
	
	对于球模型:$\varphi_0$:地心纬度,$\alpha_0$:发射方位角。\\
	对于椭球模型:$\varphi_0$:地心纬度,$\alpha_0$:发射方位角。
	
	\item \dy[发射惯性坐标系$A$]{FSGXZBX}:$O_A - x_A y_A z_A$(静系)
	\vspace*{1em}
	
	\begin{minipage}{0.6\linewidth}
		\centering
		\setlength{\tabcolsep}{8mm}{
			\begin{tabular}{cl}
				\hline
				原点 & 发射点$O_A$,起飞瞬间与发射点$O$重合\\
				\hline
				$X$轴 & $x_A$:起飞瞬间的发射水平面内指向瞄准方向\\
				\hline
				$Y$轴 & $y_A$:起飞瞬间的发射水平面指向上方\\
				\hline
				$Z$轴 & $z_A$:右手法则\\
				\hline
			\end{tabular}
		}
	\end{minipage}
	
	\vspace*{1.5em}
	\item \dy[平移坐标系$A$]{PYZBX}:$o_T - x_T y_T z_T$(动系)
	\vspace*{1em}
	
	\begin{minipage}{0.6\linewidth}
		\centering
		\setlength{\tabcolsep}{8mm}{
			\begin{tabular}{cl}
				\hline
				原点 & 发射点$O_A$,起飞瞬间与发射点$O$重合\\
				\hline
				$X$轴 & $x_A$:起飞瞬间的发射水平面内指向瞄准方向\\
				\hline
				$Y$轴 & $y_A$:起飞瞬间的发射水平面指向上方\\
				\hline
				$Z$轴 & $z_A$:右手法则\\
				\hline
			\end{tabular}
		}
	\end{minipage}
	
	\vspace*{1.5em}
	\item \dy[弹体坐标系$B$]{DTZBX}:$o_1 - x_1 y_1 z_1$(动系)
	\vspace*{1em}
	
	\begin{minipage}{0.6\linewidth}
		\centering
		\setlength{\tabcolsep}{8mm}{
			\begin{tabular}{cl}
				\hline
				原点 & 弹体质心$O_1$\\
				\hline
				$X$轴 & $x_1$:沿弹体对称轴指向头部\\
				\hline
				$Y$轴 & $y_1$:位于主对称面内,垂直于$X$轴\\
				\hline
				$Z$轴 & $z_1$:右手法则,顺着发射方向看向右为正\\
				\hline
			\end{tabular}
		}
	\end{minipage}
	
	\vspace*{1.5em}
	\item \dy[速度坐标系$V$]{SDZBX}:$o_1 - x_v y_v z_v$(动系)
	\vspace*{1em}
	
	\begin{minipage}{0.6\linewidth}
		\centering
		\setlength{\tabcolsep}{8mm}{
			\begin{tabular}{cl}
				\hline
				原点 & 弹体质心$O_1$\\
				\hline
				$X$轴 & $x_v$:弹体的速度方向\\
				\hline
				$Y$轴 & $y_v$:位于主对称面内,垂直于$X$轴\\
				\hline
				$Z$轴 & $z_v$:右手法则\\
				\hline
			\end{tabular}
		}
	\end{minipage}

	\vspace*{1.5em}
	\item \dy[半速度坐标系$H$]{BSDZBX}:$o_1 - x_h y_h z_h$(动系)
	\vspace*{1em}
	
	\begin{minipage}{0.6\linewidth}
		\centering
		\setlength{\tabcolsep}{5mm}{
			\begin{tabular}{cl}
				\hline
				原点 & 弹体质心$O_1$\\
				\hline
				$X$轴 & $x_h$:弹体的速度方向\\
				\hline
				$Y$轴 & $y_h$:包含速度矢量的铅锤面内垂直于$x_h$,向上为正\\
				\hline
				$Z$轴 & $z_h$:右手法则\\
				\hline
			\end{tabular}
		}
	\end{minipage}
\end{enumerate}

\subsection{各坐标系间的转换关系}
\begin{enumerate}
	\item $I \to E$:$Z$轴重合,$X$轴处于赤道面内相差角$\varOmega_G = \omega_e \cdot t$(时角),则转换矩阵为
	\begin{equation}
		E_I = M_z [\varOmega_G]
	\end{equation}
	
	\item $E \to G$:设地球为圆球,发射点可用经纬度$(\lambda_0, \varphi_0)$来描述,即
	\begin{equation}
		G_E = M_y\big[-(90\degree + \alpha_0)\big]\cdot M_x[\varphi_0] \cdot M_z \big[-(90 \degree - \lambda_0)\big]
	\end{equation}

	\item $G \to B$:设$G \to B$转序为$321$,旋转角为$\varphi, \psi ,\gamma$,则
	\begin{equation}
		B_G = M_x[\gamma]\cdot M_y[\psi] \cdot M_z[\varphi]
	\end{equation}
	\hspace*{1em}\defination[新的欧拉角{\RMN[1]}]
	{
		\dy[俯仰角$\varphi$]{FYJ}:轴$ox_1$在发射面$xoy$上的投影与$x$的夹角,投影在$x$的上方为正。\\
		\hspace*{2.2em}\dy[偏航角$\psi$]{PHJ}:轴$ox_1$与发射面$xoy$的夹角,$ox_1$在发射面左边为正。\\
		\hspace*{2.2em}\dy[滚动角$\gamma$]{GDJ}:旋转角速度矢量与$ox_1$轴方向一致时为正。
	}

	\item $G \to V$:设$G \to V$转序为$321$,旋转角为$\theta, \sigma ,\nu$,则
	\begin{equation}
		V_G = M_x[\nu]\cdot M_y[\sigma] \cdot M_z[\theta]
	\end{equation}
	\hspace*{1em}\defination[新的欧拉角{\RMN[2]}]
	{
		\dy[速度倾角$\theta$]{SDQJ}:轴$ox_v$在发射面$xoy$上的投影与$x$的夹角,投影在$x$的上方为正。\\
		\hspace*{2.2em}\dy[航迹偏航角$\sigma$]{HJPHJ}:轴$ox_v$与发射面$xoy$的夹角,$ox_v$在发射面左边为正。\\
		\hspace*{2.2em}\dy[倾侧角$\nu$]{QCJ}:旋转角速度矢量与$ox_v$轴方向一致时为正。
	}

	
	\item $V \to B$:由于速度系$y_v$轴位于主对称面内,因此$V \to B$只有两个欧拉角,设为$\alpha,\beta$,设定$V \to B$的转序为23,则
	\begin{equation}
		B_V = M_z[\alpha] \cdot M_y [\beta]
	\end{equation}
	
	\hspace*{1em}\defination[新的欧拉角{\RMN[3]}]
	{
		\dy[测滑角$\beta$]{CHJ}:速度轴$x_v$与弹体主对称面的夹角,右方为正。\\
		\hspace*{2.2em}\dy[攻角$\alpha$]{GJ}:速度轴$x_v$与在主对称面投影与弹体纵轴的夹角,下方为正。
	}
	
	
	\item $A \to G$:由于在发射时刻$A,G$坐标系重合,因此其转换角与飞行时间$t$相关,发射系绕地轴旋转角为$\omega_e t$,则
	\begin{equation}
		G_A = M_y[-\alpha_0] \cdot M_z[-\phi_0] \cdot M_x[\omega_e t] \cdot M_z[\phi_0] \cdot M_y[\alpha_0]
	\end{equation}
	如果火箭飞行时间较短,认为$\omega_e t$为小量,在转换矩阵中取其一次项,则
	\begin{equation}
		G_A = 
		\begin{bmatrix}
			1 & \omega_{ez}t & -\omega_{ey} t\\
			-\omega_{ez}t & 1 &\omega_{ex} t \\
			\omega_{ey} t & -\omega_{ex}t & 1 
		\end{bmatrix}
	\end{equation}
	其中,
	\begin{equation}
		\begin{cases}
			\,\omega_{ex} = \omega_e \cos \phi_0 \cos \alpha_0 \\
			\,\omega_{ey} = \omega_e \sin \phi_0\\
			\,\omega_{ez} = - \omega_e \cos \phi_0 \sin \alpha_0
		\end{cases}
	\end{equation}
\end{enumerate}

\subsection{常用欧拉角的联系方程}
由于各坐标系之间定义有欧拉角,必然存在一定的联系。

\begin{enumerate}
	\item \textbf{$B,G,V$之间的联系}
	\begin{equation}
		V_G[\theta, \sigma, \nu] = V_B [\alpha, \beta] \cdot B_G [\varphi, \psi, \gamma]
	\end{equation}
	利用姿态角$\varphi, \psi ,\gamma$和攻角侧滑角$\alpha, beta$来确定速度角$\theta, \sigma , \nu$.如果侧向角为小量,则
	\begin{equation}
		\begin{cases}
			\, \sigma= \psi \cos \alpha + \gamma \sin \alpha - \beta \\
			\, \nu = - \psi \sin \alpha + \gamma \cos \alpha \\
			\, \theta = \varphi - \alpha
		\end{cases}
	\end{equation}
	若功角$\alpha$也为小量,则
	\begin{equation}
		\begin{cases}
			\, \theta= \varphi - \alpha \\
			\, \sigma = \psi - \beta \\
			\, \nu = \gamma
		\end{cases}
	\end{equation}

\item \textbf{$B,G,V$之间的联系}\\
	\hspace*{2em}弹体相对发射系姿态角为$\varphi, \psi, \gamma$,相对平移系姿态角为$\varphi_T, \psi_T, \gamma_T$,相对系与发射惯性系矩阵$G_T$,有
	\begin{equation}
		B_T[\varphi_T, \psi_T, \gamma_T] = B_G [\varphi, \psi, \gamma] \cdot G_T[\alpha_0, \phi_0, \omega_e t]
	\end{equation}
	由此可利用姿态角$\varphi,\psi, \gamma $和飞行时间$t$来确定惯性姿态角$\varphi_T, \psi_T, \gamma_T$。如果认为侧向角及时间为小量,则
	\begin{equation}
		\begin{cases}
			\, \varphi_T = \varphi + \omega_{ez}t\\
			\, \psi_T = \psi + (\omega_{ey}\cos \varphi - \omega_{ex}\sin \varphi)\cdot t\\
			\, \gamma_t = \gamma + (\omega_{ey} \sin \varphi + \omega_{ex}\cos \varphi)\cdot t
		\end{cases}
	\end{equation}

\end{enumerate}

\section{变质量力学基本原理}
\subsection{变质量质点基本方程}
	\textbf{1. 火箭质量}
	
	由于发动机工作,飞行中有大量质点从发动机中喷出,因此必须规定一个表面,以此表面内质量作为火箭的总质量。通常此表面取为火箭外表面和发动机喷口断面。因此火箭是一个存在质点流动的变质量物体。
	
	\vspace*{1em}
	\textbf{2. 变质量质点基本方程}
	
	设当前质量为$m(t)$,当前绝对速度为$\bm{V}$,则其动量为
	\begin{equation}
		\bm{Q}(t) = m(t) \cdot \bm{V}
	\end{equation}
	质点在$\d t$时间内,有外界作用力$\bm{F}$,且向外以相对速度$\bm{V}_r$喷射质量元$-\d m$。设质点速度变化为$\d \bm{V}$,则有
	\begin{equation}
		\begin{split}
			\bm{Q}(t+\d t) &= \big(m - (-\d m)\big)\cdot \big(\bm{V} + \d \bm{V}\big)+(-\d m) \cdot (\bm{V} + \bm{V}_r)\\
			& = m(t)(\bm{V} + \d \bm{V}) - \d m \bm{V}_r
		\end{split}
	\end{equation}
	则
	\begin{equation}
		\begin{split}
			\d \bm{Q} &= \bm{Q}(t + \d t) - \bm{Q}(t) \\
			&= m(\bm{V} + \d \bm{V}) - \d m \bm{V}_r - m \cdot \bm{V} \\
			& = m \d \bm{V} -\d m \bm{V}_r
		\end{split}
	\end{equation}
	对常质量质点有动量定律
	\begin{equation}
		\d \bm{Q} = \bm{F} \d t
	\end{equation}
	所以
	\begin{equation}
		\bm{F} \d t = m \d \bm{V} -\d m \bm{V}_r
	\end{equation}
	由此可以得到
	
	\theorem[变质量质点基本方程(密歇尔斯基方程)\index{BZLZDJBFC@变质量质点基本方程}\index{MXESJFC@密歇尔斯基方程}]
	{
		\vspace*{-1em}
		\begin{equation}
			m \dfrac{\d \bm{V}}{\d t} = \bm{F}+\dfrac{\d m}{\d t} \bm{V}_r = \bm{F} + \textcolor{red}{\bm{P}_r}
		\end{equation}
		其中,$\bm{F}$为牛顿第二定律的外力;$\textcolor{red}{\bm{P}_r}$为喷射反作用力,加速力。
	}
	假设质点不受外力作用,且假设有$\bm{V}_r$与$\bm{V}$反向,则
	\begin{equation}
		m \dfrac{\d v}{\d t} = - \dfrac{\d m}{\d t} v_r \quad \Rightarrow \quad \d v = - \d v_r \dfrac{\d m}{m}
	\end{equation}
	再假设质量元喷射速度为常值,则有质点速度为
	\begin{equation}
		v = v_0 + v_r \ln \dfrac{m_0}{m}
	\end{equation}
	设初始速度为0,则可以得到
	
	\theorem[齐奥尔科夫斯基公式\index{QAEKFSJGS@齐奥尔科夫斯基公式}]
	{
		\quad \vspace*{-1em}
		\begin{equation}
			v_k = v_r \ln \dfrac{m_0}{m_k}
		\end{equation}
		设\dy[结构比]{JGB}为$\mu_k = \dfrac{m_k}{m_0}$,则
		\begin{equation}
			v_k = -v_r \ln \mu_k
		\end{equation}
		其中,$v_k$为理想速度。
	}

\subsection{变质量质点系的运动方程}
	对于变质量质点系,除了质点随物体作牵连运动外,在物体内部还有相对运动,会对物体运动有影响,应用密歇尔斯基方程存在近似性。
	
	在惯性参考系内,质点系总外力为$\bm{F}_s$,总力矩为$\bm{M}_s$,则运动方程为
	\begin{align}
		\bm{F}_s &= \sum_{i=1}^N m_i \dfrac{\d^2 \bm{r}_i}{\d t^2} \\[0.5em]
		\bm{M}_s &= \sum_{i=1}^N m_i\bm{r}_i \times \dfrac{\d^2 \bm{r}_i}{\d t^2} 
	\end{align}
	对于连续质点系(物体)的运动方程,则有
	\begin{align}
		\bm{F}_s &= \int_m \dfrac{\d^2 \bm{r}}{\d t^2} \, \d m \label{相对牛二}\\[0.5em]
		\bm{M}_s &= \int_m \bm{r} \times \dfrac{\d^2 \bm{r}}{\d t^2} \, \d m
	\end{align}

	\textbf{1. 质心运动方程}
	
	任一质点在惯性系中的矢径为
	\begin{equation}
		\bm{r} = \bm{r}_{c.m} + \bm{\rho}
	\end{equation}
	其中$\bm{r}_{c.m}$为质心的矢径,$\bm{\rho}$是相对位置矢径,则有绝对加速度为
	\begin{equation}
		\dfrac{\d^2 \bm{r}}{\d t^2} = \dfrac{\d^2 \bm{r}_{c.m}}{\d t^2} + 2 \bm{\omega}_T \times \dfrac{\delta \bm{\rho}}{\delta t} + \dfrac{\delta^2 \bm{\rho}}{\delta t^2} + \dfrac{\bm{\omega}_T}{\d t}\times \bm{\rho} + \bm{\omega}_T \times (\bm{\omega}_T \times \bm{\rho}) 
		\label{绝对加速度}
	\end{equation}
	将\eqref{绝对加速度}代入\eqref{相对牛二},可以得到
	\begin{align*}
		\bm{F}_s & = \int_m \Bigg[\dfrac{\d \bm{r}_{c.m}}{\d t^2} + 2 \bm{\omega}_T \times \dfrac{\delta \bm{\rho}}{\delta t} + \dfrac{\delta^2 \bm{\rho}}{\delta t^2} + \dfrac{\bm{\omega}_T}{\d t}\times \bm{\rho} + \bm{\omega}_T \times (\bm{\omega}_T \times \bm{\rho}) \Bigg]\,\d m \\[0.5em]
		& = m \dfrac{\d^2 \bm{r}_{c.m}}{\d t^2} + 2 \bm{\omega}_T \times \int_m \dfrac{\delta \bm{\rho}}{\delta t}\, \d m + \int_m \dfrac{\delta^2 \bm{\rho}}{\delta t^2}\, \d m + \bm{\omega}_T \times \left(\bm{\omega}_T \times \int_m \bm{\rho}\, \d m \right)
	\end{align*}
由质心的定义,有$\displaystyle \int_m \bm{\rho} \, \d m = 0$,则可以得到

\theorem[任意变质量物体的一般运动方程]
{
	 \vspace*{-1em}
\begin{equation}
	\bm{F}_S = m \dfrac{\d^2 \bm{r}_{c.m}}{\d t^2} + 2 \bm{\omega}_T \times \int_m \dfrac{\delta \bm{\rho}}{\delta t}\, \d m + \int_m \dfrac{\delta^2 \bm{\rho}}{\delta t^2}\, \d m 
\end{equation}
}

\noindent 相应地可以得到

\theorem[任意变质量物体的质心运动方程]
{
	\vspace*{-1em}
	\begin{equation}
		m \dfrac{\d^2 \bm{r}_{c.m}}{\d t^2}  = \bm{F}_s + \bm{F}'_{k} + \bm{F}'_{rel} 
		\label{运动方程}
	\end{equation}
	其中,
	{
		\begin{enumerate}[\hspace*{2em}]
			\item \dy[附加哥氏力]{FJGSL}\quad $\displaystyle \bm{F}'_k = - 2 \bm{\omega}_T \times \int_m \dfrac{\delta \bm{\rho}}{\delta t}\, \d m$
			\item \dy[附加相对力]{FJXDL} \quad $\displaystyle \bm{F}'_{rel} = - \int_m \dfrac{\delta^2 \bm{\rho}}{\delta t^2}\, \d m$
		\end{enumerate}
	}
}

\textbf{2. 绕质心运动方程}

系统$S$绕质心的力矩方程为
\begin{equation}
	\bm{M}_{c.m} = \int_m \bm{\rho} \times \dfrac{\d^2 \bm{r}}{\d t^2} \, \d m
\end{equation}
将加速度的表达式\eqref{绝对加速度}代入,可以得到
\begin{align*}
	\bm{M}_{c.m} = \int_m \bm{\rho} \times \dfrac{\d^2 \bm{r}_{c.m}}{\d t^2}\, \d m 
	+ 2 \int_m \bm{\rho} \times \left(\bm{\omega}_T \times \dfrac{\delta \bm{\rho}}{\delta t}\right)\, \d m 
	+ \int_m \bm{\rho} \times \dfrac{\delta^2 \bm{\rho}}{\delta t^2}\, \d m 
	+ \int_m \bm{\rho}\times \left(\dfrac{\d \bm{\omega}_T}{\d t} \times \bm{\rho}\right)\, \d m 
	+ \int_m \bm{\rho}\times \big[\bm{\rho} \times (\bm{\omega}_T \times \bm{\rho})\big]\, \d m
\end{align*}

由于质心的定义,$\displaystyle \int_m \bm{\rho} \, \d m = 0$且$\dfrac{\d^2 \bm{r}_{c.m}}{\d t^2}$与质量无关,所以$\displaystyle \int_m \bm{\rho} \times \dfrac{\d^2 \bm{r}_{c.m}}{\d t^2}\, \d m = 0$,即化简为
\begin{align}
	\bm{M}_{c.m} = 2 \int_m \bm{\rho} \times \left(\bm{\omega}_T \times \dfrac{\delta \bm{\rho}}{\delta t}\right)\, \d m 
	+ \int_m \bm{\rho} \times \dfrac{\delta^2 \bm{\rho}}{\delta t^2}\, \d m 
	+ \int_m \bm{\rho}\times \left(\dfrac{\d \bm{\omega}_T}{\d t} \times \bm{\rho}\right)\, \d m 
	+ \int_m \bm{\rho}\times \big[\bm{\rho} \times (\bm{\omega}_T \times \bm{\rho})\big]\, \d m
	\label{转动方程}
\end{align}
记
\begin{align*}
	\bm{M}'_k &= - 2 \int_m \bm{\rho} \times \left(\bm{\omega}_T \times \dfrac{\delta \bm{\rho}}{\delta t}\right)\, \d m \\[0.5em]
	\bm{M}'_{rel} &= - \int_m \bm{\rho} \times \dfrac{\delta^2 \bm{\rho}}{\delta t^2}\, \d m
\end{align*}
则\eqref{转动方程}可以写为
\begin{equation}
	\int_m \bm{\rho}\times \left(\dfrac{\d \bm{\omega}_T}{\d t} \times \bm{\rho}\right)\, \d m 
	+ \int_m \bm{\rho}\times \big[\bm{\rho} \times (\bm{\omega}_T \times \bm{\rho})\big]\, \d m 
	= \bm{M}_{c.m} + \bm{M}'_k + \bm{M}'_{rel}
	\label{转动方程2}
\end{equation}
公式\eqref{转动方程2}左端第二项可以处理为
\begin{equation}
	\int_m \bm{\rho} \times \big[\bm{\omega}_T \times (\bm{\omega}_T \times \bm{\rho})\big]\, \d m = \bm{\omega}_T \times \int_m \bm{\rho} \times (\bm{\omega}_T \times \bm{\rho}) \, \d m \xlongequal[]{ \textstyle \hspace*{0.5em} \Delta \hspace*{0.5em}} \bm{\omega}_T \times \bm{H}_{c.m}
\end{equation}
其中$\bm{H}_{c.m}$为将系统视为刚体后,刚体对质心的角动量。

建立与物体固连的坐标系$o_1 - xyz$,有
\begin{equation*}
	\bm{\omega}_T = 
	\begin{bmatrix}
		\omega_{Tx} & \omega_{Ty} & \omega_{Tz}
	\end{bmatrix}^{\text{T}} \qquad \bm{\rho} = 
\begin{bmatrix}
	x & y & z
\end{bmatrix}^{\text{T}}
\end{equation*}
则角动量为
\begin{align*}
	\bm{H}_{c.m} & = \int_m \big[\bm{\rho} \times (\bm{\omega}_T \times \bm{\rho})\big]\, \d m 
	= \int_m \big[(\bm{\rho \cdot \bm{\rho}}) \bm{\omega}_T - (\bm{\rho \cdot \bm{\omega_T}})\bm{\rho }\big]\, \d m\\[0.5em]
	& = \int_m \big[(\bm{\rho}\cdot \bm{\rho})\bm{\omega}_T - (\bm{\rho} \cdot \bm{\rho}^{\text{T}})\bm{\omega}_T\big]\, \d m\\[0.5em]
	& = \int_m 
	\begin{bmatrix}
		y^2 + z^2 & -xy & -xz \\
		-xy & z^2 + x^2 & -yz \\
		-zx & -zy & x^2 + y^2 
	\end{bmatrix}
	\,
	\begin{bmatrix}
		\omega_{Tx}\\
		\omega_{Ty}\\
		\omega_{Tz}
	\end{bmatrix}
	\, \d m\\
	& =  \bm{I}\cdot \bm{\omega}_T
\end{align*}

其中,$\bm{I}$为\dy[惯性张量]{GXZL}
\begin{equation}
	\bm{I} = 
	\begin{bmatrix}
		I_{xx} & - I_{xy} & -I_{xz} \\
		-I_{xy} & I_{yy} & -I_{yz} \\
		-I_{zx} & -I_{zy} & I_{zz}
	\end{bmatrix}
\end{equation}

定义\dy[转动惯量]{ZDGL}$I_{xx},I_{yy},I_{zz}$和\dy[惯量积]{GLJ}
\begin{equation}
	\begin{cases}
		\, \displaystyle I_{xx} = \int_m(y^2 + z^2)\, \d m \\[0.8em]
		\, \displaystyle I_{yy} = \int_m(x^2 + z^2)\, \d m \\[0.8em]
		\,  \displaystyle I_{zz} = \int_m(y^2 + x^2)\, \d m \\[0.8em]
		\,  \displaystyle I_{xy} = I_{yx} = \int_m xy \, \d m\\[0.8em]
		\,  \displaystyle I_{xz} = I_{zx} = \int_m xz \, \d m\\[0.8em]
		\,  \displaystyle I_{xy} = I_{yx} = \int_m yz \, \d m
	\end{cases}
\end{equation}
则
\begin{equation}
	\bm{H}_{c.m} = \int_m \bm{\rho} \times (\bm{\omega}_T \times \bm{\rho})\, \d m =  \bm{I}\cdot \bm{\omega}_T 
	\label{角动量}
\end{equation}

同理可以对方程\eqref{转动方程2}左边第一项处理得到
\begin{equation}
	\int_m \bm{\rho} \times \left(\dfrac{\d \bm{\omega}_T}{\d t} \times \bm{\rho}\right) = \int_m \begin{bmatrix}
		y^2 + z^2 & -xy & -xz \\
		-xy & z^2 + x^2 & -yz \\
		-zx & -zy & x^2 + y^2 
	\end{bmatrix}
	\,
	\begin{bmatrix}
		\dfrac{\d \omega_{Tx}}{\d t}\\[0.8em]
		\dfrac{\d \omega_{Ty}}{\d t}\\[0.8em]
		\dfrac{\d \omega_{Tz}}{\d t}
	\end{bmatrix}
	\, \d m = \bm{I}\cdot \dfrac{\d \bm{\omega}_T}{\d t}
	\label{角动量导}
\end{equation}

则最终的转动方程为

\theorem[任意变质量物体的绕质心运动方程]
{
	\vspace*{-1em}
	\begin{equation}
		\bm{I}\cdot \dfrac{\d \bm{\omega}_T}{\d t} + \bm{\omega}_T\times(\bm{I}\cdot \bm{\omega}_T) = \bm{M}_{c.m}+\bm{M}'_k+\bm{M}'_{rel}
		\label{绕质心的运动方程}
	\end{equation}
	其中,
	{
		\begin{enumerate}[\hspace*{2em}]
			\item \dy[附加哥氏力矩]{FJGSLJ}\quad $\displaystyle \bm{M}'_k = -2 \int_m \bm{\rho}\times \left(\bm{\omega}_T \times \dfrac{\delta \bm{\rho}}{\delta t}\right)\, \d m$
			\item \dy[附加相对力矩]{FJXDLJ} \quad $\displaystyle \bm{F}'_{rel} = - \int_m \bm{\rho} \times \dfrac{\delta^2 \bm{\rho}}{\delta t^2} \, \d m$
			\item \dy[惯量张量]{GXZL} \quad $\displaystyle \bm{I} = 
			\int_m \begin{bmatrix}
				y^2 + z^2 & -xy & -xz \\
				-xy & z^2 + x^2 & -yz \\
				-zx & -zy & x^2 + y^2 
			\end{bmatrix} =
		\begin{bmatrix}
			I_{xx} & - I_{xy} & -I_{xz} \\
			-I_{xy} & I_{yy} & -I_{yz} \\
			-I_{zx} & -I_{zy} & I_{zz}
		\end{bmatrix}$
		\end{enumerate}
	}
}
\vspace*{1em}

\textbf{3. 钢化原理}

\theorem[钢化原理]
{
	一般情况下,任意变质量系统的运动方程,可用一个刚体的运动方程表示。这个刚体的质量等于系统瞬时质量,其受力除真实的外力与外力矩外,还要加上两个附加力和两个附加力矩。
}






























	
	%第三章-不定积分
	\chapter{平面不可压缩势流理论}
\thispagestyle{empty}
\section{基本方程}
\subsection{引言}
理想流动的连续(质量)方程和欧拉(动量)方程:
\begin{equation*}
	\dfrac{\D \rho}{\D t} + \rho (\nabla \bm{V}) = 0 \qquad \bm{f} - \dfrac{1}{\rho} \nabla p = \dfrac{\D \bm{V}}{\D t}
\end{equation*}
在不可压缩的条件下,可以简化为
\begin{equation}
	\begin{cases}
		\, \nabla \bm{V} = 0\\[0.5em]
		\, \dfrac{\D \bm{V}}{\D t} = \bm{f} - \dfrac{1}{\rho} \nabla p
	\end{cases}
\end{equation}
\vspace*{0.5em}

\noindent
\begin{minipage}{0.4\linewidth}
	给定初始条件
	\begin{equation*}
		\begin{cases}
			\, \bm{V}(x,y,z,t_0)\\
			\, p(x,y,z,t_0)
		\end{cases}
	\end{equation*}
\end{minipage}
\begin{minipage}{0.6\linewidth}
	边界条件\vspace*{-0.5em}
	\begin{itemize}
		\item 物体表面:$\bm{V}_n = 0$\vspace*{-0.5em}
		\item 无穷远处:$\bm{V} = \bm{V}, p = p_\infty$
	\end{itemize}
\end{minipage}
\vspace*{1em}

\noindent
\begin{minipage}{0.4\linewidth}
	\noindent 求解困难:\vspace*{-0.5em}
	\begin{itemize}
		\item 存在非线性项\vspace*{-0.5em}
		\item 速度和压强耦合\vspace*{-0.5em}
		\item 物体(如飞行器)边界复杂
	\end{itemize}
\end{minipage}
\begin{minipage}{0.6\linewidth}
	\noindent 求解思路:\vspace*{-0.5em}
	\begin{itemize}
		\item 基本流动的速度位或流函数\vspace*{-0.5em}
		\item 对于简单边界条件,基本流动叠加即可。\vspace*{-0.5em}
		\item 对于复杂边界条件,基本流动叠加并利用数值方法求解。
	\end{itemize}
\end{minipage}
\vspace*{1em}

\noindent 求解途径:\vspace*{-0.5em}
\begin{itemize}
	\item 利用无旋条件简化方程(线性化)\vspace*{-0.5em}
	\item 解耦速度和压强(即分别求解速度和压强)\vspace*{-0.5em}
\end{itemize}
\vspace*{1em}


\subsection{位函数和流函数、叠加原理和边界条件}

\sssection[位函数]

位函数存在的条件是流场无旋:
\begin{equation*}
	\text{rot} \bm{V} = 0 \qquad \bm{\omega}_z = 0 \qquad \dfrac{\partial u}{\partial y} = \dfrac{\partial v}{\partial x}
\end{equation*}
即存在\dy[位函数]{WHS}$\d \phi = u \d x + v \d y$,其中
\begin{equation}
	\begin{cases}
		\, u = \dfrac{\partial \phi}{\partial x} \\[0.5em]
		\, v = \dfrac{\partial \phi}{\partial y}
	\end{cases}
\end{equation}
又由连续方程
\begin{equation}
	\dfrac{\partial u}{\partial x} + \dfrac{\partial v}{\partial y} = 0 \quad \Rightarrow \quad \dfrac{\partial^2 \phi}{\partial x^2} + \dfrac{\partial^2 \phi}{\partial y^2} = 0 \quad \Rightarrow \quad \nabla^2 \phi = 0
\end{equation}
得到速度位,即可求解速度分量;一系列的速度位等值线称为\dy[等位线]{DWX}。
\vspace*{1em}

\sssection[流函数]

由连续方程
\begin{equation*}
	\dfrac{\partial u}{\partial x} + \dfrac{\partial v}{\partial y} = 0 \quad \Rightarrow \quad \dfrac{\partial u}{\partial x} = -\dfrac{\partial v}{\partial y} = \dfrac{\partial (-v)}{\partial x}
\end{equation*}
即存在\dy[流函数]{LHS}$\d \psi = -v \d x + u \d y$,其中
\begin{equation}
	\begin{cases}
		\, v = -\dfrac{\partial \psi}{\partial x} \\[0.5em]
		\, u = \dfrac{\partial \psi}{\partial y}
	\end{cases}
\end{equation}
若流场无旋,则
\begin{equation}
	\dfrac{\partial u}{\partial y} - \dfrac{\partial v}{\partial x} = 0 \quad \Rightarrow \quad  \dfrac{\partial^2 \psi}{\partial y^2} + \dfrac{\partial^2 \psi}{\partial^2 x^2} = 0 \quad \Rightarrow \quad \nabla^2 \psi = 0
\end{equation}
得到流函数,即可求解速度分量;一系列的流函数等值线称为\dy[流线]{LX}。
\vspace*{1em}

\sssection[叠加原理]

拉普拉斯方程是线性方程,可以用\dy[叠加原理]{DJYL}求复合解。

如果多个位函数分别满足拉普拉斯方程,即
\begin{equation}
	\nabla^2 \phi_i = 0
\end{equation}
则这些位函数的线性组合也必定满足别满足拉普拉斯方程,即
\begin{equation}
	\phi = \sum_{i=1}^{n} a_i \phi_i \quad \Rightarrow \quad \nabla^2 \phi = \sum_{i=1}^{n} a_i \nabla^2 \phi_i
\end{equation}
由于速度分量和位函数也是线性关系,则速度分量也满足叠加原理,即
\begin{equation}
	\begin{cases}
		\, \displaystyle u = \dfrac{\partial \phi}{\partial x} = \sum_{i = 1}^{n}a_i \dfrac{\partial \phi_i}{\partial x} = \sum_{i=1}^n a_iu_i\\[1em]
		\, \displaystyle v = \dfrac{\partial \phi}{\partial y} = \sum_{i = 1}^{n}a_i \dfrac{\partial \phi_i}{\partial y} = \sum_{i=1}^n a_iv_i
	\end{cases}
\end{equation}

\sssection[边界条件]
\begin{equation*}
	\mbox{\dy[边界条件]{BJTJ}}
	\, 
	\begin{cases}
		\, \mbox{\blue[外边界条件](足够远处)}\\
		\, \mbox{\blue[内边界条件](物体表面)}
	\end{cases}
\end{equation*}

\noindent 对于空气动力学问题,已知速度位$\phi$,则
\begin{itemize}
	\item 外边界条件(无穷远处):
	\begin{equation}
		\dfrac{\partial \phi}{\partial x} = V_\infty \quad \dfrac{\partial \phi}{\partial y} = \dfrac{\partial \phi}{\partial z} = 0
	\end{equation}
	\item 内边界条件:(物体表面法向速度分量为零)
	\begin{equation}
		\dfrac{\partial \phi}{\partial \bm{n}} = 0
	\end{equation}
\end{itemize}
在边界条件下,拉普拉斯方程的解唯一。

\subsection{位函数和流函数的性质及其相互联系}

\sssection[位函数的性质]
\vspace*{-1em}
\begin{enumerate}[\hspace*{1.5em} (1) ]
	\item 有无旋流动定义得到;位函数值可以相差任意常数;\vspace*{-0.5em}
	
	\item 对于理想不可压无旋流动,位函数满足拉普拉斯方程和叠加原理;\vspace*{-0.5em}
	
	\item 位函数沿某一方向的偏导数等于该方向的速度分量;($\bm{s}$为速度切向)
	\begin{equation}
		\begin{aligned}
			v_s &= u \cos(\bm{x},\bm{s}) + v \cos (\bm{y}, \bm{s}) \\
			& = \dfrac{\partial \phi}{\partial x} \cdot \dfrac{\d x}{\d s} + \dfrac{\partial \phi}{\partial y}\cdot \dfrac{\d y}{\d s} \\
			&= \dfrac{\partial \phi}{\partial \bm{s}}
		\end{aligned}
	\quad \Rightarrow \quad v_s = \dfrac{\partial \phi}{\partial \bm{s}}
	\end{equation}

	\item 速度位函数沿着流线方向增加
	\begin{equation}
			\begin{aligned}
			\d \phi &= \dfrac{\partial \phi}{\partial x}\, \d x + \dfrac{\partial \phi}{\partial y}\d y\\
			& = u\, \d x + v \, \d y\\
			& = \bm{V}\cdot \d \bm{s}
		\end{aligned} 
	\quad \Rightarrow \quad \d \phi = \bm{V}\cdot \bm{s}
	\end{equation}

	\item 速度位函数值相等的点连成的线称为\dy[等位(势)线]{DWX},与速度方向垂直;
	\begin{equation}
		\begin{cases}
			\, \d \phi = 0\\
			\, \d \phi = \bm{V} \cdot \bm{s}
		\end{cases}
		\quad \Rightarrow \quad 
		\bm{V}  \d \bm{s}
	\end{equation}
	
	\item 任意两点的速度线积分等于这两点的速度位函数之差;速度线积分与路径无关,仅决定于两点的位置;对封闭曲线,速度环量为零。
	\begin{align}
		\int_A^B \bm{V}\cdot \d \bm{s} & = \int_A^B (u\,d x + v\,d y + w \, \d z) \notag \\[0.5em]
		& = \int_A^B \, \d \phi = \phi_B - \phi_A
	\end{align}
\end{enumerate}

\sssection[流函数的性质]
\vspace*{-1em}
\begin{enumerate}[\hspace*{1.5em}(1) ]
	\item 
\end{enumerate}




















	
	%第四章-定积分
	\chapter{粘性流体动力学基础}
\thispagestyle{empty}
\section{流体的粘性及其对流动的影响}
\subsection{流体的粘性}
由于流体粘性影响,均匀流经平板时,贴着平板表面的流体速度降为零,称为\red{流体与板面间 “无滑移” 边界条件}。由于收到内层流体的摩擦力、外层流体的速度有变慢趋势,反过来,由于收到外层流体的摩擦力,内层流体的速度有变快趋势。流层简单“互相牵扯”作用一层层向外传递,在距离板面一定距离后,这种作用逐步消失,速度分布变为均匀。

\defination[流体粘性]
{流层之间“阻碍”流体相对变形趋势的能力称为\dy[流体粘性]{LTNX},相对错动(剪切)流层间的一对摩擦力即\dy[粘性剪切力]{NXJQL}。}

\begin{figure}[!htb]
	\centering
	\begin{minipage}{0.45 \linewidth}
		\centering
		\includegraphics[width=\linewidth]{pic/流体粘性实验.pdf}
		\caption{流体剪切实验}
		\label{流体剪切实验2}
	\end{minipage}
	\begin{minipage}{0.45 \linewidth}
		\centering
		\includegraphics[width=0.7\linewidth]{pic/粘性几何关系.pdf}
		\vspace*{0.1em}
		\caption{流体剪切几何关系}
		\label{流体剪切几何关系2}
	\end{minipage}
\end{figure}
\vspace*{-1em}

如图\ref{流体剪切实验2}所示,可以得到剪切力和粘性剪切应力的表达式
\begin{equation}
	F = \mu \dfrac{U}{h}A, \qquad \tau = \dfrac{F}{A} = \mu \dfrac{U}{h}
\end{equation}

\theorem[牛顿粘性应力公式]
{
	\quad \vspace*{-1em}
	\begin{equation}
		\tau = \mu \dfrac{\d u}{\d y}
		\label{牛顿粘性应力公式2}
	\end{equation}
	其中,$\mu$是流体的粘性系数,$u$是流体的运动速度。\dy[牛顿粘性应力公式]{NDNXYLGS}表明粘性剪切应力不仅与速度梯度有关,而且与物性有关。
}
\noindent 从牛顿粘性应力公式可以看出:\vspace*{-0.5em}
\begin{itemize}
	\item 流体的剪应力与压强$p$无关。\vspace*{-0.5em}
	\item 当$\tau \neq 0$时,$\dfrac{\d u}{\d y} \neq 0$,即无论剪应力多小,只要存在剪应力,流体就会发生变形运动,呈现速度梯度。\vspace*{-0.5em}
	\item $\dfrac{\d u}{\d y}=0$时,$\tau = 0$,即只要流体静止或无变形,就不存在剪应力,流体不存在静摩擦力。
\end{itemize}
因此,\textbf{牛顿粘性应力公式可看成流体易流性的数学表达}。

如图\ref{流体剪切几何关系2}所示,可以找到几何关系
\begin{equation}
	\d \theta \d y = \d u \d t , \quad \dfrac{\d \theta }{\d t} = \dfrac{\d u}{\d y}
\end{equation}
即微团垂直线在单位时间内顺时针的转角$=$速度梯度,速度梯度也表示流体微团的\dy[剪切变形速度]{JQBXSD}或\dy[角变形率]{JBXL}。

一般地,流体剪切应力与速度梯度的关系表示为:
\begin{equation}
	\tau = A + B \left(\dfrac{\d u}{\d y}\right)^n
\end{equation}
但是我们一般只研究\dy[牛顿流体]{NDLT}(如水、空气、汽油、酒精等),即满足牛顿粘性应力公式\eqref{牛顿粘性应力公式2}的流体。

\warn[\textbf{液体和气体产生粘性的物理原因不同}\\
\hspace*{2em}液体的粘性主要来自于液体分子间的{\blue[内聚力]},气体的粘性主要来自于气体分子的{\blue[热运动]}。因此液体与气体{\red[动力粘性系数随温度变化的趋势相反]}(气体粘性系数随温度的升高而升高,液体粘性系数随温度的升高而降低),但动力粘性系数与压强基本无关。]

\defination[动力学粘性系数和运动学粘性系数]
{
	在许多空气动力学问题里,粘性力和惯性力同时存在,在式子中$\mu$和$\rho$往往以$\dfrac{\mu}{\rho}$的组合形式出现,用符号$\nu$表示:(注意$\mu$和$\nu$的物理区别)
	{
		\begin{itemize}
			\item \dy[动力学粘性系数]{DLXNXXS}:$\mu \quad \left[\dfrac{\text{N}\cdot \text{s}}{\text{m}^2}\right]\qquad \mu_{\text{a}} = 1.7894\times 10^{-5} \, \text{kg/m/s}, \quad \mu_{\text{w}} = 1.139 \times 10^{-3}\, \text{kg/m/s}$  
			
			\item \dy[运动学粘性系数]{YDXNXXS}:$\nu = \dfrac{\mu}{\rho} \quad \left[\dfrac{\text{m}^2}{\text{s}}\right]\qquad \nu_\text{a} = 1.461\times 10^{-5} \, \text{m}^2/\text{s} \quad \nu_\text{w} = 1.139\times 10^{-6} \, \text{m}^2/\text{s}$
		\end{itemize}
	}
}
可以看出:空气动力粘性不大,初步近似研究时可忽略其粘性作用,忽略粘性的流体称为\dy[理想流体]{LXLT}。
\vspace*{0.5em}

\noindent 流体粘性的特点总结如下:
\vspace*{-0.5em}
\begin{itemize}
	\item 流体的 剪切变形 是指流体质点之间出现相对运动 例如流体层间的相对运动、产生速度梯度;\vspace*{-0.5em}
	\item 流体的粘性是指流体抵抗剪切变形或质点之间的相对运动的能力,是流体的物理属性;\vspace*{-0.5em}
	\item 流体的粘性力是抵抗流体质点之间相对运动,例如流体层间的相对运动 的剪应力或内摩擦力;\vspace*{-0.5em}
	\item 在静止状态下流体不能承受剪切力;但是在运动状态下流体可以承受剪力,剪切力大小与流体速度梯度有关 而且与流体种类有关;\vspace*{-0.5em}
	\item 粘性流体在流动过程中必然要克服内摩擦力做功,因此流体粘性是流体发生机械能损失的根源 。
\end{itemize}

\subsection{粘性对流动的影响}
\noindent \textbf{1. 绕平板的直匀流}


































	
	%第五章-向量与空间解析几何
    \chapter{自由飞行段弹道特性分析}
\thispagestyle{empty}
\section{自由飞行段弹道方程}
\subsection{自由飞行段假设}
火箭经过动力飞行段在关机点具有一定的位置和速度后,转入无动力、无控制的自由飞行状态。通常作如下假设:

 (1) 载荷在自由段处于真空飞行状态,不受空气动力作用、不必考虑姿态,将火箭看成\red[质点];

 (2) 认为载荷只受\red[]均质圆球]的引力作用,不考虑其他星体的引力影响。

\subsection{轨道方程}
圆形均值地球段引力为
\begin{equation}
	\bm{F}_T = - \dfrac{\mu m}{r^3}\bm{r} = m \dfrac{\d^2 \bm{r}}{\d t^2} \quad \Rightarrow \quad \dfrac{\d^2 \bm{r}}{\d t^2} = - \dfrac{\mu}{r^3} \bm{r}
	\label{4.1}
\end{equation}
引力始终指向$\bm{r}$反方向,$\bm{r}$是由地球中心$O_s$至载荷质心的矢径,所以引力$\bm{F}_T$为一个\dy[有心矢量场]{YXSLC}。

用$\bm{v}$点乘公式\eqref{4.1},可以得到
\begin{equation*}
	\bm{v} \cdot \dfrac{\d \bm{v}}{\d t} = - \dfrac{\mu}{r^3}(\bm{v} \cdot \bm{r}) \quad \Rightarrow \quad \dfrac{1}{2} \dfrac{\d \bm{v}^2}{\d t} = - \dfrac{\mu}{r^3}\left(\dfrac{1}{2}\dfrac{\d \bm{r}^2}{\d t}\right) \quad \Rightarrow \quad \dfrac{1}{2} \dfrac{\d v^2}{\d t} = -\dfrac{\mu}{r^2}\dfrac{\d r}{\d t} = \dfrac{\d }{\d t} \left(\dfrac{\mu}{r}\right)
\end{equation*}
积分后得

\theorem[机械能守恒定律]
{
	\vspace*{-2em}
	\begin{equation}
		\dfrac{1}{2}v^2 = \dfrac{\mu}{r} + E \quad \Rightarrow \quad E = \dfrac{1}{2}v^2 - \dfrac{\mu}{r}
		\label{E}
	\end{equation}
}

进一步,用地心矢叉乘动力学方程\eqref{4.1},有
\begin{equation*}
	\dfrac{\d \bm{r}^2}{\d t^2} = - \dfrac{\mu}{r^3}\bm{r} \quad \Rightarrow \quad \bm{r} \times \dfrac{\d^2 \bm{r}}{\d t^2} = - \dfrac{\mu}{r^3} \bm{r} \times \bm{r} = 0 \quad \Rightarrow \quad \dfrac{\d }{\d t}\left(\bm{r} \times \dfrac{\d \bm{r}}{\d t}\right) = \dfrac{\d }{\d t}(\bm{r} \times \bm{v}) = 0
\end{equation*}
即得到

\theorem[动量矩守恒]
{
	\vspace*{-2em}
	\begin{equation}
		\bm{r} \times \bm{v} = \bm{h}
	\end{equation}
}
\noindent 由于火箭动量矩守恒,因此自由段运动为平面运动,由关机点参数$\bm{r}_k,\bm{v}_k$决定。

动力学方程两边叉乘动量矩,则有\\
左式
\begin{equation*}
	\dfrac{\d^2 \bm{r}}{\d t^2} \times \bm{h} = \dfrac{\d }{\d t}\left(\dfrac{\d \bm{r}}{\d t} \times \bm{h}\right)
\end{equation*}
右式
\begin{align*}
	- \dfrac{\mu}{r^3} \bm{r} \times \bm{h} &= - \dfrac{\mu}{r^3}\bm{r} \times (\bm{r} \times \bm{v})\\
	& = -\dfrac{\mu}{r^3} \big[\bm{r} \cdot (\bm{r} \cdot \bm{v}) - \bm{v} \cdot (\bm{r} \cdot \bm{r})\big] \\
	& = - \dfrac{\mu}{r^3}\big[r \dot{r} \bm{r} - r^2 \bm{v}\big]\\
	& = - \mu \left[\dfrac{\bm{r}}{r^2}\dfrac{\d r}{\d t} - \dfrac{1}{r} \dfrac{\d \bm{r}}{\d t}\right]\\
	& = \mu \dfrac{\d}{\d t}\left(\dfrac{\bm{r}}{r}\right)
\end{align*}
即
\begin{equation}
	\dfrac{\d }{\d t}\left(\dfrac{\d \bm{r}}{\d t} \times \bm{h}\right) = \mu \dfrac{\d }{\d t}\left(\dfrac{\bm{r}}{r}\right)
\end{equation}
积分可得
\begin{equation}
	\dfrac{\d \bm{r}}{\d t} \times \bm{h} = \mu \left(\dfrac{\bm{r}}{r} + \bm{e}\right)
\end{equation}

为了获得标量方程,可以点乘$\bm{r}$,有
\begin{equation*}
	\begin{cases}
		\, \mu \bm{r} \cdot \left(\dfrac{\bm{r}}{r} + \bm{e}\right) = \mu \big[r + re\cos <\bm{r}, \bm{e}>\big]\\[1em]
		\, \bm{r}\cdot\left(\dfrac{\d \bm{r}}{\d t} \times \bm{h}\right) = \bm{h} \cdot \left(\bm{r} \times \dfrac{\d \bm{r}}{\d t}\right) = h^2 
	\end{cases}
\end{equation*}
即

\theorem[自由飞行段的轨道方程式]
{
	\vspace*{-2em}
	\begin{align}
		\dfrac{\d \bm{r}}{\d t} \times \bm{h} &= \mu \left(\dfrac{\bm{r}}{r} + \bm{e}\right)\\
		r = \dfrac{h^2 / \mu}{1 + e \cos <\bm{r}, \bm{e}>} &\triangleq \dfrac{P}{1 + e \cos <\bm{r}, \bm{e}>}
	\end{align}
}

\section{弹道方程的分析}
\subsection{符号$e,P$的意义及其确定}
轨道方程式对应为圆锥截线方程式,$e$为偏心率,决定圆锥截线形状,$P$为半通径,和$e$共同决定截线尺寸。建立关机点当地坐标系$\bm{K-ijk}$,显然$\bm{k}$与动量矩$\bm{h}$方向一致。
\begin{equation}
	\dfrac{\bm{r}}{\d t} \times \bm{h} = \mu \left(\dfrac{\bm{r}}{r} + \bm{e}\right) \quad \Rightarrow \quad \bm{v} \times \dfrac{\bm{h}}{\mu} = \dfrac{\bm{r}}{r} + \bm{e}
\end{equation}
其中,
\begin{equation}
	h = \left|\bm{r} \times \bm{v}\right| = rv \cos \varTheta
	\label{h}
\end{equation}
其中$\varTheta$为\dy[当地速度倾角]{DDSDQJ},则
\begin{equation}
	\bm{v} \times \dfrac{\bm{h}}{\mu} = 
	\begin{vmatrix}
		\bm{i} & \bm{j} & \bm{k}\\
		v_k\sin \varTheta & v_k \cos \varTheta & 0 \\
		0 & 0 & r_kv_k\cos \varTheta/\mu
	\end{vmatrix}
	= 
	\begin{bmatrix}
		r_kv_k^2 \cos^2 \varTheta / \mu \\
		-r_kv_k^2 \sin \varTheta \cos \varTheta / \mu \\
		0
	\end{bmatrix}
\end{equation}
定义\dy[能量参数]{NLCS}(表示动能与势能之比)为
\begin{equation}
	\nu_k = \dfrac{v_k^2}{\mu / r_k}
\end{equation}
进一步,可以计算得到
\begin{equation}
	\bm{e} = \bm{v} \times \dfrac{\bm{h}}{\mu} - \dfrac{\bm{r}}{r} = 
	\begin{bmatrix}
		\nu \cos^2 \varTheta - 1\\
		-\nu \cos \varTheta \cos \varTheta\\
		0
	\end{bmatrix}
\end{equation}
则

\theorem[$e,P$的计算值]
{
	\vspace*{-2em}
	\begin{align}
		e &= \sqrt{1 + \nu_k(\nu_k - 2)\cos^2 \varTheta_k} \label{e}\\[0.5em]
		P = \dfrac{h^2}{\mu} &= \dfrac{r_k^2 v_k^2 \cos^2 \varTheta_k}{\mu} = r_k v_k \cos^2 \varTheta_k \label{P}
	\end{align}
	在$e,P$已知的情况下,轨道上任意一点的地心矢径大小$r$仅与夹角$<\bm{r},\bm{e}>$有关,记
	\begin{itemize}
		\item $f = <\bm{r}, \bm{e}>$ \quad 定义由$\bm{e}$矢量顺飞行方向到r矢量为正角,称为\dy[真近点角]{ZJDJ}
		\item $\cos f = \dfrac{P - r}{re}$ \quad 由关机点$\bm{r}_k$矢量反向转$f$角即可确定$\bm{e}$矢量方向
	\end{itemize}
	当$f = 0$时,轨道地心矢径$\bm{r}_p$最小,称点$p$为近地点,因此$\bm{e}$矢量与$\bm{r}_p$矢径方向一致。
}

则各个物理量可以表示为

\sssection[地心距(轨道方程)]
\begin{equation}
	r = \dfrac{P}{1 + e \cos f}
\end{equation}

\sssection[径向速度]
\begin{equation}
	\begin{cases}
		\, v_r = \dot{r} = \dfrac{P}{\left(1 + e \cos f\right)^2} e \sin f \cdot \dot{f} \\[0.8em]
		\, \dot{f} = \dfrac{h}{r^2} = \dfrac{1}{r} \sqrt{\dfrac{\mu}{P}}(1 + e \cos f)
	\end{cases}
	\quad \Rightarrow \quad v_r = \sqrt{\dfrac{\mu}{P}}e \sin f
\end{equation}

\sssection[周向速度]
\begin{equation}
	v_f = r\dot{f} = \sqrt{\dfrac{\mu}{P}}(1 + e \cos f)
\end{equation}

\sssection[速度及其倾角]
\begin{align}
	v = \sqrt{\dfrac{\mu}{P} \left(1 + 2 e \cos f + e^2\right)} \\
	\tan \varTheta = \dfrac{v_r}{v_f} = \dfrac{e \sin f}{1 + e \cos f}
\end{align}

\subsection{轨道根数}
对于导弹而言,关机点参数用于确定被动段参数,从而获取落点参数。但对于运载火箭,其有效载荷为入轨的人造卫星,因此需要给出其轨道根数。

\defination[轨道根数]
{
	\vspace*{-2em}
	{
		\begin{enumerate}[\hspace*{2em}]
			\item \dy[升交点]{SJD} \quad 卫星从南向北穿越赤道点\vspace*{-0.5em}
			\item \dy[春分点]{CFD} \quad 2000年1月1.5日的平春分点\vspace*{-0.5em}
			\item \dy[长半轴$a$]{CBZ} \quad 圆锥截线轨道大小参数\vspace*{-0.5em}
			\item \dy[偏心率$e$]{PXL} \quad 圆锥截线轨道形状参数\vspace*{-0.5em}
			\item \dy[轨道倾角$i$]{GDQJ} \quad 轨道面与赤道面的夹角\vspace*{-0.5em}
			\item \dy[近地点角距$w$]{JDDJJ} \quad 升交点与近地点夹角\vspace*{-0.5em}
			\item \dy[升交点角距$\Omega$]{SJDJJ} \quad 升交点与春分点夹角\vspace*{-0.5em}
			\item \dy[近地点时刻$t_p$]{JDDSK} \quad 飞越近地点的时刻
		\end{enumerate}
	
	}
}

\subsection{轨道形状与关机点参数}
轨道所对应的圆锥截线形状由偏心率$e$决定。由\peref[E]和\peref[h],注意到有
\begin{equation}
	E = \dfrac{1}{2} v_k^2 - \dfrac{\mu}{r_k}  \quad \Rightarrow \quad \nu_k = \dfrac{r_k v_k^2}{\mu} = 2\left(1 + \dfrac{r_kE}{\mu}\right), \quad \cos^2 \varTheta_k = \dfrac{h^2}{r_k^2 v_k^2}
\end{equation}
则偏心率可改写为
\begin{equation}
	e = \sqrt{1 + \nu_k (\nu_k - 2)\cos^2 \varTheta_k} = \sqrt{1 + 2 \dfrac{h^2}{\mu^2}E} = \sqrt{1 + 2 \dfrac{P}{\mu} E}
\end{equation}

\sssection[$e = 0$时轨道为圆]

\noindent 轨道参数如下:

(1) 地心距
\begin{equation}
	r = \dfrac{P}{1 + e \cos f} = P = r_k
\end{equation}

(2) 速度倾角
\begin{equation}
	\varTheta = \varTheta_k = 0
\end{equation}

(3) 偏心率
\begin{equation}
	e =  \sqrt{1 + \nu_k (\nu_k - 2)\cos^2 \varTheta_k} = 0
\end{equation}

(4) 能量参数
\begin{equation}
	\nu = \nu_k = 1
\end{equation}

(5) 速度
\begin{equation}
	v = v_k = \sqrt{\dfrac{\mu}{r_k}} \triangleq v_{\text{\RMN[1]}}\, \mbox{(\dy[第一宇宙速度]{DYYZSD})}
\end{equation}

\sssection[$e = 1$时轨道为抛物线]

\noindent 轨道参数如下:

(1) 能量参数
\begin{equation}
	e =  \sqrt{1 + \nu_k (\nu_k - 2)\cos^2 \varTheta_k} = 1 \quad \Rightarrow \quad \nu_k = 2
\end{equation}

(2) 能量
\begin{equation}
	e = \sqrt{1 + 2 \dfrac{P}{\mu}E} = 1 \quad \Rightarrow \quad E = 0
\end{equation}

(3) 速度
\begin{equation}
	v = v_k = \sqrt{2\dfrac{\mu}{r_k}} \triangleq v_{\text{\RMN[2]}}\, \mbox{(\dy[第二宇宙速度]{DEYZSD})}
\end{equation}

\sssection[$e > 1$时轨道为双曲线]

\noindent 轨道参数如下:

(1) 能量参数
\begin{equation}
	e =  \sqrt{1 + \nu_k (\nu_k - 2)\cos^2 \varTheta_k} > 1 \quad \Rightarrow \quad \nu_k > 2
\end{equation}

(2) 双曲线\dy[剩余速度]{SYSD}
\begin{equation}
	E>0 \quad \Rightarrow \quad \dfrac{v_k^2}{2} - \dfrac{\mu}{r_k} = \dfrac{v_\infty^2}{2}
\end{equation}

\sssection[$e < 1$时轨道为椭圆]

\noindent 轨道参数如下:

(1) 能量参数
\begin{equation}
	e =  \sqrt{1 + \nu_k (\nu_k - 2)\cos^2 \varTheta_k} \in (0,1) \quad \Rightarrow \quad 1 < \nu_k < 2
\end{equation}

(2) 速度
\begin{equation}
	E<0 \quad \Rightarrow \quad v_k < v_{\text{\RMN[2]}}
\end{equation}
质点动能不足以将质点移动到无穷远处,地心矢径有限。
\vspace*{0.5em}

\sssection[总结]

(1) \dy[弹道]{DD} \quad 运载火箭及其载荷的飞行轨迹,不闭合。

(2) \dy[轨道]{GD}\quad 人造卫星等绕地球闭合飞行轨迹活动。

(3) \dy[星际航行]{XJHX}\quad 能量参数$\nu_k \ge 2$。

(4) \dy[绕地飞行]{RDFX} \quad 能量参数$\nu_k < 2$:
$
\begin{cases}
	\, \mbox{圆形轨道}\quad \varTheta_k = 0 \quad v_k = v_{\text{\RMN[1]}}\\
	\, \mbox{椭圆轨道}\quad 
	\begin{cases}
		\, \mbox{闭合轨道} \\
		\, \mbox{不闭合轨道}
	\end{cases}
\end{cases}
$

\sssection[椭圆轨道的进一步分析]

椭圆直角坐标方程为
\begin{equation}
	\dfrac{x^2}{a^2} + \dfrac{y_2}{b^2} = 1, \quad c= \sqrt{a^2 + b^2}
\end{equation}
其中,$a,b,c$分别为椭圆的长半轴、短半轴、半焦距。根据轨道方程的近地点和远地点,可得
\begin{equation}
	\begin{cases}
		\, f = 0 \quad r_p = r_{\min} = \dfrac{P}{1 + e} \\
		\, f = \pi \quad r_a = r_{\max} = \dfrac{P}{1 - e}
	\end{cases}
\end{equation}
可以求得椭圆轨道的参数
\begin{align}
	a &= \dfrac{r_a + r_p}{2} = \dfrac{P}{1 - e^2}\\[0.5em]
	c &= \dfrac{r_a - r_p}{2} = \dfrac{eP}{1 - e^2} = ea \\[0.5em]
	b &= \sqrt{a^2 - c^2} = \dfrac{P}{\sqrt{1 - e^2}}\\[0.5em]
	e &= \dfrac{a}{c} = \sqrt{1 - \left(\dfrac{b}{a}\right)^2}\\[0.5em]
	P &= \dfrac{b^2}{a}
\end{align}
且
\begin{equation}
	e = \sqrt{1 + 2 \dfrac{P}{\mu}E} = \sqrt{1 - \left(\dfrac{b}{a}\right)^2},\quad P = \dfrac{b^2}{a} \quad \Rightarrow \quad a = -\dfrac{\mu}{2E} = - \dfrac{\mu r_k}{r_kv_k^2 - 2\mu}
\end{equation}
则

\theorem[活力公式]
{
	椭圆的长半轴只与主动段关机点的机械能$E$有关,且$E<0$。$E$越大,$a$也越大,同时有
	\begin{equation}
		v^2 = \mu \left(\dfrac{2}{r} - \dfrac{1}{a}\right)
		\label{活力公式}
	\end{equation}
}

\sssection[人造卫星条件]

根据轨道形状与关机点参数的关系可知,在基本假设条件下,满足以下条件可使轨道不与地球相交:
\begin{equation}
	\begin{cases}
		\, \mbox{圆形轨道}\quad \varTheta_k = 0 , \nu_k = 1\\
		\, \mbox{椭圆轨道}\quad 1 < \nu_k < 2, r_{\min} > R_{\text{e}}
	\end{cases}
\end{equation}
但考虑到地球外的大气层,即使100km处大气密度十分稀薄,但卫星速度很高,大气会显著地阻碍卫星运动,使其速度降低,从而使轨道近地点高度下降,最终卫星失去功能。因此必须使卫星运行在离地面的一定高度上,称此高度为\dy[生存高度]{SCGD},记为$h_L$,该高度与卫星轨道停留时间相关。所以有
\begin{equation}
	r_p \ge r_L = R_{\text{e}} + h_L
\end{equation}
那么,各个参数满足以下关系:

(1) 地心距
\begin{equation*}
	r_k \ge r_p \ge r_L
\end{equation*}

(2) 速度倾角(由\peref[e]及\peref[P])
\begin{equation*}
	r_p = \dfrac{P}{1 + e} \ge r_L \quad \Rightarrow \quad \cos \varTheta_k \ge \dfrac{r_L}{r_k} \sqrt{1 + \dfrac{2 \mu}{v_k^2}\left(\dfrac{1}{r_L} - \dfrac{1}{r_k}\right)} \quad \Rightarrow \quad \cos \varTheta_k \ge \dfrac{r_L}{r_k} \sqrt{1 + \dfrac{2}{\nu_k}\left(\dfrac{r_k}{r_L} - 1\right)} 
\end{equation*}

(3) 速度
\begin{equation*}
	\dfrac{r_L}{r_k} \sqrt{1 + \dfrac{2 \mu}{v_k^2}\left(\dfrac{1}{r_L} - \dfrac{1}{r_k}\right)} \le \cos \varTheta_k \le 1 \quad \Rightarrow \quad v_k^2 \ge \dfrac{2 \mu r_L}{r_k(r_k + r_L)}  \quad \Rightarrow \quad \nu_k \ge \dfrac{2}{1 + r_k/r_L}
\end{equation*}


因此只有关机点参数满足上述(1)和(2)或(3)项条件,才能使其载荷称为人造卫星。

\section{射程与主动段终点参数的关系}
\noindent 需要解决的问题:\vspace*{-0.5em}
\begin{itemize}
	\item 已知关机点的状态参数,计算导弹的射程。\vspace*{-0.3em}
	\item 已知导弹的射程,反求导弹的关机条件。
\end{itemize}

\subsection{被动段射程的计算}
\defination[导弹射程]
{
	考虑地球为均质圆球,自由段弹道处于关机点平面内,弹道平面与地球表面的截痕为大圆弧,则有\dy[射程]{SC}
	\begin{align}
		L_{kc} &= L_{ke} + L_{ec} \\
		L_{ke} &= R \beta_e
	\end{align}
	由上式则射程可用相应的地心角描述,称为\dy[角射程]{JSC}。导弹在再入段受到气动力作用,其弹道不是椭圆的一部分,但其射程所占比例很小,可近似看成椭圆弹道的延续,则整个被动段弹道用椭圆来描述。
	\begin{equation}
		L_{kc} \approx R\beta_c \quad \beta_c = \beta_e + \beta_{ec}
	\end{equation}
	若能找到射程角与关机点状态关系即找到了射程与关机点状态的关系。
}
\begin{figure}[!htb]
	\centering
	\includegraphics[width=0.3\linewidth]{pic/导弹射程.jpg}
	\caption{自由段、被动段的射程角}
	\label{射程角}
\end{figure}

如图 \ref{射程角} 所示,已知$K,C$是椭圆弹道上的两点,他们的矢径与近地点极轴之间的夹角(真近点角)记为$f_k, f_c$,结合椭圆方程的轨道表达式,可以得到
\begin{equation}
	\beta_c = f_c - f_k,\quad \cos f = \dfrac{P - r}{er}
\end{equation}
当主动段终点参数给定,那么$f$仅仅是$r$的函数,即
\begin{equation}
	\begin{cases}
		\, \cos f_k = \dfrac{P - r_k}{e r_k}\\[0.5em]
		\, \cos f_c = \dfrac{P - r_c}{er_c} \approx \dfrac{P - R}{eR} 
	\end{cases}
\end{equation}
考虑到椭圆弹道的对称性,有:$\angle KO_E a = \angle aO_Ee = \dfrac{\beta_e}{2}$,则
\begin{equation}
	\begin{cases}
		\, \cos f_k = \cos \left(\pi - \dfrac{\beta_e}{2}\right) = \dfrac{P - r_k}{e r_k}\\[0.5em]
		\, \cos f_c = \cos\left(f_k + \beta_c\right) = \cos \left(\pi + \beta_c - \dfrac{\beta_e}{2}\right) = - \cos \left(\beta_c - \dfrac{\beta_e}{2}\right) = \dfrac{P - R}{eR} 
	\end{cases}
\end{equation}
化简,得
\begin{equation*}
	 \cos \dfrac{\beta_e}{2} = \dfrac{r_k - P}{e r_k} \quad \Rightarrow \quad \sin \dfrac{\beta_e}{2} = \dfrac{1}{e} \sqrt{e^2 - \left(1 - \dfrac{P}{r_k}\right)^2}
\end{equation*}
将$e, P$的表达式代入,得
\begin{equation}
	\begin{cases}
		\, e = \sqrt{1 + \nu_k(\nu_k - 2)\cos^2 \varTheta_k} \\[0.5em]
		P = \dfrac{r_k^2 v_k^2\cos ^2 \varTheta_k}{\mu} = r_k\nu_k\cos^2 \varTheta_k
	\end{cases}
	\quad \Rightarrow \quad \sin \dfrac{\beta_e}{2} = \dfrac{1}{e} \sqrt{e^2 - \left(1 - \dfrac{P}{r_k}\right)^2} = \dfrac{P}{er_k}\tan \varTheta_k
	\label{beta_e}
\end{equation}
将$\beta_e$的表达式代入
\begin{equation*}
	\begin{cases}
		\cos \dfrac{\beta_e}{2} = \dfrac{r_k - P}{e r_k}\\[0.5em]
		\sin \dfrac{\beta_e}{2} = \dfrac{P}{er_k}\tan \varTheta_k
	\end{cases}
\quad \xrightarrow{\,\,\,\mbox{\scriptsize 代入}\,\,\,}	\quad \cos \left(\beta_c - \dfrac{\beta_e}{2}\right) = \cos \beta_c \cos \dfrac{\beta_e}{2} + \sin\beta_c \sin \dfrac{\beta_e}{2} = \dfrac{R-P}{eR}
\end{equation*}
可以得到
\begin{equation*}
	\begin{split}
		&\left(1 - \dfrac{P}{r_k}\right)\cos \beta_c + \dfrac{P}{r_k} \tan \varTheta_k\sin \beta_c = 1 - \dfrac{P}{R}\\[0.5em]
		\xrightarrow{\,\, \textstyle P = r_k\nu_k\cos^2 \varTheta_k\,\,}\quad & (1-\nu_k \cos^2 \varTheta_k)\cos \beta_c + \nu_k \cos^2 \varTheta_k\tan \varTheta_k \tan \varTheta_k \sin \beta_c = 1 - \dfrac{r_k \nu_k \cos^2 \varTheta_k}{R}\\[0.5em]
		\Rightarrow \quad & \dfrac{r_k}{R} = \dfrac{1 - (1-\nu_k\cos^2 \varTheta_k) \cos \beta_c + \nu_k \cos^2 \varTheta_k\tan\varTheta_k\sin \beta_c}{\nu_k \cos^2 \varTheta_k}
	\end{split}
\end{equation*}
最终,我们可以得到

\theorem[命中方程]
{
	\vspace*{-1.5em}\index{MZFC@命中方程}
	\begin{equation}
		\dfrac{r_k}{R} = \dfrac{1 - \cos \beta_c}{\nu_k \cos^2 \varTheta_k} + \dfrac{\cos \left(\beta_c + \varTheta_k\right)}{\cos \varTheta_k}
	\end{equation}
	三角函数转换为半角形式有
	\begin{equation*}
		\cos \beta_c = \dfrac{1 - \tan^2 \dfrac{\beta_c}{2}}{1 + \tan^2 \dfrac{\beta_c}{2}}\qquad \sin \beta_c = \dfrac{2 \tan \dfrac{\beta_c}{2}}{1 + \tan^2 \dfrac{\beta_c}{2}}
	\end{equation*}
	代入命中方程,得到
	\begin{equation}
		\big[2R\left(1 + \tan^2\varTheta_k\right) - \nu_k (R + r_k)\big]\tan^2 \dfrac{\beta_c}{2} - 2\nu_k R \tan \varTheta_k \tan \dfrac{\beta_c}{2} + \nu_k \left(R - r_k\right) = 0
		\label{命中方程}
	\end{equation}
	由于轨道参数已知,则方程可以简化为
	\begin{equation}
		A \cdot \tan^2 \dfrac{\beta_c}{2} - B \cdot \tan \dfrac{\beta_c}{2} + C = 0
	\end{equation}
}
其中,
\begin{equation*}
	\begin{split}
		A &= 2r_c \left(1 + \tan^2 \varTheta_k\right) - \nu_k (r_c + r_k) \\
		& \ge 2r_c \left(1 + \tan^2 \varTheta_k\right) -2 \nu_k r_k \\
		& = 2r_c \left(1 + \tan^2 \varTheta_k\right) -2P/\cos^2 \varTheta_k \\
		& = 2r_c \left(1 + \tan^2 \varTheta_k\right) \left(1 - \dfrac{P}{r_c}\right)\\
		& = 2r_c \left(1 + \tan^2 \varTheta_k\right) -\big[1 - (1 + e\cos f_c)\big]\\
		& = -2r_c \left(1 + \tan^2 \varTheta_k\right)e \cos f_c\\
		& \ge 0\\
		C &=  \nu_k (r_c - r_k) \le 0
	\end{split}
\end{equation*}

\theorem[被动段射程计算公式]
{
	射程通常小于$180\degree$,因此被动段射程计算公式为
	\begin{equation}
		\tan \dfrac{\beta_c}{2} = \dfrac{B + \sqrt{B^2 - 4AC}}{2A} , \quad L_{kc} = R \beta_c
		\label{被动段}
	\end{equation}	
	其中,
	\begin{equation}
		\begin{cases}
			\, A = 2 R \big(1 + \tan^2 \varTheta_k\big) - \nu_k \big(R + r_k\big)\\
			\, B = 2 \nu_k R \tan \varTheta_k \\
			\, C = \nu_k \big(R - r_k\big)
		\end{cases}
		\label{被动段系数}
	\end{equation}
}

\subsection{自由段射程的计算}

由被动段段射程公式,只需要把$r_e = r_k$替换掉$R$即可,即

\theorem[自由段射程计算公式]
{
	\vspace*{-2em}
	\begin{equation}
		\tan \dfrac{\beta_e}{2} = \dfrac{B + \sqrt{B^2 - 4AC}}{2A} 
	\end{equation}	
	其中,
	\begin{equation}
		\begin{cases}
			\, A = 2 r_k \big(1 + \tan^2 \varTheta_k\big) - 2\nu_k r_k\\
			\, B = 2 r_k \nu_k  \tan \varTheta_k \\
			\, C = \nu_k \big(r_k - r_k\big) = 0
		\end{cases}
	\label{自由段系数}
	\end{equation}
	简化为
	\begin{equation}
		\tan \dfrac{\beta_c}{2} \approx \tan \dfrac{\beta_e}{2} = \dfrac{B}{A} = \dfrac{\nu_k \sin \varTheta_k \cos \varTheta_k}{1 - \nu_k \cos^2 \varTheta_k}
		\label{自由段方程}
	\end{equation}
	实际上,将\peref[P]$\,\, P = r_k \nu_k \cos^2 \varTheta_k$代入\peref[beta_e],可以得到
	\begin{equation}
		\sin \dfrac{\beta_c}{2} \approx \sin \dfrac{\beta_e}{2} = \dfrac{\nu_k}{2 e}\sin 2 \varTheta_k
		\label{自由段}
	\end{equation}
	工程上常用这个形式进行计算,其比较简单。自由段射程为
	\begin{equation}
		L_{ke} = R \cdot \beta_e
	\end{equation}
}

\clearpage
观察自由段射程公式\eqref{自由段},可以发现

\textbf{(1) \hspace*{0.5em} 能量系数给定}\\
\hspace*{2em} 存在\red[最佳速度倾角使射程最大]。也就是说,在给定关机点矢径及速度下,亦即给定机械能$E$下,最佳速度倾角使射程最大。
\begin{equation*}
	\varTheta_{k\text{.opt}} \quad \Rightarrow \quad \beta_{c\text{.max}}
\end{equation*}

\textbf{(2) \hspace*{0.5em} 射程给定}\\
\hspace*{2em} 存在\red[最佳速度倾角使能量系数最小],当关机点位置定,即使关机点速度最小,也就是说导弹的机械能E最小。称为\dy[最小能量弹道]{ZXNLDD}。


\subsection{最佳倾角确定}

已知$r_k, v_k$,确定$\varTheta_{k\text{.opt}}, \beta_{c\text{.max}}$。

被动段射程可以描述为$\beta_c = \beta_c\big(r_k, v_k, \varTheta_k\big)$,由于$r_k, v_k$确定,则$\beta_c = \beta_c\big(\varTheta_k\big)$,极值点(最佳倾角)处导数为0,即
\begin{equation}
	\dfrac{\partial \beta_c}{\partial \varTheta_k} = 0
\end{equation}

对方程\eqref{自由段方程}求导可得
\begin{equation}
	\dfrac{\partial A}{\partial \varTheta_k} \tan^2 \dfrac{\beta_c}{2} - \dfrac{\partial B}{\partial \varTheta_k} \tan \dfrac{\beta_c}{2} + \left(2 A \tan \dfrac{\beta_c }{2} - B\right)\dfrac{\partial \tan \dfrac{\beta_c}{2}}{\partial \varTheta_k} = 0
\end{equation}
对自由段方程的系数\eqref{被动段系数}求导,可以得到
\begin{equation}
	\begin{cases}
		\, \dfrac{\partial A}{\partial \varTheta_k} = 4R\tan \varTheta_k\sec^2\varTheta_k \\[0.8em]
		\, \dfrac{\partial B}{\partial \varTheta_k} = 2R\nu_k \sec^2 \varTheta_k
	\end{cases}
\end{equation}
且
\begin{equation}
	\dfrac{\partial \tan \dfrac{\beta_c}{2}}{\partial \varTheta_k} = \dfrac{1}{2} \sec^2 \dfrac{\beta_c}{2} \dfrac{\partial \beta_c}{\partial \varTheta_k} = 0(\varTheta_k = \varTheta_{k\text{.opt}}, \beta_c = \beta_{c\text{.max}})
\end{equation}

取$\varTheta_k = \varTheta_{k\text{.opt}}, \beta_c = \beta_{c\text{.max}}$,则
\begin{align*}
	& 4 R \tan \varTheta_{k\text{.opt}} \sec^2\varTheta_{k\text{.opt}} \tan^2 \dfrac{\beta_{c\text{.max}}}{2} - 2 R \nu_k \sec^2 \varTheta_{k\text{.opt}} \tan \dfrac{\beta_{c\text{.max}}}{2} = 0 \\
	\Longrightarrow \quad & 2R\sec^2 \varTheta_{k\text{.opt}} \tan \dfrac{\beta_{c\text{.max}}}{2} \left(2 \tan \varTheta_{k\text{.opt}} \tan \dfrac{\beta_{c\text{.max}}}{2} - \nu_k\right) = 0 \\
	\xrightarrow{ \,\, \textstyle 2R\sec^2 \varTheta_{k\text{.opt}} \tan \frac{\textstyle \beta_{c\text{.max}}}{\textstyle 2} \neq 0 \,\, } \quad & 2 \tan \varTheta_{k\text{.opt}} \tan \dfrac{\beta_{c\text{.max}}}{2} - \nu_k = 0
\end{align*}
即
\begin{equation}
	\tan \dfrac{\beta_{c\text{.max}}}{2} = \dfrac{\nu_k}{2 \tan \varTheta_{k\text{.opt}}}
\end{equation}
将计算结果代入到\peref[命中方程],化简后可以得到
\begin{equation}
	\big[4(R - r_k) - 2 R \nu_k\big] \tan^2 \varTheta_{k\text{.opt}} = \nu_k^2 (R + r_k) - 2 R \nu_k
\end{equation}

\theorem[被动段最佳速度倾角]
{
	\vspace*{-1em}
	\begin{equation}
		\tan \varTheta_{k\text{.opt}} = \sqrt{\dfrac{\nu_k \big[2R - \nu_k (R + r_k)\big]}{2 R \nu_k - 4(R - r_k) }}, \qquad 
		\tan \dfrac{\beta_{c\text{.max}}}{2} = \sqrt{\dfrac{\nu_k \big[R \nu_k - 2(R - r_k)\big]}{2 \big[2 R - \nu_k (R + r_k)\big]}}
		\label{被动段最佳倾角}
	\end{equation}
}

自由段的最佳速度倾角,只需要在公式\eqref{被动段最佳倾角}中令$R = r_e = r_k$,即
 
 \theorem[自由段最佳速度倾角]
 {
 	\vspace*{-1em}
 	\begin{equation}
 		\tan \varTheta_{ek\text{.opt}} = \sqrt{1 - \nu_k}, \qquad 
 		\tan \dfrac{\beta_{e\text{.max}}}{2} = \dfrac{1}{2} \dfrac{\nu_k}{\sqrt{1 - \nu_k}}
 		\label{自由段最佳倾角}
 	\end{equation}
 	将$\tan \varTheta_{ek\text{.opt}}$的表达式代入$\tan \dfrac{\beta_{e\text{.max}}}{2}$的表达式中,可以得到$\beta_{e\text{.max}}$和$\varTheta_{ek\text{.opt}}$之间的关系
 	\begin{equation}
 		\varTheta_{ek\text{.opt}} = \dfrac{1}{4} \big(\pi - \beta_{e\text{.max}}\big)
 	\end{equation}
 }

可以看出:当被动段射程越大,再入段射程比例就越小,被动段最佳倾角越接近自由段最佳倾角。

对于$h_k = 0$的自由段,当射程很小时,最佳倾角接近于$45 \degree$,与炮兵选用的最佳射击角一致。


\subsection{最小能量的最佳倾角确定}

已知$r_k, \beta_c$求$\varTheta_{k\text{.opt}}, \nu_{k\text{.min}}$.用图解法来寻找在$r_k, \beta_c$条件下的$\varTheta_{k\text{.opt}}, \nu_{k\text{.min}}$.
\begin{figure}[!htb]
	\centering
	\includegraphics[width=0.3\linewidth]{pic/最小能量-最佳倾角.jpg}
	\vspace*{-1em}
	\caption{最小能量弹道图解法示意图}
	\label{最小能量}
\end{figure}

将再入段看成自由段的延续,被动段弹道为平面弹道。给定射程时,点$O_E, K, C$相对位置确定。设与地心$O_E$对应的椭圆虚焦点为$O$,满足
\begin{equation}
	\begin{cases}
		r_k + OK = 2a_k \\
		r_c + OC = 2a_c
	\end{cases} \qquad 
	\begin{cases}
		OK = 2a_k - r_k \\
		OC = 2a_c - r_c
	\end{cases}
\end{equation}

给定椭圆长半轴$a$,以$K,C$为圆心,分别以$OK$与$OC$为半径画圆,则有两个交点$O,O'$,即为椭圆虚焦点,对应椭圆的焦距、偏心率不同,如图\ref{最小能量}所示。可以得到

(1) \hspace*{0.5em} 两个虚焦点$O,O'$对称于$KC$连线。

(2) \hspace*{0.5em} 随着长半轴$a$减小,虚焦点逐渐靠近$KC$线,最终重合于$OE'$点。此时有
\begin{equation}
	a_{\min} = \dfrac{1}{4} \big(KC + r_k + r_c\big)
\end{equation}
长半轴最小,即能量最小,对应于最小能量弹道。

(3) \hspace*{0.5em} 对于给定的$a$所画的椭圆上任意一点的法线必平分该点至该椭圆两焦点连线的夹角,如图\ref{最小能量1}所示。而对于最小能量弹道,如图\ref{最小能量2}所示,由速度倾角的定义
\begin{equation}
	\angle \bm{x}K\bm{v} = \varTheta_k
\end{equation}
而由于$K\bm{x} \perp KP$,$K \perp O_E$,那么
\begin{equation}
	\angle O_E K P = \angle \bm{x}K\bm{v} = \varTheta_{k, \min}
\end{equation}
由速度$\bm{v}$的方向沿椭圆切线方向,那么
\begin{equation}
	\angle C K O_E = 2 \angle P K O_E = 2 \angle PKC = 2 \varTheta_{k, \min}
\end{equation}
由几何关系
\begin{equation}
	\begin{split}
		\tan \angle CKO_E = \tan 2\varTheta_k = \dfrac{CQ}{KQ} = \dfrac{CQ}{KO_E + O_EQ} = \dfrac{O_EC \sin \beta_c}{KO_E - OE_C \cos \beta_c} = \dfrac{r_c \sin \beta_c}{r_k - r_c \cos \beta_c}
	\end{split}
\end{equation}
对$\triangle KCO_E$应用余弦定理,得
\begin{equation}
	KC = \sqrt{\left|O_EK\right|^2 + \left|O_EC\right|^2 - 2\left|O_EK\right|\left|O_EC\right|\cos \beta_c} = \sqrt{r_k^2 + r_c^2 - 2r_cr_k\cos \beta_c}
\end{equation}
又
\begin{equation}
	KC = OK + OC = 2a - r_k + 2a - r_c = 4a - r_k - r_c
\end{equation}
可以解得最小能量弹道对应对半长轴为
\begin{equation}
	a_{\min} = \dfrac{\sqrt{r_k^2 + r_c^2 - 2r_cr_k\cos \beta_c} + r_k + r_c}{4}
\end{equation}
由\peref[活力公式],可以计算得到所需要的最小速度为
\begin{equation}
	v_{k,\min} = \sqrt{\mu \left(\dfrac{2}{r_k} - \dfrac{1}{a_{\min}}\right)} = \sqrt{\mu \left(\dfrac{1}{r_k} - \dfrac{4}{\sqrt{r_k^2 + r_c^2 - 2r_cr_k\cos \beta_c} + r_k + r_c}\right)} = 2 \tan \varTheta_{k, \min} \cdot \tan \dfrac{\beta_c}{2}
\end{equation}

\begin{figure}[!htb]
	\centering
	\begin{minipage}{0.49\linewidth}
		\centering
		\includegraphics[width=0.63\linewidth]{pic/最小能量.pdf}
		\caption{给定射程的飞行轨迹}
		\label{最小能量1}
	\end{minipage}
	\begin{minipage}{0.49\linewidth}
		\centering
		\includegraphics[width=0.75\linewidth]{pic/最小能量2.pdf}
		\caption{最小能量的飞行轨迹}
		\label{最小能量2}
	\end{minipage}
\end{figure}

 \theorem[自由段最小能量]
{
	自由段最小能量的最优倾角和最小速度分别为
	\begin{equation}
		\begin{cases}
			\, \varTheta_{k,\min} = \dfrac{1}{2}\arctan \left(\dfrac{r_c \sin \beta_c}{r_k - r_c \cos \beta_c}\right)\\[1em]
			\, v_{k,\min} = \sqrt{\mu \left(\dfrac{1}{r_k} - \dfrac{4}{\sqrt{r_k^2 + r_c^2 - 2r_cr_k\cos \beta_c} + r_k + r_c}\right)} = 2 \tan \varTheta_{k, \min} \cdot \tan \dfrac{\beta_c}{2}
		\end{cases}
	\end{equation}
}













	
	%第七章-重积分
	\chapter{重积分}
\section{二重积分}
\subsection{二重积分的定义}
\thispagestyle{empty}

\tdefination[二重积分的定义]
设$z=f(x,y)$是定义在平面上的有界闭区域$D$上的函数,若对$D$的任意分割$\{D_1,D_2,\cdots,D_n\}$及任意选择的$(x_i,y_i) \in D_i (i=1,2,\cdots,n)$,当$\lambda \rightarrow 0$时,极限
\begin{equation}
\lim_{\lambda \rightarrow 0} \sum^{n}_{i=1} f(x_i,y_i)\,\Delta \sigma_i
\footnote{$\lambda$表示$n$个区域$D_i$其中的最大直径,$\Delta \sigma_i$表示$D_i$的最大面积.}
\end{equation}

总存在,则这个极限称为$f(x,y)$在$D$上的二重积分,记做
\begin{equation}
\iint\limits_{D}f(x,y) \, \, \d \sigma \huo  \iint\limits_{D}f(x,y) \, \, \d x \d y
\end{equation}

\par 其中,$D$称作积分区域,而$f(x,y)$称作被积函数,$\d\sigma$称为面积元素.

\subsection{二重积分的性质}\label{二重积分的性质}

\ttheorem[二重积分的三个基本性质]
\vspace*{-1.5em}
\begin{enumerate}
	\setlength{\itemindent}{1em}
	\setlength{\topsep}{0.01em}
	\setlength{\itemsep}{0.01em}
	
	\item 常数因子可以提取:($k$为常数)
	\begin{equation}
	\iint\limits_{D}kf(x,y) \, \, \d \sigma =k\iint\limits_{D}f(x,y) \, \, \d \sigma 
	\end{equation}
	\eqsj
	\item 被积函数的可拆可合性:
	\begin{equation}
	\iint\limits_{D}\left[ f(x,y)\pm g(x,y)\right]  \, \, \d \sigma =\iint\limits_{D}f(x,y) \, \, \d \sigma \pm \iint\limits_{D}g(x,y) \, \, \d \sigma
	\end{equation}
	\eqsj
	\item 积分区域的可拆可合性:(设$D \rightarrow D_1+D_2$)
	\begin{equation}
	\iint\limits_{D}f(x,y) \, \, \d \sigma =\iint\limits_{D_1}f(x,y) \, \, \d \sigma + \iint\limits_{D_2}f(x,y) \, \, \d \sigma 
	\end{equation}
\end{enumerate}

\theorem[积分的保号性]
若函数$f$及$g$在$D$上满足不等式
\[
f(x,y) \le g(x,y), \quad \forall (x,y) \in D 
\]
则
\begin{equation}
\iint\limits_{D}f(x,y) \, \, \d \sigma \le \iint\limits_{D}g(x,y) \, \, \d \sigma
\label{积分的保号性}
\end{equation}
\par 特别地,由于$-|f(x,y)| \le f(x,y) \le |f(x,y)|$,带入式\eqref{积分的保号性},得到
\begin{equation}
\left| \iint\limits_{D}f(x,y) \, \, \d \sigma \right|  \le \iint\limits_{D}|f(x,y)| \, \, \d \sigma
\end{equation}

\theorem[积分中值定理]
若函数$f(x,y)$在有界闭区域$D$上连续,则在$D$上至少存在一点$(x_0,y_0)$,使
\begin{equation}
\iint\limits_{D}f(x,y) \, \, \d \sigma=f(x_0,y_0) \cdot S
\end{equation}
\par 其中$S$为区域$D$的面积.

\subsection{二重积分的计算}

\ttheorem[$X$型积分与$Y$型积分]
对于不同的积分区域主要可以划分为两种:$X$型积分与$Y$型积分
\begin{equation}
\begin{split}
\iint\limits_{D}f(x,y)\,\, \d x \d y &= \int_{a}^{b}\left[ \int_{\varphi_1(x)}^{\varphi_2(x)}f(x,y)\,\, \d x\right] \d y \\
&= \int_{a}^{b}\left[ \int_{\varphi_1(y)}^{\varphi_2(y)}f(x,y)\,\, \d y\right] \d x 
\end{split}
\end{equation}

\ttheorem[极坐标变换]
设$x,y$的极坐标方程为
$
\begin{cases}
x = r \cos \theta,\\
y = r\sin \theta . \\
\end{cases}
$
则
\begin{equation}
\iint\limits_{D}f(x,y)\,\, \d x \d y= \int_{\alpha}^{\beta }\d \theta \int_{r_1(\theta)}^{r_2(\theta)}f(r \cos \theta , r \sin \theta )r\,\,\d r
\end{equation}

\ttheorem[广义极坐标变换]
设$x,y$的极坐标方程为
$
\begin{cases}
x = ar \cos \theta,\\
y = br\sin \theta . \\
\end{cases}
$
则
\begin{equation}
\iint\limits_{D}f(x,y)\,\, \d x \d y= \int_{\alpha}^{\beta }\d \theta \int_{r_1(\theta)}^{r_2(\theta)}f(ar \cos \theta , br \sin \theta )abr\,\,\d r
\end{equation}

\ttheorem[一般变换]
设$x,y$满足
$
\begin{cases}
x = x(\xi,\eta),\\
y = y(\xi,\eta). \\
\end{cases}
$
则
\begin{equation}
\iint\limits_{D}f(x,y)\,\, \d x \d y= \iint\limits_{D‘}f[x(\xi,\eta),y(\xi,\eta)]\,|J|\, \d \xi \d \eta
\end{equation}
其中$J$是变换的雅克比行列式,即
\renewcommand{\arraystretch}{1.5}
\begin{equation*}
|J|=\frac{D(x,y)}{D(\xi,\eta)}=
\left| 
\begin{array}{cc}
\displaystyle \frac{\partial x}{\partial \xi} & \displaystyle \frac{\partial y}{\partial \xi} \\
\displaystyle \frac{\partial x}{\partial \eta} & \displaystyle \frac{\partial y}{\partial \eta} 
\end{array}
\right| 
\end{equation*}
\renewcommand{\arraystretch}{1}

\subsection{二重积分的几何应用}
\sj
\example[求隐函数的平面面积]
由二重积分的定义,记隐函数所围成的封闭曲面的面积为$S_D$,那么可以得到
\begin{equation}
\iint\limits_{D} \, \, \d \sigma=\iint\limits_{D}\, \, \d x \d y=S_D
\end{equation}

\example[求空间曲面的面积]
若$S$由参数方程
$
\begin{cases}
x = x(u,v),\\
y = y(u,v),\\
z= z(u,v).
\end{cases}
$
确定,记
$
\begin{cases}
E = x_u^2 + y_u^2 +z_u^2,\\
F = x_ux_v + y_uy_v + z_uz_v,\\
G = x_v^2 + y_v^2 +z_v^2.
\end{cases}
$
则
\begin{equation}
S=\iint\limits_{D'} \sqrt{EG-F^2}\, \, \d \sigma
\end{equation}

\section{三重积分}
\subsection{三重积分的定义}

\tdefination[三重积分的定义]
设三元函数$f(x,y,z)$是定义在光滑曲面所围成的空间区域$\Omega$上,若对$\Omega$的任意分割$\{\Omega_1,\Omega_2,\cdots,\Omega_n\}$及任意选择的$(x_i,y_i,z_i) \in \Omega_i (i=1,2,\cdots,n)$,当$\lambda \rightarrow 0$时,极限
\begin{equation}
\lim_{\lambda \rightarrow 0} \sum^{n}_{i=1} f(x_i,y_i,z_i)\,\Delta V_i
\footnote[1]{$\lambda$表示$n$个区域$\Omega_i$其中的最大直径,$\Delta V_i$表示$\Omega_i$的最大体积.}
\end{equation}

总存在,则这个极限称为$f(x,y)$在$D$上的三重积分,记做
\begin{equation}
\iiint\limits_{\Omega}f(x,y,z) \, \, \d V \huo  \iiint\limits_{\Omega}f(x,y,z) \, \, \d x \d y \d z
\end{equation}

\par 其中,$\Omega$称作积分区域,而$f(x,y,z)$称作被积函数,$\d V$称为体积元素.

\subsection{三重积分的性质}
三重积分的基本性质和二重积分完全类似。具体请参见\ref{二重积分的性质}.


\subsection{三重积分的计算}

\ttheorem[投影法]
投影法可以认为是平行于$z$轴的线在投影区域内运动,连续地切割立体得到得到一条条立体内的线段$z_1(x,y) \rightarrow z_2(x,y)$,然后再把所有在投影区域内的所有线段进行积分,即
\begin{equation}
\iiint\limits_{\Omega} \,\d x \d y  \d z = \iint\limits_{D_{xOy}}\,\d x \d y \int_{z_1(x,y)}^{z_2(x,y)}f(x,y,z) \,\d z
\end{equation}

\theorem[切片法]
切片法可以认为是用平行于$xOy$的平面$z=z_0\in [a,b]$去截立体得到的截面$D_{z_0}$,求出$D_{z_0}$后再把一片片截面积分拼成一个立体,即
\begin{equation}
\iiint\limits_{\Omega} \,\d x \d y  \d z = \int_{a}^{b} \, \d z  \iint\limits_{D_{z}} f(x,y,z) \,\d x \d y
\end{equation}


\theorem[柱坐标变换]
柱坐标变换
$
\begin{cases}
x =r \cos \theta,\\
y = r \sin \theta ,\\
z = z.
\end{cases}
$
下的三重积分计算公式为
\begin{equation}
\iiint\limits_{\Omega} \,\d x \d y  \d z = \iiint\limits_{\Omega'} f(r\cos\theta,r\sin\theta,z)\,r \,\,\d r \d \theta  \d z
\end{equation}

\ttheorem[球坐标变换]
球坐标变换
$
\begin{cases}
x =\rho \,\sin \varphi \cos \theta,\\
y = \rho \,\sin \varphi \sin \theta ,\\
z = \rho \,\cos \varphi .
\end{cases}
$
下的三重积分计算公式为
\begin{equation}
\iiint\limits_{\Omega} \,\d x \d y  \d z = \iiint\limits_{\Omega'} f(\rho \,\sin \varphi \cos \theta,\rho \,\sin \varphi \sin \theta , \rho \,\cos \varphi)\, \rho^2 \sin \varphi \,\,\d \rho \d \varphi  \d \theta 
\end{equation}

\ttheorem[一般变换]
设$x,y,z$满足
$
\begin{cases}
x = x(u,v,w),\\
y = y(u,v,w), \\
z = z(u,v,w).
\end{cases}
$
则
\begin{equation}
\iiint\limits_{\Omega }f(x,y,z)\,\, \d x \d y= \iiint\limits_{\Omega‘}f[x(u,v,w),y(u,v,w),z(u,v,w)]\,|J|\, \,\d u \d v \d w
\end{equation}
其中$J$是变换的雅克比行列式,即
\renewcommand{\arraystretch}{1.5}
\begin{equation*}
|J|=\frac{D(x,y,z)}{D(u,v,w)}=
\left| 
\begin{array}{ccc}
\displaystyle \frac{\partial x}{\partial u} & \displaystyle \frac{\partial y}{\partial u} & \displaystyle \frac{\partial z}{\partial u} \\
\displaystyle \frac{\partial x}{\partial v} & \displaystyle \frac{\partial y}{\partial v} & \displaystyle \frac{\partial z}{\partial v} \\
\displaystyle \frac{\partial x}{\partial w} & \displaystyle \frac{\partial y}{\partial w} & \displaystyle \frac{\partial z}{\partial w} 
\end{array}
\right| 
\end{equation*}
\renewcommand{\arraystretch}{1}


\subsection{三重积分的几何应用}
\sj
\example[求立体的体积]
由三重积分的定义,记隐函数围成的封闭立体的体积为$V$,那么可以得到
\begin{equation}
\iiint\limits_{\Omega} \, \, \d V=\iiint\limits_{\Omega}\, \, \d x \d y \d z=V
\end{equation}
	
	%第八章-曲线积分和曲面积分
	\chapter{系统采样理论}
\thispagestyle{empty}

\section{采样过程与采样定理}
\subsection{采样过程}
如图\ref{信号采样},将连续信号转换成离散信号的过程,称为\dy[采样过程]{CYGC}。这个过程可以看成是一个信号的\dy[调制过程]{TZGC}。
\begin{figure}[!htb]
	\centering
	\includegraphics[width=0.6\linewidth]{pic/信号采样.jpg}
	\caption{信号的采样过程}
	\label{信号采样}
\end{figure}
\vspace*{-1em}
\begin{equation}
	f_\tau^*(t) = p(t) \cdot f(t)
\end{equation}
实现信号采样过程的装置称为\dy[采样开关]{CYKG},如图\ref{采样开关}.
\begin{figure}[!htb]
	\centering
	\begin{tikzpicture}
		\draw[arrows={-Stealth}] (0cm,0cm) --+(1cm,0cm)node[very near start, above = -4.5mm, xshift = -1.5mm]{$\tau$};
		\draw (-1.5cm, 0cm) -- (-0.5cm,0cm);
		\draw (-0.5cm, -0.2cm) -- (0cm,0cm)node[midway, above = 1mm]{$T$};
	\end{tikzpicture}
	\caption{采样开关}
	\label{采样开关}
\end{figure}

由于载波信号$p(t)$是周期函数,故可以展成如下Fourier级数
\begin{equation}
	p(t) = \sum_{n = -\infty}^{+\infty }C_n \e^{\j n \omega_\text{S} t}
\end{equation}
其中,$\omega_\text{S}= \dfrac{2\pi}{T}$,
\clearpage
\begin{align}
	C_n &= \dfrac{1}{T}\int_{T/2}^{-T/2}p(t)\e^{- \j n \omega_{\text{S}}t}\, \d t = \dfrac{1}{T}\int_{0}^{\tau} \e ^{- \j n \omega_{\text{S}}t}\, \d t \notag \\[0.5em]
	& = \dfrac{1}{T\tau}\cdot \dfrac{1}{- \j n \omega_{\text{S}}} \left(\e^{-\j n \omega_{\text{S}} \tau} - 1\right) = \dfrac{1}{T \tau} \cdot \dfrac{1}{-\j n \omega_{\text{S}}} \big(\cos n \omega_{\text{S}} \tau - \j \sin n \omega_{\text{S}} \tau - 1\big)\notag\\[0.5em]
	& = \dfrac{1}{T\tau}\cdot \dfrac{1}{-\j n \omega_{\text{S}}}\Bigg[1 - 2 \sin^2 \left(\dfrac{n \omega_{\text{S}} \tau}{2} \right)- 2 \j \sin \left(\dfrac{n \omega_{\text{S}} \tau}{2} \right) \cos \left(\dfrac{n\omega_{\text{S}}\tau}{2}\right) - 1\Bigg] \notag \\[0.5em]
	& = \dfrac{1}{T \tau} \cdot \dfrac{1}{-\j n \omega_{\text{S}}}(- 2 \j) \sin \left(\dfrac{n \omega_{\text{S}}\tau}{2}\right) \Bigg[\cos\left(\dfrac{n \omega_{\text{S}} \tau}{2}\right)- \j \sin \left(\dfrac{n \omega_{\text{S}} \tau}{2}\right)\Bigg]\notag\\[0.5em]
	& = \dfrac{1}{T} \dfrac{\sin\big(n\omega_{\text{S}}\tau /2\big)}{n \omega_{\text{S}} \tau / 2}\e^{-\j m \omega_\s \tau/2}
\end{align}

若连续信号$f(t)$的Fourier变换为$F(\j \omega)$,则采样信号$f_\tau^*(t)$的Fourier变换为
\begin{align}
	F_\tau^*(\j \omega) &= \int_{-\infty}^{+\infty} F_\tau^*(t)\e^{-\j \omega t}\, \d t = \int_{- \infty}^{+\infty} \Bigg[\sum_{n = -\infty}^{+ \infty}C_n f(t)\e^{\j n \omega_{\text{S}}t}\Bigg]\e^{-\j \omega t}\, \d t \notag \\[0.5em]
	& = \sum_{n = -\infty}^{+\infty} C_n \int_{-\infty}^{+\infty}f(t)\e^{-\j (\omega - n\omega_{\text{S}})t}\, \d t \notag \\[0.5em]
	& = \sum_{n = -\infty}^{+\infty} C_nF(\j \omega - \j n \omega_{\text{S}})
\end{align}

如图\ref{连续与离散信号},对于$n = 0$的部分,称为$F_\tau^*(\j \omega)$的\dy[主分量]{ZFL},其余的部分称为$F_\tau^*(\j \omega)$的\dy[补分量]{BFL}。
\vspace*{-1em}
\begin{figure}[!htb]
	\centering
	\includegraphics[width=0.7\linewidth]{pic/连续与离散信号.jpg}
	\vspace*{-2em}
	\caption{连续信号与离散信号的频谱($\omega_{\text{s}} < 2 \omega_{\text{max}}$)}
	\label{连续与离散信号}
\end{figure}

\subsection{采样定理}
若存在一个理想的低通滤波器(其频率特性如图\ref{理想低通滤波器}),就可以将采样信号完全恢复成原连续信号。由此可得\dy[香农采样定理]{XNCYDL}:

如果采样频率$\omega_{\text{s}}$满足一下条件
\begin{equation}
	\omega_{\text{s}} \ge 2\omega_{\max}
\end{equation}
其中,$\omega_\max$为连续信号频谱的上限频率,此时离散信号和连续信号的频谱如图\ref{连续与离散信号2}.

则经采样得到的脉冲序列可以无失真地恢复为原连续信号。

\begin{figure}[!htb]
	\begin{minipage}{0.6\linewidth}
		\includegraphics[width=\linewidth]{pic/连续与离散信号2.jpg}
		\vspace*{-1em}
		\caption{连续信号与离散信号的频谱($\omega_{\text{s}} \ge 2 \omega_{\text{max}}$)}
		\label{连续与离散信号2}
	\end{minipage}
	\begin{minipage}{0.4\linewidth}
		\vspace*{5em}
		\includegraphics[width=\linewidth]{pic/理想低通滤波器.pdf}
		\vspace*{6em}
		\vspace*{-1em}
		\caption{理想低通滤波器}
		\label{理想低通滤波器}
	\end{minipage}
\end{figure}
但理想的低通滤波器在物理上是不可实现的,在实际应用中只能用非理想的低通滤波器来代替理想的低通滤波器。

\subsection{理想采样过程}
为了简化采样过程的数学描述,引入\dy[理想采样开关]{LXCYKG}的概念。
\begin{figure}[!htb]
	\centering
	\includegraphics[width = 0.6\linewidth]{pic/理想采样开关.jpg}
	\vspace*{-0.5em}
	\caption{理想采样开关的采样过程}
	\label{理想采样开关的采样过程}
\end{figure}

\defination[理想采样开关]
载波信号$p(t)$可以近似成\dy[理想脉冲序列]{LXMCXL}
\begin{equation}
	\delta_T(t) = \sum_{k = -\infty}^{+\infty}\delta(t - kT)
\end{equation}
设当$t<0,f(t) = 0$,则
\begin{equation}
	f^*(t) = f(t) \cdot \delta_T(t) = \sum_{k = 0}^{\infty} f(t) \cdot \delta(t - kT)
\end{equation}
同样,$\delta_T(t)$可以展开成如下Fourier级数
\begin{equation}
	\delta_T (t) = \sum_{n = -\infty}^{+\infty}C_n \e^{\j n \omega_{\text{s}}t}
\end{equation}
其中,$C_n = \lim\limits_{\tau \to 0} \Bigg(\dfrac{1}{T}\dfrac{\sin(n\omega_\text{s}\tau /2)}{n \omega_\text{s}\tau/2}\e^{- \j n \omega_\text{s}\tau / 2}\Bigg) = \dfrac{1}{T}$.则有
\begin{align}
	f^*(t) = \dfrac{1}{T}\sum_{n = -\infty}^{+\infty}f(t)\e^{\j n \omega_\text{s}t}\\[0.5em]
	F^*(\j \omega) = \dfrac{1}{T}\sum_{n = -\infty}^{+\infty}F(\j \omega - \j n \omega_\text{s})
	\label{采样函数}
\end{align}

\section{信号的恢复与零阶保持器}
\subsection{信号恢复的基本概念}
\dy[信号恢复]{XHHF}是指采样信号恢复为连续信号的过程,能够实现这一过程的装置称为\dy[保持器]{BCQ}。对于$kT<t<(k+1)T$时,可将$f(t)$展开成泰勒级数
\begin{equation}
	f(t) = f(kT) + f'(t)\big|_{t = kT}\cdot(t - kT) + \cdots + \dfrac{1}{n!}f^{(n)}(t)\big|_{t = kT} \cdot (t-kT)^n + \cdots
\end{equation}
其中,各阶导数的近似值为
\begin{align}
	f'(kT) &\approx \dfrac{f(kT) - f(kT - T)}{T}\\[0.5em]
	f''(Kt) & \approx \dfrac{f'(kT) - f'(kT - T)}{T}\\
	& \quad \quad \cdots \cdots \notag
\end{align}
由此类推,计算$n$阶导数的近似值需已知$n+1$个采样时刻的瞬时值。

\subsection{零阶保持器}
\begin{figure}[!htb]
	\centering
	\includegraphics[width = 0.9\linewidth]{pic/零阶保持.pdf}
	\caption{零阶保持器采样示意图}
	\label{零阶保持器采样}
\end{figure}
零阶保持器的数学表达式为
\begin{equation}
	f(t) = f(kT), \quad kT < t<(k+1)T
\end{equation}
则理想采样开关的输出函数为
\begin{equation}
	f^*(t) = f(t)\cdot \delta_T(t) = \sum_{k = 0}^{\infty} f(kt) \cdot \delta(t - kT)
\end{equation}
其Laplace变换为
\begin{equation}
	F^*(s) = \sum_{k = 0}^{+\infty} f(kT) \e^{- kTs}
\end{equation}

\begin{figure}[!htb]
	\centering
	\begin{tikzpicture}
		\node (A)[draw, inner sep = 5pt, xshift = 2.5cm]{零阶保持器};
		
		\draw[arrows={-Stealth}] (0cm,0cm) --(A)node[midway, above = 0mm]{$f^*(t)$};
		\draw[arrows={-Stealth}] (A) -- +(2.2cm,0cm)node[midway, above = 0mm]{$f_\text{h} (t)$};
		\draw (-1.5cm, 0cm) -- (-0.5cm,0cm)node[very near start, above = 0cm]{$f(t)$};
		\draw (-0.5cm, -0.2cm) -- (0cm,0cm)node[midway, above = 1mm]{$T$};
		\draw[dashed] (-0.7cm, 0.5cm) -- (-0.7cm, -0.5cm) -- (0.2cm, -0.5cm)node[midway, above = -6mm]{\small 理想采样开关} -- (0.2cm,0.5cm) -- (-0.7cm, 0.5cm);
	\end{tikzpicture}
	\caption{零阶保持器}
	\label{零阶保持器}
\end{figure}
零阶保持器的输出为
\begin{equation}
	f_\text{h}(t) = \sum_{k = 0}^{+\infty} f(kT)\big[1(t- kT) - 1(t - kT - T)\big]
\end{equation}
其Laplace变换为
\begin{equation}
	F_\text{h}(s) = \sum_{k = 0}^{+ \infty}f(kT)\Bigg[\dfrac{\e^{-kTs} - \e^{-(k+1)Ts}}{s}\Bigg] = \Bigg(\dfrac{1 - \e^{- Ts}}{s}\Bigg) \sum_{k = 0}^{+\infty}f(kT)\e^{-kTs}
\end{equation}

则零阶保持器的传递函数为
\begin{equation}
	G_\text{h}(s) = \dfrac{F_{\text{h}}(s)}{F^*(s)} = \dfrac{1 - \e^{-Ts}}{s}
\end{equation}

\noindent 零阶保持器的频率特性为
\begin{align}
	G_\text{h} (\j \omega) = \dfrac{1 - \e^{\j \omega T}}{\j \omega} &= T \dfrac{\sin \big(\omega T / 2\big))}{\omega T /2} \e^{\textstyle -\frac{1}{2} \j \omega T} \notag \\[0.5em]
	&= T \dfrac{\sin \big(\pi \omega / \omega_\text{s}\big)}{\pi \omega / \omega_\text{s}}\e^{- \j \pi\omega /\omega_\text{s}}
\end{align}
\begin{itemize}
	\item 幅频特性
	\begin{equation}
		\big|G_\text{h}(\j \omega)\big| = T \Bigg|\dfrac{\sin \big(\pi \omega / \omega_\text{s}\big)}{\pi \omega / \omega_\text{s}}\Bigg|
	\end{equation}
	\item 相频特性
	\begin{equation}
		\angle G_\text{h}(\j \omega) = - \dfrac{\pi \omega}{\omega_\text{s}} + \angle \sin \big(\pi \omega / \omega_\text{s}\big)
	\end{equation}
	其中,
	\begin{equation}
		\angle \sin \big(\pi \omega / \omega_\text{s}\big) =
		\, \begin{cases}
			\,0, & 2n\omega_\text{s}<\omega < (2n+1)\omega_\text{s}\\[0.5em]
			\, \pi, & (2n+1)\omega_\text{s} < \omega < 2(n+1)\omega_\text{s}
		\end{cases}
		\quad (n = 0,1,2,\cdots)
	\end{equation}
\end{itemize}

零阶保持器的频率特性曲线如图\ref{零阶保持器幅频特性}所示,对比图\ref{理想低通滤波器}可知零阶保持器是一个低通滤波器,但不是理想的低通滤波器,它除了允许信号的主频谱分量通过外,还允许部分高频分量通过。
\begin{figure}[!htb]
	\centering
	\includegraphics[width = 0.56\linewidth]{pic/零阶保持器幅频特性.jpg}
	\vspace*{-1em}
	\caption{零阶保持器的频率特性曲线}
	\label{零阶保持器幅频特性}
\end{figure}

\section{$z$变换与$z$逆变换}
\subsection{$z$变换}
连续信号$f(t)$经采样后得到的脉冲序列为
\begin{align}
	f^*(t) = \sum_{k = 0}^{+\infty}f(kT)\cdot \delta(t -kT)
\end{align}
对其进行Laplace变换,得
\begin{equation}
	F^*(s) = \sum_{k = 0}^{+\infty}f(kT)\e^{-kTs}
\end{equation}
引入一个新变量
\begin{equation}
	z = \e^{Ts}
\end{equation}
可得$z$变换的定义式
\begin{equation}
	F(z) = F^*(s)\big|_{s=(1/T)\ln z} = \sum_{k=0}^{\infty} f(kT)z^{-k}
\end{equation}
称$F(z)$为$f^*(s)$的\dy[$z$变换]{ZBH},记作$Z\big[f^*(t)\big]=Z\big[f(kT)\big]=F(z).$\\
求解方法:
\begin{enumerate}[\textbf{方法} 1 ]
	\item \textbf{级数求和法}\\
	展开以后是一个等比数列,求和以后再求极限即可求得$z$变换。
	\item \textbf{部分分式法}\\
	设连续函数的Laplace变换为有理函数,将其展开成\dy[部分分式]{BFFS}的形式为
	\begin{equation}
		F(s) = \sum_{i = 1}^{n} \dfrac{a_i}{s+s_i}
	\end{equation}
	因此,连续函数的$z$变换可以由有理函数求出
	\begin{equation}
		F(z) = \sum_{i=1}^{n} \dfrac{a_i z}{z - \e^{-s_iT}}
	\end{equation}
\end{enumerate}

\examples 求$f(t) = a^{t/T}$函数的$z$变换。

\solve 由$z$变换的定义有
\begin{align*}
	F(z) &= \sum_{k=0}^{\infty} f^*(kT)z^{-k} = \sum_{k = 0}^{\infty} a^{k} z^{-k}= \lim\limits_{k \to \infty} \dfrac{1 - (az^{-1})^{k+1}}{1-az^{-1}} = \dfrac{1}{1 - az^{-1}} = \dfrac{z}{z-a}
\end{align*}
\clearpage
\vspace*{-2.5em}

\examples 求$F(s) = \dfrac{a}{s(s+a)}$的$z$变换。

\solve 将$F(s)$写成部分分式之和的形式
\[
F(s) = \dfrac{a}{s(s+a)} = \dfrac{1}{s} - \dfrac{1}{s+a}
\]
其中,$a_1 = 1,\quad a_2 = -1,\quad s_1 = 0, \quad s_2 = -a.$所以,
\[
F(z) = \dfrac{z}{z-1} - \dfrac{z}{z - \e^{-aT}} = \dfrac{(1 - \e^{-aT})z}{z^2 - (1 + \e^{-aT})z+\e^{-aT}}
\]

\subsection{常用信号的$z$变换}
常见函数的Laplace变换和Z变换如表\ref{常用函数的z变换}\footnote{请注意背诵此表,作者表示期末考栽在没有记住$Z\bigg[\dfrac{1}{s^3}\bigg]$上。}.

{
	\centering
	\setlength{\tabcolsep}{11mm}{
		\begin{longtable}{cccc}
			
			\toprule
			函数名 & $F(s)$ & $f(t)$ & $F(z)$\\
			\midrule
			\endfirsthead
			
			\multicolumn{3}{r}{续表}\\
			\toprule
			函数名 & $F(s)$ & $f(t)$ & $F(z)$\\
			\midrule
			\endhead
			
			% 表格“尾页前”,表格最后显示内容
			\bottomrule
			\endfoot
			
			% 表格“尾页”,表格最后显示内容
			\bottomrule
			\endlastfoot
			
			单位脉冲信号& 1& $\delta t$ & 1\\[1em]
			单位阶跃信号&$\dfrac{1}{s}$ & $1(t)$ & $\dfrac{z}{z - 1}\quad |z|>1$\\[1em]
			单位斜坡信号 & $\dfrac{1}{s^2}$ & $t$ & $\dfrac{Tz}{(z-1)^2} \quad |z|>1$\\[1em]
			单位加速度信号 & $\dfrac{1}{s^3}$ & $\dfrac{1}{2}t^2$ & $\dfrac{T^2z(z+1)}{2(z-1)^3} \quad |z|>1$\\[1em]
			指数函数 & $\dfrac{1}{s+a}$ & $\e^{-at}$ & $\dfrac{z}{z - \e^{-aT}}$\\[1em]
			正弦函数 & $\dfrac{\omega}{s^2 + \omega^2}$ & $\sin \omega t$ & $\dfrac{z \cdot \sin\omega T}{z^2 - 2z\cos \omega T + 1}$\vspace*{0,5em}
		\end{longtable}
	}
	\vspace*{-2em}
	\captionof{table}{常用函数的$z$变换}
	\label{常用函数的z变换}
}


\subsection{$z$变换的基本定理}
\begin{enumerate}[\hspace*{2em} \textbf{定理} 1 ]
	\item \textbf{线形定理}\\
	设$a_1,a_2$为任意常数,连续时间函数$f_1(t),f_2(t)$的$z$变换分别为$F_1(z),F_2(z)$,则有
	\begin{equation}
		Z\big[a_1f_1(t) + a_2f_2(t) \big] = a_1F_1(z)+a_2F_2(z)
	\end{equation}
	
	\item \textbf{滞后定理}\\
	设连续时间函数在$t < 0$时,$f(t) = 0$,且$Z\big[f(t)\big] = F(z)$,则有
	\begin{equation}
		Z\big[f(t-kT)\big] = z^{-k}F(z)
	\end{equation}
	
	\item \textbf{超前定理}\\
	设连续时间函数$f(t)$的$z$变换为$F(z)$,则有
	\begin{equation}
		Z\big[f(t+kT)\big] = z^kF(z) - \sum_{n = 0}^{k - 1} f(nT)z^{k-n}
	\end{equation}
	
	\item \textbf{初值定理}\\
	设连续时间函数$f(t)$的$z$变换为$F(z)$,则有
	\begin{equation}
		f(0) = \lim\limits_{z \to \infty} F(z)
	\end{equation}
	
	\item \textbf{终值定理}\\
	设连续时间函数$f(t)$的$z$变换为$F(z)$,则有
	\begin{equation}
		f(\infty) = \lim\limits_{z \to 1} (1 - z^{-1})F(z) = \lim\limits_{z \to 1}(z - 1)F(z)
	\end{equation}
	
	\item \textbf{位移定理}\\
	设$a$为任意常数,连续时间函数$f(t)$的$z$变换为$F(z)$,则有
	\begin{equation}
		Z\big[f(t)\e^{-at}\big] = F(z\cdot \e^{aT})
	\end{equation}
\end{enumerate}

\subsection{$z$逆变换}
$z$逆变换是$z$变换的逆运算。其目的是由像函数$F(z)$求出所对应的采样脉冲序列$f^*(t)$($f(kT)$),记作
\begin{equation}
	Z^{-1}\big[F(z)\big] = f^*(t)
\end{equation}
\vspace*{-2em}

\warn[\hspace*{2em}$z$逆变换只能求出采样信号$f^*(t)$,但不能求出连续信号$f(t)$。]

\noindent 求解方法:
\begin{enumerate}[\textbf{方法} 1 ]
	\item \textbf{部分分式法}\\
	若象函数$F(z)$是复变量$z$的有理分式,且$\dfrac{F(z)}{z}$的极点$z_i = \e^{-a_iT}$互异,则
	\begin{equation}
		\dfrac{F(z)}{z} = \dfrac{K_1}{z - \e^{-a_1T}} + \dfrac{K_2}{z - e^{-a_2T}} + \cdots + \dfrac{K_m}{z - \e^{-a_mT]}}
	\end{equation}
	其中,
	\begin{equation}
		K_i = \lim\limits_{z \to z_i} (z - z_i) \dfrac{F(z)}{z}
	\end{equation}
	两边同时乘以$z$,再取反变换得
	\begin{align}
		Z^{-1}\big[F(z)\big] &= Z^{-1}\Bigg[\dfrac{K_1z}{z-\e^{-a_1T}}\Bigg] + Z^{-1}\Bigg[\dfrac{K_2z}{z-\e^{-a_2T}}\Bigg] + \cdots + Z^{-1}\Bigg[\dfrac{K_mz}{z-\e^{-a_mT}}\Bigg]\\[1em]
		f(kT) &= K_1\e^{-a_1kT} + K_2 \e^{-a_2nT} + \cdots + K_m\e^{-a_mkT}
	\end{align}
	
	\item \textbf{长除法}\\
	若$z$变换函数$F(z)$是复变量$z$的有理函数,则可将$F(z)$展成$z^{-1}$的无穷级数,即
	\begin{equation}
		F(z) = f_0 + f_1z^{-1} + \cdots + f_k z^{-k} + \cdots
	\end{equation}
	则
	\begin{align}
		f(kT) &= f_k, \quad k = 0,1,2,\cdots\\
		f^*(t) &= \sum_{k = 0}^{+ \infty}f_k \delta(t - kT)
	\end{align}
	
	\item \textbf{留数计算法}\\
	由$z$变换的定义可知
	\begin{align}
		F(z) &= \sum_{k=0}^{+\infty} f(kT)z^{-k}\\
		F(z)z^{n-1} &= \sum_{k=0}^{+\infty}f(kT)z^{n-k-1}\\
		\oint_\Gamma F(z)z^{m-1}\,\d z &=\oint_\Gamma \Bigg[\sum_{k=0}^{+\infty}f(kT)z^{n-k-1}\Bigg]\,\d z = \sum_{k=0}^{+\infty} f(kT)\oint_\Gamma z^{n-k-1}\,\d z
	\end{align}
	根据柯西定理$\displaystyle \oint_\Gamma z^{n-1}\, \d z = 
	\,
	\begin{cases}
		\, 2\pi \j, &n = 0\\
		\, 0, & n \neq 0
	\end{cases}
	$,化简得
	\begin{align}
		f(kT) = \dfrac{1}{2\pi\j} \oint_\Gamma F(z)z^{k-1} \, \d z
	\end{align}
	其中,$\Gamma$包围了$F(z)z^{k-1}$的所有极点。设$F(z)z^{k-1}$的极点为$z_i,i=1,2,\cdots,n$,则
	\begin{align}
		f(kT) = \sum_{i=0}^n \text{Res}\big[F(z)z^{k-1},z_i\big]
	\end{align}
\end{enumerate}

\examples 求$F(z)=\dfrac{z}{(z-1)(z-\e^{-T})}$的逆变换。

\solve 首先将$\dfrac{F(z)}{z}$展成部分分式
\[
\dfrac{F(z)}{z} = \dfrac{K_1}{z-1} + \dfrac{K_2}{z-\e^{-T}}
\]
其中
\begin{align*}
	K_1 = \lim\limits_{z \to 1} (z-1)\dfrac{F(z)}{z} = \dfrac{1}{1- \e^{-T}}\\[0.5em]
	K_2 = \lim\limits_{z \to \e^{-T}} = - \dfrac{1}{1 - \e^{-T}}
\end{align*}
从而得到
\begin{align}
	F(z) = \dfrac{1}{1 - \e^{-T}}\Bigg(\dfrac{z}{z-1} - \dfrac{z}{z-\e^{-T}}\Bigg)\\[0.5em]
	f(nT) = \dfrac{1}{1 - \e^{-T}}\big(1 - \e^{-nT}\big)\\[0.5em]
	f^*(t) = \dfrac{1}{1 - \e^{-T}}\sum_{k = 0}^{+\infty} \big(1 - \e^{-kT}\big)\delta (t - kT)
\end{align}

\examples 求$F(z)=\dfrac{z}{(z-2)(z-3)}$的逆变换。

\solve 由于
\[
F(z) = \dfrac{z}{(z-2)(z-3)} = \dfrac{z}{z^2 -5z + 6} 
\]
运用长除法得
\[\left( {{z^2} - 5z + 6} \right)\mathop{\left){\vphantom{1\begin{array}{l}
z\\
\underline {z - 5 + 6{z^{ - 1}}} \\
\quad {\mkern 1mu} {\mkern 1mu} {\mkern 1mu} 5 - 6{z^{ - 1}}\\
\quad {\mkern 1mu} {\mkern 1mu} {\mkern 1mu} \underline {5 - 25{z^{ - 1}} + 30{z^{ - 2}}} \\
\quad \quad {\mkern 1mu} {\mkern 1mu} {\mkern 1mu} {\mkern 1mu} {\mkern 1mu} {\mkern 1mu} 19{z^{ - 1}} - 30{z^{ - 2}}\\
\quad \quad {\mkern 1mu} {\mkern 1mu} {\mkern 1mu} {\mkern 1mu} {\mkern 1mu} {\mkern 1mu} \underline {19{z^{ - 1}} - 95{z^{ - 2}} + 114{z^{ - 3}}} \\
\quad \quad \quad \quad \quad {\mkern 1mu} {\mkern 1mu} {\mkern 1mu} {\mkern 1mu} {\mkern 1mu} 65{z^{ - 2}} - 114{z^{ - 3}}\\
\quad \quad \quad \quad \quad {\mkern 1mu} {\mkern 1mu} {\mkern 1mu} {\mkern 1mu} {\mkern 1mu} \underline {65{z^{ - 2}} - 325{z^{ - 3}} + 390{z^{ - 4}}} \\
\quad \quad \quad \quad \quad {\mkern 1mu} {\mkern 1mu} {\mkern 1mu} {\mkern 1mu} \quad \quad {\mkern 1mu} {\mkern 1mu} {\mkern 1mu} {\mkern 1mu} {\mkern 1mu} {\mkern 1mu} {\mkern 1mu} {\mkern 1mu} {\mkern 1mu} {\mkern 1mu} {\mkern 1mu} {\mkern 1mu} 211{z^{ - 3}} - 390{z^{ - 4}}\\
\quad \quad \quad \quad \quad {\mkern 1mu} {\mkern 1mu} {\mkern 1mu} {\mkern 1mu} \quad \quad {\mkern 1mu} {\mkern 1mu} {\mkern 1mu} {\mkern 1mu} {\mkern 1mu} {\mkern 1mu} {\mkern 1mu} {\mkern 1mu} {\mkern 1mu} {\mkern 1mu} {\mkern 1mu} {\mkern 1mu} \quad \quad  \cdots  \cdots 
\end{array}}}\right.
\!\!\!\!\overline{\,\,\,\vphantom 1{\begin{array}{l}
z\\
\underline {z - 5 + 6{z^{ - 1}}} \\
\quad {\mkern 1mu} {\mkern 1mu} {\mkern 1mu} 5 - 6{z^{ - 1}}\\
\quad {\mkern 1mu} {\mkern 1mu} {\mkern 1mu} \underline {5 - 25{z^{ - 1}} + 30{z^{ - 2}}} \\
\quad \quad {\mkern 1mu} {\mkern 1mu} {\mkern 1mu} {\mkern 1mu} {\mkern 1mu} {\mkern 1mu} 19{z^{ - 1}} - 30{z^{ - 2}}\\
\quad \quad {\mkern 1mu} {\mkern 1mu} {\mkern 1mu} {\mkern 1mu} {\mkern 1mu} {\mkern 1mu} \underline {19{z^{ - 1}} - 95{z^{ - 2}} + 114{z^{ - 3}}} \\
\quad \quad \quad \quad \quad {\mkern 1mu} {\mkern 1mu} {\mkern 1mu} {\mkern 1mu} {\mkern 1mu} 65{z^{ - 2}} - 114{z^{ - 3}}\\
\quad \quad \quad \quad \quad {\mkern 1mu} {\mkern 1mu} {\mkern 1mu} {\mkern 1mu} {\mkern 1mu} \underline {65{z^{ - 2}} - 325{z^{ - 3}} + 390{z^{ - 4}}} \\
\quad \quad \quad \quad \quad {\mkern 1mu} {\mkern 1mu} {\mkern 1mu} {\mkern 1mu} \quad \quad {\mkern 1mu} {\mkern 1mu} {\mkern 1mu} {\mkern 1mu} {\mkern 1mu} {\mkern 1mu} {\mkern 1mu} {\mkern 1mu} {\mkern 1mu} {\mkern 1mu} {\mkern 1mu} {\mkern 1mu} 211{z^{ - 3}} - 390{z^{ - 4}}\\
\quad \quad \quad \quad \quad {\mkern 1mu} {\mkern 1mu} {\mkern 1mu} {\mkern 1mu} \quad \quad {\mkern 1mu} {\mkern 1mu} {\mkern 1mu} {\mkern 1mu} {\mkern 1mu} {\mkern 1mu} {\mkern 1mu} {\mkern 1mu} {\mkern 1mu} {\mkern 1mu} {\mkern 1mu} {\mkern 1mu} \quad \quad  \cdots  \cdots 
\end{array}}}}
\limits^{\displaystyle\hfill\,\,\, {{z^{ - 1}} + 5{z^{ - 2}} + 19{z^{ - 3}} + 65{z^{ - 4}} +  \cdots }}\]

则$f(0) = 0, \quad f(T) = 1, \quad f(2T) = 5, \quad f(3T) = 19, \quad f(4T) = 65, \cdots$,即
\[
f^*(t) = \delta(t-T) + 5\delta(t-2T) + 19\delta(t-3T) + 65\delta(t-4T) + \cdots
\]

\examples 求$F(z) = \dfrac{10z}{(z-1)(z-2)}$的逆变换。

\solve 采用留数计算法。
\[
F(z) z^{k-1} = \dfrac{10z^k}{(z-1)(z-2)}
\]
有两个极点$z_1 = 1,z_2 = 2$,且
\begin{align*}
	\text{Res}\big[F(z)z^{k-1}, 1\big] &= \lim\limits_{z \to 1}(z-1)F(z)z^{k-1} = -10\\
	\text{Res}\big[F(z)z^{k-1}, 2\big] &= \lim\limits_{z \to 2}(z-2)F(z)z^{k-1} = 10\cdot2^k
\end{align*}
所以
\[
f(kT) = 10(2^k - 1) \quad (k = 0,1,2,\cdots)
\]

\subsection{用$z$变换法解线性常系数差分方程}
\noindent \textbf{1. 差分的定义}

\defination[差分]
对误差信号$e(t)$进行采样,并将瞬时值$e(kT)$记为$e_k$或$e(k)$,则$e_k$的\\[0.5em]
一阶前向\dy[差分]{CF}定义为
\begin{equation}
	\Delta e_k = e_{k+1} - e_k
\end{equation}
二阶前向差分定义为
\begin{align}
	\Delta^2 \e_k &= \Delta\big(\Delta e_k\big) = \Delta e_{k+1} - \Delta e_k \notag\\
	&=e_{k+2} - 2e_{k+1}  + e_k
\end{align}
$n$阶前向差分定义为
\begin{equation}
	\Delta^n e_k = \Delta^{n-1}e_{k+1} - \Delta^{n-1}e_k
\end{equation}
一阶后向\dy[差分]{CF}定义为
\begin{equation}
	\nabla e_k = e_{k} - e_{k-1}
\end{equation}
$n$阶后向差分定义为
\begin{equation}
	\nabla^n e_k = \nabla^{n-1}e_{k} - \nabla^{n-1}e_{k-1}
\end{equation}

\noindent \textbf{2. 线性定常差分方程}

采样得到的到$k$时刻的所以输入数据记为$e_0,e_1,\cdots e_k$,对应的一直到$k-1$时刻的所有输出数据记为$u_0,u_1,\cdots u_{k-1}$,根据差分算法可以得到关于$u_k$的函数表达式
\begin{equation}
	u_k = f(e_0,e_1,\cdots, e_k,u_0,u_1,\cdots, u_{k-1})
\end{equation}
假设$u_k$是线性的,则其$n$阶\dy[线性差分方程]{XXCFFC}可表示为
\begin{equation}
	a_0u_k+a_1u_{k-1}+\cdots+a_nu_{k-n} = b_0e_k+b_1e_{k-1}+\cdots + b_me_{k-m}
\end{equation}

\noindent \textbf{3. 线性差分方程的求解}

\example[求解线性差分方程]\vspace*{0.5em}
\noindent  \hspace*{0.2em}  \tcbox[colframe =black, colback =black!10!white,boxrule=0.5mm,size=small,on line]{\color{black}{{ 解题步骤}}\hspace*{0.25em}}\hspace{1.5em}
\vspace*{0.5em}
\begin{enumerate}
	\item 对方程两边同时做$z$变换,注意利用$z$变换的超前定理
	\[
	Z\big[f(t+kT)\big] = z^kF(z) - \sum_{n = 0}^{k - 1} f(nT)z^{k-n}
	\]
	\item 求$C(z)$的$z$逆变换(三种方法,一般用留数计算法),得到$c(k)$。
\end{enumerate}

\examples 求解线性差分方程
\[
c(k+2) + 3c(k+1) + 2c(k) = 0
\]
初始条件为$c(0) = 0, c(1) = 1$.

\solve 对方程两边进行$z$变换,得
\[
z^2C(z) - z^2c(0)-zc(1)+3zC(z)-3zc(0)+2C(z) = 0
\]
代入初始条件$c(0) = 0, c(1) = 1$,化简后得
\[
C(z) = \dfrac{z}{z^2 + 3z + 2} = \dfrac{z}{(z+1)(z+2)} = \dfrac{z}{z+1} - \dfrac{z}{z+2}
\]
由留数计算法,求$C(z)z^{k-1}$的留数为
\begin{align*}
	\text{Res} \big[C(z)z^{k-1}, -1\big] &= \lim\limits_{z \to -1} (z+1)C(z)z^{k-1} = \lim\limits_{z \to -1} (z+1)\left(\dfrac{z}{z+1} - \dfrac{z}{z+2}\right)z^{k-1} = (-1)^k\\[0.5em]
	\text{Res} \big[C(z)z^{k-1}, -2\big] &= \lim\limits_{z \to -2} (z+2)C(z)z^{k-1} = \lim\limits_{z \to -2} (z+2)\left(\dfrac{z}{z+1} - \dfrac{z}{z+2}\right)z^{k-1} = 2(-2)^{k-1}=-(-2)^k
\end{align*}
所以最终解为
\[
c(k) = (-1)^k - (-2)^k\quad (k = 0,1,2,\cdots)
\]
\clearpage

\section{脉冲传递函数}
\subsection{脉冲传递函数的定义}

\begin{figure}[!htb]
	\centering
	\begin{tikzpicture}
		\node (A)[draw, inner sep = 5pt, xshift = 1.6cm]{$G(s)$};
		
		\draw[arrows={-Stealth}] (0cm,0cm) --(A)node[midway, above = 0mm]{$r^*(t)$};
		\draw[arrows={-Stealth}] (A) -- +(3cm,0cm)node[very near end, xshift = 0.6cm]{$c(t)$};
		\draw (-1.5cm, 0cm) -- (-0.5cm,0cm)node[midway, above = 0cm]{$r(t)$};
		\draw (-0.5cm, -0.2cm) -- (0cm,0cm);
		\draw[dashed] (2.5cm, 0cm) -- (2.5cm, 0.7cm) -- (3cm, 0.7cm);
		\draw (3cm, 0.5cm) -- (3.5cm,0.7cm);
		\draw[arrows={-Stealth}, dashed] (3.5cm, 0.7cm) -- (4.6cm,0.7cm)node[very near end, xshift = 0.5cm]{$c^*(t)$};

	\end{tikzpicture}
	\caption{开环采样系统}
	\label{开环采样}
\end{figure}

\defination[脉冲传递函数]
如图\ref{开环采样},输出采样信号$c^*(t)$的$z$变换与输入采样信号的$z$变换之比称为\dy[脉冲传递函数]{MCCDHS}
\begin{equation}
	G(z) = \dfrac{C(z)}{R(z)}
\end{equation}

\subsection{开环脉冲传递函数}
\noindent \textbf{1. 开环脉冲传递函数的推导}

由第\pageref{采样函数}页的公式\eqref{采样函数}可得
\begin{align}
	r^*(t) = \dfrac{1}{T}\sum_{n = -\infty}^{+\infty}r(t)\e^{\j n \omega_\text{s}t}\\[0.5em]
	R^*(s) = \dfrac{1}{T}\sum_{n = -\infty}^{+\infty}R(s - \j n \omega_\text{s})
\end{align}
由周期函数的定义可知$R^*(s)$是周期为$\j \omega_\text{s}$的周期函数。如图\ref{开环采样},连续环节的输出函数为
\begin{equation}
	C(s) = G(s)R^*(s)
\end{equation}
而
\begin{align}
	c^*(s) &= \dfrac{1}{T}\sum_{n = -\infty}^{+\infty}C(s - \j n \omega_\text{s}) = \dfrac{1}{T}\sum_{n = -\infty}^{+\infty}G(s - \j n \omega_\text{s}) R^*(s - \j n \omega_\text{s}) \notag \\[0.5em]
	& = \Bigg[\dfrac{1}{T}\sum_{n = -\infty}^{+\infty}G(s - \j n \omega_\text{s})\Bigg] R^*(s) =G^*(s)\cdot R^*(s)
\end{align}
即
\begin{equation}
	C(z) = G(z)R(z)
\end{equation}

\noindent \textbf{2. 串联环节的脉冲传递函数}
\begin{enumerate}[\hspace*{2em} (1) ]
	\item \textbf{串联环节间无采样开关时的脉冲传递函数}
	\begin{figure}[!htb]
		\centering
		\begin{tikzpicture}
			\node (A)[draw, inner sep = 5pt, xshift = 1.6cm]{$G_1(s)$};
			\node (B) [draw, inner sep =5pt, right of = A, node distance = 2.4cm]{$G_2(s)$};
			
			\draw[arrows={-Stealth}] (0cm,0cm) --(A);
			\draw[arrows={-Stealth}] (A) -- (B);
			\draw[arrows={-Stealth}] (B) --+(3cm,0cm);
			\draw (-1.5cm, 0cm) -- (-0.5cm,0cm);
			\draw (-0.5cm, 0cm) -- (0cm,0.2cm);
			\draw[dashed] (5cm, 0cm) -- (5cm, 0.7cm) -- (5.5cm, 0.7cm);
			\draw (5.5cm, 0.7cm) -- (6cm, 0.9cm);
			\draw [arrows={-Stealth},dashed] (6cm,0.7cm) -- (6.6cm,0.7cm); 
			\draw (0.3cm,1cm) --+(0cm,0.6cm);
			\draw (6.3cm,1cm) --+(0cm,0.6cm);
			\draw [arrows={Stealth-Stealth}] (0.3cm, 1.3cm) -- (6.3cm, 1.3cm)node[midway, above = 0cm]{$G(z)$};
			
			
		\end{tikzpicture}
		\caption{串联环节间无采样开关时的开环采样系统}
		\label{串联无开关}
	\end{figure}

	相当于把$G_1G_2$看作一个整体做$z$变换,其脉冲传递函数为
	\begin{equation}
		G(z) = Z\big[G_1(s)G_2(s)\big]=G_1G_2(z)
	\end{equation}
	\item \textbf{串联环节间无采样开关时的脉冲传递函数}
	\begin{figure}[!htb]
		\centering
		\begin{tikzpicture}
			\node (A)[draw, inner sep = 5pt, xshift = 1.6cm]{$G_1(s)$};
			\node (B) [draw, inner sep =5pt, right of = A, node distance = 2.4cm]{$G_2(s)$};
			
			\draw[arrows={-Stealth}] (0cm,0cm) --(A);
			\draw (A) -- +(0.9cm,0cm) --+(1.4cm, 0.2cm);
			\draw[arrows={-Stealth}] (3.cm, 0cm) -- (B);
			\draw[arrows={-Stealth}] (B) --+(3cm,0cm);
			\draw (-1.5cm, 0cm) -- (-0.5cm,0cm);
			\draw (-0.5cm, 0cm) -- (0cm,0.2cm);
			\draw[dashed] (5cm, 0cm) -- (5cm, 0.7cm) -- (5.5cm, 0.7cm);
			\draw (5.5cm, 0.7cm) -- (6cm, 0.9cm);
			\draw [arrows={-Stealth},dashed] (6cm,0.7cm) -- (6.6cm,0.7cm); 
			\draw (0.3cm,1cm) --+(0cm,0.6cm);
			\draw (6.3cm,1cm) --+(0cm,0.6cm);
			\draw [arrows={Stealth-Stealth}] (0.3cm, 1.3cm) -- (6.3cm, 1.3cm)node[midway, above = 0cm]{$G(z)$};
			
			
		\end{tikzpicture}
		\caption{串联环节间有采样开关时的开环采样系统}
		\label{串联有开关}
	\end{figure}
	
	其脉冲传递函数为各个连续环节z变换的乘积,记为
	\begin{equation}
		G(z) = Z\big[G_1(s)\big]Z\big[G_2(s)\big]=G_1(z)G_2(z)
	\end{equation}

	\item \textbf{有零阶保持器时的脉冲传递函数}
	\begin{figure}[!htb]
		\centering
		\begin{tikzpicture}
			\node (A)[draw, inner sep = 5pt, xshift = 1.6cm]{ZOH};
			\node (B) [draw, inner sep =5pt, right of = A, node distance = 2.4cm]{$G_2(s)$};
			
			\draw[arrows={-Stealth}] (0cm,0cm) --(A);
			\draw[arrows={-Stealth}] (A) -- (B);
			\draw[arrows={-Stealth}] (B) --+(3cm,0cm);
			\draw (-1.5cm, 0cm) -- (-0.5cm,0cm);
			\draw (-0.5cm, 0cm) -- (0cm,0.2cm);
			\draw[dashed] (5cm, 0cm) -- (5cm, 0.7cm) -- (5.5cm, 0.7cm);
			\draw (5.5cm, 0.7cm) -- (6cm, 0.9cm);
			\draw [arrows={-Stealth},dashed] (6cm,0.7cm) -- (6.6cm,0.7cm); 
			\draw (0.3cm,1cm) --+(0cm,0.6cm);
			\draw (6.3cm,1cm) --+(0cm,0.6cm);
			\draw [arrows={Stealth-Stealth}] (0.3cm, 1.3cm) -- (6.3cm, 1.3cm)node[midway, above = 0cm]{$G(z)$};
			
			
		\end{tikzpicture}
		\caption{含零阶保持器的开环采样系统}
		\label{串联零阶保持器开关}
	\end{figure}
\end{enumerate}

其开环传递函数为
\begin{align}
	G(z) &= \Bigg[\dfrac{1-\e^{-Ts}}{s}\cdot G(s)\Bigg] = Z\Bigg[\dfrac{1}{s}G(s)\Bigg] - Z\Bigg[\dfrac{1}{s}G(s)\e^{-Ts}\Bigg]\notag \\[0.5em]
	& = (1 - z^{-1})\cdot Z\Bigg[\dfrac{1}{s}G(s)\Bigg]
\end{align}

\subsection{闭环脉冲传递函数}
\begin{figure}[!htb]
	\centering
	\begin{tikzpicture}[circuit ee IEC]
		\node[bulb] (O) [draw, inner sep = 5pt, label = -85:$-$]{};
		\node (A) [draw, inner sep =6pt, right of = O, node distance =3cm]{$G(s)$};
		\node (B) [draw, inner sep = 6pt, below of = A, node distance = 1.5cm]{$H(s)$};
		
		\draw[arrows = {-Stealth}] (-1cm,0cm) -- (O)node[near start, above = 0mm]{$r(t)$};
		\draw (O) -- (1cm, 0cm) node[midway, above = 0mm]{$e(t)$} -- (1.5cm, 0.2cm);
		\draw[arrows = {-Stealth}] (1.5cm,0cm) -- (A)node[midway, above = 0cm]{$e^*(t)$};
		\draw[arrows = {-Stealth}] (A) --+(3cm, 0cm) node[very near end, xshift = 0.6cm]{$c(t)$};;
		\draw[arrows = {-Stealth}] (4.3cm, 0cm) -- +(0cm, -1.5cm) -- (B);
		\draw[arrows = {-Stealth}] (B) -- (0cm, -1.5cm) -- (O);
		\draw[dashed] (4.3cm,0cm) -- (4.3cm, 0.7cm)  -- (4.8cm, 0.7cm);
		\draw (4.8cm, 0.7cm) -- (5.3cm, 0.9cm);
		\draw[arrows = {-Stealth}, dashed] (5.3cm, 0.7cm) -- (6cm, 0.7cm)node[very near end, xshift = 0.5cm]{$c^*(t)$};;
		
	\end{tikzpicture}
\end{figure}

采样开关的输入和系统的输出分别为
\begin{align}
	E(s) &= R(s) - G(s)H(s)E^*(s)\\
	C(s) &= G(s) E^*(s)
\end{align}
可以得到
\begin{align*}
	E^*(s) &= R^*(s) - GH^*(s)E^*(s)\\
	C^*(s) &= G^*(s)E^*(s)
\end{align*}
整理得
\begin{equation}
	C^*(s) = \dfrac{G^*(s)}{1+GH^*(s)}R^*(s)
\end{equation}
即
\begin{align}
	C(z) &= \dfrac{G(z)}{1+GH(z)}R(z)\\[0.5em]
	\varPhi(z) &= \dfrac{C(z)}{R(z)} = \dfrac{G(z)}{1+GH(z)}
\end{align}

\example[求脉冲传递函数]
\examples \label{7.7}已知系统的结构图如图\ref{7.7.1},求输出脉冲函数$C(z)$.
\begin{figure}[!htb]
	\centering
	\begin{tikzpicture}[circuit ee IEC]
		\node[bulb] (O) [draw, inner sep = 5pt, label = -85:$-$]{};
		\node (A) [draw, right of = O, inner sep = 6pt, node distance = 2cm]{$G_1(s)$};
		\node (B) [draw, right of = A, inner sep = 6pt, node distance = 3cm]{$G_2(s)$};
		\node (C) [draw, below of = B, inner sep = 6pt, node distance = 1.5cm, xshift = -1.5cm]{$H(s)$};
		
		\draw [arrows={-Stealth}] (-1.5cm,0cm) -- (O)node[above = 0cm, near start]{$R(s)$};
		\draw [arrows={-Stealth}] (O) -- (A);
		\draw (A) -- (3.2cm,0cm) -- (3.7cm, 0.2cm);
		\draw [arrows={-Stealth}] (3.7cm, 0cm) -- (B);
		\draw [arrows={-Stealth}] (B) -- + (2cm,0cm)node[above = 0cm, very near end]{$C(s)$};
		\draw [arrows={-Stealth}] (6.3cm,0cm) -- +(0cm,-1.5cm) -- (C);
		\draw [arrows={-Stealth}] (C) -- (0cm,-1.5cm) -- (O);
		
	\end{tikzpicture}
	\caption{\ref{7.7}$\,\,$题系统结构图}
	\label{7.7.1}
\end{figure}

\solve 设采样前的信号为$E(s)$,采样后的信号为$E^*(s)$,则可以列出
\begin{align*}
	C(s) &= E^*(s) G_2(s)\\
	E(s) &= \big[R(s) - C(s)H(s)\big]G_1(s) = \big[R(s) - E^*(s)G_2(s)H(s)\big]G_1(s) = R(s)G_1(s) - E^*(s)G_2(s)H(s)G_1(s)
\end{align*}
对上面两个式子同时进行Z变换,可得
\begin{equation*}
	\begin{aligned}
		C(z) &= E(z)G_2(z)\\
		E(z) &= RG_1(z) - E(z)G_1G_2H(z)
	\end{aligned}
	\quad \Longrightarrow \quad
	E(z) = \frac{RG_1(z)}{1+G_1G_2H(z)}
	\quad \Longrightarrow \quad
	C(z) = \frac{G_2(z)RG_1(z)}{1+G_1G_2H(z)}
\end{equation*}


\section{采样系统的性能分析}
\subsection{稳定性}

\noindent \textbf{1. 从$s$平面到$z$平面的映射关系}

由$z$变换的定义
\begin{equation}
	z=\e^{Ts}
\end{equation}
令
\begin{equation}
	s = \sigma + \j \omega
\end{equation}
则
\begin{equation}
	z=\e^{\sigma T}\e^{\j \omega T}
\end{equation}
当$\sigma = 0$时,
\begin{equation}
	z = \e^{\j \omega T}
\end{equation}

此时$\omega: -\infty \to +\infty$,则$z$平面上的点绕单位圆逆时针绕无穷多圈。而当$\omega : -\dfrac{\omega_{\text{s}}}{2} \to \dfrac{\omega_\text{s}}{2}$时,则$z$平面上的点绕单位圆逆时针绕一圈。

由于闭环系统的特征根落在$s: \sigma < 0$的时候,系统稳定。即\textbf{如果闭环系统的特征根在$z$平面内的单位圆内,系统稳定。}其中,左半$s$平面上$ -\dfrac{\omega_{\text{s}}}{2} \le \omega \le \dfrac{\omega_\text{s}}{2}$的带称为\dy[主带]{ZD},其他部分称为\dy[次带]{CD}。
\vspace*{1.5em}

\noindent \textbf{2. $z$域的稳定条件和稳定性判据}

在$z$平面上系统稳定的充分必要条件是:系统的特征根必须全部位于$z$平面的单位圆内。

设采样系统的闭环脉冲传递函数为
\begin{equation}
	\varPhi(z) = \dfrac{C(z)}{R(z)} = \dfrac{M(z)}{D(z)}
\end{equation}
则闭环特征方程为$D(z) = 0.$
\begin{enumerate}[\hspace*{2em} (1) ]
	\item \dy[朱利稳定判据]{ZLWDPJ}\index{JURYWDPJ@Jury稳定判据}
	\begin{equation}
		D(z) = a_0 + a_1 z + a_2 z^2 + \cdots + a_n z^n
	\end{equation}
	且$a_n>0$,根据特征方程的系数构造朱利阵列,则方程$D(z) = 0$的根位于单位圆内的充分必要条件为
	\begin{equation}
		D_(1)>0, \quad (-1)^nD(-1)>0, \quad 
		\begin{cases}
			\, |a_0| < a_n\\
			\, |b_0| > |b_{n-1}|\\
			\, |c_0| > |c_{n-2}|\\
			\, \cdots\\
			\, |q_0| > |q_2|
		\end{cases}
	\end{equation} 
	
	\begin{table}
		\centering
		\setlength{\tabcolsep}{6.5mm}{
		\begin{tabular}{c|c|c|c|c|c|c|c}
			\hline
			行数 & $z^0$ & $z^1$ & $z^2$ & $\cdots$ & $\cdots$ & $z^{n-1}$ & $z^n$ \\
			\hline
			1& $a_0$ & $a_1$ & $a_2$ & $\cdots$ & $\cdots $ & $a_{n-1}$ & $a_n$\\
			2& $a_n$ & $a_{n-1}$ & $a_{n-2}$ & $\cdots$ & $\cdots $ &$a_1$ & $a_0$ \\
			3& $b_0$ & $b_1$ & $b_2$ & $\cdots$ & $\cdots $ & $b_{n-1}$ &  \\
			4& $b_{n-1}$ & $b_{n-2}$ & $\cdots$ & $\cdots $ &$b_1$ & $b_0$ & \\
			5& $c_0$ & $c_1$ & $c_2$ & $\cdots$ &  $c_{n-2}$ & &  \\
			6&$b_{n-2}$ & $\cdots$ & $\cdots $ &$b_1$ & $b_0$ & & \\
			$\cdots$ & $\cdots $& $\cdots$ &  $\cdots$ & $\cdots$ &  $\cdots$ & $\cdots$ &  $\cdots$ \\
			$2n-2$ & $q_2$ & $q_1$ & $q_0$ & & & & \\
			\hline
		\end{tabular}
		}
	\caption{朱利阵列}
	\label{朱利阵列}
	\end{table}
	其中,
	\begin{equation}
		b_k = 
		\begin{vmatrix}
			\,\, a_0 & a_{n-k} \,\,\\
			\,\, a_n & a_k \,\,
		\end{vmatrix}
		\quad \quad
		c_k = 
		\begin{vmatrix}
			\,\, b_0 & b_{n-1-k} \,\,\\
			\,\, b_{n-1} & b_k \,\,
		\end{vmatrix}
		\quad \quad \cdots
	\end{equation}

\item \dy[劳斯稳定判据]{RSWDPJ}\index{ROUTHWDPJ@Rough稳定判据}
\\
对于采样系统,也可用Routh判据分析其稳定性,但由于在z域中稳定区域是单位圆内,而不是左半平面,因此不能直接应用Routh判据。

引入双线性变换
\begin{equation}
	z = \dfrac{w + 1}{w - 1}
\end{equation}
此时可用Routh判据判断采样系统的稳定性。$s$平面、$z$平面、$w$平面的稳定区域如图\ref{三平面}所示。
\begin{figure}[!htb]
	\centering
	\includegraphics[width=0.7\linewidth]{pic/三平面.pdf}
	\caption{$s$平面、$z$平面、$w$平面的稳定区域}
	\label{三平面}
\end{figure}

\item $z$平面的根轨迹方法

根轨迹是当系统当特征方程当某些实参数由零变化到无穷时,特征根在复平面上移动的轨迹。
\begin{figure}[!htb]
	\centering
	\begin{tikzpicture}[circuit ee IEC]
		\node[bulb] (O) [draw, inner sep = 5pt, label = -85:$-$]{};
		\node (A) [draw, inner sep =6pt, right of = O, node distance =3cm]{$G(s)$};
		\node (B) [draw, inner sep = 6pt, below of = A, node distance = 1.5cm]{$H(s)$};
		
		\draw[arrows = {-Stealth}] (-1cm,0cm) -- (O)node[near start, above = 0mm]{$r(t)$};
		\draw (O) -- (1cm, 0cm) node[midway, above = 0mm]{$e(t)$} -- (1.5cm, 0.2cm);
		\draw[arrows = {-Stealth}] (1.5cm,0cm) -- (A)node[midway, above = 0cm]{$e^*(t)$};
		\draw[arrows = {-Stealth}] (A) --+(3cm, 0cm) node[very near end, xshift = 0.6cm]{$c(t)$};;
		\draw[arrows = {-Stealth}] (4.3cm, 0cm) -- +(0cm, -1.5cm) -- (B);
		\draw[arrows = {-Stealth}] (B) -- (0cm, -1.5cm) -- (O);
		\draw[dashed] (4.3cm,0cm) -- (4.3cm, 0.7cm)  -- (4.8cm, 0.7cm);
		\draw (4.8cm, 0.7cm) -- (5.3cm, 0.9cm);
		\draw[arrows = {-Stealth}, dashed] (5.3cm, 0.7cm) -- (6cm, 0.7cm)node[very near end, xshift = 0.5cm]{$c^*(t)$};;
		
	\end{tikzpicture}
	\caption{闭环采样系统}
	\label{闭环采样}
\end{figure}

如图\ref{闭环采样}所示,以基本的闭环采样系统为例,其特征方程为
\begin{equation}
	1+G(z) = 1+GH(z) = 0
\end{equation}
其与$s$平面所得到的根轨迹方程(见第\pageref{GH}页的公式\eqref{GH})形式完全一致。所以,\textbf{$z$平面和$s$平面中绘制根轨迹图的方法完全一致,只是在分析系统稳定性时,特征根位置的意义不同}。
\end{enumerate}

\examples \label{7.8}已知采样系统的闭环特征方程为
\vspace*{-0.5em}
\[
D(z) = -0.125+0.75z-1.5z^2+z^3
\vspace*{-0.5em}
\]
判断系统的稳定性。

\solve 由于$D(1) = 0.125 > 0, \quad (-1)^3D(-1) = 3.375>0$,求得朱利阵列如表\ref{7.8.1}.
\begin{table}[!htb]
	\centering
	\setlength{\tabcolsep}{6.5mm}{
		\begin{tabular}{c|c|c|c|c}
			\hline
			行数 & $z^0$ & $z^1$ & $z^2$ & $z^3$ \\
			\hline
			1& -0.125&0.75&-1.5&1\\
			2& 1&-1.5&0.75&-0.125\\
			3& -0.98&1.41&-0.56&\\
			4&-0.56 & 1.41 & -0.96&\\
			\hline
		\end{tabular}
	}
	\caption{\ref{7.8}$\,$的朱利阵列}
	\label{7.8.1}
\end{table}

由$|a_0| = 0.125<a_3=1,\quad |b_0| = 0.98 > |b_2| = 0.56$,所以系统是稳定的。
\vspace*{1em}

\examples 已知系统的闭环特征方程为
\vspace*{-0.5em}
\[
D(z) = 45z^3 -117z^2+119z-39 = 0
\vspace*{-0.5em}
\]
判断系统的稳定性。
\clearpage
\vspace*{-2.5em}

\solve 将$z = \dfrac{w+1}{w-1}$代入特征方程式,得
\begin{align*}
	D(w) = 45 \left(\dfrac{w+1}{w-1}\right)^3 - 117 \left(\dfrac{w+1}{w-1}\right)^2 + 119\left(\dfrac{w+1}{w-1}\right) - 39 = 0\\
	45(w+1)^3-117(w+1)^2(w-1)+119(w+1)(w-1)^2-39(w-1)^3=0
\end{align*}
整理得
\[
w^3 + 2w^2 + 2w + 40 = 0
\]
列出劳斯表如下:
\begin{center}
	\begin{tabular}{ccc}
		$w^3$ & 1&2\\
		$w^2$ & 2&40\\
		$w^1$ & -18&\\
		$w^0$ & 40 &\\
	\end{tabular}
\end{center}
由劳斯判据可得:有2个根在$w$右半平面,即有2个根在$z$平面上的单位圆外,故系统为不稳定。
\vspace*{1em}

\examples \label{7.10}若已经求得系统的$G(z) = GH(z) = \dfrac{0.632Kz}{(z-1)(z-0.368)}$,绘制系统的根轨迹。

\solve 由根轨迹绘制的基本法则(见第\pageref{根轨迹的基本法则}页的章节\ref{根轨迹的基本法则}\nameref{根轨迹的基本法则}),可以绘制根轨迹如图\ref{7.10-1}。
\begin{figure}[!htb]
	\centering
	\vspace*{-1em}
	\includegraphics[width=0.3\linewidth]{pic/7-10根轨迹.jpeg}
	\vspace*{-1em}
	\caption{\ref{7.10}$\,$ 系统的根轨迹}
	\label{7.10-1}
\end{figure}

\subsection{闭环极点与瞬态响应之间的关系}
设采样系统的闭环传递函数为
\begin{align}
	\varPhi(z) &= \dfrac{b_0 z^m + b_1 z^{m-1}+\cdots + b_{m-1}z+b_m}{a_0 z^n + a_1 z^{n-1}+\cdots + a_{n-1}z +a_n}\\[0.5em]
	& = \dfrac{b_0(z-z_1)(z-z_2)\cdots(z-z_m)}{a_0(z-p_1)(z-p_2)\cdots(z-p_n)} = \dfrac{M(z)}{D(z)}
\end{align}
若输入信号为单位阶跃,则
\begin{equation}
	C(z) = \varPhi(z)\cdot R(z)  = \dfrac{M(z)}{D(z)}\cdot \dfrac{z}{z-1}
\end{equation}

将$\dfrac{C(z)}{z}$按部分分式展开,得
\begin{align}
	C(z) &= \dfrac{M(1)}{D(1)}\cdot\dfrac{z}{z-1} + \sum_{k=1}^{n} \dfrac{c_kz}{z - p_k}\\
	c(mT) &= \dfrac{M(1)}{D(1)} + \sum_{k=1}^{n}c_kp_k^m \quad (m = 0,1,2,\cdots)
	\label{稳态分量}
\end{align}
\begin{figure}[!htb]
	\centering
	\includegraphics[width=0.58\linewidth]{pic/极点与瞬态响应.jpg}
	\vspace*{-1.5em}
	\caption{不同极点所对应的瞬态响应}
	\label{极点与瞬态响应}
\end{figure} 
公式\eqref{稳态分量}中第一项为\dy[稳态分量]{WTFL},第二项为\dy[瞬态分量]{STFL}。可以看出,稳态分量的变化规律取决于极点在$z$平面中的位置。如图\ref{极点与瞬态响应},下面分五种情况进行讨论
\vspace*{-0.5em}
\begin{enumerate}[\hspace*{2em}(a) ]
	\item $0<p_k<1$\quad 极点所对应的瞬态分量是单调收敛的 \vspace*{-0.5em}
	\item $p_k>1$\quad 极点所对应的瞬态分量是单调发散的 \vspace*{-0.5em}
	\item $-1<p_k<0$ \quad 极点所对应的瞬态分量是正负交替的 \vspace*{-0.5em}
	\item $p_k<-1$\quad 极点所对应的瞬态分量是正负交替的 \vspace*{-0.5em}
	\item $p_k, p_{k+1}$为一对共轭复根,且$|p_k|<1$\quad 极点所对应的瞬态分量是按余弦规律振荡 \vspace*{-0.5em}
	\item $p_k, p_{k+1}$为一对共轭复根,且$|p_k|>1$\quad 极点所对应的瞬态分量是按余弦规律振荡 \vspace*{-0.5em}
\end{enumerate}

\subsection{稳态误差}

\begin{figure}[!htb]
	\centering
	\begin{tikzpicture}[circuit ee IEC]
		\node[bulb] (O) [draw, inner sep = 5pt, label = -85:$-$]{};
		\node (A) [draw, inner sep =6pt, right of = O, node distance =3cm]{$G(s)$};
		\node (B) [draw, inner sep = 6pt, below of = A, node distance = 1.5cm]{$H(s)$};
		
		\draw[arrows = {-Stealth}] (-1cm,0cm) -- (O)node[near start, above = 0mm]{$r(t)$};
		\draw (O) -- (1cm, 0cm) node[midway, above = 0mm]{$e(t)$} -- (1.5cm, 0.2cm);
		\draw[arrows = {-Stealth}] (1.5cm,0cm) -- (A)node[midway, above = 0cm]{$e^*(t)$};
		\draw[arrows = {-Stealth}] (A) --+(3cm, 0cm) node[very near end, xshift = 0.6cm]{$c(t)$};;
		\draw[arrows = {-Stealth}] (4.3cm, 0cm) -- +(0cm, -1.5cm) -- (B);
		\draw[arrows = {-Stealth}] (B) -- (0cm, -1.5cm) -- (O);
		\draw[dashed] (4.3cm,0cm) -- (4.3cm, 0.7cm)  -- (4.8cm, 0.7cm);
		\draw (4.8cm, 0.7cm) -- (5.3cm, 0.9cm);
		\draw[arrows = {-Stealth}, dashed] (5.3cm, 0.7cm) -- (6cm, 0.7cm)node[very near end, xshift = 0.5cm]{$c^*(t)$};
		
	\end{tikzpicture}
	\caption{闭环系统的稳态误差}
	\label{闭环稳态误差}
\end{figure}
如图\ref{闭环稳态误差},在输入信号$r(t)$作用下,误差的$z$变换的表达式为
\begin{equation}
	E(z) = \dfrac{1}{1+G(z)}R(z)
\end{equation}

\noindent \textbf{1. 当输入为阶跃函数时}
\begin{equation}
	R(z) = \dfrac{z}{z-1}
\end{equation}

定义\dy[静态位置误差系数]{JTWZWCXS}为
\begin{equation}
	K_p = \lim\limits_{z \to 1} G(z)
\end{equation}
则根据终值定理,有
\begin{equation}
	e(\infty) = \lim\limits_{z \to 1} (z - 1)\dfrac{1}{1 + G(z)} \cdot \dfrac{z}{z-1} = \dfrac{1}{1+K_p}
\end{equation}
\vspace*{0.5em}

\noindent \textbf{2. 当输入为单位斜坡函数时}
\begin{equation}
	R(z) = \dfrac{Tz}{(z-1)^2}
\end{equation}

定义\dy[静态速度误差系数]{JTSDWCXS}为
\begin{equation}
	K_v = \lim\limits_{z \to 1} (z-1)G(z)
\end{equation}
则根据终值定理,有
\begin{equation}
	e(\infty) = \lim\limits_{z \to 1} (z - 1)\dfrac{1}{1 + G(z)} \cdot \dfrac{Tz}{(z-1)^2} = \dfrac{T}{K_v}
\end{equation}
\vspace*{0.5em}

\noindent \textbf{3. 当输入为单位加速度信号时}
\begin{equation}
	R(z) = \dfrac{T^2z(z+1)}{2(z-1)^3}
\end{equation}

定义\dy[静态加速度误差系数]{JTJSDWCXS}为
\begin{equation}
	K_a = \lim\limits_{z \to 1} (z-1)^2G(z)
\end{equation}
则根据终值定理,有
\begin{equation}
	e(\infty) = \lim\limits_{z \to 1} (z - 1)\dfrac{1}{1 + G(z)} \cdot \dfrac{T^z(z+1)}{2(z-1)^3} = \dfrac{T^2}{K_a}
\end{equation}

总结如下表\ref{三个误差与输入信号的关系}.
\begin{table}[!htb]
	\centering
	\setlength{\tabcolsep}{8mm}{
		\begin{tabular}{ccccc}
			\toprule
			& & & & \vspace*{-1.3em} \\ 
			系统型别 & 静态误差系数 & $r(t) = 1(t)$ & $r(t) = t$ & $r(t) = \dfrac{1}{2}t^2$ \\
			& & & & \vspace*{-1.3em} \\
			\midrule
			& & & & \vspace*{-1.3em} \\
			0型系统 & 
			$\begin{aligned}
				K_\text{p} &= K\\
				K_\text{v} &= 0\\
				K_\text{a} &= 0
			\end{aligned}$
			& $e_{\text{ss}} = \dfrac{1}{1+K}$
			& $e_{\text{ss}} = \infty$
			& $e_{\text{ss}} = \infty$\\
			& & & & \vspace*{-1.2em} \\
			\hline
			& & & & \vspace*{-1.2em} \\
			\RMN[1] 型系统 &
			$\begin{aligned}
				K_\text{p} &= \infty\\
				K_\text{v} &= K\\
				K_\text{a} &= 0
			\end{aligned}$
			& $e_{\text{ss}} = 0$
			& $e_{\text{ss}} = \dfrac{T}{K}$
			& $e_{\text{ss}} = \infty$\\
			& & & & \vspace*{-1.2em} \\
			\hline
			& & & & \vspace*{-1.2em} \\
			\RMN[2] 型系统 &
			$\begin{aligned}
				K_\text{p} &= \infty\\
				K_\text{v} &= \infty \\
				K_\text{a} &= K
			\end{aligned}$
			& $e_{\text{ss}} = 0$
			& $e_{\text{ss}} = 0$
			& $e_{\text{ss}} = \dfrac{T^2}{K}$\\
			& & & & \vspace*{-1.2em} \\
			\hline
			& & & & \vspace*{-1.2em} \\
			\RMN[3] 型系统 &
			$\begin{aligned}
				K_\text{p} &= \infty\\
				K_\text{v} &= \infty\\
				K_\text{a} &= \infty
			\end{aligned}$
			& $e_{\text{ss}} = 0$
			& $e_{\text{ss}} = 0$
			& $e_{\text{ss}} = 0$\\
			& & & & \vspace*{-1.2em} \\
			\bottomrule
		\end{tabular}
	}
	\caption{稳态误差、静态误差系数与输入信号之间的关系}
	\label{三个误差与输入信号的关系}
\end{table}
\clearpage
\vspace*{-2.5em}

\examples \label{7.11}已知采样系统的结构如图所示,其中
$$
G(s) = \dfrac{2(0.5s + 1)}{s^2}
$$
采样周期$T=0.2\,\text{s}$,求在输入信号$r(t) = 1+t+0.5t^2(t>0)$的作用下,系统的稳态误差。
	\begin{figure}[!htb]
	\centering
	\begin{tikzpicture}[circuit ee IEC]
		\node (A)[draw, inner sep = 6pt, xshift = 1.6cm]{ZOH};
		\node (B) [draw, inner sep =5pt, right of = A, node distance = 2.4cm]{$G_2(s)$};
		\node[bulb] (O)[draw, inner sep = 6pt, xshift = -1.8cm, label = -85:$-$]{};
		
		\draw[arrows={-Stealth}] (-3cm,0cm) -- (O)node[near start, above = 0cm]{$r(t)$};
		\draw[arrows={-Stealth}] (0cm,0cm) --(A);
		\draw[arrows={-Stealth}] (A) -- (B);
		\draw[arrows={-Stealth}] (B) --+(3cm,0cm)node[midway, above = 0cm]{$c(t)$};
		\draw (-1.5cm, 0cm) -- (-0.5cm,0cm);
		\draw (-0.5cm, 0cm) -- (0cm,0.2cm);		
		\draw (6cm,0cm) -- +(0cm,-1cm) -- (-1.8cm,-1cm) -- (O);
		
	\end{tikzpicture}
	\caption{\ref{7.11}$\,$ 系统结构图}
	\label{7.11-1}
\end{figure}

\solve 采样系统的开环传递函数为
\begin{align*}
	G(z) &= \left(1 - z^{-1}\right)\cdot Z\Bigg[\dfrac{1}{s}\cdot \dfrac{2(0.5s+1)}{s^2}\Bigg]\\[0.5em]
	& = \dfrac{z-1}{z}\Bigg[\dfrac{Tz}{(z-1)^2}+ \dfrac{T^2z(z+1)}{(z-1)^3}\Bigg]\\[0.5em]
	& = \dfrac{0.24z - 0.16}{(z-1)^2}
\end{align*}

\textbf{(求稳态误差前一定要判断系统的稳定性)}采样系统的闭环特征方程为
\[
D(z) = z^2 - 1.76z + 0.84 = 0
\]
所以,$D(1) = 0.08>0,\quad D(-1) = 3.6>0\quad |a_0|=0.84<a_2=1$,即系统稳定。

由于系统的传递函数存在两个$(z-1)^{-1}$,即此为\RMN[2]型系统,则
\begin{align*}
	K_\text{p} &= 0\\
	K_\text{v} &= 0\\
	K_\text{a} & = \lim\limits_{z \to 1}(z-1)^2G(z) = \lim\limits_{z \to 1}(0.24z-0.16) = 0.08
\end{align*}
所以,在输入$r(t)=1+t+0.5t^2$作用下的稳态误差为
\[
e(\infty) = \dfrac{1}{1+K_\text{p}} + \dfrac{T}{K_\text{v}}+\dfrac{T^2}{K_\text{a}} = 0 + 0 + \dfrac{0.04}{0.08} = 0.5
\]

\section{采样系统的数字校正}
\begin{figure}[!htb]
	\centering
	\begin{tikzpicture}[circuit ee IEC]
		\node[bulb] (O) [draw, inner sep =6pt, label = -85:$-$]{};
		\node (A) [draw, right of = O, inner sep = 5pt, node distance = 3cm]{$D(z)$};
		\node (B) [draw, right of = A, inner sep = 5pt, node distance = 3cm]{$G(s)$};
		
		\draw[arrows={-Stealth}] (-1.2cm,0cm) -- (O)node[near start, above = 0cm]{$r(t)$};
		\draw (O) -- (1cm,0cm)node[midway, above = 0cm]{$e(t)$} -- (1.5cm,0.2cm);
		\draw [arrows={-Stealth}] (1.5cm, 0cm) -- (A)node[midway, above = 0cm]{$e^*(t)$};
		\draw (A) -- +(1.2cm,0cm)node[midway, above = 0cm, xshift = 1mm]{$u(t)$} -- +(1.7cm,0.2cm);
		\draw [arrows={-Stealth}] (4.7cm, 0cm) -- (B)node[midway, above = 0cm]{$u^*(t)$};
		\draw [arrows={-Stealth}](B) --+ (2.5cm, 0cm)node[midway, above = 0cm]{$c(t)$};
		\draw [arrows={-Stealth}](7.5cm,0cm) --+(0cm, -1cm) -- (0cm ,-1cm) -- (O);
		
	\end{tikzpicture}
\caption{含数字校正装置的采样系统}
\label{数字校正}
\end{figure}

如图\ref{数字校正}所示的闭环采样系统,闭环脉冲传递函数为
\begin{equation}
	\varPhi(z) = \dfrac{G(z)D(z)}{1+G(z)D(z)}
\end{equation}
系统的误差为
\begin{equation}
	E(z) = R(z) - C(z) = \left[1 - \varPhi(z)\right]R(z)
\end{equation}

设输入为时间的幂函数$At^q(t>0)$,其中$q$是正整数,则
\begin{equation}
	R(z) = \dfrac{B(z)}{\big(1-z^{-1}\big)^{q+1}}
\end{equation}
其中,$B(z)$是$z^{-1}$的有限次多项式.
\vspace*{0.5em}

为了使稳态误差
\begin{equation}
	e(\infty) = \lim\limits_{z\to 1}\big(1-z^{-1}\big)E(z) = \lim\limits_{z\to 1}\big(1-z^{-1}\big)\left[1-\varPhi(z)\right]R(z) = \lim\limits_{z\to 1}\left[1-\varPhi(z)\right]\dfrac{B(z)}{\big(1-z^{-1}\big)^{q}} = 0
\end{equation}
则$1-\varPhi(z)$项至少有$q+1$个$(1-z^{-1})$因子,不妨取$q+1$个,即可以写为
\begin{equation}
	1-\varPhi(z) = \big(1-z^{-1}\big)^{q+1}\varphi(z)
\end{equation}
其中,$\varphi(z)$是关于$z^{-1}$的多项式,并且不含因子$\big(1-z^{-1}\big)$.
\vspace*{0.5em}

又为了使系统能在尽可能少的周期内实现对输入的完全跟踪,应使中$\varphi(z)$所含$z^{-1}$项的数目最少,为此应取$\varphi(z) = 1$,即
\begin{equation}
	\varPhi(z) = 1 - \big(1-z^{-1}\big)^{q+1}\varphi(z) \xlongequal{\textstyle \quad \varphi(z) = 1\quad}  1 - \big(1-z^{-1}\big)^{q+1}
\end{equation}
代入脉冲传递函数,可得
\begin{equation}
	D(z) = \dfrac{1}{G(z)}\dfrac{\varPhi(z)}{1 - \varPhi(z)}
\end{equation}

特别地,此时误差函数的$z$变换为
\begin{equation}
	E(z) = R(z) - C(z) = \left[1 - \varPhi(z)\right]R(z) = \big(1-z^{-1}\big)^{q+1}\varphi(z) \cdot \dfrac{B(z)}{\big(1-z^{-1}\big)^{q+1}} = \varphi(z)B(z)\xlongequal{\textstyle \quad \varphi(z) = 1\quad} B(z)
\end{equation}

\noindent 下面以输入为阶跃、斜坡和等加速度函数时的情形。
\begin{enumerate}[\hspace*{2em} 1. ]
	\item $\bm{r(t) = 1(t)}$
	\begin{equation}
		R(z) = \dfrac{z}{z-1} = \dfrac{1}{\big(1-z^{-1}\big)^{0+1}} = \dfrac{B(z)}{\big(1-z^{-1}\big)^{q+1}}
	\end{equation}
	最少拍无差系统的闭环传递函数为
	\begin{equation}
		\varPhi(z) = 1 - \big(1-z^{-1}\big)^{0+1} = z^{-1}
	\end{equation}
	此时误差信号的$z$变换为
	\begin{equation}
		E(z) = B(z) = 1 = 1\cdot z^0
	\end{equation}
	系统经过\textcolor{blue}{1拍}便可以完全跟踪上输入信号。
	\vspace*{0.5em}
	
	\item $\bm{r(t) = t}$
	\begin{equation}
		R(z) = \dfrac{Tz}{(z-1)^2} = \dfrac{Tz^{-1}}{\big(1-z^{-1}\big)^{1+1}} = \dfrac{B(z)}{\big(1-z^{-1}\big)^{q+1}}
	\end{equation}
	最少拍无差系统的闭环传递函数为
	\begin{equation}
		\varPhi(z) = 1 - \big(1-z^{-1}\big)^{1+1} = 2z^{-1} - z^{-2}
	\end{equation}
	此时误差信号的$z$变换为
	\begin{equation}
		E(z) = B(z) = T z^{-1}
	\end{equation}
	系统经过\textcolor{blue}{2拍}便可以完全跟踪上输入信号。
	\vspace*{1em}
	
	\item $\bm{r(t) = \dfrac{1}{2}t^2}$
	\begin{equation}
		R(z) = \dfrac{T^2z(z+1)}{2(z-1)^3} = \dfrac{0.5T^2(z^{-1}+z^{-2})}{\big(1-z^{-1}\big)^{2+1}} = \dfrac{B(z)}{\big(1-z^{-1}\big)^{q+1}}
	\end{equation}
	最少拍无差系统的闭环传递函数为
	\begin{equation}
		\varPhi(z) = 1 - \big(1-z^{-1}\big)^{2+1} = 3z^{-1} - 2z^{-2} + z^{-3}
	\end{equation}
	此时误差信号的$z$变换为
	\begin{equation}
		E(z) = B(z) = 0.5T^2(z^{-1}+z^{-2})
	\end{equation}
	系统经过\textcolor{blue}{3拍}便可以完全跟踪上输入信号。
\end{enumerate}


























	
	%第九章-常微分方程
	\chapter{常微分方程}
\section{一阶微分方程}
\sj
\begin{equation}
\frac{\d y}{\d x}=f(x,y)
\end{equation} 
\subsection{可分离变量的方程}
\sj
一般地,如果一个一阶微分方程能写成
\begin{equation}
P(x,y)\d x=Q(x,y) \d y
\end{equation}
那么原方程称为可分离变量的微分方程.

\example[可分离变量的方程1]
形如
\begin{equation}
\frac{\d y}{\d x}=f(ax+by+c)
\end{equation}
\par 解法:作变量替换$z=ax+by+c$即可.
\begin{equation}
\frac{\d z}{\d x}=a+b\frac{\d y}{\d x}=a+bf(z)
\end{equation}

\example[可分离变量的方程2]
形如
\begin{equation}
\frac{\d y}{\d x}=f(x,y)
\end{equation}
其中,$f(x,y)$是齐次函数.
\par 解法:将$f(x,y)$写成$\displaystyle \frac{y}{x}$或$\displaystyle h\left( \frac{y}{x}\right) $的形式
\begin{equation}
y'=h(\frac{y}{x})
\end{equation}
作变量替换$\displaystyle u=\frac{y}{x}$,
\begin{equation}
y'=u+xu'=h(u)
\end{equation}
即
\begin{equation}
\frac{\d u}{\d x}=\frac{h(u)-u}{x}
\end{equation}
这是一个可分离变量的方程.
\newpage 

\example[可分离变量的方程3]
形如
\begin{equation}
\frac{\d y}{\d x}=f\left(\frac{a_1x+b_1y+c_1}{a_2x+b_2y+c_2}\right) 
\end{equation}
分两种情况讨论:
\jg
\par 1. $\displaystyle \frac{a_1}{a_2}=\frac{b_1}{b_2}$.则存在常数$k$,使得$(a_2,b_2)=k(a_1,b_1)$,作变量替换$z=a_1x+b_1y$,则
\begin{equation*}
\frac{\d z}{\d x}=a_1+b_1\frac{\d y}{\d x}=a_1+b_1f\left( \frac{z+c_1}{kz+c_2}\right) 
\end{equation*}
这是一个可分离变量的方程.
\jg 
\par 2. $\displaystyle \frac{a_1}{a_2} \ne \frac{b_1}{b_2}$.
\jg
\par \quad \quad (1) 当$c_1=c_2$时,$f(x,y)$是齐次方程.
\par \quad \quad (2) 当$c_1\ne c_2$时,将式子改写为:(可以用待定系数法求出$x_0,y_0$)
\begin{equation*}
f\left( \frac{a_1x+b_1y+c_1}{z_2x+b_2y+c_2}\right) =f\left( \frac{a_1(x-x_0)+b_1(y-y_0)}{a_2(x-x_0)+b_2(y-y_0)}\right) 
\end{equation*}
\par 作变量替换$u=x-x_0,v=y-y_0.$
\begin{equation*}
\frac{\d y}{\d x}=f\left( \frac{a_1u+b_1v}{a_2u+b_2v}\right) 
\end{equation*}
\par 这是一个齐次方程.

\section{一阶线性微分方程}
形如
\begin{equation}
\frac{\d y}{\d x}+P(x)y=Q(x)
\end{equation}
的方程称为一阶线性微分方程.
\jg

\subsection{一阶线性齐次微分方程}
\begin{equation}
\frac{\d y}{\d x}+P(x)y=0
\end{equation}
\par 解法:可分离变量的微分方程.
\par 通解:
\begin{equation}
y^*(x)=C_0\e^{-\int_{x_0}^{x}P(t)\,\,\d t}
\end{equation}

\theorem[线性齐次微分方程通解定理]
一阶线性齐次微分方程的通解包含了它的一切解.
\jg

\subsection{一阶线性非齐次微分方程}
\begin{equation}
\frac{\d y}{\d x}+P(x)y=Q(x)
\end{equation}
是对应齐次方程$\displaystyle  \frac{\d y}{\d x}+P(x)y=0$的线性非齐次方程.
\par 解法:常数变易法.作变量替换
\begin{equation}
y(x)=u(x)\e^{-\int_{x_0}^{x}P(t)\,\,\d t}
\end{equation}
求出一阶导数$y'$然后将$y,y'$反代回$y'+P(x)y=Q(x)$求出$u(x)$再反代回$y(x)$即可求出一阶线性非齐次微分方程的通解.
\jg

\theorem[线性非齐次微分方程通解定理]
线性非齐次方程的一个特解与相应的齐次方程的通解之和,构成非齐次方程的通解.
\par 线性非齐次方程的两个特解之差构成齐次方程的通解.
\par 线性非齐次方程的两个特解之和仍为 线性非齐次方程的特解.
\jg

\example[伯努利方程]
\begin{equation}
\frac{\d y}{\d x}+P(x)y=Q(x)y^\alpha
\end{equation}
其中$\alpha \ne 0,1$且为常数.
\par 解法:两端同时除以$y^\alpha$
\begin{equation*}
y^{-\alpha}\frac{\d y}{\d x}+P(x)\cdot y^{1-\alpha}=Q(x) \quad \Longleftrightarrow \quad \frac{1}{1-\alpha}\cdot \frac{\d y^{1-\alpha}}{\d x}+P(x)\cdot y^{1-\alpha}=Q(x)
\end{equation*}
\par 作变量替换$z=y^{1-\alpha}$
\begin{equation}
\frac{1}{1-\alpha}\cdot \frac{\d z}{\d x}+P(x)\cdot z=Q(x)
\end{equation}
\par 这是一个一阶线性非齐次方程.

\section{全微分方程与积分因子}
\begin{equation}
\frac{\d y}{\d x}=\frac{-P(x,y)}{Q(x,y)}
\end{equation}
它可以写成$P(x,y)\d x+Q(x,y)\d y=0$的形式.
\jg
\subsection{全微分方程}
如果$P(x,y)\d x+Q(x,y)\d y=0$满足
\begin{equation}
\frac{\partial P}{\partial y} = \frac{\partial Q}{\partial x}
\end{equation}
则可以表示成全微分方程的形式.求全微分方程的方法有:
\par 1. 路径无关的曲线积分
\begin{equation}
\int_{(x_0,y_0)}^{(x,y)}P\,\d x+Q\,\d y
\end{equation}
\par 2. 现将$P(x,y)$看作是$x$的函数,由$\displaystyle \frac{\partial u}{\partial x}=P$求出$P(x,y)$关于$x$的原函数$u_1(x,y)$,令
\begin{equation*}
u(x,y)=u_1(x,y)+\varphi (y)
\end{equation*}
然后将$u(x,y)$代入$\displaystyle \frac{\partial u}{\partial y}=Q$求出$\varphi(y)$后代入$u(x,y)$即可.
\jg

\subsection{积分因子}
\tdefination[积分因子]
设方程
$$
M(x,y)\d x+N(x,y)\d y=0
$$
不是全微分方程.若存在函数$\mu (x,y) \ne 0$,使
$$
\mu M\d x+\mu N\d y=0
$$
是全微分方程.则称$\mu$是积分因子.

\theorem[特殊积分因子的求法1]
如果
\begin{equation}
\frac{\displaystyle \frac{\partial M}{\partial y}-\frac{\partial N}{\partial x}}{N(x,y)}
\end{equation}
仅是$x$的函数,记为$F(x)$,则
\begin{equation}
\mu(x)=\e ^{\int_{x_0}^{x}F(t) \,\d t}
\end{equation}

\theorem[特殊积分因子的求法2]
如果
\begin{equation}
\frac{\displaystyle \frac{\partial N}{\partial x}-\frac{\partial M}{\partial y}}{M(x,y)}
\end{equation}
仅是$y$的函数,记为$G(y)$,则
\begin{equation}
\mu(y)=\e ^{\int_{y_0}^{y}G(t) \,\d t}
\end{equation}

\section{可降阶的二阶微分方程}
\sj
\example[不显含未知函数$y$的方程]
不显含未知函数$y$的方程
\begin{equation}
F(x,y',y'')=0
\end{equation}
\par 解法:作变量替换$z=y'$,得到关于新未知函数$z$的一阶方程
\begin{equation}
F(x,z,z')=0
\end{equation}
求出$z$后再求积分$\displaystyle y=\int z \d x$即可.

\example[不显含未知函数$x$的方程]
不显含未知函数$x$的方程
\begin{equation}
F(yy',y'')=0
\end{equation}
\par 解法:作变量替换$p=y'$,并将$y$看作是自变量
\begin{equation}
y''=\frac{\d p}{\d x}=\frac{\d p}{\d y}\cdot \frac{\d y}{\d x}=p \frac{\d p}{\d y}
\end{equation}
反代回原方程,
\begin{equation}
F(x,p,p\frac{\d p}{\d y})=0
\end{equation}
求出$p$后再求积分$\displaystyle y=\int p \d x$即可.

\section{高阶线性微分方程}
高阶线性微分方程的形式如下:
\begin{equation}
y^{(n)}(x)+p_1(x)\,y^{n-1}(x)+\cdots+p_{n_1}\,y'(x)+p_n(x)\,y(x)=f(x) 
\end{equation}
\subsection{二阶线性齐次微分方程}
二阶线性微分方程的形式为
\begin{equation}
y''(x)+p(x)\,y'(x)+q(x)\,y(x)=f(x)
\end{equation}
其中$f(x)\equiv 0$,则为二阶线性齐次微分方程,若$f(x) \not \equiv 0$,则为二阶线性非齐次微分方程.
\jg

\theorem[二阶线性微分方程解的唯一性定理]
设函数$p(x),q(x),f(x)$在$[a,b]$上连续,则初值问题
\begin{equation}
\begin{cases}
y''+p(x)+q(x)y=f(x),\\
y(x_0)=y_0,\,\,y'(x_0)=y_1
\end{cases}
\end{equation}
在区间$[a,b]$内存在唯一解$y(x)$.可以推广到一般$n$阶线性微分方程解的唯一性.
\jg

\defination[线性相关性]
设$m$个函数$\varphi_1(x),\varphi_2(x),\cdots,\varphi_m(x)$在区间$[a,b]$上有定义,若存在$m$个不全为0的常数$k_1,k_2,\cdots,k_m$,使
\begin{equation}
k_1\varphi_1(x)+k_2\varphi_2(x)+\cdots+k_m\varphi_m(x)\equiv 0,\quad x\in[a,b]
\end{equation}
则称函数组$\varphi_1(x),\varphi_2(x),\cdots,\varphi_m(x)$在区间$[a,b]$上线性相关,否则称函数组$\varphi_1(x),\varphi_2(x),\cdots,\varphi_m(x)$在区间$[a,b]$上线性无关.
\jg
\newpage
\theorem[二阶线性齐次方程解的性质]
若$y_1(x),y_2(x)$是二阶线性齐次方程的两个通解,则它们的任意一个线性组合($C_1,C_2$为任意常数)
\begin{equation}
C_1y_1(x)\pm C_2y_2(x)
\end{equation}
也是这个二阶线性齐次方程的解.


\theorem[二阶线性齐次方程通解的线性相关性]
设$\varphi_1(x),\varphi_2(x),\,\, x\in(a,b)$是二阶线性齐次方程的两个解.则$\varphi_1(x),\varphi_2(x)$在$(a,b)$上线性相关的充要条件是:它们确定的朗斯基行列式
\begin{equation}
W(x)=
\left| 
\begin{array}{cc}
\varphi_1(x) & \varphi_2(x)\\
\varphi'_1(x) & \varphi'_2(x)
\end{array}
\right| 
\equiv 0
\quad 
x \in (a,b)
\end{equation}
当$W(x)\not \equiv 0$时,$\varphi_1(x),\varphi_2(x)$在$(a,b)$上线性无关.
\jg

\theorem[二阶线性齐次方程通解的结构]
若$\varphi_1(x),\varphi_2(x)$是二阶线性齐次方程的两个线性无关解,则它们的任意一个线性组合($C_1,C_2$为任意常数)
\begin{equation}
C_1\varphi_1(x)+C_2\varphi_2(x)
\end{equation}
也是这个二阶线性齐次方程的通解.
\jg

\theorem[$n$阶线性齐次方程通解的结构]
若$\varphi_1(x),\varphi_2(x),\cdots,\varphi_n(x)$是$n$阶线性齐次方程
$$
y^{(n)}(x)+p_1(x)\,y^{n-1}(x)+\cdots+p_{n_1}\,y'(x)+p_n(x)\,y(x)=f(x) 
$$
的$n$个线性无关解,则它们的任意一个线性组合
\begin{equation}
C_1\varphi_1(x)+C_2\varphi_2(x)+\cdots+C_n\varphi_n(x)
\end{equation}
是这个$n$阶线性齐次方程的通解,其中$C_1,C_2,\cdots,C_n$为任意常数.
\jg
\jg

\subsection{二阶线性非齐次微分方程}
\ttheorem[二阶线性非齐次方程通解的结构]
若$y^*(x)$是二阶线性非齐次方程的一个特解,又$C_1\varphi_1(x)+C_2\varphi_2(x)$是对应的二阶线性齐次方程的通解($C_1,C_2$为任意常数),则
\begin{equation}
y(x)=C_1\varphi_1(x)+C_2\varphi_2(x)+y^*(x)
\end{equation}
是这个二阶线性非齐次方程的通解.
\jg
\newpage
\theorem[二阶线性非齐次方程解的性质1]
若$y_1(x),y_2(x)$是二阶线性非齐次方程的两个特解,则
\begin{equation*}
y_1(x)+y_2(x)
\end{equation*}
是这个二阶线性非齐次方程的解,而
\begin{equation*}
y_1(x)-y_2(x)
\end{equation*}
是这个二阶线性非齐次方程对应的二阶线性齐次方程的解.
\jg

\theorem[二阶线性非齐次方程解的性质2]
若$y_1(x),y_2(x)$分别是二阶线性非齐次方程
\begin{equation*}
\begin{split}
y''+py'+q=f_1(x)\\
y''+py'+q=f_2(x)
\end{split}
\end{equation*}
的解,则函数$y(x)=y_1(x)+y_2(x)$是二阶线性非齐次方程
\begin{equation}
y''+py'+q=f_1(x)+f_2(x)
\end{equation}
的解.
\par 也就是说求方程$y''+py'+q=f_1(x)+f_2(x)$的特解可以先分别求出
\begin{equation*}
\begin{split}
y''+py'+q=f_1(x)\\
y''+py'+q=f_2(x)
\end{split}
\end{equation*}
的特解$y_1(x),y_2(x)$再相加.

\section{二阶线性常系数微分方程}
\subsection{二阶线性常系数齐次微分方程}
\tdefination[特征根与与特征方程]
考虑方程
\begin{equation}
y''+py'+qy=0
\end{equation}
其中,$p,q$是常数.考虑这个方程解的形式为
\begin{equation*}
y=\e^{\lambda x}
\end{equation*}
代入原方程,消去$\e^{\lambda x}$则得到特征方程
\begin{equation}
\lambda^2+p\lambda+q=0
\end{equation}
特征方程的根
\begin{equation}
\lambda=\frac{1}{2}(-p\pm\sqrt{p^2-4q})
\end{equation}
称为特征根.
\jg
\newpage

\theorem[二阶线性常系数齐次微分方程的通解]
\begin{table}[!htb]
	\centering
	\renewcommand{\arraystretch}{1}
	\setlength{\tabcolsep}{20mm}{
		\begin{tabular}{cc}
			\toprule[1.5pt] 
			\rowcolor[gray]{0.9}   特征根 &  通解形式 \\  
			\midrule
			两相异实根$\lambda_1,\lambda_2$& $\displaystyle C_1\e^{\lambda _1x}+C_2\e^{\lambda _2x}$\\
			二重根$\lambda_1$ & $\displaystyle (C_1+C_2x)\e^{\lambda _1x}$ \\
			共轭复根$\lambda_{1,2}=\alpha \pm \beta \rm{i}$& $\displaystyle \e^{\alpha x}(C_1\cos \beta x+C_2\sin \beta x)$ \\
			\bottomrule[1.5pt]
		\end{tabular}  
	}
	\caption{二阶线性常系数齐次微分方程的通解}
	\renewcommand{\arraystretch}{1}
	\label{二阶线性常系数齐次微分方程的通解}
\end{table} 

\theorem[$n$阶线性常系数齐次微分方程的通解]
对于$n$阶线性齐次常系数微分方程
\begin{equation}
y^{(n)}+a_1y^{(n-1)}+a_2y^{(n-2)}+\cdots+a_{n-1}y'+a_ny=0
\end{equation}
对应的特征方程为
\begin{equation}
\lambda^n+a_1\lambda^{n-1}+a_2\lambda^{n-2}+\cdots +a_{n-1}\lambda +a_n=0
\end{equation}
每个特征根所对应的线性无关的特解如下表\ref{n阶线性常系数齐次微分方程的通解}.
\begin{table}[!htb]
	\centering
	\renewcommand{\arraystretch}{1}
	\setlength{\tabcolsep}{15mm}{
		\begin{tabular}{cc}
			\toprule[1.5pt] 
			\rowcolor[gray]{0.9} 特征根 &  对应的线性无关的特解 \\  
			\midrule
			单实根$\lambda$& $\displaystyle e^{\lambda x}$\\
			$k$重实根$\lambda(k>1)$ & $\displaystyle e^{\lambda x},xe^{\lambda x},\cdot ,x^{k-1}e^{\lambda x}$ \\
			单共轭复根$\lambda_{1,2}=\alpha \pm \beta \rm{i}$& $\displaystyle \e^{\alpha x}\cos \beta x,\e^{\alpha x}\sin \beta x$ \\
			\makecell[c]{$m$重共轭复根$(m>1)$\\$\lambda_{1,2}=\alpha \pm \beta \rm{i}$}& \makecell[c]{$\displaystyle \e^{\alpha x}\cos \beta x,\e^{\alpha x}\sin \beta x,x\e^{\alpha x}\cos \beta x,x\e^{\alpha x}\sin \beta x,\cdots$,\\$x^m\e^{\alpha x}\cos \beta x,x^m\e^{\alpha x}\sin \beta x$}\\
			\bottomrule[1.5pt]
		\end{tabular}  
	}
	\caption{$n$阶线性常系数齐次微分方程的通解}
	\renewcommand{\arraystretch}{1}
	\label{n阶线性常系数齐次微分方程的通解}
\end{table} 
\par 由下表\ref{n阶线性常系数齐次微分方程的通解}可得到相应的$n$个线性无关的特解.然后作它们的线性组合,即可得到$n$阶线性齐次常系数微分方程的通解.

\subsection{二阶线性常系数非齐次微分方程}
\begin{table}[!htb]
	\centering
	\renewcommand{\arraystretch}{1}
	\setlength{\tabcolsep}{9mm}{
		\begin{tabular}{ccc}
			\toprule[1.5pt] 
			\rowcolor[gray]{0.9}  $f(x)$的形式& 条件 &  特解的形式 \\  
			\midrule
			\multirow{3}{*}{$P_n(x)$} & 0不是特征根& $Q_n(x)$\\
			& 0是单特征根 &$xQ_n(x)$ \\
			& 0是重特征根& $x^2Q_n(x)$\\
			\midrule
			\multirow{3}{*}{$a\e^{\alpha x} $} & $\alpha $不是特征根& $A\e^{\alpha x} $\\
			& $\alpha $是单特征根 &$Ax\e^{\alpha x} $ \\
			& $\alpha $是重特征根& $Ax^2\e^{\alpha x} $\\
			\midrule
			\multirow{2}{*}{$a\cos \beta x+b\sin \beta x$} & $\pm \beta \rm{i}$不是特征根 &$A\cos \beta x+B\sin \beta x$ \\
			& $\pm \beta \rm{i}$是特征根 &$x(A\cos \beta x+B\sin \beta x)$ \\
			\midrule
			\multirow{3}{*}{$P_n(x)\e^{\alpha x}$} & $\alpha $不是特征根& $Q_n(x)\e^{\alpha x} $\\
			& $\alpha $是单特征根 &$xQ_n(x)\e^{\alpha x} $ \\
			& $\alpha $是重特征根& $x^2Q_n(x)\e^{\alpha x} $\\
			\midrule
			\multirow{2}{*}{\makecell[c]{$P_n(x)\e^{\alpha x}(a\cos \beta x+b\sin \beta x)$\\$\beta \ne 0$}} & $\alpha \pm \beta \rm{i}$不是特征根 &$\e^{\alpha x }[Q_n\cos \beta x+R_n\sin \beta x]$ \\
			& $\alpha \pm \beta \rm{i}$是特征根 &$x\e^{\alpha x }[Q_n\cos \beta x+R_n\sin \beta x]$ \\
			\bottomrule[1.5pt]
		\end{tabular}  
	}
	\caption{多种特殊线性常系数非齐次微分方程的特解}
	\renewcommand{\arraystretch}{1}
	\label{多种特殊线性常系数非齐次微分方程的特解}
\end{table} 

\tdefination[二阶线性常系数非齐次微分方程]
方程
\begin{equation}
y''+py'+qy=f(x)
\end{equation}
其中,$p,q$为常数,$f(x)\not \equiv 0$.
\jg

\theorem[多种特殊线性常系数非齐次微分方程的特解]
$f(x)$满足一定特殊条件的情况下,可以求得特解如上页表\ref{多种特殊线性常系数非齐次微分方程的特解}.其中$P_n,Q_n,R_n$是$n$次多项式.可以用待定系数法代定系数然后反代回原线性常系数非齐次微分方程通过比对系数可以求出所有参数.
\par 常系数非齐次微分方程的通解就等于其对应的常系数齐次微分方程的通解加上常系数非齐次微分方程的一个特解.
\jg

\inference[常数变易法求二阶线性常系数非齐次微分方程的通解]
1. 求出相应二阶线性常系数齐次微分方程$y''+p(x)y'+q(x)y=f(x)$的两个线性无关的解$\varphi_1(x),\varphi_1(x)$,即齐次方程的通解为($C_1,C_2$为任意常数)
\begin{equation}
y^*(x)=C_1\varphi_1(x)+C_2\varphi_2(x)
\end{equation}
\par 2. 将上述的任意常数$C_1,C_2$替换为待定的函数$C_1(x),C_2(x)$,即
\begin{equation}
y(x)=C_1(x)\varphi_1(x)+C_2(x)\varphi_2(x)
\label{常数变易法求二阶线性常系数非齐次微分方程的通解}
\end{equation}
\par 3. 解下列方程,求出待定的函数$C_1(x),C_2(x)$
\begin{equation}
\begin{cases}
C'_1(x)\varphi_1(x)+C'_2(x)\varphi_2(x)=0\\
C'_1(x)\varphi'_1(x)+C'_2(x)\varphi'_2(x)=f(x)
\end{cases}
\end{equation}
\par 4. $C_1(x),C_2(x)$代入原式\eqref{常数变易法求二阶线性常系数非齐次微分方程的通解}即求出通解
\begin{equation}
y(x)=C_1(x)\varphi_1(x)+C_2(x)\varphi_2(x)
\end{equation}
\newpage

\example[欧拉方程]
形如
\begin{equation}
a_0x^ny^{(n)}+a_1x^{n-1}y^{(n-1)}+\cdots +a_{n-1}xy'+a_n=9
\end{equation}
的方程称为欧拉方程.
\par 解法:当$x>0$时,令$\displaystyle x=\e^{t}$ ,当$x<0$时,令$\displaystyle x=-\e^{t}$ 即可化为线性常系数微分方程.
\begin{equation}
b_0\,\frac{\d^n y}{\d t^n}+b_1\,\frac{\d^{n-1} y}{\d t^{n-1}}+\cdots+b_{n-1}\,\frac{\d y}{\d t}+b_n=0
\end{equation}
注:由$\displaystyle x=\e^{t}$可得
\begin{equation}
\frac{\d^k y}{\d x^k}=\left(C_1\,\frac{\d y}{\d t}+C_2\frac{\d^2 y}{\d t^2}+\cdots+C_k\frac{\d^k y}{\d x^k} \right) \e^{-kt},\quad k=1,2,\cdots,n
\end{equation}
其中,$C_i(i=1,2,\cdots,k)$为常数.



	
	%附章
	% 附章专用计数器
%\newcounter{CFsection}[chapter]
%\renewcommand{\theCFsection}{\stepcounter{CFsection} \textbf{F.\arabic{CFsection}}}
%\newcounter{CFsubsection}[section]
%\renewcommand{\theCFsubsection}{\stepcounter{CFsubsection} \textbf{F.\arabic{CFsection}.\arabic{CFsubsection}}}
%\renewcommand{\theequation}{F.\arabic{equation}}


% 附章专用标题格式
%\titleformat{\chapter}{\bfseries\Huge\color{titlepurple}}{附章 \quad}{0pt}{}
%\titleformat{\section}{\Large\color{titlepurpleb}}{\bfseries{\theCFsection}\quad  }{0pt}{}
%\titleformat{\subsection}{\large\color{titlepurplec}}{\bfseries{\theCFsubsection}\quad  }{0pt}{}


\chapter{补充公式及基本化简方法}
\thispagestyle{empty}
\section{三角函数公式}
\subsection{和差化积}
\begin{equation}
	\boxed{\sin \alpha + \sin \beta = 2 \sin \dfrac{\alpha + \beta}{2}  \cos  \dfrac{\alpha - \beta}{2} }
	\vspace*{0.5em}
\end{equation}
\renewcommand{\arraystretch}{1.6}
\begin{tabular}{l}
\proof \quad $\displaystyle \sin \alpha + \sin \beta = \sin  \left( \dfrac{\alpha + \beta}{2} + \dfrac{\alpha - \beta}{2}\right) + \sin \left( \dfrac{\alpha + \beta}{2} - \dfrac{\alpha - \beta}{2} \right)$\\[-1em]
$\displaystyle = \sin  \dfrac{\alpha + \beta}{2} \cos  \dfrac{\alpha - \beta}{2}  + \cos  \dfrac{\alpha + \beta}{2}\sin  \dfrac{\alpha - \beta}{2} + \sin \dfrac{\alpha + \beta}{2}\cos \dfrac{\alpha - \beta}{2} - \cos \dfrac{\alpha + \beta}{2} \sin \dfrac{\alpha - \beta}{2}$\\
$= 2 \sin \dfrac{\alpha + \beta}{2}  \cos  \dfrac{\alpha - \beta}{2} $
\end{tabular}

\vspace*{1.2em}

\begin{equation}
	\boxed{\sin \alpha - \sin \beta = 2 \cos \dfrac{\alpha + \beta}{2}\sin \dfrac{\alpha- \beta}{2}}
	\vspace*{0.5em}
\end{equation}
\begin{tabular}{l}
	\proof \quad $\displaystyle \sin \alpha - \sin \beta = \sin \left( \dfrac{\alpha + \beta }{2} + \dfrac{\alpha - \beta}{2}\right) - \sin \left( \dfrac{\alpha - \beta}{2} - \dfrac{\alpha - \beta}{2} \right)$ \\[-1em]
	$=\sin \dfrac{\alpha + \beta}{2} \cos \dfrac{\alpha - \beta}{2} + \cos \dfrac{\alpha + \beta}{2}\sin \dfrac{\alpha - \beta}{2} - \sin \dfrac{\alpha + \beta}{2}\cos\dfrac{\alpha - \beta}{2} + \cos \dfrac{\alpha + \beta}{2} \sin \dfrac{\alpha - \beta}{2}$\\
	$= 2 \cos \dfrac{\alpha + \beta}{2} \sin \dfrac{\alpha - \beta}{2}$
\end{tabular}

\vspace*{1.2em}

\begin{equation}
	\boxed{\cos \alpha + \cos \beta = 2 \cos \dfrac{\alpha + \beta}{2}\cos \dfrac{\alpha- \beta}{2}}
	\vspace*{0.5em}
\end{equation}
\begin{tabular}{l}
	\proof \quad $\displaystyle \cos \alpha + \cos \beta = \cos \left( \dfrac{\alpha + \beta }{2} + \dfrac{\alpha - \beta}{2}\right) + \cos \left( \dfrac{\alpha - \beta}{2} - \dfrac{\alpha - \beta}{2} \right)$ \\[-1em]
	$=\cos \dfrac{\alpha + \beta}{2} \cos \dfrac{\alpha - \beta}{2} - \sin \dfrac{\alpha + \beta}{2}\sin \dfrac{\alpha - \beta}{2} + \cos \dfrac{\alpha + \beta}{2} \cos\dfrac{\alpha - \beta}{2} - \sin \dfrac{\alpha + \beta}{2} \sin \dfrac{\alpha - \beta}{2}$\\
	$= 2 \cos \dfrac{\alpha + \beta}{2} \cos \dfrac{\alpha - \beta}{2}$
\end{tabular}

\vspace*{2em}

\begin{equation}
	\boxed{\cos \alpha - \cos \beta = - 2 \sin \dfrac{\alpha + \beta}{2}\sin \dfrac{\alpha- \beta}{2}}
	\vspace*{0.5em}
\end{equation}
\begin{tabular}{l}
	\proof \quad $\displaystyle \cos \alpha - \cos \beta = \cos \left( \dfrac{\alpha + \beta }{2} + \dfrac{\alpha - \beta}{2}\right) - \cos \left( \dfrac{\alpha - \beta}{2} - \dfrac{\alpha - \beta}{2} \right)$ \\[-1em]
	$=\cos \dfrac{\alpha + \beta}{2} \cos \dfrac{\alpha - \beta}{2} - \sin \dfrac{\alpha + \beta}{2}\sin \dfrac{\alpha - \beta}{2} - \cos \dfrac{\alpha + \beta}{2} \cos\dfrac{\alpha - \beta}{2} + \sin \dfrac{\alpha + \beta}{2} \sin \dfrac{\alpha - \beta}{2}$\\
	$= - 2 \sin \dfrac{\alpha + \beta}{2} \sin \dfrac{\alpha - \beta}{2}$
\end{tabular}

\begin{equation}
	\boxed{\tan \alpha + \tan \beta = \dfrac{\sin \big(\alpha + \beta \big)}{\cos \alpha \cos \beta} }
	\vspace*{0.5em}
\end{equation}
\quad \proof \quad $\displaystyle \tan \alpha + \tan \beta = \dfrac{\sin \alpha}{\cos \alpha} + \dfrac{\sin \beta}{\cos \beta} = \dfrac{\sin \alpha \cos \beta + \cos \alpha + \sin \beta}{\cos \alpha \cos \beta} = \dfrac{\sin \big( \alpha + \beta)}{\cos \alpha \cos \beta}$


\subsection{积化和差}
\vspace*{-1em}
\begin{align}
	\boxed{\sin \alpha \cos \beta = \dfrac 1 2 \Big[\sin \big( \alpha + \beta \big) + \sin \big( \alpha - \beta \big) \Big]} \\[0.5em]
	\boxed{\sin \alpha \sin \beta = -\dfrac 1 2 \Big[\cos \big(\alpha + \beta \big) - \sin \big( \alpha - \beta \big) \Big]} \\[0.5em]
	\boxed{\cos \alpha \sin \beta = \dfrac 1 2 \Big[\sin \big( \alpha + \beta \big) - \sin \big( \alpha - \beta \big) \Big]} \\[0.5em]
	\boxed{\cos \alpha \cos \beta = \dfrac 1 2 \Big[\cos \big(\alpha + \beta \big) + \cos \big( \alpha - \beta \big) \Big]}
\end{align}

\subsection{降幂公式(升幂公式)}
由二倍角公式,得
\begin{equation*}
	\cos 2 \alpha = \cos^2 \alpha - \sin^2 \alpha = 1 - 2\sin^2 \alpha = 2 \cos^2 \alpha - 1
\end{equation*}

\noindent 从而推导出以下公式
\begin{equation}
	\boxed{\sin^2 \alpha = \dfrac{1 - \cos 2 \alpha }{2}} \qquad \qquad \boxed{\cos^2 \alpha = \dfrac{1 + \cos 2 \alpha }{2}} \qquad \qquad \boxed{\tan^2 \alpha = \dfrac{1 - \cos 2 \alpha }{1 + \cos 2 \alpha}}
	\vspace*{0.5em}
\end{equation}

\subsection{万能公式}
\begin{equation}
	\boxed{\sin 2 \theta = \dfrac{2 \tan \theta}{1 + \tan^2 \theta}} \qquad \qquad \boxed{\cos 2 \theta = \dfrac{1 - \tan^2 \theta}{1 + \tan^2 \theta}} \qquad \qquad \boxed{\tan 2 \theta = \dfrac{2 \tan \theta}{1 - \tan^2 \theta}}
	\vspace*{0.5em}
\end{equation}
\proof 由二倍角公式,得
\vspace*{-1em}
\begin{align*}
	& \sin 2 \theta = 2 \sin \theta \cos \theta = \dfrac{2 \sin \theta \cos \theta }{\sin^2 \theta + \cos^2 \theta} = \dfrac{2 \tan \theta}{1 + \tan^2 \theta}. \\[0.5em]
	&\cos 2 \theta = \cos^2 \theta - \sin^2 \theta = \dfrac{\cos^2 \theta - \sin^2 \theta}{\cos^2 \theta + \sin^2 \theta} = \dfrac{1 - \tan^2 \theta}{1 + \tan^2 \theta}.\\[0.5em]
	&\tan 2 \theta = \tan (\theta + \theta) = \dfrac{2 \tan\theta}{1 - \tan^2 \theta}.
\end{align*}


\subsection{其他公式}
\vspace*{-1em}
\begin{align}
	\dfrac{1 + \sin \theta}{1 - \sin \theta} = \dfrac{\big(1 + \sin \theta \big)^2}{1 - \sin^2 \theta} = \left( \dfrac{1 + \sin \theta}{\cos \theta} \right)^2 = \big(\csc \theta + \tan \theta \big)^2 \\[0.5em]
	\dfrac{1 + \cos \theta}{1 - \cos \theta} = \dfrac{\big( 1 + \cos \theta \big)^2}{1 - \cos^2 \theta} = \left( \dfrac{1 + \cos \theta}{\sin \theta} \right)^2 = \big( \sec \theta + \cot \theta \big)^2
\end{align}

\subsection{双曲函数公式}
双曲函数有着特别的性质(与三角函数类似)。具体如下:【注:以下性质用双曲函数的定义可以直接证明,故证明省略】
\begin{align}
	\boxed{\ch^2 x - \sh^2 x = 1} \qquad \quad\\[0.5em]
	\boxed{\sh (x + y) = \sh x \,\ch y + \ch x \,\sh y} \\[0.5em]
	\boxed{\sh (x - y) = \sh x \,\ch y - \ch x \,\sh y} \\[0.5em]
	\boxed{\ch (x + y) = \ch x \,\ch y + \sh x \,\sh y} \\[0.5em]
	\boxed{\ch (x - y) = \ch x \,\ch y - \sh x \,\sh y}  
\end{align}


















	%打印索引—————————————
	\newpage
	\addcontentsline{toc}{chapter}{附录}
	\addcontentsline{toc}{section}{索引}
	\color{titlepurplec}
	\appendix
	\CJKfamily{kai}
	\printindex
	%———————————————
	
\end{document}