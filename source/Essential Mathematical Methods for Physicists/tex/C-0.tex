\chapter{绪论}
数学物理方程课程主要研究以下三个方程。

\defination[泊松方程]
\index{PSFC@泊松方程}
\begin{equation}
	\frac{\partial^2u}{\partial x^2} + \frac{\partial^2 u}{\partial y^2} +\frac{\partial^2 u}{\partial z^2} = f
\end{equation}

\defination[波动方程]
\index{BDFC@波动方程}
\begin{equation}
	\frac{\partial^2 u}{\partial t^2} = a^2\left(	\frac{\partial^2u}{\partial x^2} + \frac{\partial^2 u}{\partial y^2} +\frac{\partial^2 u}{\partial z^2} \right)
\end{equation}

\defination[热传导方程]
\index{RCDFC@热传导方程}
\begin{equation}
		\frac{\partial u}{\partial t} = a^2\left(	\frac{\partial^2u}{\partial x^2} + \frac{\partial^2 u}{\partial y^2} +\frac{\partial^2 u}{\partial z^2} \right)
\end{equation}

可以发现,以上方程都是\dy[二阶线性偏微分方程]{EJXXPWFFC}.
\begin{enumerate}[\hspace*{3em}$-$]
	\item 二阶:导数的最高阶数为2;
	\item 线性:只存在对于未知函数的\dy[线性运算]{XXYS}\footnote{线性运算:对于运算$\mathcal{L}(u_1,u_2)$,满足:$\dfrac{\partial^n \mathcal{L}(\alpha_1 u_1+\alpha_2 u_2)}{\partial t^n} = \dfrac{\alpha_1 \partial^n \mathcal{L}(u_1)}{\partial t^n}+ \dfrac{\alpha_2 \partial^n \mathcal{L}(u_2)}{\partial t^n}$}。
\end{enumerate}

