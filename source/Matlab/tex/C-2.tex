\chapter{Matlab矩阵处理}
\thispagestyle{empty}
\section{特殊矩阵}
\subsection{常用的特殊矩阵}
常用的产生通用的特殊矩阵的函数如下表。
	\begin{table}[!htb]
	\centering
	\setlength{\tabcolsep}{10mm}{
		\begin{tabular}{cc}
			\toprule
			函数 & 说明\\
			\midrule
			\lstinline|zeros| & 产生全0矩阵,即零矩阵 \\
			\lstinline|ones| & 产生全1矩阵,即幺矩阵 \\
			\lstinline|eye| & 产生单位矩阵\\
			\lstinline|rand| & 产生$(0,1)$区间均匀分布的随机矩阵\\
			\lstinline|randn| & 产生均值为0,方差为1的标准正态分布随机矩阵\\
			\bottomrule
		\end{tabular}
	}
\end{table}

这些函数的调用格式相似,以\lstinline|zeros|函数为例子:
	\begin{table}[!htb]
	\centering
	\setlength{\tabcolsep}{10mm}{
		\begin{tabular}{cc}
			\toprule
			函数 & 说明\\
			\midrule
			\lstinline|zeros(m)| & 产生$m \times m$零矩阵 \\
			\lstinline|zeros(m,n)| & 产生$m \times n$零矩阵 \\
			\lstinline|zeros(size(A))| & 产生与矩阵$\bm{A}$同样大小的零矩阵 \\
			\bottomrule
		\end{tabular}
	}
\end{table}

\warn[
建立随机矩阵
\begin{myitemize}
	\item 已知满足在$(0,1)$区间上均匀分布的随机数$x_i$,则任意$(a,b)$区间上均匀分布的随机数为
	\begin{equation}
		y_i = a+(b-a)x_i
	\end{equation}
	\item 已知满足均值为0,方差为1的标准正态分布的随机数$x_i$,则均值为$\mu$,方差为$\sigma^2$的随机数为
	\begin{equation}
		y_i = \mu + \sigma x_i
	\end{equation}
\end{myitemize}
]
\newpage
\vspace*{-2.5em}
\examples 建立随机矩阵。
\begin{enumerate}
	\item 在区间$(20,50)$内均匀分布的5阶随机矩阵。
	\item 均值为0.6,方差为0.1的5阶正态分布随机矩阵。
\end{enumerate}

\begin{lstlisting}
>> x = 20 + (50 - 20)*rand(5)
x = 
		21.6185		37.0647		43.8285		27.8891		22.5146
		35.9239		34.0817		29.3365		39.6224		26.8693
		43.3750		20.3571		35.8560		40.6764		47.4001
		48.0203		30.1137		24.9695		42.4445		24.5713
		23.8972		24.8655		38.0595		33.5162		44.7745
>> y = 0.6 + sqrt(0.1)*randn(5)
y =
		0.6392		1.5196		0.5138		0.4881		0.6106
		1.0543		0.8610		0.9474		0.3396		0.1783
		-0.0201		1.0361		0.5121		0.1013		0.9565
		0.5375		0.2654		0.8218		0.7606		0.7107
		0.2180		0.4518		-0.0488		0.6892		0.5054
\end{lstlisting}

\subsection{用于专门学科的特殊矩阵}
\begin{enumerate}
	\item 魔方矩阵
	\begin{itemize}
		\item 性质:
		\begin{itemize}
			\item $n$阶魔方矩阵,其元素由$1, 2, 3, \cdots, n^2$共$n^2$个整数组成。
			\item 每行、每列及两条对角线上的元素和都等于$\displaystyle \frac{n(n^2+1)}{2}$.
		\end{itemize}
	\item 函数:\lstinline|magic(n)|,注意:$n>3$.
	\end{itemize}
\item 范德蒙矩阵
	\begin{itemize}
		\item 性质:范德蒙矩阵的最后一列为1,倒数第二列为一个指定的向量,其他各列是最后一列与倒数第二列对应元素的乘积。即
		\begin{equation}
			\bm{A} =
			\begin{bmatrix}
				a_1^{n-1} & \cdots & a_1^2 & a_1 & 1\\
				a_2^{n-1} & \cdots & a_2^2 & a_2 & 1\\
				a_3^{n-1} & \cdots & a_3^2 & a_3 & 1\\
				\vdots & \ddots & \vdots & \vdots & \vdots\\
				a_n^{n-1} & \cdots & a_n^2 & a_n & 1
			\end{bmatrix}
		\label{vander}
		\end{equation}
	\item 函数:\lstinline|vander(V)|,生成的矩阵与式\eqref{vander}完全相同,其中输入的列向量$\bm{V}$为:
	\begin{equation*}
		\bm{V} =
		\begin{bmatrix}
			a_1\\
			a_2\\
			\vdots\\
			a_n
		\end{bmatrix}
	\end{equation*}
	\end{itemize}

\item 希尔伯特矩阵
\begin{itemize}
	\item 性质:每个元素为
	\begin{equation}
		h_{ij}=\frac{1}{i+j-1}
	\end{equation}
这个矩阵是一个高度病态的矩阵,即任何一个元素法身微笑变动,整个矩阵的值和你矩阵都会发生很大变化,病态程度和阶数相关。
\item 函数:\lstinline|hilb(n)|,逆矩阵\lstinline|invhlib(n)|。
\end{itemize}
\item  托普利兹矩阵
\begin{itemize}
	\item 性质:除第一行第一列外,其他每个元素都与左上角的元素相同。
	\item 函数:\lstinline|toeplitz(x,y)|,其中$\bm{x},\bm{y}$分别为第一行和第一列,均为向量。\lstinline|toeplitz(x)|用向量$\bm{x}$生成一个对称的托普利兹矩阵。
	\item 例子:
	\begin{lstlisting}
>> T = toeplitz(1:6)
T =
		1		2		3		4		5		6
		2		1		2		3		4		5
		3		2		1		2		3		4
		4		3		2		1		2		3
		5		4		3		2		1		2
		6		5		4		3		2		1
	\end{lstlisting}
\end{itemize}
\item 伴随矩阵
\begin{itemize}
	\item 定义:设多项式$p(x)$为
	\begin{equation}
		p(x)=a_nx^n+a_{n-1}x^{n-1}+\cdots +a_1x+a_0
	\end{equation}
	称矩阵
	\begin{equation}
		\bm{A} =
		\begin{bmatrix}
			-\dfrac{a_{n_1}}{a_n} & -\dfrac{a_{n-2}}{a_{n}} & -\dfrac{a_{n-3}}{a_{n}} & \cdots & -\dfrac{a_2}{a_n} & -\dfrac{a_1}{a_n}\\
			1 & 0 & 0 & \cdots & 0 & 0\\
			0 & 1 & 0 & \cdots & 0 & 0\\
			\vdots & \vdots & \vdots & \ddots & \vdots & \vdots\\
			0 & 0 & 0 & \cdots & 0 & 0\\
			0 & 0 & 0 & \cdots & 1 & 0
		\end{bmatrix}
	\end{equation}
为多项式$p(x)$的伴随矩阵,$p(x)$称为矩阵$\bm{A}$的特征多项式,方程$p(x)=0$的根称为矩阵$\bm{A}$的特征值。
\item 函数:\lstinline|compan(p)|,其中$\bm{p}$是一个多项式的系数向量,即
\begin{equation*}
	\bm{p} = 
	\begin{bmatrix}
		a_n\\
		a_{n-1}\\
		\vdots\\
		a_1\\
		a_0
	\end{bmatrix}
\end{equation*}
\end{itemize}
\item 帕斯卡矩阵
\begin{itemize}
	\item 定义:二次项$(x+y)^n$展开后的系数随$n$的增大组成一个三角形表,称为杨辉三角形。由杨辉三角形表组成的矩阵称为帕斯卡)矩阵。
	\item 函数:\lstinline|pascal(n)|生成一个$n$阶帕斯卡矩阵($(x+y)^{n - 1}$的展开式)。
\end{itemize}
\end{enumerate}

\section{矩阵变换}
\subsection{对角阵与三角阵}
\begin{enumerate}
	\item 对角阵
	\begin{itemize}
		\item 定义:只有对角线上有非0元素的矩阵称为对角矩阵,对角线上的元素相等的对角矩阵称为数量矩阵,对角线上的元素都为1的对角矩阵称为单位矩阵。
		\item 函数
		\begin{itemize}
			\item 提取矩阵的对角阵元素:\lstinline|diag(A), diag(A, k)|.
			\par \hspace*{2em}用于提取主对角线和第$k$条对角线\footnote{
				主对角线$k=0$,主对角线上方$k>0$,主对角线下方$k<0$,下同}上的元素。
			\item 构造对角阵:\lstinline|diag(V), diag(V,k)|.
			\par \hspace*{2em} 用于生成以向量$\bm{V}$为主对角线的方阵和以向量$\bm{V}$为第$k$条对角线的矩阵。
		\end{itemize}
	\end{itemize}

\examples 先建立$5× \times 5 $矩阵A,然后将$\bm{A}$的第一行元素乘以1,第二行乘以2,…,第五行乘以5。\vspace*{1em}

\begin{lstlisting}
>> A = [17, 0, 1, 0, 15; 23, 5, 7, 14, 16 ; 4, 0, 13, 0, 22; 10, 12, 19, 21, 3;...
	11, 18, 25, 2, 19];
>> D = diag(1:5);
>> D*A		% 用D左乘A,对A的每行乘以一个指定常数,左行右列
ans =
			17		0		1		0		15
			46		10		14		28		32
			12		0		39		0		66
			40		48		76		84		12
			55		90		125	10		95
\end{lstlisting}
\item 三角阵
\begin{itemize}
	\item 上三角阵
	\begin{itemize}
		\item 定义:矩阵的对角线以下的元素全为0的一种矩阵
		\item 函数:\lstinline|triu(A), triu(A, k)|.
		\par \hspace*{2em} 用于求矩阵$\bm{A}$的上三角阵和第$k$条对角线以上的元素。
	\end{itemize}
	\item 下三角阵
	\begin{itemize}
		\item 定义:矩阵的对角线以上的元素全为0的一种矩阵
		\item 函数:\lstinline|tril(A), tril(A, k)|.
		\par \hspace*{2em} 用于求矩阵$\bm{A}$的下三角阵和第$k$条对角线以下的元素。
	\end{itemize}
\end{itemize}
\end{enumerate}

\subsection{矩阵的转置与旋转}
矩阵的转置与旋转操作函数和操作说明见下表:
\begin{table}[!htb]
	\centering
	\setlength{\tabcolsep}{6mm}{
		\begin{tabular}{ccl}
			\toprule
			功能 & 函数名 & \hspace*{9em}说明\\
			\midrule
			转置 & \lstinline|A.'| & 矩阵转置\\
			\hline
			共轭转置 & \lstinline|A'| &  转置的基础上还要取每个数的复共轭\\
			\hline
			矩阵旋转 & \lstinline|rot90(A, k)| & 将矩阵$\bm{A}$旋转$90\degree$的$k$倍,当$k$为1时可省略\\
			\hline
			矩阵的左右翻转 & \lstinline|fliplr(A)| & \makecell[l]{将原矩阵的第一列和最后一列调换,\\第二列和倒数第二列调换,…,依次类推}\\
			\hline
			矩阵的上下翻转 & \lstinline|flipud(A)| & \makecell[l]{将原矩阵的第一行和最后一行调换,\\第二行和倒数第二行调换,…,依次类推}\\
			\bottomrule
		\end{tabular}
	}
\end{table}

\subsection{矩阵的逆和伪逆}
矩阵的逆和伪逆函数和操作说明见下表:
\begin{table}[!htb]
	\centering
	\setlength{\tabcolsep}{6mm}{
		\begin{tabular}{ccl}
			\toprule
			功能 & 函数名 & \hspace*{9em}说明\\
			\midrule
			逆矩阵 & \lstinline|B = inv(A)| & 求矩阵的逆\\
			\hline
			\makecell[c]{伪逆矩阵\\(广义逆矩阵)} & \lstinline|B = pinv(A)| &\makecell[l]{如果矩阵$\bm{A}$不是一个方阵,或者$\bm{A}$是一个非满秩的方阵时,\\矩阵$\bm{A}$没有逆矩阵,但可以找到一个与$\bm{A}$的转置矩阵$\bm{A}'$\\同型的矩阵$\bm{B}$,使得:$
				\bm{A} \cdot \bm{B} \cdot \bm{A} = \bm{A}, \,\, \bm{B} \cdot \bm{A} \cdot \bm{B} = \bm{B}$
			}\\
			\bottomrule
		\end{tabular}
	}
\end{table}

\section{矩阵求值}
\subsection{方阵的行列式}
把一个方阵看作一个行列式,并按行列式的规则来求值的函数:
\begin{center}
	\lstinline|det(A)|
\end{center}

\subsection{矩阵的迹和秩}
\begin{enumerate}
	\item 矩阵的秩
	\begin{itemize}
		\item 定义:矩阵线性无关的行数与列数称为矩阵的秩。通常,对于一组向量$\bm{x}_1,\bm{x}_2,\cdots,\bm{x}_p$,若存在一组不全为零的数$k_i\,\,(i=1,2,\cdots,p)$,使
		\begin{equation}
			k_1\bm{x}_1+k_2\bm{x}_2+\cdots k_p\bm{x}_p = 0
		\end{equation}
	那么这$p$个向量线性相关,反之线性无关。
	\item 函数:\lstinline|rank(A)|.
	\end{itemize}
\item 矩阵的迹
\begin{itemize}
	\item 定义:矩阵的迹等于矩阵的对角线元素之和,也等于矩阵的特征值之和。
	\item 函数:\lstinline|trace(A)|,矩阵$\bm{A}$必须是方阵。
\end{itemize}
\end{enumerate}

\subsection{向量和矩阵的范数}
\begin{table}[!htb]
	\centering
	\setlength{\tabcolsep}{3mm}{
		\begin{tabular}{ccccl}
			\toprule
			类型 & 范数类型 & 函数 & 数学表达式 & 说明\\
			\midrule
			\multirow{5}*{向量} & 1--范数 & \lstinline|norm(V,1)| & $\displaystyle \big|\big|\bm{V}\big|\big|_1=\sum_{n}^{i = 1} |v_i|$ & 向量元素的绝对值之和\\
			\specialrule{0em}{0.5em}{0.5em} 
			& 2--范数 & \lstinline|norm(V,2); norm(V)| & $\displaystyle \big|\big|\bm{V}\big|\big|_2=\sqrt{\sum_{n}^{i = 1} |v_i|^2}$ & 向量元素绝对值的平方和的平方根\\
			\specialrule{0em}{0.5em}{0.5em} & $\infty$--范数 & \lstinline|norm(V,inf);| & $\displaystyle \big|\big|\bm{V}\big|\big|_\infty=\max_{1 \le i \le n }\big\lbrace |v_i| \big\rbrace$ & 向量元素绝对值中的最大值\\
			\specialrule{0em}{0.5em}{0.5em} 
			\hline
			\specialrule{0em}{0.5em}{0.5em} 
			\multirow{5}*{矩阵} & 1--范数 & \lstinline|norm(A,1)| & $\displaystyle \big|\big|\bm{V}\big|\big|_1=\sum_{n}^{i = 1} |v_i|$ & 向量元素的绝对值之和\\
			\specialrule{0em}{0.5em}{0.5em} 
			& 2--范数 & \lstinline|norm(A,2); norm(A)| & $\displaystyle \big|\big|\bm{V}\big|\big|_2=\sqrt{\sum_{n}^{i = 1} |v_i|^2}$ & 向量元素绝对值的平方和的平方根\\
			\specialrule{0em}{0.5em}{0.5em} 
			& $\infty$--范数 & \lstinline|norm(A,inf);| & $\displaystyle \big|\big|\bm{V}\big|\big|_\infty=\max_{1 \le i \le n }\big\lbrace |v_i| \big\rbrace$ & 向量元素绝对值中的最大值\\
			\specialrule{0em}{0.5em}{0.5em} 
			\bottomrule
		\end{tabular}
	}
\end{table}

\subsection{矩阵的条件数}
矩阵$\bm{A}$的条件数等于$\bm{A}$的范数与$\bm{A}$的逆矩阵的范数的乘积,即
\begin{equation}
	\text{cond}(\bm{A}) = \big|\big| \bm{A} \big|\big| \cdot \big|\big| \bm{A}^{-1} \big|\big|
\end{equation}
定义的条件数总是大于1的。条件数越接近1,矩阵的性能越好。反之,矩阵的性能越差。
\begin{table}[!htb]
	\centering
	\setlength{\tabcolsep}{7mm}{
		\begin{tabular}{ccc}
			\toprule
			范数类型 & 函数 & 数学表达式 \\
			\midrule
			1-范数下的条件数 & \lstinline|cond(A,1)| & $\displaystyle \text{cond}(\bm{A},1) = \big|\big| \bm{A} \big|\big|_1 \cdot \big|\big| \bm{A}^{-1} \big|\big|_1$  \\
			\specialrule{0em}{0.5em}{0.5em} 
			2-范数下的条件数 & \lstinline|cond(A,2); cond(A)| & $\displaystyle \text{cond}(\bm{A},2) = \big|\big| \bm{A} \big|\big|_2 \cdot \big|\big| \bm{A}^{-1} \big|\big|_2$ \\
			\specialrule{0em}{0.5em}{0.5em} $\infty$-范数下的条件数 & \lstinline|cond(A,inf);| & $\displaystyle \text{cond}(\bm{A},\infty ) = \big|\big| \bm{A} \big|\big|_{\infty} \cdot \big|\big| \bm{A}^{-1} \big|\big|_{\infty}$ \\
			\specialrule{0em}{0.5em}{0.5em} 
			\bottomrule
		\end{tabular}
	}
\end{table}

\subsection{矩阵的特征值与特征向量}
\tdefination[特征值与特征向量]
对于$n$阶方阵$\bm{A}$,求数$\bm{\lambda}$和向量$\bm{\xi}$,使得
\begin{equation}
	\bm{A}\bm{\xi} \,=\,\lambda \bm{\xi}
\end{equation}
即
\begin{equation}
	\big( \bm{A}-\lambda \bm{I} \big)\bm{\xi}=0
	\label{特征}
\end{equation}
则满足式\eqref{特征}的向量$\bm{\xi}$称为\dya[特征向量],数$\lambda$称为\dya[特征值]。

由于式\eqref{特征}有非零解,所以必须使其系数行列式为0,即
\begin{equation}
	\big| \bm{A} - \lambda \bm{I} \big| = 0
\end{equation}

Matlab求特征值和特征向量的函数如下表。
\begin{table}[!htb]
	\centering
	\setlength{\tabcolsep}{10mm}{
		\begin{tabular}{ccc}
			\toprule
			函数名\&用法 & 说明\\
			\midrule
			\lstinline|V = eig(A)| & 求矩阵$\bm{A}$的全部特征值,构成向量$\bm{V}$\\
			\lstinline|[X, D] = eig(A)| & \makecell[c]{求矩阵$\bm{A}$的全部特征值,构成对角阵$\bm{D}$,\\并产生矩阵$\bm{X}$,$\bm{X}$各列是相应的特征向量。}\\
			\lstinline|[X, D] = eig(A, 'nobalance')| & 同上\\
			\bottomrule
		\end{tabular}
	}
\end{table}

\warn[
{\lstinline|[X, D] = eig(A, 'nobalance')|} 与 {\lstinline|[X, D] = eig(A)| }的区别\\
\hspace*{2em}{\lstinline|[X, D] = eig(A, 'nobalance')|}是直接计算矩阵$\bm{A}$的特征值和特征向量,
而 {\lstinline|[X, D] = eig(A)|}在计算矩阵$\bm{A}$的特征值和特征向量之前先求矩阵$\bm{A}$的相似变换。
]











