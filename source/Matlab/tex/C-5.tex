\chapter{Matlab数据分析与多项式计算}
\thispagestyle{empty}
\section{数据统计分析}
\subsection{最大值与最小值}
以最大值函数为例子,介绍函数的用法,最小值函数的调用完全一致。
\begin{table}[!htb]
	\centering
	\setlength{\tabcolsep}{10mm}{
	\begin{tabular}{cl}
		\toprule
		 函数 & 说明\\
		\midrule
		 \lstinline|max(A)| & 返回一个行向量,向量的第$i$个元素是矩阵$\bm{A}$第$i$列的最大值\\
		 \hline
		 \lstinline|max(A,[ ],2)| & 返回一个列向量,向量的第$i$个元素是矩阵$\bm{A}$第$i$行的最大值\\
		 \hline
		 \lstinline|[Y,U] = max(A)| & \makecell[l]{$\bm{Y}$行向量记录矩阵$\bm{A}$每列的最大值\\$\bm{U}$矩阵记录每列最大值的行号}\\
		 \hline
		\lstinline|U = max(A,B)| & \makecell[l]{$\bm{A}, \bm{B}$是同型的向量或矩阵\\$\bm{U}$的每个元素等于$\bm{A}, \bm{B}$对应元素的最大值}\\
		\hline
		\lstinline|U = max(A,n)| & \makecell[l]{$n$是一个标量,$\bm{U}$的每个元素等于$\bm{A}$对应元素与$n$中的最大值}\\
		\bottomrule
	\end{tabular}	
	}
\end{table}

\subsection{求和与求积}
求和函数用\lstinline|sum|,求积函数用\lstinline|prod|.两者调用方式完全一致,以\lstinline|sum|为例:
\begin{table}[!htb]
	\centering
	\setlength{\tabcolsep}{10mm}{
		\begin{tabular}{cl}
			\toprule
			函数 & 说明\\
			\midrule
			\lstinline|sum(X)| & 返回向量$\bm{X}$的元素之和\\
			\hline
			\lstinline|sum(A)| & 返回一个行向量,向量的第$i$个元素是矩阵$\bm{A}$第$i$列元素之和\\
			\hline
			\lstinline|sum(A,2)| & 返回一个列向量,向量的第$i$个元素是矩阵$\bm{A}$第$i$行元素之和\\
			\bottomrule
		\end{tabular}	
	}
\end{table}

\subsection{平均值和中值}
平均值函数用\lstinline|mean|,中值函数用\lstinline|median|.两者调用方式完全一致,以\lstinline|mean|为例:
\begin{table}[!htb]
	\centering
	\setlength{\tabcolsep}{10mm}{
		\begin{tabular}{cl}
			\toprule
			函数 & 说明\\
			\midrule
			\lstinline|mean(X)| & 返回向量$\bm{X}$的平均值\\
			\hline
			\lstinline|mean(A)| & 返回一个行向量,向量的第$i$个元素是矩阵$\bm{A}$第$i$列元素的平均值\\
			\hline
			\lstinline|mean(A,2)| & 返回一个列向量,向量的第$i$个元素是矩阵$\bm{A}$第$i$行元素的平均值\\
			\bottomrule
		\end{tabular}	
	}
\end{table}

\subsection{累加和与累乘积}
设$\bm{U} = (u_1, u_2, \cdots, u_n)$是一个向量,$\bm{V},\bm{W}$是与$\bm{U}$等长的另外两个向量,并且
\begin{align}
	\bm{V} = \left(\sum_{i = 1}^{1} u_i,\sum_{i = 1}^{2} u_i, \cdots ,\sum_{i = 1}^{n} u_i  \right)\\[0.5em]
	\bm{W} = \left(\prod_{i = 1}^{1} u_i,\prod_{i = 1}^{2} u_i, \cdots ,\prod_{i = 1}^{n} u_i  \right)
\end{align}
称$\bm{V}$为$\bm{U}$的累加和向量,$\bm{W}$为$\bm{U}$的累乘积向量。求矩阵元素的累加和的函数为\lstinline|cumsum|,累乘积的函数为\lstinline|cumprod|.两者调用方式完全一致,以\lstinline|cumsum|为例:
\begin{table}[!htb]
	\centering
	\setlength{\tabcolsep}{10mm}{
		\begin{tabular}{cl}
			\toprule
			函数 & 说明\\
			\midrule
			\lstinline|cumsum(X)| & 返回向量$\bm{X}$的累加和向量\\
			\hline
			\lstinline|cumsum(A)| & 返回一个矩阵,其第$i$列是矩阵$\bm{A}$第$i$列元素的累加和\\
			\hline
			\lstinline|cumsum(A,2)| & 返回一个矩阵,其第$i$行是矩阵$\bm{A}$第$i$行元素的累加和\\
			\bottomrule
		\end{tabular}	
	}
\end{table}

\subsection{标准差与相关系数}
标准差的计算公式
\begin{align}
	S_1 = \sqrt{\dfrac{1}{n -1} \sum_{i = 1}^{n} \left(x_i - \overline{x}\right)^2}\\
	S_2 =  \sqrt{\dfrac{1}{n} \sum_{i = 1}^{n} \left(x_i - \overline{x}\right)^2}
\end{align}
其中
\begin{align*}
	\overline{x} = \dfrac{1}{n} \sum_{i = 1}^{n} x_i
\end{align*}
相关系数的计算公式为
\begin{align}
	r = \dfrac{\displaystyle \sum_{i = 1}^{n}(x_i - \overline{x})(y_i - \overline{y})}{\displaystyle \sqrt{\sum_{i = 1}^{n}(x_i - \overline{x})^2\sum_{i = 1}^n (y_i - \overline{y})^2}}
\end{align}
\begin{table}[!htb]
	\centering
	\setlength{\tabcolsep}{8mm}{
		\begin{tabular}{ccl}
			\toprule
			功能 & 函数 & 说明\\
			\midrule
			标准差 & \lstinline|std(A,flag,dim)| & \makecell[l]{dim = 1 \quad 求各列元素的标准差\\ dim = 2 \quad 求各行元素的标准差\\ flag = 0 \quad 按$S_1$计算标准差 \\ flag = 1 \quad 按$S_2$计算标准差}\\
			\hline
			& & \vspace*{-1.5em }\\
			相关系数 & \lstinline|corrcoef(X, Y)| & 返回一个$2 \times 2$矩阵:
			$
			\begin{bmatrix}
				r(\bm{X}) & r(\bm{X, Y})\\
				r(\bm{Y, X}) & r(\bm{Y})
			\end{bmatrix}
			$
			\\
			\bottomrule
		\end{tabular}	
	}
\end{table}
\newpage

相关系数是反映两组数据序列之间的相互关系的指标,类似的指标还有协方差,计算公式为
\begin{align}
	c = \dfrac{1}{n - 1} \sum_{i = 1}^n (x_i - \overline{x})(y_i - \overline{y})
\end{align}
Matlab的函数为\lstinline|cov|,其调用格式与\lstinline|corrcoef|函数类似。
\vspace*{0.5em}

\subsection{排序}
\lstinline|sort|函数对矩阵$\bm{A}$对各列或各行重新排序,其调用格式为
\begin{center}
	\lstinline|[Y, I] = sort(A, dim, mode)|
\end{center}
其中
\begin{itemize}
	\item $\bm{Y}$为排序后对矩阵
	\item $\bm{I}$记录$\bm{Y}$中对元素在$A$中对位置
	\item \lstinline|dim|指明对$\bm{A}$进行列/行排序
	\begin{itemize}
		\item \lstinline|dim = 1| \quad 列排序
		\item \lstinline|dim = 2| \quad 行排序
	\end{itemize}
	\item \lstinline|mode|表示升序/降序
	\begin{itemize}
		\item \lstinline|mode = 'ascend'| \quad 升序(默认)
		\item \lstinline|mode = 'descend'| \quad 降序
	\end{itemize}
\end{itemize}



\section{多项式计算}
$n$次多项式
\begin{align}
	P(x) = a_nx^n + a_{n-1}x^{n-1} + a_{n-2}x^{n-2} + \cdots a_1x + a_0
\end{align}
在Matlab中,$P(x)$表达为向量形式:
\begin{align}
	[a_n,a_{n-1},a_{n-2},\cdots,a_1,a_0]
\end{align}
\subsection{多项式的四则运算}
多项式的四则运算等价于多项式系数\lstinline|P1, P2|之间的运算,详细见下表
\begin{table}[!htb]
		\centering
	\setlength{\tabcolsep}{10mm}{
		\begin{tabular}{ccc}
			\toprule
			功能 & 函数 & 说明 \\
			\midrule
			多项式加法 & \lstinline|P = P1 + P2| & 直接相加即可\\
			多项式减法 & \lstinline|P = P1 - P2| & 直接相减即可\\
			多项式乘法 & \lstinline|P = conv(P1, P2)| & ——\\
			多项式除法 & \lstinline|[Q, r] = deconv(P1, P2)| & Q返回商值,r返回余式\\
			\bottomrule
		\end{tabular}
	}
\end{table}

\subsection{多项式的导函数}
求多项式的导函数用\lstinline|polyder|函数,其调用格式为
\begin{enumerate}[\hspace*{3em} (1)]
	\item \lstinline|p = polyder(P)|\quad 求多项式P的导函数
	\item \lstinline|p = polyder(P, Q)|\quad 求P$\cdot$Q的导函数
	\item \lstinline|[p, q] = polyder(P, Q)|\quad 求P/Q的导函数,导函数分子存入p,分母存入q
\end{enumerate}


\subsection{多项式求值}
\begin{enumerate}
	\item \textbf{代数多项式求值}
	\begin{center}
		\lstinline|polval(P, x)|
	\end{center}
	若x为一数值,则求多项式在该点的值;若x为向量或矩阵,则对向量或矩阵中的每个元素求其多项式的值。
	\item \textbf{矩阵多项式求值}
	\begin{center}
		\lstinline|polyvalm(P, A)|
	\end{center}
	其中$A$为方阵。
\end{enumerate}

\subsection{多项式求根}
求根函数和已知根$x$构造多项式的函数:
\begin{center}
	\lstinline|x = roots(P)|\\
	\lstinline|P = poly(x)|
\end{center}


\section{数据插值}
\begin{table}[!htb]
	\centering
	\setlength{\tabcolsep}{4mm}{
		\begin{tabular}{ccc}
			\toprule
			功能 & 函数 & 说明 \\
			\midrule
			一维插值 & \lstinline|Y1 = interp1(X, Y, X1, method)| & $\bm{X,Y}$为等长向量,分别为采样点和采样值\\
			二维插值 & \lstinline|Y2 = interp2(X, Y, Z, X1, X2, method)| & $X,Y$是采样点,$Z$是采样值\\
			\bottomrule
		\end{tabular}
	}
\end{table}
method有四种,见下表
\begin{table}[!htb]
	\centering
	\setlength{\tabcolsep}{4mm}{
		\begin{tabular}{cc}
			\toprule
			方法 & 说明 \\
			\midrule
			\lstinline|'linear'| & 线性插值(默认),两点连线,不外插\\
			\lstinline|'nearest'| & 最近点插值,插值点优先选择较近的数据点插值,不外插 \\
			\lstinline|'pchip'| & 分段3次埃米尔插值,不适用于二维插值\\
			\lstinline|'spline'| & 3次样条插值,每个分段内构造一个3次多项式,可外插\\
			\bottomrule
		\end{tabular}
	}
\end{table}

\section{曲线拟合}
最小二乘实现曲线拟合,使用\lstinline|polyfit|函数:
\begin{center}
	\lstinline|P = polyfit(X, Y, m)|\\
	\lstinline|[P, S] = polyfit(X, Y, m)|\\
	\lstinline|[P, S, mu] = polyfit(X, Y, m)|
\end{center}
函数根据采样点$\bm{X}$和采样点函数$\bm{Y}$,产生一个$m$次多项式$\bm{P}$及其在采样点的误差向量$\bm{S}$。其中
\begin{itemize}
	\item $\bm{X, Y}$ \quad 等长向量
	\item $\bm{P}$ \quad 长度为$m + 1$的向量,其为多项式系数
	\item \lstinline|mu| \quad 二元向量,mu(1)是mean(X),而mu(2)是std(X)
\end{itemize}



























