\chapter{流体运动学与动力学基础}
\thispagestyle{empty}

\section{流场及其描述}
由连续介质假设,空间中存在无穷多个流体质点。对于无数多的流体质点,当其发生运动时,如何正确描述整个流场的运动行为?主要解决两个问题:\vspace*{-0.5em}
\begin{itemize}
	\item 怎样跟踪和区分每一个流体质点\vspace*{-0.5em}
	\item 如何描述每一个流体质点的运动特征及其变化
\end{itemize}

\subsection{拉格朗日法(质点法或质点系法)}

拉格朗日法的核心思想:着眼于单个流体质点运动,跟踪并观察其运动状态(运动历程、空间轨迹、速度和加速度等),从而获得整个流场的运动规律。

\defination[流体质点的标识]
{
	对于某一流体质点$(a,b,c)$的空间位置表示为
	\begin{equation}
		x(a,b,c,t),\quad y(a,b,c,t), \quad z(a,b,c,t)
		\label{流体质点标识符}
	\end{equation}
	其中,$t$表示时间,$(a,b,c)$是\blue[流体质点的标识符],用来区分和表示各个流体质点,可用流体质点的初始位置坐标表示,$a,b,c,t$称为\dy[拉格朗日变数]{LGLRBS}。\vspace*{-0.5em}
	{
		\begin{itemize}
			\item $t$ 给定,则公式\eqref{流体质点标识符}表示的是$t$时刻不同质点的空间位置\vspace*{-0.5em}
			\item $a,b,c$给定,则公式\eqref{流体质点标识符}表示的是指定质点$a,b,c$的轨迹
		\end{itemize}
	}
}

因为质点的坐标是时间$t$的函数,所以对于给定的流体质点$(a,b,c)$,其速度和加速度为
\begin{equation}
	\begin{cases}
		\, u = \dfrac{\partial x(a,b,c,t)}{\partial t} \\[0.7em]
		\, v = \dfrac{\partial y(a,b,c,t)}{\partial t} \\[0.7em]
		\, w = \dfrac{\partial z(a,b,c,t)}{\partial t}
	\end{cases}
	\qquad 
	\begin{cases}
		\, u = \dfrac{\partial^2 x(a,b,c,t)}{\partial t^2} \\[0.7em]
		\, v = \dfrac{\partial^2 y(a,b,c,t)}{\partial t^2} \\[0.7em]
		\, w = \dfrac{\partial^2 z(a,b,c,t)}{\partial t^2}
	\end{cases}
\end{equation}

\warn
[
	\hspace*{2em}流体质点标识符$(a,b,c)$是空间位置$(x,y,z)$的函数,求导是针对同一个流体质点$(a,b,c)$进行。除此之外,流体质点的其他物理量也都是$a,b,c,t$的函数,例如流体质点的温度可以表示为$T(a,b,c,t).$
]

\defination[迹线]
{
	对于某一个流体质点$(a,b,c)$随着时间$t$的运动轨迹称为\dy[迹线]{JX}。
}
由拉格朗日法的速度、加速度的表达式,可以得到迹线的微分方程

\theorem[迹线的微分方程]
{
	\quad \vspace*{-1em}
	\begin{equation}
		\dfrac{\d x}{u} = \dfrac{\d y}{v} = \dfrac{\d z}{w} = \d t
	\end{equation}
}

\subsection{欧拉法(空间点法、流场法)}


\subsection{流线与流普、流面、流管和流量}


\subsection{欧拉法下的流体质点加速度表达式}


\section{流体微团运动分析}
\subsection{流体微团的基本运动形式}
\noindent 在理论力学中,研究对象是质点和刚体(无变形体),其基本的运动形式为
\begin{enumerate}
	\item 质点(无体积大小的物质点)只有平移运动\vspace*{-0.5em}
	\item 刚体(具有一定体积大小但无变形但物体)运动包括平移运动和旋转运动
\end{enumerate}

\defination[流体微团的基本运动形式]
{
	流体力学研究的对象是质点和流体微团(具有体积和形状的质点团),其运动包括平移、转动、线变形和角变形,如图所示。即
	\begin{equation*}
		\begin{aligned}[c]
			& \mbox{刚体运动:平动} + \mbox{转动}\\
			& \mbox{流体运动:平动} + \mbox{转动} + \mbox{线变形} + \mbox{角变形}
		\end{aligned}
	\end{equation*}
}


流场中任取一平面流体微团,设微团$A$的速度为$u(x,y), v(x,y)$,根据泰勒级数在$A$点的一节展开,
$$
f(x) = \dfrac{f(x_0)}{0!} + \dfrac{f(x_0)}{1!}(x-x_0) + R_n(x_0)
$$
可以得到
\vspace*{1em}

\begin{minipage}{0.7\linewidth}
	\centering
	\setlength{\tabcolsep}{8mm}{
		\begin{tabular}{ccc}
			\hline
			 点&$x$分量 & $y$分量 \\
			\hline
			&&\quad \\[-1.3em]
			$B$ & $u + \dfrac{\partial u}{\partial x} \d x$ & $v + \dfrac{\partial v}{\partial x} \d x$\\
			&&\quad \\[-1.3em]
			\hline
			&&\quad \\[-1.3em]
			$C$ & $u + \dfrac{\partial u}{\partial y} \d y$ & $v + \dfrac{\partial v}{\partial y} \d y$\\
			&&\quad \\[-1.3em]
			\hline
			&&\quad \\[-1.3em]
			$D$ & $u + \dfrac{\partial u}{\partial x} \d x + \dfrac{\partial u}{\partial y} \d y$ & $v + \dfrac{\partial u}{\partial y} \d y + \dfrac{\partial v}{\partial y} \d y$ \\
			&&\quad \\[-1.3em]
			\hline
		\end{tabular}
	}
\end{minipage}

\begin{enumerate}
	\item 平动\\
	
	\item 线变形\\
	
	\item 角变形和转动\\
	\hspace*{1em} \dy[角变形速率]{JBXSL}(\dy[切变速率]{QBSL}):单位时间内角变形量,用$\gamma_z$表示\\
	\hspace*{1em} \dy[转动角速度]{ZDJSD}:单位时间内角平分线的转动量,用$\omega_z$表示\\
	由图分析可得
	\begin{equation*}
		\begin{cases}
			\, \angle BAB' = \beta + \alpha \\
			\, \angle CAC' = -\beta + \alpha
		\end{cases}
		\quad \Rightarrow \quad 
		\begin{cases}
			\, \alpha = \dfrac{\angle BAB' + \angle CAC'}{2}\\[0.5em]
			\, \beta = \dfrac{\angle BAB' - \angle CAC'}{2}
		\end{cases}
	\end{equation*}
	又
	\begin{equation*}
		\begin{aligned}
			\angle BAB' \approx \tan \angle BAB' = \dfrac{BB'}{\d x} = \dfrac{\left(v + \dfrac{\partial v}{\partial x} \d x - v\right)\d t}{\d x} = \dfrac{\partial v}{\partial x}\d t \\[1em]
			\angle CAC' \approx \tan \angle CAC' = -\dfrac{CC'}{\d x} = -\dfrac{\left(u + \dfrac{\partial u}{\partial y} \d y - v\right)\d t}{\d y} = - \dfrac{\partial u}{\partial y}\d t \\[1em]
		\end{aligned}
	\end{equation*}
	则
	\begin{equation}
		\begin{aligned}
			\gamma_z =  \dfrac{\beta}{\d t} = \dfrac{1}{2}\left(\dfrac{\partial v}{\partial x} + \dfrac{\partial u}{\partial y}\right)\\[0.5em]
			\omega_z =  \dfrac{\alpha }{\d t} = \dfrac{1}{2} \left(\dfrac{\partial v}{\partial x} - \dfrac{\partial u}{\partial y}\right)
		\end{aligned}
	\end{equation}
\end{enumerate}
同样的,对于三维的微元体,有

\theorem[流体微团的运动形式]
{
	\quad \vspace*{-2em}
	{
		\begin{itemize}
			\item 平动:平移速度\quad $u(x,y,z,t),\, v(x,y,z,t),\, w(x,y,z,t)$
			\item 线变形:线变形速率(正应变率)
			\begin{equation}
				\begin{cases}
					\,\theta_x = \dfrac{\partial u}{\partial x}\\[0.5em]
					\, \theta_y = \dfrac{\partial v}{\partial y}\\[0.5em]
					\, \theta_z = \dfrac{\partial w}{\partial z}
				\end{cases}
			\end{equation}
			\item 角变形:角变形速率(切变速率)
			\begin{equation}
				\begin{cases}
					\, \gamma_x = \dfrac{1}{2}\left(\dfrac{\partial w}{\partial y} + \dfrac{\partial v}{\partial z}\right)\\[0.7em]
					\, \gamma_y = \dfrac{1}{2}\left(\dfrac{\partial u}{\partial z} + \dfrac{\partial w}{\partial x}\right)\\[0.7em]
					\, \gamma_z = \dfrac{1}{2}\left(\dfrac{\partial v}{\partial x} + \dfrac{\partial u}{\partial y}\right)
				\end{cases}
			\end{equation}
			
			\item 转动:转动角速度
			\begin{equation}
				\begin{cases}
						\, \omega_x = \dfrac{1}{2}\left(\dfrac{\partial w}{\partial y} - \dfrac{\partial v}{\partial z}\right)\\[0.7em]
					\, \omega_y = \dfrac{1}{2}\left(\dfrac{\partial u}{\partial z} - \dfrac{\partial w}{\partial x}\right)\\[0.7em]
					\, \omega_z = \dfrac{1}{2}\left(\dfrac{\partial v}{\partial x} - \dfrac{\partial u}{\partial y}\right)
				\end{cases}	
			\end{equation}
		\end{itemize}
	}
}

\section{理想流体运动微分方程组}
\subsection{系统与控制体}
\vspace*{-1em}
\defination[系统]
{
	 \dy[系统]{XT}:包含着确定不变物质的任何集合体。对流体力学(或空气动力学)而言,\red[系统是指由任何确定流体质点组成的集合体]。
}

\noindent 系统的基本特点:
\begin{itemize}
	\item 系统质量保持恒定、边界上没有质量交换;\vspace*{-0.5em}
	\item 边界随流体一起运动\vspace*{-0.5em}
	\item 边界上受到外界的表面力作用\vspace*{-0.5em}
	\item 边界上存在能量交换
\end{itemize}

\defination[控制体]
{
	\dy[控制体]{KZT}:有物质(如流体)穿过,相对于某个坐标系固定不变的任何体积。其边界称为\dy[控制面]{KZM}。
}

\noindent 控制体的基本特点
\begin{itemize}
	\item 形状形状可根据需要而定,边界相对于坐标系而言是固定的;\vspace*{-0.5em}
	\item 控制体本身不变,但是穿过的物质(流体质点)随时间是变化的;\vspace*{-0.5em}
	\item 控制面可发生质量交换:\vspace*{-0.5em}
	\item 控制面上受到外界作用力;\vspace*{-0.5em}
	\item 控制面可存在能量交换
\end{itemize}

\subsection{连续(质量)方程}
\noindent \textbf{1. 考虑流场中的一个六面体微团(控制体)}
\begin{itemize}
	\item 体积:$\d x \d y \d z$;\vspace*{-0.5em}
	\item 中心:$O(x,y,z)$的速度$\bm{v}_o = (u,v,w)$;\vspace*{-0.5em}
	\item 密度:$\rho(x,y,z,t)$是时间和空间的函数;\vspace*{-0.5em}
	\item $t$时刻的质量流量(密流或通量):$x,y,z$方向分别为$\rho u , \rho v , \rho w$
\end{itemize}
\vspace*{0.5em}

\noindent \textbf{2. 考虑净流量}
\begin{itemize}
	\item $\d t$时刻内从流体微团左侧变进入控制体的质量为
	\begin{equation*}
		m_1 = \left[\rho u - \dfrac{\partial (\rho u)}{\partial x} \dfrac{\d x}{2}\right]\d y \d z \d t
	\end{equation*}

		\item $\d t$时刻内从流体微团左侧变进入控制体的质量为
	\begin{equation*}
		m_1 = \left[\rho u + \dfrac{\partial (\rho u)}{\partial x} \dfrac{\d x}{2}\right]\d y \d z \d t
	\end{equation*}
	
	\item $\d t$时间内$x$方向上的净流量为
	\begin{equation*}
		\d m_x = m_1 - m_2 = - \dfrac{\partial (\rho u)}{\partial x} \d x\d y \d z \d t
	\end{equation*}
\end{itemize}
同样的,各个方向上的净流量为
\begin{equation}
	\d m_x = - \dfrac{\partial (\rho u)}{\partial x} \d x\d y \d z \d t, \qquad 
	\d m_y = - \dfrac{\partial (\rho v)}{\partial y} \d x\d y \d z \d t, \qquad 
	\d m_z = - \dfrac{\partial (\rho w)}{\partial z} \d x\d y \d z \d t
\end{equation}
即整个流体微团的净流量为
\begin{equation}
	\d m = \d m_x + \d m_y + \d m_z = - \left[\dfrac{\partial (\rho u)}{\partial x} +  \dfrac{\partial (\rho v)}{\partial y} +  \dfrac{\partial (\rho w)}{\partial z} \right] \d x \d y \d z \d t
\end{equation}
\vspace*{0.5em}

\noindent \textbf{3. 考虑质量增量}

由$\d t$时间内密度的变化,可以得到质量的变化为
\begin{equation}
	\d m_t = \left[\rho + \dfrac{\partial \rho}{\partial t}\d t\right]\d x \d y \d z - \rho \d x \d y \d z = \dfrac{\partial \rho}{\partial t}\d x \d y \d z\d t
\end{equation}

\noindent \textbf{4. 考虑质量守恒}
 
 根据\blue[质量守恒定律],净流量的总质量应该等于质量增量,即
 \begin{equation}
 	- \left[\dfrac{\partial (\rho u)}{\partial x} +  \dfrac{\partial (\rho v)}{\partial y} +  \dfrac{\partial (\rho w)}{\partial z} \right] \d x \d y \d z \d t = \dfrac{\partial \rho}{\partial t} \d x \d y \d z\d t
 \end{equation}
化简得到
\begin{equation}
	\dfrac{\partial \rho}{\partial t} + \dfrac{\partial (\rho u)}{\partial x} +  \dfrac{\partial (\rho v)}{\partial y} +  \dfrac{\partial (\rho w)}{\partial z}  = 0
\end{equation}
利用微分运算法则,得
\begin{equation}
	\dfrac{\partial \rho}{\partial t} + u \dfrac{\partial \rho }{\partial x} +  v \dfrac{\partial \rho }{\partial y} +  w \dfrac{\partial \rho }{\partial w} + \rho \left(\dfrac{\partial u}{\partial x} + \dfrac{\partial v}{\partial y}  + \dfrac{\partial w}{\partial z} \right) = 0
\end{equation}
即得到连续(质量方程)\footnote{在方程中$\dfrac{\text{D} \rho}{\text{D} t}$中微分算子$\text{D}$代表的是欧拉方法中的随体导数,下同。}

\theorem[连续(质量)方程\index{LXFC@连续方程} \index{ZLFC@质量方程}]
{
	\quad \vspace*{-1em}
	\begin{equation}
	 \dfrac{\text{D} \rho}{\text{D} t} + \rho (\nabla \bm{V}) = \dfrac{\partial \rho}{\partial t} + \nabla \cdot (\rho \bm{V}) = 0
	\end{equation}
}

\noindent \textbf{5. 连续(质量)方程的物理意义}
\begin{itemize}
	\item $\dfrac{\partial \rho}{\partial t}$:流体微团(控制体)密度的\red[局部增长率]
	\item $\displaystyle \nabla \cdot (\rho \bm{V}) = \lim\limits_{\text{V} \to 0} = \dfrac{ \displaystyle \oiint \rho (\bm{V} \cdot \bm{n})\, \d s}{\text{V}}$:流体微团(控制体)\blue[单位体积流出的质量流量]
\end{itemize}
\vspace*{1em}

\noindent \textbf{6. 连续(质量)方程的特殊情况}
\begin{enumerate}[\hspace*{1.5em} (1) ]
	\item 流体质点密度是\textbf{常数(不随时间和空间变化)},则
	\begin{equation*}
		\begin{aligned}
			&\dfrac{\partial \rho}{\partial t} = 0\\
			&\bm{V} \cdot \nabla \rho = 0
		\end{aligned} \quad \Longrightarrow \quad \rho (\nabla \bm{V}) = 0 \quad \Longrightarrow \quad \nabla \bm{V} = 0
	\end{equation*}
	说明此时单位体积流出的体积流量(相对体积膨胀率,速度散度)为0.
	
	\item 流体质点密度\textbf{在流动中保持不变(随体导数为零;不可压缩流动)},则
	\begin{equation*}
		\dfrac{\D \rho}{\D t} = 0 \quad \Longrightarrow \quad \rho (\nabla \bm{V}) = 0 \quad \Longrightarrow \quad \nabla \bm{V} = 0
	\end{equation*}
	说明不可压缩流动中,流体微团的相对体积膨胀率保持为零,或从流体微团(控制体)流出的单位体积流量为零。
\end{enumerate}
\vspace*{-1em}
\warn[\hspace*{1.5em}\textbf{不可压缩流动的密度并不一定处处为常数。}]
\vspace*{0.5em}

\noindent \textbf{7. 连续(质量)方程的应用}

连续方程是流动首先应该满足的基本关系。根据某方向的速度分布和连续方程,确定出其他方向的速度分布。

\examples 设不可压缩流体在$xoy$平面内流动,速度沿$x$轴方向的分量$u = Ax$ ($A$ 为常数),求速度在$y$轴方向的分量$v$。

\solve 对于不可压缩流动,有$\dfrac{\D \rho}{\D t}$

\subsection{欧拉方程——理想(无粘)流体运动的动量方程}

\noindent \textbf{1. 动量定理与牛顿第二定律}
\begin{itemize}
	\item \dy[动量定理]{DLDL}:物体动量的改变量等于物体合外力的冲量
	\begin{equation}
		\bm{I} = \bm{F} \d t = m \d \bm{V}
	\end{equation}

	\item \dy[牛顿第二定理]{NDDEDL}:针对流体来说,加速度采用随体导数给出,即
	\begin{equation}
		\bm{F} = m \dfrac{\D \bm{V}}{\D t}
	\end{equation}
	
\end{itemize}
\vspace*{0.5em}

\noindent \textbf{2. 考虑流场中的一个六面体微团(控制体)}
\begin{itemize}
	\item 体积:$\d x \d y \d z$;\vspace*{-0.5em}
	\item 中心:$O(x,y,z)$的速度$\bm{v}_o = (u,v,w)$;\vspace*{-0.5em}
	\item 密度:$\rho(x,y,z,t)$是时间和空间的函数;\vspace*{-0.5em}
	\item 压强:$p(x,y,z,t)$是时间和空间的函数;\vspace*{-0.5em}
	\item $t$时刻的质量流量(密流或通量):$x,y,z$方向分别为$\rho u , \rho v , \rho w$
\end{itemize}
\vspace*{0.5em}

\noindent \textbf{3. 计算合外力}
\begin{itemize}
	\item $x$方向上表面力:
	\begin{equation*}
		\left(p - \dfrac{\partial p}{\partial x} \dfrac{\d x}{2}\right) \d y \d z - \left(p + \dfrac{\partial p}{\partial x} \dfrac{\d x}{2}\right) \d y \d z = -\dfrac{\partial p}{\partial x} \d x \d y \d z
	\end{equation*}
	
	\item $x$方向上质量力:$f_x\rho \d x \d y \d z$
	
	\item $x$方向上的合外力
	\begin{equation}
		F_x = f_x\rho \d x \d y \d z - \dfrac{\partial p}{\partial x} \d x \d y \d z
	\end{equation}
\end{itemize}
\vspace*{0.5em}

\noindent \textbf{4. 利用牛顿第二定律得到方程}
\vspace*{0.5em}

	由$x$方向上的加速度$\dfrac{\D u}{\D x}$及牛顿第二定律,可以得到
	\begin{equation}
		 f_x\rho \d x \d y \d z - \dfrac{\partial p}{\partial x} \d x \d y \d z = (\rho \d x \d y \d z) \dfrac{\D u}{\D x}
	\end{equation}
	化简得到
	\begin{equation}
		f_x\rho - \dfrac{\partial p}{\partial x} = \rho \dfrac{\D u}{\D x} \quad \Longleftrightarrow \quad f_x - \dfrac{1}{\rho} \dfrac{\partial p}{\partial x} = \dfrac{\partial u}{\partial t} + u \dfrac{\partial u}{\partial x} + v\dfrac{\partial u}{\partial y} + w \dfrac{\partial u}{\partial z}
	\end{equation}
同样地,其他方向上也有类似的方程,可以得到\dy[欧拉方程]{OLFC}。

\theorem[欧拉方程]
{
	\vspace*{-1em}
	\begin{equation}
		\begin{cases}
			\, f_x - \dfrac{1}{\rho} \dfrac{\partial p}{\partial x} = \dfrac{\partial u}{\partial t} + u \dfrac{\partial u}{\partial x} + v\dfrac{\partial u}{\partial y} + w \dfrac{\partial u}{\partial z}\\[0.8em]
			\, f_y - \dfrac{1}{\rho} \dfrac{\partial p}{\partial y} = \dfrac{\partial v}{\partial t} + u \dfrac{\partial v}{\partial x} + v\dfrac{\partial v}{\partial y} + w \dfrac{\partial v}{\partial z}\\[0.8em]
			\, f_z - \dfrac{1}{\rho} \dfrac{\partial p}{\partial z} = \dfrac{\partial w}{\partial t} + u \dfrac{\partial w}{\partial x} + v\dfrac{\partial w}{\partial y} + w \dfrac{\partial w}{\partial z}
		\end{cases}
	\end{equation}
	写成矢量形式为
	\begin{equation}
		\bm{f} - \dfrac{1}{\rho}\nabla p = \dfrac{\D \bm{V}}{\D t}
	\end{equation}
}

\section{理想流体运动积分方程组}
\subsection{概述}
\begin{equation*}
	\mbox{研究对象}\,
	\begin{cases}
		\, \,\,\, \mbox{\blue[系统] —— 拉格朗日型积分方程} \, 
			\begin{cases}
				\, \mbox{确定的封闭系统}\\
				\, \mbox{方程包含某物理量对时间的变化率}\\
				\, \mbox{系统大小和形状均随时间变化,追踪系统有困难}\\
				\, \mbox{不容易表达物理量随时间的变化率}
			\end{cases}\\
		\hspace*{1.3em}\Bigg\updownarrow \hspace*{1em}\mbox{\red[雷诺输运方程]}\\
		\, \mbox{\blue[控制体] —— 欧拉型积分方程}\, 
		\begin{cases}
			\, \mbox{把流体系统视为控制体,建立物理学的普遍性规律}\\
			\, \mbox{把适用于流体系统的各个物理普遍定律用控制体的方式表达出来}
		\end{cases}
	\end{cases}
\end{equation*}

\subsection{拉格朗日型积分方程}

\noindent \textbf{1. 质量(连续)方程}

对于封闭系统,质量恒定,不随时间变化,则\footnote[1]{注:$\V$代表微元体的体积,$\bm{V}$代表微元体的速度}
\begin{equation}
	\dfrac{\d m}{\d t} =\dfrac{\d }{\d t} \iiint \limits_{\V} \rho \,\d \V = 0
\end{equation}
\vspace*{1em}

\noindent \textbf{2. 动量方程}

对于封闭系统,系统所受合外力等于系统动量对时间的变化率,则
\begin{equation}
	\sum \bm{F} = \dfrac{\d \bm{K}}{\d t} = \dfrac{\d (m \bm{V})}{\d t} = \dfrac{\d }{\d t} \iiint\limits_{\V} \rho \bm{V} \, \d \V
\end{equation}
\vspace*{1em}

\noindent \textbf{3. 能量方程}

对于封闭系统,单位时间内由外界传入系统对热量与外界对系统所做对功之和,等于该系统对总能量对时间的变化率。即
\begin{equation}
	\dot{Q} + \dot{W} = \dfrac{\d E}{\d t} \iiint\limits_{\V} \rho \left(e + \dfrac{\bm{V}^2}{2}\right)\, \d \V
\end{equation}
其中,$E$为系统的总能量,$e$为微元体单位体积的内能,$\dfrac{\bm{V}^2}{2}$为单位微元体体积的动能。
\vspace*{1em}

\subsection{雷诺输运方程}
对于\blue[系统]的物理量$N$,假设\red[单位质量流体]中含有的物理量为$\sigma$,则
\begin{equation}
	\sigma = \dfrac{\d N}{\d m} = \dfrac{\d N}{\rho \d \V}
\end{equation}
系统的物理量$N$可以通过积分得到
\begin{equation}
	N = \iiint\limits_{\V} \sigma \rho \, \d \V
\end{equation}
\begin{itemize}
	\item 如果$\sigma = 1$,则$N = m$代表质量\vspace*{-0.5em}
	\item 如果$\sigma = \bm{V}$,则$N = \bm{V}$代表系统的速度\vspace*{-0.5em}
	\item 如果$\sigma = e + \bm{V}^2/2$,则$N = e + \bm{V}^2/2$代表系统的总能量
\end{itemize}

设$t$时刻时,控制体和系统重合。而在$t+\Delta t$时刻,控制体和系统仍重合的部分记为$\V_{\sys}^1$,系统超出控制体的部分记为$\V_{\sys}^2$,控制体剩下与系统不重合的部分记为$\V_{\sys}^3$.

则
\begin{align*}
	\Delta N & = N(t + \Delta t) - N(t)\\[0.5em]
	& = \iiint\limits_{\V_{\sys}^1 + \V_{\sys}^2} \rho \sigma \, \d \V - \iiint\limits_{\V_{\sys}^1 + \V_{\sys}^3} \rho \sigma \, \d \V \\[0.5em]
	& = \mathop{
			\underbrace{
				\mathop{\underbrace{\iiint\limits_{\V_{\sys}^1} \rho \big[\sigma(t + \Delta t) - \sigma(t)\big]\, \d \V}}_{\scriptsize \mbox{\makecell[c]{体积不变的部分\\[-0.1em] 物理量随时间变化}}} 
				+ \mathop{\underbrace{\iiint\limits_{\V_{\sys}^2} \rho \sigma(t + \Delta t)\, \d \V - \iiint\limits_{\V_{\sys}^3} \rho \sigma(t)\, \d \V}}_{\scriptsize \mbox{\makecell[c]{体积变化的部分\\[-0.1em]所引起的物理量变化}}} 
			}_{\scriptsize \mbox{物理量的\blue[随体变化]}}
		}
\end{align*}

则
\begin{equation}
	\dfrac{\d N}{\d t} = \lim\limits_{t \to 0} \dfrac{\Delta N}{\Delta t}
\end{equation}
分别考虑每一项的极限:\vspace*{-0.5em}
\begin{itemize}
	\item 由于$\Delta t \to 0$,则$\V_{\sys}^1$体积几乎不变,“似动非动”,即$\V_{\sys}^1 \to \V_{\con}$,则
	\begin{equation*}
		\lim\limits_{\Delta t \to 0} \iiint\limits_{\V_{\sys}^1} \rho \big[\sigma(t + \Delta t) - \sigma(t)\big]\, \d \V = \iiint\limits_ {\V_{\con}} \rho \dfrac{\partial \sigma}{\partial t} \, \d \V = \dfrac{\partial }{\partial t}\iiint\limits_ {\V_{\con}} \rho \sigma \, \d \V 
	\end{equation*}
	
	\item 对于$\V_{\sys}^2$来说,其体积的变化量$\d \V$等于流体流出交界面$\S_{12}$的流量,记交界面$\S_{12}$的法向量为$\bm{n}$,则有$\d \V = \bm{V} \cdot  \bm{n} \, \d \S$,即
	\begin{equation*}
		\lim\limits_{\Delta t \to 0} \iiint\limits_{\V_{\sys}^2} \rho \sigma(t + \Delta t)\, \d \V = \iint\limits_{\S_{12}} \rho \sigma (\bm{V} \cdot \bm{n})\, \d \S
	\end{equation*}
	
	\item 对于$\V_{\sys}^3$来说,其体积的变化量$\d \V$等于流体流入交界面$\S_{13}$的流量,记交界面$\S_{13}$的法向量为$\bm{n}$,则有$\d \V = - \bm{V} \cdot  \bm{n} \, \d \S$(流入的方向与法向相反),即
	\begin{equation*}
		\lim\limits_{\Delta t \to 0} \iiint\limits_{\V_{\sys}^3} \rho \sigma(t)\, \d \V = -\iint\limits_{\S_{13}} \rho \sigma (\bm{V} \cdot \bm{n})\, \d \S
	\end{equation*}
\end{itemize}
将每一项的结果代回$\Delta N$的表达式,得
\begin{align*}
	\Delta N & = \iiint\limits_{\V_{\sys}^1} \rho \big[\sigma(t + \Delta t) - \sigma(t)\big]\, \d \V + \iiint\limits_{\V_{\sys}^2} \rho \sigma(t + \Delta t)\, \d \V - \iiint\limits_{\V_{\sys}^3} \rho \sigma(t)\, \d \V \\[0.5em]
	& =  \dfrac{\partial }{\partial t}\iiint\limits_ {\V_{\con}} \rho \sigma \, \d \V + \iint\limits_{\S_{12}} \rho \sigma (\bm{V} \cdot \bm{n})\, \d \S - \left(-\iint\limits_{\S_{13}} \rho \sigma (\bm{V} \cdot \bm{n})\, \d \S \right) \\[0.5em]
	& = \dfrac{\partial }{\partial t}\iiint\limits_ {\V_{\con}} \rho \sigma \, \d \V \,\,+ \oiint\limits_{\S} \rho \sigma (\bm{V} \cdot \bm{n})\, \d \S  
\end{align*}

得到\dy[雷诺输运方程]{LNSYFC}如下

\theorem[雷诺输运方程]
{
	\quad \vspace*{-1em}
	\begin{equation}
		\mathop{
			\underbrace{\dfrac{\d }{\d t} \iiint\limits_{\V_{\sys}} \rho \sigma \, \d \V} 
		}_{\small \mbox{\makecell[c]{\blue[系统]\\[-0.5em] \red[拉氏观点]}}}
		=\mathop{ \underbrace{\dfrac{\partial }{\partial t}\iiint\limits_ {\V_{\con}} \rho \sigma \, \d \V  + \oiint\limits_{\S} \rho \sigma (\bm{V} \cdot \bm{n})\, \d \S}}_{\small \mbox{\makecell[c]{\blue[控制体]\\[-0.5em] \red[欧拉观点]}}}
		\label{雷诺输运方程}
	\end{equation}
	其中,
	{
		\begin{itemize}
			\item $\displaystyle \dfrac{\partial }{\partial t}\iiint\limits_ {\V_{\con}} \rho \sigma \, \d \V$代表控制体内物理量随时间的变化率(表征流场的\red[非定常]特性)。
			\item $\displaystyle \oiint\limits_{\S} \rho \sigma (\bm{V} \cdot \bm{n})\, \d \S $代表单位时间内,通过控制面流出的物理量净增量(流场的\red[不均匀性]引起的),可以看成是由于流动对物理量的输运引起控制体内物理量的净变化 。
		\end{itemize}
	}
}
\noindent 雷诺输运方程的总体思想:\vspace*{-0.5em}
\begin{enumerate}
	\item 在$t$时刻,系统恰好和控制体重合;\vspace*{-0.5em}
	\item 在$t$时刻,系统随流体运动到某一具体的位置,将此时刻的系统固化为控制体;\vspace*{-0.5em}
	\item 系统在此时刻的所含物理量的变化率,可以用\blue[控制体内所含物理量的时间变化率] $+$ \blue[流场欧拉速度对控制体内物理量的输运]来表述 。
\end{enumerate}

\subsection{欧拉型积分方程}
\sssection[质量方程]

令$\sigma = 1$,代入雷诺输运方程\eqref{雷诺输运方程},可以得到
\begin{equation}
	\dfrac{\d m}{\d t} = \dfrac{\partial }{\partial t} \iiint\limits_{\V_{\con}} \rho \, \d \V + \oiint\limits_{\S}\rho (\bm{V} \cdot \bm{n}) \, \d S 
\end{equation}
由质量守恒,可得$\dfrac{\d m}{\d t} = 0$,即
\begin{equation}
	\dfrac{\partial }{\partial t} \iiint\limits_{\V_{\con}} \rho \, \d \V = - \oiint\limits_{\S}\rho (\bm{V} \cdot \bm{n}) \, \d S 
\end{equation}
其物理意义为\red[控制体内的质量增加率$=$净流入控制面的质量流量]。
\vspace*{1em}

\sssection[动量方程]

令$\sigma = \bm{V}$,代入雷诺输运方程\eqref{雷诺输运方程},可以得到
\begin{equation}
	\dfrac{\d \bm{K}}{\d t} = \dfrac{\partial }{\partial t} \iiint\limits_{\V_{\con}} \rho \bm{V}\, \d \V + \oiint\limits_{\S}\rho \bm{V}(\bm{V} \cdot \bm{n}) \, \d S 
\end{equation}
由动量定理,可得$\displaystyle \sum \bm{F} = \dfrac{\d \bm{K}}{\d t}$,即
\begin{equation}
	\sum \bm{F} = \dfrac{\partial }{\partial t} \iiint\limits_{\V_{\con}} \rho \bm{V}\, \d \V \,\,+ \oiint\limits_{\S}\rho (\bm{V} \cdot \bm{n}) \, \d S  = \dfrac{\partial }{\partial t} \iiint\limits_{\V_{\con}} \rho\bm{V} \, \d \V + \oiint\limits_{\S} \bm{p}_n \, \d S 
\end{equation}
其物理意义为\red[控制体所受合外力$=$控制体内动量的增加率$+$净流出控制面的动量流量]。
\vspace*{1em}

\sssection[能量方程]

令$\sigma = e + \dfrac{V^2}{2}$,代入雷诺输运方程\eqref{雷诺输运方程},可以得到
\begin{equation}
	\dfrac{\d E}{\d t} = \dfrac{\partial }{\partial t} \iiint\limits_{\V_{\con}} \rho \left(\rho + \dfrac{V^2}{2}\right) \, \d \V + \oiint\limits_{\S}\rho \left(\rho + \dfrac{V^2}{2}\right) (\bm{V} \cdot \bm{n}) \, \d S 
\end{equation}
由能量守恒,可得$\dfrac{\d E}{\d t} = \dot{Q} + \dot{W}$,即
\begin{equation}
	\dot{Q} + \dot{W} = \dfrac{\partial }{\partial t} \iiint\limits_{\V_{\con}} \rho \left(\rho + \dfrac{V^2}{2}\right) \, \d \V + \oiint\limits_{\S}\rho \left(\rho + \dfrac{V^2}{2}\right) (\bm{V} \cdot \bm{n}) \, \d S 
\end{equation}
其物理意义为\red[外界对控制体的传热率和净输入功率$=$控制体内能量的增加率$+$净流出控制面的能量流量]。


\section{漩涡运动基础}
\subsection{漩涡现象}
\sssection[自然界]

龙卷风(旋风)、台风:空气强烈对流而产生的一种高速旋转的自然现象。
\vspace*{1em}

\sssection[工程应用]

机翼的翼尖漩涡、机翼边条翼分离漩涡、卡门涡街

\subsection{基本概念}
\sssection[涡量]

\defination[涡量]
{
	\dy[涡量]{WL}:流场中任何一个流体微团绕其中心作刚性旋转的角速度的两倍。【注:涡量是一个纯运动学的概念】
	\begin{align}
		\bm{\omega} = \omega_x \bm{i} + \omega_y \bm{j} + \omega_z \bm{k}, \omega= \sqrt{\omega_x^2 + \omega_y^2 + \omega_z^2}\\
		\bm{\varOmega} = \text{rot} \bm{V} = 2 \bm{\omega} = \nabla \times \bm{V} = 
		\begin{pmatrix}
			\bm{i} & \bm{j} & \bm{k} \\
			\dfrac{\partial }{\partial x} & \dfrac{\partial }{\partial y} & \dfrac{\partial }{\partial z}\\
			u & v & w
		\end{pmatrix}
	\end{align}
}

\noindent 涡量是表征流体微团选装快慢的特征量,是刻画漩涡局部运动的物理量。

\begin{itemize}
	\item $\bm{\varOmega} = 0$ \quad 无旋流场,存在势函数
	\item $\bm{\varOmega} \neq 0$ \quad 有旋流场
\end{itemize}

\defination[涡线及其微分方程]
{
	\dy[涡线]{WX}:和流线一样,在某一瞬时涡量场中存在一条光滑的曲线,在曲线上的每一点的涡量与该曲线相切。其微分方程为
	\begin{equation}
		\dfrac{\d x}{\omega_x} = \dfrac{\d y}{\omega_y} = \dfrac{\d z}{\omega_z}
	\end{equation}
}

\defination[涡面与涡管]
{
	\dy[涡面]{WM}:任意取非涡线的曲线,过这个曲线上的每一点都做一条涡线,形成的一个由涡线组成的曲面。\\[0.5em]
	\hspace*{2.25em}\dy[涡管]{WG}:任意取非涡线的封闭曲线,过这个曲线上的每一点都做一条涡线,形成的一个由涡线组成的封闭曲面。
}

\sssection[涡通量]

\defination[涡通量]
{
	\dy[涡通量]{WTL}:通过某个面积的涡量总和。\vspace*{-0.5em}
	{
		\begin{itemize}
			\item 平面(二维)问题
			\begin{equation}
				I = \iint\limits_{A} 2 \omega_z \, \d A
			\end{equation}
			\item 曲面(三维)问题
			\begin{equation}
				I = \iint\limits_{S} \bm{\varOmega}\cdot \, \d \bm{S} = \iint \text{rot} \bm{V}\cdot \d \bm{S} = \iint\limits_{S} \bm{\varOmega}\cdot \bm{n} \d S
			\end{equation}
		\end{itemize}
	}
}
\vspace*{1em}

\sssection[环量]

\defination[环量]
{
	\dy[环量]{HL}:速度矢量沿着流场中的任意一条封闭曲线的积分。
	\begin{equation}
		\varGamma = \oint\limits_L \bm{V} \cdot \, \d\bm{s} = \oint\limits_L \bm{V}\cos \alpha \, \d s = \oint \limits_L (u\,\d x + v\, \d y + w\, \d z)
	\end{equation}
}
\noindent 环量被用来表征漩涡的强度。
\vspace*{1em}

\sssection[环量和涡通量的关系]

对于二维流动,有
\begin{align*}
	\d \varGamma & = \int\limits_{ABCDA} \bm{V}\, \d \bm{s}\\[0.5em]
	& = \left(u + \dfrac{\partial u}{\partial x}\dfrac{\d x}{2}\right)\, \d x + \left(v + \dfrac{\partial v}{\partial x}\, \d x  + \dfrac{\partial v}{\partial y} \dfrac{\d y}{2}\right)\d y - \left(u + \dfrac{\partial u}{\partial y}\, \d y + \dfrac{\partial u}{\partial x} \dfrac{\d x}{2}\right)\d x - \left(v + \dfrac{\partial v}{\partial y} \dfrac{\d y}{2}\right)\d y\\[0.5em]
	& = \left(\dfrac{\partial v}{\partial x} - \dfrac{\partial u}{\partial y}\right)\, \d x \d y = 2 \omega_z \, \d x \d y
\end{align*}

对所有网格区域的速度环量求和,可以得到绕整个封闭曲线的速度环量值。

\theorem[格林公式]
{
	\vspace*{-1em}
	\begin{equation}
		\varGamma = \oint\limits_{L} \bm{V} \cdot \d \bm{s} = \oint\limits_{L}(u\,\d x + v\, \d y) = \iint\limits_A \left(\dfrac{\partial v}{\partial x} - \dfrac{\partial u}{\partial y}\right)\, \d S = \iint\limits_A 2 \omega_z \, \d \bm{S}
	\end{equation}
}

\subsection{涡的诱导速度}

\subsection{理想流中的涡定理}
\vspace*{-1em}
\theorem[涡强保持定理]
{
	沿涡线或涡管的涡强度保持不变。其推论为:涡管不能在流体中中断,可以延伸到无限远去,可以自相连接成一个涡环(不一定是圆环),也可以止于边界,固体的边界或自由边界。
}

\theorem[涡线或涡面保持定理]
{
	在某时刻构成涡线或涡管的流体质点,在以后运动过程中仍将构成涡线或涡管。
}

\theorem[涡强守恒定理]
{
	涡的强度不随时间变化,既不会增强,也不会削弱或消失。
}



















