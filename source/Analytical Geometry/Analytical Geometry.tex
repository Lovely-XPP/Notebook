%模板
\documentclass[12pt,a4paper,twoside]{book}

\usepackage{ctex}
\usepackage{ulem}
\usepackage{makecell}
\usepackage{verbatim}
\usepackage{enumerate}%罗列专用宏包
\usepackage{graphicx}%插入图片的宏包
\usepackage{subfigure} 
\usepackage{newtxtext}
\usepackage{amsmath,ntheorem}
\usepackage{newtxmath,bm}
\usepackage{makeidx}%索引专用
\makeindex  %添加索引
\usepackage{amsmath}
\usepackage{fancyhdr}
\usepackage{framed}
\usepackage{wrapfig}
\usepackage{textcomp}%树叶图案在这个包里
\usepackage{bbding}%很多漂亮的图案
\usepackage[dvipsnames, svgnames, x11names,table,xcdraw]{xcolor}%导入了所有颜色配置文件的宏包
\usepackage{emptypage}
\usepackage{geometry}%页边距调整
\geometry{left=2cm,right=2cm,bottom=2cm,top=2cm}
\usepackage{titletoc}%目录页的宏包
\usepackage{titlesec}%改变章节或标题的样式的宏包
\usepackage[bookmarks=true,colorlinks,linkcolor=black]{hyperref}
\usepackage{enumerate}%使用改宏包优化罗列环境
\usepackage{tcolorbox}%box宏包
\usepackage{multirow} 
\usepackage{longtable}%跨页表格
\usepackage{supertabular}%跨页表格
\usepackage{tikz}
\usetikzlibrary{shapes.geometric}
\usetikzlibrary{arrows,arrows.meta}

%各类设置

%字体设置
\setCJKmainfont[BoldFont=PingFangSC-Semibold]{PingFangSC-Regular}%需要查看电脑字体查找对应字体的文件英文文件名

%章节或标题的样式
\titleformat{\chapter}{\bfseries\Huge\color{titlepurple}}{第\ \thechapter\ 章\ \quad}{0pt}{}
\titleformat{\section}{\Large\color{titlepurpleb}}{\bfseries{\thesection}\quad  }{0pt}{}
\titleformat{\subsection}{\large\color{titlepurplec}}{\bfseries{\thesubsection}\quad  }{0pt}{}
\titlespacing{\subsection}{1.5em}{0.1em}{1em}[1em]
%格式如下:\titlespacing*{章节名称}{左间距}{(前)行间距}{(后)行间距}[右间距(一般都没用,填0.1em即可,但不能不填)]
\titlespacing*{\subsubsection}{2em}{3em}{1em}[1em]

%目录调整
\newcounter{mycontents}
\newcommand{\thecontents}{\refstepcounter{mycontents} \alph{mycontents}.}
%\titlecontents{标题名}[左间距]{标题格式}{标题标志}{无序号标题}{指引线与页码}[下间距]
\titlecontents{chapter}
[0cm]
{\bf \large \vspace{0.8em} }{\contentspush{第 \thecontentslabel\ 章 \hspace*{0.8em}}}{}{\titlerule*[0.5pc]{$\cdot$}\contentspage}
\titlecontents{section}[1.7cm]{\bf  \vspace{0.5em} }{\contentslabel{2.4em}}{\hspace*{-2.5em} \thecontents \hspace*{0.8em}}{\titlerule*[0.5pc]{$\cdot$}\contentspage}
\titlecontents{subsection}[2.5cm]{\small \vspace{0.2em} }{\contentslabel{3em}}{}{\titlerule*[0.5pc]{$\cdot$}\contentspage}

%定义颜色
%定义某个颜色,对应颜色代号查表
\definecolor{titlepurple}{HTML}{5758BB}%一级标题(目前:蓝紫色)
\definecolor{titlepurpleb}{HTML}{3A006F}%二级标题(目前:深紫色)
\definecolor{titlepurplec}{HTML}{006266}%三级标题(目前:墨绿色)
\definecolor{tab1}{HTML}{9698ED}%表格1
\definecolor{tab2}{HTML}{DBDCFF}%表格2
\definecolor{dy0}{HTML}{EA7500}%小标题定义专用(目前:橙黄色)
\definecolor{dl}{HTML}{007500}%小标题定理专用(目前:深绿色)
\definecolor{inference}{HTML}{343300}%小标题推论专用(目前:墨绿色)
\definecolor{ex}{HTML}{7158e2}%小标题例专用(目前:紫色)
\definecolor{dy}{HTML}{BF0060}%夹杂在文本中的定义词的颜色1(目前:深红色)
\definecolor{dy2}{HTML}{6C3365}%夹杂在文本中的定义词的颜色2(目前:红紫色)
\definecolor{超链接}{HTML}{0000C6}%含超链接的文字专用色(目前:蓝紫色)
\definecolor{文字底色}{HTML}{F8FF00}%强调的文字底色(目前:黄色)


%定义计数器
\newcounter{theorem}[section]
\newcounter{defination}[section]
\newcounter{example}[section]
\newcounter{inference}[section]
\newcounter{E}[section]
\newcounter{F}[section]
\renewcommand{\thetheorem}{定理 \thesection.\arabic{theorem}}
\renewcommand{\thedefination}{定义 \thesection.\arabic{defination}}
\renewcommand{\theexample}{例 \thesection.\arabic{example}}
\renewcommand{\theinference}{推论 \thesection.\arabic{inference}}


%定义环境
\newcommand{\mybox}[2][]{
	\begin{tcolorbox}[on line,
		arc=0pt,outer arc=0pt,colback=#1!10!white,colframe=#1,
		boxsep=0pt,left=3pt,right=3pt,top=3.5pt,bottom=3.5pt,
		boxrule=0pt,leftrule=1.5pt]#2
\end{tcolorbox}}

%命令格式说明:正常情况的命令就是中文对应的英文名,以下有几个特殊情况进行了微调
%1. 小标题在列表上方,使用enup+英文名;小标题在列表下方,使用enbelow+英文名
%2. 标题间隔太大,采用t+英文名
%3. 间距太小,用add+英文名
%4. 在列举环境中间距太小用adden+英文名

%定理类
\newcommand{\theorem}[1][]{\vspace{1em}\noindent\refstepcounter{theorem}\label{#1} \mybox[dl]{\color{dl}\bf{\thetheorem}\hspace{1em}#1}\vspace{0.5em}  \par}
\newcommand{\enuptheorem}[1][]{\vspace{1em}\noindent\refstepcounter{theorem}\label{#1} \mybox[dl]{\color{dl}\bf{\thetheorem}\hspace{1em}#1}\vspace*{-0.8cm}}
\newcommand{\enbelowtheorem}[1][]{\hspace*{-1.5em}\noindent\refstepcounter{theorem}\label{#1} \mybox[dl]{\color{dl}\bf{\thetheorem}\hspace{1em}#1}  \par}
\newcommand{\ttheorem}[1][]{\noindent\refstepcounter{theorem}\label{#1} \mybox[dl]{\color{dl}\bf{\thetheorem}\hspace{1em}#1} \vspace*{-1em}\par }
\newcommand{\addtheorem}[1][]{\vspace{1.2em}\noindent\refstepcounter{theorem}\label{#1} \mybox[dl]{\color{dl}\bf{\thetheorem}\hspace{1em}#1}\vspace*{0.5em}  \par}
\newcommand{\addentheorem}[1][]{\vspace{1.2em}\hspace*{-1.5em}\noindent\refstepcounter{theorem}\label{#1} \mybox[dl]{\color{dl}\bf{\thetheorem}\hspace{1em}#1}\vspace{0.5em}  \par}

%推论类
\newcommand{\inference}[1][]{\vspace{1em}\noindent\refstepcounter{inference}\label{#1} \mybox[inference]{\color{inference}\bf{\theinference}\hspace{1em}#1}\vspace{0.5em}   \par}
\newcommand{\enupinference}[1][]{\vspace{1em}\noindent\refstepcounter{inference}\label{#1} \mybox[inference]{\color{inference}\bf{\theinference}\hspace{1em}#1}\vspace*{-0.8cm}  }
\newcommand{\enbelowinference}[1][]{\hspace*{-1.5em}\noindent\refstepcounter{inference}\label{#1} \mybox[inference]{\color{inference}\bf{\theinference}\hspace{1em}#1}  \par}
\newcommand{\tinference}[1][]{\noindent\refstepcounter{inference}\label{#1} \mybox[inference]{\color{inference}\bf{\theinference}\hspace{1em}#1}\vspace*{0.5em} \par }
\newcommand{\addinference}[1][]{\vspace{1.2em}\noindent\refstepcounter{inference}\label{#1} \mybox[inference]{\color{inference}\bf{\theinference}\hspace{1em}#1}\vspace{0.5em} \par}
\newcommand{\addeninference}[1][]{\vspace{1.2em}\hspace*{-1.5em}\noindent\refstepcounter{inference}\label{#1} \mybox[inference]{\color{inference}\bf{\theinference}\hspace{1em}#1}\vspace{0.5em} \par}

%定义类
\newcommand{\defination}[1][]{\vspace{1em}\noindent\refstepcounter{defination}\label{#1} \mybox[dy0]{\color{dy0}\bf{\thedefination}\hspace{1em}#1}\vspace{0.5em} \par}
\newcommand{\enupdefination}[1][]{\vspace{1em}\noindent\refstepcounter{defination}\label{#1} \mybox[dy0]{\color{dy0}\bf{\thedefination}\hspace{1em}#1}\vspace*{-0.8cm}}
\newcommand{\enbelowdefination}[1][]{\hspace*{-1.5em}\noindent\refstepcounter{defination}\label{#1} \mybox[dy0]{\color{dy0}\bf{\thedefination}\hspace{1em}#1} \par }
\newcommand{\tdefination}[1][]{   \noindent\refstepcounter{defination}\label{#1} \mybox[dy0]{\color{dy0}\bf{\thedefination}\hspace{1em}#1}\vspace*{0.5em}  \par }
\newcommand{\adddefination}[1][]{\vspace{1.2em}\noindent\refstepcounter{defination}\label{#1} \mybox[dy0]{\color{dy0}\bf{\thedefination}\hspace{1em}#1}\vspace{0.5em} \par}
\newcommand{\addendefination}[1][]{\vspace{1.2em}\hspace*{-1.5em}\noindent\refstepcounter{defination}\label{#1} \mybox[dy0]{\color{dy0}\bf{\thedefination}\hspace{1em}#1}\vspace{0.5em} \par}

%例类
\newcommand{\example}[1][]{\vspace{1em}\noindent\refstepcounter{example}\label{#1} \mybox[ex]{\color{ex}\bf{\theexample}\hspace{1em}#1}\vspace{0.5em} \par }
\newcommand{\enupexample}[1][]{\vspace{1em}\noindent\refstepcounter{example}\label{#1} \mybox[ex]{\color{ex}\bf{\theexample}\hspace{1em}#1}\vspace*{-0.8cm}}
\newcommand{\enbelowexample}[1][]{\hspace*{-1.5em}\noindent\refstepcounter{example}\label{#1} \mybox[ex]{\color{ex}\bf{\theexample}\hspace{1em}#1} \par }
\newcommand{\texample}[1][]{  \noindent\refstepcounter{example}\label{#1} \mybox[ex]{\color{ex}\bf{\theexample}\hspace{1em}#1} \vspace*{0.5em} \par }
\newcommand{\addexample}[1][]{\vspace{1.2em}\noindent\refstepcounter{example}\label{#1} \mybox[ex]{\color{ex}\bf{\theexample}\hspace{1em}#1}\vspace{0.5em} \par }
\newcommand{\addenexample}[1][]{\vspace{1.2em}\hspace*{-1.5em}\noindent\refstepcounter{example}\label{#1} \mybox[ex]{\color{ex}\bf{\theexample}\hspace{1em}#1}\vspace{0.5em} \par }

%\theoremstyle{break}
%\theoremindent0.2cm
%\newtheorem*{theorem}{\hspace{0.2cm}\color{dl}\label{#1} \mybox[dl]{\color{dl}定理\addtocounter{A}{1} \thesection.\arabic{A}}}
%\newtheorem*{defination}{\hspace{0.2cm}\color{dy0}\label{#1} \mybox[dy0]{\color{dy0}定义\addtocounter{B}{1} \thesection.\arabic{B}}}
%\newtheorem*{feature}{\hspace{-0.16cm}\color{ffa725}\label{#1} \mybox[ffa725]{\color{ffa725}性质\addtocounter{C}{1} \thesection.\arabic{C}}}
%\newtheorem*{inference}{\hspace{-0.16cm}\color{1a9850}\label{#1} \mybox[1a9850]{\color{1a9850}推论\addtocounter{D}{1} \thesection.\arabic{D}}}
%\newtheorem*{method}{\hspace{-0.16cm}\color{6a3d9a}\label{#1} \mybox[6a3d9a]{\color{6a3d9a}方法\addtocounter{E}{1} \thesection.\arabic{E}}}
%\newtheorem*{example}{\hspace{-0.16cm}\color{53a9ab}\label{#1} \mybox[53a9ab]{\color{53a9ab}例题\addtocounter{F}{1} \thesection.\arabic{F}}}

%文章标题
		\title{\vspace*{-2cm}
			\Huge{空间解析几何总结}\thanks{本笔记已经开源,可以免费下载,github地址:https://github.com/Lovely-XPP/Notebook\\ \hspace*{6.36cm} gitee分流地址:https://gitee.com/sysu\_xpp/Notebook}\\
			\quad\\
\begin{center}
	\includegraphics[width=15cm]{picture/cover6.png}
	%\includegraphics[width=8cm]{cover2.png}
	%\includegraphics[width=8cm]{cover3.png}
	%\includegraphics[width=12cm]{cover4.png}
\end{center}
}
		\author{
		{  \large {易鹏}}\\
		{  \large 中山大学}\vspace*{0.5em}\\
		内部版本号:V4.62.84(正式版)
}


%调整间距(倍数)
\linespread{1.5}

%自定义页眉页脚---------------
\pagestyle{fancy}
\renewcommand{\chaptermark}[1]{\markboth{\;第\ \thechapter\ 章\quad#1\;}{}}
\renewcommand{\sectionmark}[1]{\markright{\;\thesection\ #1\;}}
\fancyhf{}
%\fancyfoot[C]{\bfseries\thepage}
\fancyhead[LO]{\small\rightmark}
\fancyhead[RE]{\small\leftmark}
\fancyhead[RO,LE]{\;\thepage\;}
\fancyfoot[RO,LE]{\footnotesize {空间解析几何}}
\fancyfoot[RE,LO]{\footnotesize Analytic Geometry of Space}
\renewcommand{\headrulewidth}{0.4pt} % 注意不用\setlength
%\renewcommand{\footrulewidth}{0pt}
\fancyheadoffset[LE,RO]{0cm}
\fancyfootoffset[LE,RO]{0cm}
% 注意不用\setlength
%\renewcommand{\footrulewidth}{0pt}

%自定义命令
\newcommand{\link}[1][]{\hyperref[#1] {\color{超链接}#1}}%超链接简化命令
\newcommand{\sj}{\vspace*{-1em}}
\newcommand{\sja}{\vspace*{-0.5em}}
\newcommand{\jg}{\vspace*{1em}}
\newcommand{\eqkg}{$\,$}
\newcommand{\kg}{\hspace*{1em}}
\newcommand{\huo}{\mbox{或}}


%文档开始
\begin{document}

%标题及目录
\pagenumbering{Roman}
\clearpage {\pagestyle{empty}}
\renewcommand{\thefootnote}{*}
\maketitle
\renewcommand{\thefootnote}{\arabic{footnote}}
\setcounter{page}{1}
\tableofcontents

%正文部分
\newpage
\setcounter{page}{1}
\pagenumbering{arabic}


%第一章

\chapter{火箭发动机推进原理}
\thispagestyle{empty}
\section{概述}
\noindent \textbf{1. 火箭发动机工作过程与能量转化}
{
	\begin{center}
		\begin{tikzpicture}[node distance=1.2cm]
			%定义流程图具体形状
			\node(O) [minimum height=0cm,draw, xshift = -9cm,,inner sep=8pt] {燃烧室};
			\node (B) [minimum height=0cm,draw, xshift = -4.25cm,node distance=3.5cm, inner sep=8pt] {喷管};
			\node (C) [minimum height=0cm,draw, xshift = 0cm,node distance=2cm, inner sep=8pt] {推力传递系统};
			
			%连接具体形状
			\draw[arrows={-Stealth}](-12cm,0cm) -- (O)  node[midway,above=0cm]{推进剂} node[midway, below = 0cm]{化学能};
			\draw[arrows={-Stealth}](O) -- (B) node[midway, above = 0cm]{高温高压燃气} node[midway, below = 0cm]{热能};
			\draw[arrows={-Stealth}](B) -- (C) node[midway, above = 0cm]{高速燃气} node[midway, below = 0cm]{动能};
			\draw[arrows={-Stealth}](C) -- +(3.4cm,0) node[midway,above=0cm]{飞行器} node[midway, below = 0cm]{动能};
		\end{tikzpicture}
		\captionof{figure}{火箭发动机工作过程与能量转化}
		\label{火箭工作过程与能量转化}
	\end{center}
}
\vspace*{-0.5em}
火箭发动机的工作过程和能量转化如图\ref{火箭工作过程与能量转化}所示,实际的误差如表\ref*{火箭实际误差}所示。

\begin{table}[!htb]
	\centering
	\setlength{\tabcolsep}{10mm}{
	\begin{tabular}{|c|c|c|c|}
		\hline
		主要过程 & 分析方法 & 实际过程效率 & 产生原因 \\
		\hline
		燃烧放热 & $Q = \dot{m} \Delta t$ & 燃烧不完全 & 液滴,不均匀等\\
		\hline
		燃气加热 & $\Delta T = UI\dot{m}c_\gamma$ & 热损失 & 壁面传热等\\
		\hline
		膨胀加速 & $\Delta E_k  = \Delta H$ &不完全膨胀 & 分离、摩擦等\\
		\hline
		反作用推进 & $F = \dot{m}I_J$ & 分离,非对称等 & \\
		\hline
	\end{tabular}
	}
	\caption{火箭发动机的实际过程与主要误差}
	\label{火箭实际误差}
\end{table}

\noindent 在分析时,做出以下假设以简化模型\vspace*{-0.5em}
\begin{itemize}
	\item 燃烧室内:完全燃烧,化学能全部转化为热能;\vspace*{-0.5em}
	\item 燃烧室内:忽略热损失,热能全部用于燃气升温;\vspace*{-0.5em}
	\item 喷管内:燃气绝热等熵流动,热能转化为动能。
\end{itemize}
\vspace*{-0.5em}

\section{推进系统的推力}
\subsection{牛顿三大定律}
\vspace*{-1em}
\theorem[牛顿三大定律]
{
	第一运动定律
	\begin{equation}
		\sum \bm{F}_i = \dfrac{\d \bm{v}}{\d t} = 0
	\end{equation}
	\hspace*{1.8em} 第二运动定律
	\begin{equation}
		\bm{F} = \dfrac{\d \bm{p}}{\d t} \qquad \bm{F} = \dfrac{\d m}{\d t}\bm{v} + m \dfrac{\d \bm{v}}{\d t}
	\end{equation}
	\hspace*{1.8em} 第三运动定律
	\begin{equation}
		\bm{F}_{12} = \bm{F}_{21}
	\end{equation}
}

其中,$\bm{F}$为力,$\bm{v}$为速度,$m$为质量,$t$为时间,$\bm{p} = m \bm{v}$为动量。
\vspace*{1em}

\subsection{推力公式}
\noindent \textbf{1. 假设条件(理想情况)}\vspace*{-0.5em}
\begin{enumerate}[\hspace*{1.5em} (1) ]
	\item 一维定常流动;\vspace*{-0.5em}
	\item 外界大气压均匀;\vspace*{-0.5em}
	\item 忽略推进剂入口造成的动量。
\end{enumerate}
\vspace*{1em}

\noindent \textbf{2. 推力公式的推导}
\begin{figure}[!htb]
	\centering
	\includegraphics[width=0.8\linewidth]{pic/推力推导.png}
	\vspace*{-1em}
	\caption{火箭喷射示意图}
	\label{推力推导}
\end{figure}

如图\ref{推力推导}所示,火箭发动机在工作时不仅在内表面会受到燃气压强的作用,其外表面还会受到环境气压的作用,即
\begin{equation}
	\bm{F} = \bm{F}_{\text{in}} + \bm{F}_{\text{out}}
\end{equation}

取$x$方向为正方向,由于火箭是沿轴向对称的,垂直于轴线方向的力相互抵消,则只需考虑喷管的出口部分,由气体受到喷管出口向前的力(和速度方向相反),设$\bm{F}_{\text{in}}$方向为$x$轴正向,则由动量定理
\begin{equation}
		-\bm{F}_{\text{in}} - p_{\text{e}} \bm{n} A_{\text{e}} = m (\bm{u}_{\text{e}} - \bm{u}_{\text{in}})
\end{equation}
即
\begin{equation}
	\bm{F}_{\text{in}} = - m (\bm{u}_{\text{e}} - \bm{u}_{\text{in}}) - p_\e \bm{n} A_\e
\end{equation}
由气压外力与$x$轴正向方向一致,则
\begin{equation}
	\bm{F}_{\text{out}} = p_{\e}
\end{equation}
\vspace*{1em}

\noindent \textbf{3. 推力公式}

\theorem[推力公式]
{
	\quad \vspace*{-1em}
	\begin{equation}
		F = \mathop{\underbrace{\dot{m}u_\e}}_{\scriptsize \mbox{\blue[动量推力]}} + \,\,\, \mathop{\underbrace{(p_\e - p_\a)A_\e}}_{\scriptsize \mbox{\blue[压力推力]}}
	\end{equation}
	其中,\vspace*{-0.5em}
	{
		\begin{enumerate}[\hspace*{1.5em}]
			\item $\bm{F}$ \quad 推力,N \vspace*{-0.5em}
			\item $\dot{m}$ \quad 流量,kg/s \vspace*{-0.5em}
			\item $\bm{u}_\e$ \quad 火箭喷气出口速度, m/s \vspace*{-0.5em}
			\item $\bm{p}_\e$ \quad 火箭喷气出口压强,kPa \vspace*{-0.5em}
			\item $\bm{p}_\a$ \quad 当前高度的大气压强,kPa \vspace*{-0.5em}
			\item $A_\e$ \quad 火箭喷气出口的截面积,$\text{m}^2$ 
		\end{enumerate}
	}
}
\vspace*{1em}

\noindent \textbf{4. 推力公式的讨论}

(1) \hspace*{0.5em}$\dot{m}u_\e$为\dy[动量推力]{DLTL},占推力总值的90\%以上,$(p_\e - p_\a)A_\a$为\dy[压力推力]{YLTL}。

(2) \hspace*{0.5em}进入推力室燃料有初始速度?固体发动机和液体发动机。

(3) \hspace*{0.5em}火箭发动机的推力与飞行器的飞行速度无关,与高度(大气压强)相关。

(4) \hspace*{0.5em}实际飞行中,有外阻力? \quad \blue[有]

(5) \hspace*{0.5em}排气速度不均匀?非沿轴向?\quad 

(6) \hspace*{0.5em}是杏需要假设:完全气体假设?绝热等熵?无热损失和摩擦损失?\quad \blue[不需要]

\vspace*{1em}

\noindent \textbf{5. 推力相关概念}

\defination[特征推力]
{
	\dy[特征推力]{TZTL}(\dy[额定推力]{EDDL}、\dy[发动机设计状态推力]{FDJSJZTTL}):当$p_\e = p_\a$时的发动机推力。对应的高度为火箭发动机的设计高度。即$F^0 = \dot{m}u_\e$
}

\defination[真空推力]
{
	发动机在真空环境下工作时的推力,$P_\a = 0$,即$F_\text{V} = \dot{m}u_\e + P_\e A_\e$
}

\defination[地面推力(海平面推力)]
{
	\quad \vspace*{-1em}
	\begin{equation}
		F_0 = \dot{m}u_\e + A_\e (p_\e - p_{\a 0})
	\end{equation}
	 其中,$p_{\a 0}$为地面的大气压强。
}

\section{喷气速度与流量特性}
\subsection{喷气速度}

\noindent \textbf{1. 基本假设}

(1) \hspace*{0.5em}燃烧室内的燃气参数$T,P_\text{c}, \rho$处处相等,忽略燃烧室速度。

(2) \hspace*{0.5em}喷管中的流动是一维定常、等熵流动,且忽略燃气对喷管壁的传热和摩擦。

(3) \hspace*{0.5em}燃气是定压比热为常数的理想气体(量热完全气体)。

(4) \hspace*{0.5em}燃烧室内化学平衡,喷管内组分不变(冻结流)。
\vspace*{1em}

\noindent \textbf{2. 公式推导}

由假设可得燃气流动的能量守恒方程
\begin{equation}
	H + \dfrac{u^2}{2} = H_0
\end{equation}

在截面处,有
\begin{align*}
	&H_\c + \dfrac{u_\c^2}{2} = H_\e + \dfrac{u_\e^2}{2}\\
	\Rightarrow \quad& H_0 = H_\e + \dfrac{u_\e^2}{2}\\
	\Rightarrow \quad& u_\e = \sqrt{2(H_0 - H_\e)}
\end{align*}

将$H_\e = c_pT_\e, \quad H_0 =c_p T_\f$代入,得
\begin{equation}
	u_\e = \sqrt{2c_p(T_\f - T_\e)} = \sqrt{2c_p T_\f \left(1 - \dfrac{T_\e}{T_\f}\right)}
\end{equation}
对于绝热等熵流动,有
\begin{equation*}
	\dfrac{T_\e}{T_\f} = \left(\dfrac{p_\e}{p_\c}\right)^{\textstyle \frac{k-1}{k}}, \qquad c_p = \dfrac{k}{k-1}\dfrac{R_0}{MW}
\end{equation*}
则可以得到喷气速度方程。

\theorem[喷气速度方程]
{
	\vspace*{-1em}
	\begin{equation}
		u_e = \sqrt{\dfrac{2k}{k-1} \dfrac{R_0}{MW}T_\f \left[1 - \left( \dfrac{p_\e}{p_\c} \right)^{\textstyle \frac{k-1}{k}}\right]}
	\end{equation}
	其中,\vspace*{-0.5em}
	{
		\begin{enumerate}[\hspace*{1.5em}]
			\item $R_0$ \quad 通用气体常数,N \vspace*{-0.5em}
			\item $M$ \quad 燃气平均相对分子质量,g / mol \vspace*{-0.5em}
			\item $k$ \quad 比热比 \vspace*{-0.5em}
			\item $T_\f$ \quad 推进剂绝热燃烧温度,K
		\end{enumerate}
	}
}

\noindent \textbf{2. $u_\e$的影响因素}
\begin{enumerate}[\hspace*{1em}(1) \hspace*{0.5em}]
	\item $T_\f$\quad $t_\f \uparrow$,可转换成动能的热能增加$\rightarrow u_\e \uparrow$
	\item $MW$ \quad $MW \downarrow \, \rightarrow \, u_\e \uparrow$
	\item $k$ \quad $k \uparrow \, \rightarrow\, \sqrt{\dfrac{2k}{k-1}}\, \bigg \downarrow$而$\displaystyle \left[1 - \left(\dfrac{p_\e}{p_\c}\right)^{\textstyle \frac{k-1}{k}}\right] \Bigg \uparrow\, \, \rightarrow$综合考虑,$u_\e$随$k$的增大而略有减小
	\item $\dfrac{p_\e}{p_\c}$ \quad 在$MW,k,T_\f$一定的情况下,$u_\e$随$\dfrac{p_\e}{p_\c}$的减小而增大。其物理意义为燃气在喷管中的膨胀程度,膨胀压强比$\dfrac{p_\e}{p_\c}$越小,燃气膨胀得越充分,有更多的热能转换为动能,喷气速度越高。若$p_\e = 0$,则说明此时燃气的全部热能都转换为动能,喷气速度达到极限值,称为\dy[极限喷气速度]{JXPQSD}。
\end{enumerate}

	\defination[极限喷气速度]
	{
		\vspace*{-1em}
		\begin{equation}
			u_{\text{L}} = \sqrt{2h_\c} = \sqrt{\dfrac{2k}{k-1} R T_\f}
		\end{equation}
	}

\section{飞行器飞行性能指标及分析}

\subsection{飞行器加速度特性}
简化飞行过程,垂直升空的起飞加速度为
\begin{equation}
	a = \dfrac{F}{m_0} - g_0 \quad \Rightarrow \quad \dfrac{a}{g_0} = \dfrac{F}{mg_0} - 1
\end{equation}
\dy[起飞推重比]{QFTZB}定义为$\dfrac{F}{m_0g_0}$,大飞行器约为1.2$\, \sim \,$2.2,小导弹约为$50 \, \sim \, 100$.

\subsection{小结}
航天飞行器或火箭飞行器的主要性能参数:最大速度$V_{\max}$(单级火箭),速度增量$\Delta V$(多级火箭),有效载荷,最大射程,最大飞行高度等
\vspace*{0.5em}

\noindent \textbf{1. 性能最佳}
\begin{enumerate}[\hspace*{1.5em} (1)  ]
	\vspace*{-0.5em}
	\item 有效载荷确定时,缩短完成任务的时间\vspace*{-0.5em}
\end{enumerate}

\vspace*{0.5em}
\noindent \textbf{2. 改进推进系统性能提高飞行器性能途径}
\begin{enumerate}[\hspace*{1.5em} (1)  ]
	\vspace*{-0.5em}
	\item 有效出口速度或比冲:直接影响飞行性能。方法:高能推经剂,高室压及大膨胀比喷管(高空上面级)\vspace*{-0.5em}
	\item 质量数增大:对数影响效果。方法:减小结构质量,增大起飞质量,采用多级推进。\vspace*{-0.5em}
	\item 加大起飞推力,缩短动力飞行时间,减少重力损失\vspace*{-0.5em}
	\item 减小阻力\vspace*{-0.5em}
	\item 最优喷管设计\vspace*{-0.5em}
	\item 发射初速度\vspace*{-0.5em}
	\item 有效喷出速度接近飞行器主飞行段的飞行速度
\end{enumerate}

\vspace*{0.5em}
\noindent \textbf{3. 任务速度}

\defination[任务速度]
{
	\dy[任务速度]{RWSU}:
}


\section{本章总结}
\begin{figure}[!htb]
	\centering
	\begin{tikzpicture}
		\node (A) [draw, inner sep = 5pt]{发动机推力};
		\node (A1) [draw, inner sep = 5pt, right of = A, node distance = 9cm]{\makecell[c]{$F=\dot{m}u_e+A_e(P_e - P_a)$:牛顿定律;\\假设条件,推力公式的推导*及其应用}};
		\node (B) [draw, inner sep = 5pt, below of = A, node distance = 2.4cm]{排气速度};
		\node (B1) [draw, inner sep = 5pt, right of = B, node distance = 9cm]{$\displaystyle u_e = \sqrt{\dfrac{2k}{k-1}\rho_e  P_e \left[\left(\dfrac{P}{P_e}\right)^{\textstyle \frac{2}{k}} - \left(\dfrac{P}{P_e}\right)^{\textstyle \frac{k+1}{k}} \right]}$};
		\node (C) [draw, inner sep = 5pt, below of = B, node distance = 2.4cm]{流量特性};
		\node (C1) [draw, inner sep = 5pt, right of = C, node distance = 9cm]{$\dot{m} = A \sqrt{\dfrac{2k}{k-1} \rho_\e P_\e \left[ \left(\dfrac{P}{P_\text{e}}\right)^{\textstyle\frac{2}{k}} - \dfrac{P}{P_\e}^{\textstyle\frac{k+1}{k}}\right]} \Rightarrow \dot{m} = \dfrac{\Gamma}{\sqrt{RT_\f}}P_\e A_\text{t} = C_\text{D}P_\text{c}A_\text{t} = \dfrac{P_\text{c}A_\text{t}}{c^*}$};
		\node (D) [draw, inner sep = 5pt, below of = C, node distance = 2cm]{推力系数};
		\node (D1) [draw, inner sep = 5pt, right of = D, node distance = 9cm]{$F = C_\text{F}P_\text{c}A_\text{t}$,其物理意义及影响因素};
		\node (E) [draw, inner sep = 5pt, below of = D, node distance = 2.4cm]{总冲和比冲};
		\node (E1) [draw, inner sep = 5pt, right of = E, node distance = 9cm]{$\displaystyle I = \int_0^{t_a} F\, \d t, \quad I_s = \dfrac{I}{M_P} = \dfrac{\displaystyle \int_0^{t_a} F \, \d t }{\displaystyle\int_0^{t_a} \dot{m}\, \d t}$,其物理意义、相关概念及影响因素};
		\node (F) [draw, inner sep = 5pt, below of = E, node distance = 2.4cm]{\makecell[c]{推进剂的混合比\\当量比、余氧系数}};
		\node (F1) [draw, inner sep = 5pt, right of = F, node distance = 9cm]{组元确定情况下,氧化剂和燃料之间的比值,决定燃烧室温度。};
		\node (G) [draw, inner sep = 5pt, below of = F, node distance = 2cm]{飞行器运动};
		\node (G1) [draw, inner sep = 5pt, right of = G, node distance = 9cm]{$F = M\dfrac{\d V}{\d t}\quad \red[V_{\max} - V_0 = \ln \mu] \quad M_0 = M_P + M_e + M_s$};
		\node (H) [draw, inner sep = 5pt, below of = G, node distance = 2cm]{\makecell[c]{飞行器性能\\与发动机性能关系}};
		\node (H1) [draw, inner sep = 5pt, right of = H, node distance = 9cm]{$\displaystyle \Delta u_f = \sum_{i = 1}^{n} I_{s_i} \ln \mu_i$};
		
		\draw[arrows={-Stealth[scale=0.8]}] (A) -- (B);
		\draw[arrows={-Stealth[scale=0.8]}] (B) -- (C);
		\draw[arrows={-Stealth[scale=0.8]}] (C) -- (D);
		\draw[arrows={-Stealth[scale=0.8]}] (D) -- (E);
		\draw[arrows={-Stealth[scale=0.8]}] (E) -- (F);
		\draw[arrows={-Stealth[scale=0.8]}] (F) -- (G);
		\draw[arrows={-Stealth[scale=0.8]}] (G) -- (H);
		\draw (A) -- (A1);
		\draw (B) -- (B1);
		\draw (C) -- (C1);
		\draw (D) -- (D1);
		\draw (E) -- (E1);
		\draw (F) -- (F1);
		\draw (G) -- (G1);
		\draw (H) -- (H1);
	\end{tikzpicture}
	\caption{第1章总结图}
	\label{第1章总结图}
\end{figure}



































%第二章

\chapter{飞行动力学的基础知识}
\thispagestyle{empty}
\section{地球的运动及形状}
\subsection{地球的运动}
\vspace*{-1em}
\defination[地球运动]
{\dy[地球运动]{DQYD}分为\dy[质心运动]{ZXYD}(公转)和\dy[绕心运动]{RXYD}(自转)\vspace*{-0.5em}
{
\begin{itemize}
	\item \dy[公转]{GZ}:以近圆轨道绕太阳公转,周期为1年。\vspace*{-0.5em}
	\item \dy[自转]{ZZ}:地球自转轴成为\dy[地轴]{DZ},地球绕地轴自西向东匀速转动。
\end{itemize}
}
}
\noindent 地球运动的基本参数:\vspace*{-0.5em}
\begin{itemize}
	\item 地球公转周期:$T = 365.25636$个平日\vspace*{-0.5em}
	\item 地球自转周期:$t = 86164.099\,$s $=$ 23$\,$h$\,$56$\,$m$\,$4.099$\,$s\vspace*{-0.5em}
	\item 地球自转角速度:$\omega_e = \dfrac{2 \pi}{86164.1} = 7.292115\times 10^{-5} \text{rad/s}$
\end{itemize}
\vspace*{-0.5em}

\theorem[地球运动的假设]
{在导弹飞行时间内(飞行时间短),可认为\textcolor{red}{地轴在惯性空间指向不变},且\textcolor{red}{地球作匀速直线运动}。但实际上地轴的指向是变化的,存在极移(物质变化)、进动(太阳引力)和章动(月球引力),地球本身也存在加速度。}

其中,进动是指在太阳引力作用下,地轴会绕一个轴作周期约为27500年的圆周运动,类似于不平衡的陀螺,如图所示。而章动是指在月球引力的作用下,地轴并不是做完美的圆周运动,会上下浮动,浮动的周期约为18.6年,如图所示。



\subsection{地球的形状}
\vspace*{-1em}
\defination[地球形状]
{实际应用中采用简单形状描述地球:\vspace*{-0.5em}
{
\begin{itemize}
	\item \red[均质圆球]:$R=6371004$m\vspace*{-0.5em}
	\item \red[总地球椭球体]:$a_e = 6378149\text{m},\,\, b_e = 6356775\text{m}$,地球扁率(离心率)$\alpha_e = \dfrac{(a_e - b_e)}{a_e} = \dfrac{1}{298.257}$
\end{itemize}
}
}


\section{地球大气}
\subsection{地球大气分层}
\vspace*{-1em}
\defination[地球大气分层]
{\dy[地球大气分层]{DQDQFC}是按大气温度分层:\vspace*{-0.5em}
{
\begin{itemize}
	\item \dy[对流层]{DLC}:$0\, \sim \, 18\, \text{km} / 8 \, \text{km}$,75\%大气质量,95\%水汽;\vspace*{-0.5em}
	\item \dy[平流层]{PLC}:$\sim 50 \, \text{km}$,同温层$+$臭氧层,温度升高,大气密度和压强降低,只有地表的0.08\%.\vspace*{-0.5em}
	\item \dy[中间层]{ZJC}:$50\, \text{km} \, \sim \, 90\,\text{km}$,温度降低\vspace*{-0.5em}
	\item \dy[热成层]{RCC}:$90 \, \text{km} \, \sim \, 500\, \text{km}$,温度升高\vspace*{-0.5em}
	\item \dy[外逸层]{WYC}:$>500\, \text{km}$.\vspace*{-0.5em}
\end{itemize}
}
对于运载火箭,一般只考虑90km以下的大气影响。
}

\noindent 大气的物理性质分布如下:

\begin{enumerate}
	\item \textbf{温度分布}\\
	\hspace*{2em}温度随高度的变化曲线在$0 \, \sim \, 80 \, \text{km}$内可以由一系统的折线表示:
	\begin{equation}
		T(h) = T_0 + Gh
	\end{equation}
	
	对于不同的层,相应的参数$G$取值不同。
	
	\item \textbf{压强分布}\\
	\hspace*{2em}大气的实际压强与气温一样变化一场复杂,为了得到一般意义的标准分布,常采用“大气垂直平衡”假设,即认为大气在铅锤方向是静止的,处于力的平衡状态。由$p=Rg_0\rho T$得:
	\begin{equation}
		p(h) = p_0 \e^{\textstyle - \frac{1}{R}\int_0^h \frac{\d h}{T}}
	\end{equation}
	\proof 由“大气垂直平衡”假设,可以得到
	\begin{equation*}
		(p + \d p)\d S + \rho g_0 \d S \d h = p \d S
	\end{equation*}
	化简得到
	\begin{equation*}
		\d p + \rho g \d h = 0
	\end{equation*}
	代入$p=Rg_0\rho T$可得
	\begin{equation*}
		\dfrac{\d p}{p} = - \dfrac{g}{Rg_0T}\, \d h
	\end{equation*}
	积分可得
	\begin{equation}
		\frac{\ln p}{\ln p_0} = \int_{h_0}^h - \frac{g}{Rg_0T}\, \d h \quad \Rightarrow \quad \frac{p}{p_0} = \e^{\textstyle - \frac{1}{R}\int_{h_0}^h \frac{g}{g_0 T}\, \d h}
	\end{equation}
	\item \textbf{密度分布}\\
	\hspace*{2em}由气体状态方程,已知温度$T$和气压$p$,可得
	\begin{equation}
		\dfrac{\rho}{\rho_0} = \frac{pT_0}{p_0 T} = \frac{T_0}{T} \e ^{\textstyle -\frac{1}{R}\int_0^H \frac{\d H}{T}},\, H = \dfrac{1}{g_0}\int_0^h g \, \d h
	\end{equation}
	其中,$H$为\dy[地势高度]{DSGD},相当于具有同等势能的均匀重力场中的高度,其总小于几何高度$h$,但在高度不打时二者差别较小。若认为在某一高度范围内为等温过程,则:
	\begin{equation}
		\dfrac{\rho_2}{\rho_1} = \e^{\textstyle - \frac{H_2 - H_1}{H_{M_1}}}, \, H_{M_1} = RT_1
	\end{equation}
	其中,$H_{M_1}$称为\dy[基准高]{JZG}或\dy[标高]{BG}。
	
	\hspace*{2em} 如果假设在$0 \, \sim \, 80\, \text{km}$内为恒温过程,则有
	\begin{equation}
		\frac{p}{p_0} = \dfrac{\rho}{\rho_0} = \e^{-\beta h}, \quad \beta = \frac{1}{H_{\text{MCP}}}=\frac{1}{7.11\text{km}}
	\end{equation}
	这个模型称为\dy[指数大气模型]{ZSDQMX}。
\end{enumerate}

\subsection{标准大气}
导弹飞行状态随与随高度变化的大气参数有密切关系(压强、密度、温度及音速等)。

\section{坐标系间的方向余弦阵及矢量导数的关系}
由于不同坐标系对同一物理量的描述形式或者坐标投影不同,为了在统一坐标系中描述飞行器的运动,存在不同物理量到基准坐标系的转换需求。

\defination[坐标转换]
{
	同一矢量在不同坐标系下的坐标不同,将矢量$S_a$坐标系中的坐标转换到$S_b$坐标系称为\dy[坐标系转换]{ZBXZH}。
}

\subsection{坐标系之间的方向余弦阵}
\vspace*{-1em}
\defination[坐标系]
{
	\dy[笛卡尔坐标系]{DKEZBX}:由原点及过原点的两(三)条具有方向的坐标轴组成,坐标轴上的度量单位通常相等。\\
	\hspace*{2em} \dy[球坐标系]{QZBX}:球坐标系由原点、方位角、仰角和距离构成。\\
	\hspace*{2em} \dy[极坐标系]{JZBX}:极坐标系由极点、极径及极角构成。
}

考虑两个直角坐标系:$P:O_p-\bm{x}_p\bm{y}_p\bm{z}_p,\quad Q:O_q-\bm{x}_q\bm{y}_q\bm{z}_q$,定义$P_Q$为$Q$系中单位矢量$E_q$变换到$P$系中单位矢量$E_p$的转换矩阵,由
\begin{equation}
	E_p = P_QE_q, \qquad E_p =
	\begin{bmatrix}
		\bm{x}_p^0 & \bm{y}_p^0 & \bm{z}_p^0
	\end{bmatrix}^{\text{T}}
\qquad 
	E_q = 
	\begin{bmatrix}
		\bm{x}_q^0 & \bm{y}_q^0 & \bm{z}_q^0
	\end{bmatrix}^{\text{T}}
\end{equation}
由于
\begin{equation*}
	E_q \cdot E_q^{\text{T}} = 
	\begin{bmatrix}
		\bm{x}_q^0\\
		\bm{y}_q^0\\
		\bm{z}_q^0
	\end{bmatrix}
	\begin{bmatrix}
	\bm{x}_q^0 & \bm{y}_q^0 & \bm{z}_q^0
	\end{bmatrix}
	=
	\begin{bmatrix}
		\bm{x}_q^0 \cdot \bm{x}_q^0 & \bm{x}_q^0 \cdot \bm{y}_q^0 & \bm{x}_q^0 \cdot \bm{z}_q^0 \\ 
		\bm{y}_q^0 \cdot \bm{x}_q^0 & \bm{y}_q^0 \cdot \bm{y}_q^0 & \bm{y}_q^0 \cdot \bm{z}_q^0 \\ 
		\bm{z}_q^0 \cdot \bm{x}_q^0 & \bm{z}_q^0 \cdot \bm{y}_q^0 & \bm{z}_q^0 \cdot \bm{z}_q^0 
	\end{bmatrix}
	=
	\begin{bmatrix}
		1 & 0 & 0 \\
		0 & 1 & 0 \\
		0 & 0 & 1
	\end{bmatrix} = E
\end{equation*}
那么
\begin{equation}
	P_Q = E_p \cdot E_q^{\text{T}} = 
	\begin{bmatrix}
		\bm{x}_p^0 \cdot \bm{x}_q^0 & \bm{x}_p^0 \cdot \bm{y}_q^0 & \bm{x}_p^0 \cdot \bm{z}_q^0 \\ 
		\bm{y}_p^0 \cdot \bm{x}_q^0 & \bm{y}_p^0 \cdot \bm{y}_q^0 & \bm{y}_p^0 \cdot \bm{z}_q^0 \\ 
		\bm{z}_p^0 \cdot \bm{x}_q^0 & \bm{z}_p^0 \cdot \bm{y}_q^0 & \bm{z}_p^0 \cdot \bm{z}_q^0 
	\end{bmatrix}
	=
	\begin{bmatrix}
		\cos(\bm{x}_p, \bm{x}_q) & \cos(\bm{x}_p, \bm{y}_q) & \cos(\bm{x}_p, \bm{z}_q)\\
		\cos(\bm{y}_p, \bm{x}_q) & \cos(\bm{y}_p, \bm{y}_q) & \cos(\bm{y}_p, \bm{z}_q)\\
		\cos(\bm{z}_p, \bm{x}_q) & \cos(\bm{z}_p, \bm{y}_q) & \cos(\bm{z}_p, \bm{z}_q)
	\end{bmatrix}
	\triangleq
	\begin{bmatrix}
		a_{ij}
	\end{bmatrix}
	\quad i,j=1,2,3
	\label{方向余弦阵}
\end{equation}

公式\eqref{方向余弦阵}称为\dy[方向余弦阵]{FXYXZ},同理可以得到
\begin{equation}
	Q_p = E_q \cdot E_p^{\text{T}}
	\begin{bmatrix}
		\bm{x}_q^0 \cdot \bm{x}_p^0 & \bm{x}_q^0 \cdot \bm{y}_p^0 & \bm{x}_q^0 \cdot \bm{z}_p^0 \\ 
		\bm{y}_q^0 \cdot \bm{x}_p^0 & \bm{y}_q^0 \cdot \bm{y}_p^0 & \bm{y}_q^0 \cdot \bm{z}_p^0 \\ 
		\bm{z}_q^0 \cdot \bm{x}_p^0 & \bm{z}_q^0 \cdot \bm{y}_p^0 & \bm{z}_q^0 \cdot \bm{z}_p^0 
	\end{bmatrix}
	=
	\begin{bmatrix}
		\cos(\bm{x}_q, \bm{x}_p) & \cos(\bm{x}_q, \bm{y}_p) & \cos(\bm{x}_q, \bm{z}_p)\\
		\cos(\bm{y}_q, \bm{x}_p) & \cos(\bm{y}_q, \bm{y}_p) & \cos(\bm{y}_q, \bm{z}_p)\\
		\cos(\bm{z}_q, \bm{x}_p) & \cos(\bm{z}_q, \bm{y}_p) & \cos(\bm{z}_q, \bm{z}_p)
	\end{bmatrix}
\end{equation}
又
\begin{equation*}
	P_q^{-1} = Q_p = E_q \cdot E_p^{\text{T}} = \left(E_p \cdot E_q^{\text{T}} \right)^{\text{T}} = P_q^{\text{T}}
\end{equation*}
这说明:\red[方向余弦阵是正交矩阵],那么方向余弦阵只有\blue[三个独立变量]。

\defination[初等转换矩阵]
{
	当两个坐标系之间存在平行轴的时候,此时的方向余弦矩阵称为\dy[初等转换矩阵]{CDZHJZ}。分别绕$x,y,z$轴旋转的初等转换矩阵为
	\begin{equation}
		M_x(\theta) = 
		\begin{bmatrix}
			1 & 0 & 0\\
			0 & \cos \theta & \sin \theta \\
			0 & -\sin \theta & \cos \theta
		\end{bmatrix}
		\qquad
		M_y(\theta) =
		\begin{bmatrix}
			\cos \theta & 0 & -\sin\theta\\
			0 & 1 & 0\\
			\sin \theta & 0 & \cos \theta
		\end{bmatrix}
		\qquad
		M_z (\theta)= 
		\begin{bmatrix}
			\cos \theta & \sin \theta & 0 \\
			-\sin \theta & \cos \theta & 0 \\
			0 & 0 & 1
		\end{bmatrix}
	\end{equation}
}

\theorem[转换矩阵的传递性]
{
	对于直角坐标系$P,Q,S$,它们相互之间的转换矩阵为$P_Q,S_Q,S_P$,则\vspace*{-1em}
	\begin{equation}
		S_Q = S_P \cdot P_Q, \quad P_Q = P_S \cdot S_Q, \quad P_S = P_Q\cdot Q_S
	\end{equation}
}

\subsection{坐标系转换阵的欧拉角表示方法}
\vspace*{-1em}
\defination[转换矩阵的欧拉角表示]
{
	将坐标系视作刚体,则经过三次旋转后可以与另一个坐标系重合,因此可以用这三个旋转角(\dy[欧拉角]{OLJ})作为独立变量,来描述方向余弦阵。
}

\begin{figure}[!htb]
	\centering
	\includegraphics[width=0.3\linewidth]{pic/欧拉角.jpg}
	\caption{坐标系旋转转换}
	\label{欧拉角}
\end{figure}

例如,如图\ref{欧拉角}所示,$o-\bm{x}_q\bm{y}_q\bm{z}_q$分别绕$z,y,x$轴旋转三次,得到
\begin{equation*}
	\begin{split}
		o-\bm{x}_q\bm{y}_q\bm{z}_q \, \xrightarrow{\quad \textstyle M_z[\xi] \quad } o - x_1 y_1 \bm{z}_q \, \xrightarrow{\quad \textstyle M_y[\eta] \quad } o - \bm{x}_p y_1 \bm{z}_1\ \, \xrightarrow{\quad \textstyle M_x[\zeta] \quad } \,o - \bm{x}_p \bm{y}_p \bm{z}_p
	\end{split}
\end{equation*}
即
\begin{equation}
	P_Q = M_x[\xi] \cdot  M_y[\eta]\cdot  M_z[\zeta]
\end{equation}
进一步代入,得
\begin{equation}
	P_Q = 
	\begin{bmatrix}
		\cos \xi \cos \eta & \sin \xi \cos \eta & - \sin \eta \\
		\cos \xi \sin \eta \sin \zeta - \sin \xi \cos \zeta & \sin \xi \sin \eta \sin \zeta + \cos \xi \cos \zeta & \cos \eta \sin \zeta \\
		\cos \xi \sin \eta \cos \zeta + \sin \xi \sin \zeta & \sin \zeta \sin \eta \cos \zeta - \cos \xi \sin \zeta & \cos \eta \cos \zeta
	\end{bmatrix}
\end{equation}

\subsection{坐标间矢量导数的关系}
\vspace*{-1em}
\defination[矢量导数]
{
	\dy[矢量导数]{SLDS}:同一矢量在不同坐标系有不同的投影,其导数的数值不同。
}

设坐标系$O:o-xyz$相对于坐标系$P:o_p-x_py_pz_p$有角速率$\omega.$ 矢量$\bm{a}$在坐标系$O$的投影为
\begin{equation}
	\bm{a} = a_x \bm{x}^0 + a_y \bm{y}^0 + a_z \bm{z}^0
\end{equation}
取微分,得
\begin{equation}
	\dfrac{\d \bm{a}}{\d t} = \dfrac{\d a_x}{\d t}\bm{x}^0 + \dfrac{\d a_y}{\d t} \bm{y}^0 + \dfrac{\d a_z}{\d t} \bm{z}^0 + a_x \dfrac{\bm{x}^0}{\d t} + \dfrac{}{分母}
\end{equation}

\section{常用坐标系及其相互转换}
\subsection{常用坐标系}
常用的坐标系分类
\begin{itemize}
	\item 取地心为原点:地心惯性坐标系,地心固连坐标系\vspace*{-0.5em}
	\item 取发射点为原点:发射坐标系,发射惯性坐标系\vspace*{-0.5em}
	\item 取对象质心为原点:体坐标系,速度坐标系,半速度坐标系
\end{itemize}

\begin{enumerate}
	\item \dy[地心惯性坐标系$I$]{DXGXZBX}:$O_E - X_I Y_I Z_I$(静系)
	\vspace*{1em}
	
	\begin{minipage}{0.6\linewidth}
		\centering
		\setlength{\tabcolsep}{12mm}{
		\begin{tabular}{cl}
			\hline
			原点 & 地心$O_E$\\
			\hline
			$X$轴 & $X_I$:平春分点\\
			\hline
			$Y$轴 & $Y_I$:右手法则\\
			\hline
			$Z$轴 & $Z_I$:地球自转轴\\
			\hline
		\end{tabular}
	}
	\end{minipage}

	\vspace*{1.5em}
	\item \dy[地心坐标系$E$]{DXZBX}:$O_E - X_E Y_E Z_E$(动系)
	\vspace*{1em}
	
	\begin{minipage}{0.6\linewidth}
		\centering
		\setlength{\tabcolsep}{12mm}{
		\begin{tabular}{cl}
			\hline
			原点 & 地心$O_E$\\
			\hline
			$X$轴 & $X_E$:给定子午线\\
			\hline
			$Y$轴 & $Y_E$:右手法则\\
			\hline
			$Z$轴 & $Z_E$:地球自转轴\\
			\hline
		\end{tabular}
	}
	\end{minipage}
	
	\vspace*{1.5em}
	\item \dy[发射坐标系$G$]{FSZBX}:$O - xyz$(动系)
	\vspace*{1em}
	
	\begin{minipage}{0.6\linewidth}
		\centering
		\setlength{\tabcolsep}{8mm}{
			\begin{tabular}{cl}
				\hline
				原点 & 发射点$O$\\
				\hline
				$X$轴 & $x$:发射水平面内指向瞄准方向\\
				\hline
				$Y$轴 & $y$:发射水平面指向上方\\
				\hline
				$Z$轴 & $z$:右手法则\\
				\hline
			\end{tabular}
		}
	\end{minipage}
	\vspace*{0.5em}
	
	对于球模型:$\varphi_0$:地心纬度,$\alpha_0$:发射方位角。\\
	对于椭球模型:$\varphi_0$:地心纬度,$\alpha_0$:发射方位角。
	
	\item \dy[发射惯性坐标系$A$]{FSGXZBX}:$O_A - x_A y_A z_A$(静系)
	\vspace*{1em}
	
	\begin{minipage}{0.6\linewidth}
		\centering
		\setlength{\tabcolsep}{8mm}{
			\begin{tabular}{cl}
				\hline
				原点 & 发射点$O_A$,起飞瞬间与发射点$O$重合\\
				\hline
				$X$轴 & $x_A$:起飞瞬间的发射水平面内指向瞄准方向\\
				\hline
				$Y$轴 & $y_A$:起飞瞬间的发射水平面指向上方\\
				\hline
				$Z$轴 & $z_A$:右手法则\\
				\hline
			\end{tabular}
		}
	\end{minipage}
	
	\vspace*{1.5em}
	\item \dy[平移坐标系$A$]{PYZBX}:$o_T - x_T y_T z_T$(动系)
	\vspace*{1em}
	
	\begin{minipage}{0.6\linewidth}
		\centering
		\setlength{\tabcolsep}{8mm}{
			\begin{tabular}{cl}
				\hline
				原点 & 发射点$O_A$,起飞瞬间与发射点$O$重合\\
				\hline
				$X$轴 & $x_A$:起飞瞬间的发射水平面内指向瞄准方向\\
				\hline
				$Y$轴 & $y_A$:起飞瞬间的发射水平面指向上方\\
				\hline
				$Z$轴 & $z_A$:右手法则\\
				\hline
			\end{tabular}
		}
	\end{minipage}
	
	\vspace*{1.5em}
	\item \dy[弹体坐标系$B$]{DTZBX}:$o_1 - x_1 y_1 z_1$(动系)
	\vspace*{1em}
	
	\begin{minipage}{0.6\linewidth}
		\centering
		\setlength{\tabcolsep}{8mm}{
			\begin{tabular}{cl}
				\hline
				原点 & 弹体质心$O_1$\\
				\hline
				$X$轴 & $x_1$:沿弹体对称轴指向头部\\
				\hline
				$Y$轴 & $y_1$:位于主对称面内,垂直于$X$轴\\
				\hline
				$Z$轴 & $z_1$:右手法则,顺着发射方向看向右为正\\
				\hline
			\end{tabular}
		}
	\end{minipage}
	
	\vspace*{1.5em}
	\item \dy[速度坐标系$V$]{SDZBX}:$o_1 - x_v y_v z_v$(动系)
	\vspace*{1em}
	
	\begin{minipage}{0.6\linewidth}
		\centering
		\setlength{\tabcolsep}{8mm}{
			\begin{tabular}{cl}
				\hline
				原点 & 弹体质心$O_1$\\
				\hline
				$X$轴 & $x_v$:弹体的速度方向\\
				\hline
				$Y$轴 & $y_v$:位于主对称面内,垂直于$X$轴\\
				\hline
				$Z$轴 & $z_v$:右手法则\\
				\hline
			\end{tabular}
		}
	\end{minipage}

	\vspace*{1.5em}
	\item \dy[半速度坐标系$H$]{BSDZBX}:$o_1 - x_h y_h z_h$(动系)
	\vspace*{1em}
	
	\begin{minipage}{0.6\linewidth}
		\centering
		\setlength{\tabcolsep}{5mm}{
			\begin{tabular}{cl}
				\hline
				原点 & 弹体质心$O_1$\\
				\hline
				$X$轴 & $x_h$:弹体的速度方向\\
				\hline
				$Y$轴 & $y_h$:包含速度矢量的铅锤面内垂直于$x_h$,向上为正\\
				\hline
				$Z$轴 & $z_h$:右手法则\\
				\hline
			\end{tabular}
		}
	\end{minipage}
\end{enumerate}

\subsection{各坐标系间的转换关系}
\begin{enumerate}
	\item $I \to E$:$Z$轴重合,$X$轴处于赤道面内相差角$\varOmega_G = \omega_e \cdot t$(时角),则转换矩阵为
	\begin{equation}
		E_I = M_z [\varOmega_G]
	\end{equation}
	
	\item $E \to G$:设地球为圆球,发射点可用经纬度$(\lambda_0, \varphi_0)$来描述,即
	\begin{equation}
		G_E = M_y\big[-(90\degree + \alpha_0)\big]\cdot M_x[\varphi_0] \cdot M_z \big[-(90 \degree - \lambda_0)\big]
	\end{equation}

	\item $G \to B$:设$G \to B$转序为$321$,旋转角为$\varphi, \psi ,\gamma$,则
	\begin{equation}
		B_G = M_x[\gamma]\cdot M_y[\psi] \cdot M_z[\varphi]
	\end{equation}
	\hspace*{1em}\defination[新的欧拉角{\RMN[1]}]
	{
		\dy[俯仰角$\varphi$]{FYJ}:轴$ox_1$在发射面$xoy$上的投影与$x$的夹角,投影在$x$的上方为正。\\
		\hspace*{2.2em}\dy[偏航角$\psi$]{PHJ}:轴$ox_1$与发射面$xoy$的夹角,$ox_1$在发射面左边为正。\\
		\hspace*{2.2em}\dy[滚动角$\gamma$]{GDJ}:旋转角速度矢量与$ox_1$轴方向一致时为正。
	}

	\item $G \to V$:设$G \to V$转序为$321$,旋转角为$\theta, \sigma ,\nu$,则
	\begin{equation}
		V_G = M_x[\nu]\cdot M_y[\sigma] \cdot M_z[\theta]
	\end{equation}
	\hspace*{1em}\defination[新的欧拉角{\RMN[2]}]
	{
		\dy[速度倾角$\theta$]{SDQJ}:轴$ox_v$在发射面$xoy$上的投影与$x$的夹角,投影在$x$的上方为正。\\
		\hspace*{2.2em}\dy[航迹偏航角$\sigma$]{HJPHJ}:轴$ox_v$与发射面$xoy$的夹角,$ox_v$在发射面左边为正。\\
		\hspace*{2.2em}\dy[倾侧角$\nu$]{QCJ}:旋转角速度矢量与$ox_v$轴方向一致时为正。
	}

	
	\item $V \to B$:由于速度系$y_v$轴位于主对称面内,因此$V \to B$只有两个欧拉角,设为$\alpha,\beta$,设定$V \to B$的转序为23,则
	\begin{equation}
		B_V = M_z[\alpha] \cdot M_y [\beta]
	\end{equation}
	
	\hspace*{1em}\defination[新的欧拉角{\RMN[3]}]
	{
		\dy[测滑角$\beta$]{CHJ}:速度轴$x_v$与弹体主对称面的夹角,右方为正。\\
		\hspace*{2.2em}\dy[攻角$\alpha$]{GJ}:速度轴$x_v$与在主对称面投影与弹体纵轴的夹角,下方为正。
	}
	
	
	\item $A \to G$:由于在发射时刻$A,G$坐标系重合,因此其转换角与飞行时间$t$相关,发射系绕地轴旋转角为$\omega_e t$,则
	\begin{equation}
		G_A = M_y[-\alpha_0] \cdot M_z[-\phi_0] \cdot M_x[\omega_e t] \cdot M_z[\phi_0] \cdot M_y[\alpha_0]
	\end{equation}
	如果火箭飞行时间较短,认为$\omega_e t$为小量,在转换矩阵中取其一次项,则
	\begin{equation}
		G_A = 
		\begin{bmatrix}
			1 & \omega_{ez}t & -\omega_{ey} t\\
			-\omega_{ez}t & 1 &\omega_{ex} t \\
			\omega_{ey} t & -\omega_{ex}t & 1 
		\end{bmatrix}
	\end{equation}
	其中,
	\begin{equation}
		\begin{cases}
			\,\omega_{ex} = \omega_e \cos \phi_0 \cos \alpha_0 \\
			\,\omega_{ey} = \omega_e \sin \phi_0\\
			\,\omega_{ez} = - \omega_e \cos \phi_0 \sin \alpha_0
		\end{cases}
	\end{equation}
\end{enumerate}

\subsection{常用欧拉角的联系方程}
由于各坐标系之间定义有欧拉角,必然存在一定的联系。

\begin{enumerate}
	\item \textbf{$B,G,V$之间的联系}
	\begin{equation}
		V_G[\theta, \sigma, \nu] = V_B [\alpha, \beta] \cdot B_G [\varphi, \psi, \gamma]
	\end{equation}
	利用姿态角$\varphi, \psi ,\gamma$和攻角侧滑角$\alpha, beta$来确定速度角$\theta, \sigma , \nu$.如果侧向角为小量,则
	\begin{equation}
		\begin{cases}
			\, \sigma= \psi \cos \alpha + \gamma \sin \alpha - \beta \\
			\, \nu = - \psi \sin \alpha + \gamma \cos \alpha \\
			\, \theta = \varphi - \alpha
		\end{cases}
	\end{equation}
	若功角$\alpha$也为小量,则
	\begin{equation}
		\begin{cases}
			\, \theta= \varphi - \alpha \\
			\, \sigma = \psi - \beta \\
			\, \nu = \gamma
		\end{cases}
	\end{equation}

\item \textbf{$B,G,V$之间的联系}\\
	\hspace*{2em}弹体相对发射系姿态角为$\varphi, \psi, \gamma$,相对平移系姿态角为$\varphi_T, \psi_T, \gamma_T$,相对系与发射惯性系矩阵$G_T$,有
	\begin{equation}
		B_T[\varphi_T, \psi_T, \gamma_T] = B_G [\varphi, \psi, \gamma] \cdot G_T[\alpha_0, \phi_0, \omega_e t]
	\end{equation}
	由此可利用姿态角$\varphi,\psi, \gamma $和飞行时间$t$来确定惯性姿态角$\varphi_T, \psi_T, \gamma_T$。如果认为侧向角及时间为小量,则
	\begin{equation}
		\begin{cases}
			\, \varphi_T = \varphi + \omega_{ez}t\\
			\, \psi_T = \psi + (\omega_{ey}\cos \varphi - \omega_{ex}\sin \varphi)\cdot t\\
			\, \gamma_t = \gamma + (\omega_{ey} \sin \varphi + \omega_{ex}\cos \varphi)\cdot t
		\end{cases}
	\end{equation}

\end{enumerate}

\section{变质量力学基本原理}
\subsection{变质量质点基本方程}
	\textbf{1. 火箭质量}
	
	由于发动机工作,飞行中有大量质点从发动机中喷出,因此必须规定一个表面,以此表面内质量作为火箭的总质量。通常此表面取为火箭外表面和发动机喷口断面。因此火箭是一个存在质点流动的变质量物体。
	
	\vspace*{1em}
	\textbf{2. 变质量质点基本方程}
	
	设当前质量为$m(t)$,当前绝对速度为$\bm{V}$,则其动量为
	\begin{equation}
		\bm{Q}(t) = m(t) \cdot \bm{V}
	\end{equation}
	质点在$\d t$时间内,有外界作用力$\bm{F}$,且向外以相对速度$\bm{V}_r$喷射质量元$-\d m$。设质点速度变化为$\d \bm{V}$,则有
	\begin{equation}
		\begin{split}
			\bm{Q}(t+\d t) &= \big(m - (-\d m)\big)\cdot \big(\bm{V} + \d \bm{V}\big)+(-\d m) \cdot (\bm{V} + \bm{V}_r)\\
			& = m(t)(\bm{V} + \d \bm{V}) - \d m \bm{V}_r
		\end{split}
	\end{equation}
	则
	\begin{equation}
		\begin{split}
			\d \bm{Q} &= \bm{Q}(t + \d t) - \bm{Q}(t) \\
			&= m(\bm{V} + \d \bm{V}) - \d m \bm{V}_r - m \cdot \bm{V} \\
			& = m \d \bm{V} -\d m \bm{V}_r
		\end{split}
	\end{equation}
	对常质量质点有动量定律
	\begin{equation}
		\d \bm{Q} = \bm{F} \d t
	\end{equation}
	所以
	\begin{equation}
		\bm{F} \d t = m \d \bm{V} -\d m \bm{V}_r
	\end{equation}
	由此可以得到
	
	\theorem[变质量质点基本方程(密歇尔斯基方程)\index{BZLZDJBFC@变质量质点基本方程}\index{MXESJFC@密歇尔斯基方程}]
	{
		\vspace*{-1em}
		\begin{equation}
			m \dfrac{\d \bm{V}}{\d t} = \bm{F}+\dfrac{\d m}{\d t} \bm{V}_r = \bm{F} + \textcolor{red}{\bm{P}_r}
		\end{equation}
		其中,$\bm{F}$为牛顿第二定律的外力;$\textcolor{red}{\bm{P}_r}$为喷射反作用力,加速力。
	}
	假设质点不受外力作用,且假设有$\bm{V}_r$与$\bm{V}$反向,则
	\begin{equation}
		m \dfrac{\d v}{\d t} = - \dfrac{\d m}{\d t} v_r \quad \Rightarrow \quad \d v = - \d v_r \dfrac{\d m}{m}
	\end{equation}
	再假设质量元喷射速度为常值,则有质点速度为
	\begin{equation}
		v = v_0 + v_r \ln \dfrac{m_0}{m}
	\end{equation}
	设初始速度为0,则可以得到
	
	\theorem[齐奥尔科夫斯基公式\index{QAEKFSJGS@齐奥尔科夫斯基公式}]
	{
		\quad \vspace*{-1em}
		\begin{equation}
			v_k = v_r \ln \dfrac{m_0}{m_k}
		\end{equation}
		设\dy[结构比]{JGB}为$\mu_k = \dfrac{m_k}{m_0}$,则
		\begin{equation}
			v_k = -v_r \ln \mu_k
		\end{equation}
		其中,$v_k$为理想速度。
	}

\subsection{变质量质点系的运动方程}
	对于变质量质点系,除了质点随物体作牵连运动外,在物体内部还有相对运动,会对物体运动有影响,应用密歇尔斯基方程存在近似性。
	
	在惯性参考系内,质点系总外力为$\bm{F}_s$,总力矩为$\bm{M}_s$,则运动方程为
	\begin{align}
		\bm{F}_s &= \sum_{i=1}^N m_i \dfrac{\d^2 \bm{r}_i}{\d t^2} \\[0.5em]
		\bm{M}_s &= \sum_{i=1}^N m_i\bm{r}_i \times \dfrac{\d^2 \bm{r}_i}{\d t^2} 
	\end{align}
	对于连续质点系(物体)的运动方程,则有
	\begin{align}
		\bm{F}_s &= \int_m \dfrac{\d^2 \bm{r}}{\d t^2} \, \d m \label{相对牛二}\\[0.5em]
		\bm{M}_s &= \int_m \bm{r} \times \dfrac{\d^2 \bm{r}}{\d t^2} \, \d m
	\end{align}

	\textbf{1. 质心运动方程}
	
	任一质点在惯性系中的矢径为
	\begin{equation}
		\bm{r} = \bm{r}_{c.m} + \bm{\rho}
	\end{equation}
	其中$\bm{r}_{c.m}$为质心的矢径,$\bm{\rho}$是相对位置矢径,则有绝对加速度为
	\begin{equation}
		\dfrac{\d^2 \bm{r}}{\d t^2} = \dfrac{\d^2 \bm{r}_{c.m}}{\d t^2} + 2 \bm{\omega}_T \times \dfrac{\delta \bm{\rho}}{\delta t} + \dfrac{\delta^2 \bm{\rho}}{\delta t^2} + \dfrac{\bm{\omega}_T}{\d t}\times \bm{\rho} + \bm{\omega}_T \times (\bm{\omega}_T \times \bm{\rho}) 
		\label{绝对加速度}
	\end{equation}
	将\eqref{绝对加速度}代入\eqref{相对牛二},可以得到
	\begin{align*}
		\bm{F}_s & = \int_m \Bigg[\dfrac{\d \bm{r}_{c.m}}{\d t^2} + 2 \bm{\omega}_T \times \dfrac{\delta \bm{\rho}}{\delta t} + \dfrac{\delta^2 \bm{\rho}}{\delta t^2} + \dfrac{\bm{\omega}_T}{\d t}\times \bm{\rho} + \bm{\omega}_T \times (\bm{\omega}_T \times \bm{\rho}) \Bigg]\,\d m \\[0.5em]
		& = m \dfrac{\d^2 \bm{r}_{c.m}}{\d t^2} + 2 \bm{\omega}_T \times \int_m \dfrac{\delta \bm{\rho}}{\delta t}\, \d m + \int_m \dfrac{\delta^2 \bm{\rho}}{\delta t^2}\, \d m + \bm{\omega}_T \times \left(\bm{\omega}_T \times \int_m \bm{\rho}\, \d m \right)
	\end{align*}
由质心的定义,有$\displaystyle \int_m \bm{\rho} \, \d m = 0$,则可以得到

\theorem[任意变质量物体的一般运动方程]
{
	 \vspace*{-1em}
\begin{equation}
	\bm{F}_S = m \dfrac{\d^2 \bm{r}_{c.m}}{\d t^2} + 2 \bm{\omega}_T \times \int_m \dfrac{\delta \bm{\rho}}{\delta t}\, \d m + \int_m \dfrac{\delta^2 \bm{\rho}}{\delta t^2}\, \d m 
\end{equation}
}

\noindent 相应地可以得到

\theorem[任意变质量物体的质心运动方程]
{
	\vspace*{-1em}
	\begin{equation}
		m \dfrac{\d^2 \bm{r}_{c.m}}{\d t^2}  = \bm{F}_s + \bm{F}'_{k} + \bm{F}'_{rel} 
		\label{运动方程}
	\end{equation}
	其中,
	{
		\begin{enumerate}[\hspace*{2em}]
			\item \dy[附加哥氏力]{FJGSL}\quad $\displaystyle \bm{F}'_k = - 2 \bm{\omega}_T \times \int_m \dfrac{\delta \bm{\rho}}{\delta t}\, \d m$
			\item \dy[附加相对力]{FJXDL} \quad $\displaystyle \bm{F}'_{rel} = - \int_m \dfrac{\delta^2 \bm{\rho}}{\delta t^2}\, \d m$
		\end{enumerate}
	}
}

\textbf{2. 绕质心运动方程}

系统$S$绕质心的力矩方程为
\begin{equation}
	\bm{M}_{c.m} = \int_m \bm{\rho} \times \dfrac{\d^2 \bm{r}}{\d t^2} \, \d m
\end{equation}
将加速度的表达式\eqref{绝对加速度}代入,可以得到
\begin{align*}
	\bm{M}_{c.m} = \int_m \bm{\rho} \times \dfrac{\d^2 \bm{r}_{c.m}}{\d t^2}\, \d m 
	+ 2 \int_m \bm{\rho} \times \left(\bm{\omega}_T \times \dfrac{\delta \bm{\rho}}{\delta t}\right)\, \d m 
	+ \int_m \bm{\rho} \times \dfrac{\delta^2 \bm{\rho}}{\delta t^2}\, \d m 
	+ \int_m \bm{\rho}\times \left(\dfrac{\d \bm{\omega}_T}{\d t} \times \bm{\rho}\right)\, \d m 
	+ \int_m \bm{\rho}\times \big[\bm{\rho} \times (\bm{\omega}_T \times \bm{\rho})\big]\, \d m
\end{align*}

由于质心的定义,$\displaystyle \int_m \bm{\rho} \, \d m = 0$且$\dfrac{\d^2 \bm{r}_{c.m}}{\d t^2}$与质量无关,所以$\displaystyle \int_m \bm{\rho} \times \dfrac{\d^2 \bm{r}_{c.m}}{\d t^2}\, \d m = 0$,即化简为
\begin{align}
	\bm{M}_{c.m} = 2 \int_m \bm{\rho} \times \left(\bm{\omega}_T \times \dfrac{\delta \bm{\rho}}{\delta t}\right)\, \d m 
	+ \int_m \bm{\rho} \times \dfrac{\delta^2 \bm{\rho}}{\delta t^2}\, \d m 
	+ \int_m \bm{\rho}\times \left(\dfrac{\d \bm{\omega}_T}{\d t} \times \bm{\rho}\right)\, \d m 
	+ \int_m \bm{\rho}\times \big[\bm{\rho} \times (\bm{\omega}_T \times \bm{\rho})\big]\, \d m
	\label{转动方程}
\end{align}
记
\begin{align*}
	\bm{M}'_k &= - 2 \int_m \bm{\rho} \times \left(\bm{\omega}_T \times \dfrac{\delta \bm{\rho}}{\delta t}\right)\, \d m \\[0.5em]
	\bm{M}'_{rel} &= - \int_m \bm{\rho} \times \dfrac{\delta^2 \bm{\rho}}{\delta t^2}\, \d m
\end{align*}
则\eqref{转动方程}可以写为
\begin{equation}
	\int_m \bm{\rho}\times \left(\dfrac{\d \bm{\omega}_T}{\d t} \times \bm{\rho}\right)\, \d m 
	+ \int_m \bm{\rho}\times \big[\bm{\rho} \times (\bm{\omega}_T \times \bm{\rho})\big]\, \d m 
	= \bm{M}_{c.m} + \bm{M}'_k + \bm{M}'_{rel}
	\label{转动方程2}
\end{equation}
公式\eqref{转动方程2}左端第二项可以处理为
\begin{equation}
	\int_m \bm{\rho} \times \big[\bm{\omega}_T \times (\bm{\omega}_T \times \bm{\rho})\big]\, \d m = \bm{\omega}_T \times \int_m \bm{\rho} \times (\bm{\omega}_T \times \bm{\rho}) \, \d m \xlongequal[]{ \textstyle \hspace*{0.5em} \Delta \hspace*{0.5em}} \bm{\omega}_T \times \bm{H}_{c.m}
\end{equation}
其中$\bm{H}_{c.m}$为将系统视为刚体后,刚体对质心的角动量。

建立与物体固连的坐标系$o_1 - xyz$,有
\begin{equation*}
	\bm{\omega}_T = 
	\begin{bmatrix}
		\omega_{Tx} & \omega_{Ty} & \omega_{Tz}
	\end{bmatrix}^{\text{T}} \qquad \bm{\rho} = 
\begin{bmatrix}
	x & y & z
\end{bmatrix}^{\text{T}}
\end{equation*}
则角动量为
\begin{align*}
	\bm{H}_{c.m} & = \int_m \big[\bm{\rho} \times (\bm{\omega}_T \times \bm{\rho})\big]\, \d m 
	= \int_m \big[(\bm{\rho \cdot \bm{\rho}}) \bm{\omega}_T - (\bm{\rho \cdot \bm{\omega_T}})\bm{\rho }\big]\, \d m\\[0.5em]
	& = \int_m \big[(\bm{\rho}\cdot \bm{\rho})\bm{\omega}_T - (\bm{\rho} \cdot \bm{\rho}^{\text{T}})\bm{\omega}_T\big]\, \d m\\[0.5em]
	& = \int_m 
	\begin{bmatrix}
		y^2 + z^2 & -xy & -xz \\
		-xy & z^2 + x^2 & -yz \\
		-zx & -zy & x^2 + y^2 
	\end{bmatrix}
	\,
	\begin{bmatrix}
		\omega_{Tx}\\
		\omega_{Ty}\\
		\omega_{Tz}
	\end{bmatrix}
	\, \d m\\
	& =  \bm{I}\cdot \bm{\omega}_T
\end{align*}

其中,$\bm{I}$为\dy[惯性张量]{GXZL}
\begin{equation}
	\bm{I} = 
	\begin{bmatrix}
		I_{xx} & - I_{xy} & -I_{xz} \\
		-I_{xy} & I_{yy} & -I_{yz} \\
		-I_{zx} & -I_{zy} & I_{zz}
	\end{bmatrix}
\end{equation}

定义\dy[转动惯量]{ZDGL}$I_{xx},I_{yy},I_{zz}$和\dy[惯量积]{GLJ}
\begin{equation}
	\begin{cases}
		\, \displaystyle I_{xx} = \int_m(y^2 + z^2)\, \d m \\[0.8em]
		\, \displaystyle I_{yy} = \int_m(x^2 + z^2)\, \d m \\[0.8em]
		\,  \displaystyle I_{zz} = \int_m(y^2 + x^2)\, \d m \\[0.8em]
		\,  \displaystyle I_{xy} = I_{yx} = \int_m xy \, \d m\\[0.8em]
		\,  \displaystyle I_{xz} = I_{zx} = \int_m xz \, \d m\\[0.8em]
		\,  \displaystyle I_{xy} = I_{yx} = \int_m yz \, \d m
	\end{cases}
\end{equation}
则
\begin{equation}
	\bm{H}_{c.m} = \int_m \bm{\rho} \times (\bm{\omega}_T \times \bm{\rho})\, \d m =  \bm{I}\cdot \bm{\omega}_T 
	\label{角动量}
\end{equation}

同理可以对方程\eqref{转动方程2}左边第一项处理得到
\begin{equation}
	\int_m \bm{\rho} \times \left(\dfrac{\d \bm{\omega}_T}{\d t} \times \bm{\rho}\right) = \int_m \begin{bmatrix}
		y^2 + z^2 & -xy & -xz \\
		-xy & z^2 + x^2 & -yz \\
		-zx & -zy & x^2 + y^2 
	\end{bmatrix}
	\,
	\begin{bmatrix}
		\dfrac{\d \omega_{Tx}}{\d t}\\[0.8em]
		\dfrac{\d \omega_{Ty}}{\d t}\\[0.8em]
		\dfrac{\d \omega_{Tz}}{\d t}
	\end{bmatrix}
	\, \d m = \bm{I}\cdot \dfrac{\d \bm{\omega}_T}{\d t}
	\label{角动量导}
\end{equation}

则最终的转动方程为

\theorem[任意变质量物体的绕质心运动方程]
{
	\vspace*{-1em}
	\begin{equation}
		\bm{I}\cdot \dfrac{\d \bm{\omega}_T}{\d t} + \bm{\omega}_T\times(\bm{I}\cdot \bm{\omega}_T) = \bm{M}_{c.m}+\bm{M}'_k+\bm{M}'_{rel}
		\label{绕质心的运动方程}
	\end{equation}
	其中,
	{
		\begin{enumerate}[\hspace*{2em}]
			\item \dy[附加哥氏力矩]{FJGSLJ}\quad $\displaystyle \bm{M}'_k = -2 \int_m \bm{\rho}\times \left(\bm{\omega}_T \times \dfrac{\delta \bm{\rho}}{\delta t}\right)\, \d m$
			\item \dy[附加相对力矩]{FJXDLJ} \quad $\displaystyle \bm{F}'_{rel} = - \int_m \bm{\rho} \times \dfrac{\delta^2 \bm{\rho}}{\delta t^2} \, \d m$
			\item \dy[惯量张量]{GXZL} \quad $\displaystyle \bm{I} = 
			\int_m \begin{bmatrix}
				y^2 + z^2 & -xy & -xz \\
				-xy & z^2 + x^2 & -yz \\
				-zx & -zy & x^2 + y^2 
			\end{bmatrix} =
		\begin{bmatrix}
			I_{xx} & - I_{xy} & -I_{xz} \\
			-I_{xy} & I_{yy} & -I_{yz} \\
			-I_{zx} & -I_{zy} & I_{zz}
		\end{bmatrix}$
		\end{enumerate}
	}
}
\vspace*{1em}

\textbf{3. 钢化原理}

\theorem[钢化原理]
{
	一般情况下,任意变质量系统的运动方程,可用一个刚体的运动方程表示。这个刚体的质量等于系统瞬时质量,其受力除真实的外力与外力矩外,还要加上两个附加力和两个附加力矩。
}
































%第三章

\chapter{平面不可压缩势流理论}
\thispagestyle{empty}
\section{基本方程}
\subsection{引言}
理想流动的连续(质量)方程和欧拉(动量)方程:
\begin{equation*}
	\dfrac{\D \rho}{\D t} + \rho (\nabla \bm{V}) = 0 \qquad \bm{f} - \dfrac{1}{\rho} \nabla p = \dfrac{\D \bm{V}}{\D t}
\end{equation*}
在不可压缩的条件下,可以简化为
\begin{equation}
	\begin{cases}
		\, \nabla \bm{V} = 0\\[0.5em]
		\, \dfrac{\D \bm{V}}{\D t} = \bm{f} - \dfrac{1}{\rho} \nabla p
	\end{cases}
\end{equation}
\vspace*{0.5em}

\noindent
\begin{minipage}{0.4\linewidth}
	给定初始条件
	\begin{equation*}
		\begin{cases}
			\, \bm{V}(x,y,z,t_0)\\
			\, p(x,y,z,t_0)
		\end{cases}
	\end{equation*}
\end{minipage}
\begin{minipage}{0.6\linewidth}
	边界条件\vspace*{-0.5em}
	\begin{itemize}
		\item 物体表面:$\bm{V}_n = 0$\vspace*{-0.5em}
		\item 无穷远处:$\bm{V} = \bm{V}, p = p_\infty$
	\end{itemize}
\end{minipage}
\vspace*{1em}

\noindent
\begin{minipage}{0.4\linewidth}
	\noindent 求解困难:\vspace*{-0.5em}
	\begin{itemize}
		\item 存在非线性项\vspace*{-0.5em}
		\item 速度和压强耦合\vspace*{-0.5em}
		\item 物体(如飞行器)边界复杂
	\end{itemize}
\end{minipage}
\begin{minipage}{0.6\linewidth}
	\noindent 求解思路:\vspace*{-0.5em}
	\begin{itemize}
		\item 基本流动的速度位或流函数\vspace*{-0.5em}
		\item 对于简单边界条件,基本流动叠加即可。\vspace*{-0.5em}
		\item 对于复杂边界条件,基本流动叠加并利用数值方法求解。
	\end{itemize}
\end{minipage}
\vspace*{1em}

\noindent 求解途径:\vspace*{-0.5em}
\begin{itemize}
	\item 利用无旋条件简化方程(线性化)\vspace*{-0.5em}
	\item 解耦速度和压强(即分别求解速度和压强)\vspace*{-0.5em}
\end{itemize}
\vspace*{1em}


\subsection{位函数和流函数、叠加原理和边界条件}

\sssection[位函数]

位函数存在的条件是流场无旋:
\begin{equation*}
	\text{rot} \bm{V} = 0 \qquad \bm{\omega}_z = 0 \qquad \dfrac{\partial u}{\partial y} = \dfrac{\partial v}{\partial x}
\end{equation*}
即存在\dy[位函数]{WHS}$\d \phi = u \d x + v \d y$,其中
\begin{equation}
	\begin{cases}
		\, u = \dfrac{\partial \phi}{\partial x} \\[0.5em]
		\, v = \dfrac{\partial \phi}{\partial y}
	\end{cases}
\end{equation}
又由连续方程
\begin{equation}
	\dfrac{\partial u}{\partial x} + \dfrac{\partial v}{\partial y} = 0 \quad \Rightarrow \quad \dfrac{\partial^2 \phi}{\partial x^2} + \dfrac{\partial^2 \phi}{\partial y^2} = 0 \quad \Rightarrow \quad \nabla^2 \phi = 0
\end{equation}
得到速度位,即可求解速度分量;一系列的速度位等值线称为\dy[等位线]{DWX}。
\vspace*{1em}

\sssection[流函数]

由连续方程
\begin{equation*}
	\dfrac{\partial u}{\partial x} + \dfrac{\partial v}{\partial y} = 0 \quad \Rightarrow \quad \dfrac{\partial u}{\partial x} = -\dfrac{\partial v}{\partial y} = \dfrac{\partial (-v)}{\partial x}
\end{equation*}
即存在\dy[流函数]{LHS}$\d \psi = -v \d x + u \d y$,其中
\begin{equation}
	\begin{cases}
		\, v = -\dfrac{\partial \psi}{\partial x} \\[0.5em]
		\, u = \dfrac{\partial \psi}{\partial y}
	\end{cases}
\end{equation}
若流场无旋,则
\begin{equation}
	\dfrac{\partial u}{\partial y} - \dfrac{\partial v}{\partial x} = 0 \quad \Rightarrow \quad  \dfrac{\partial^2 \psi}{\partial y^2} + \dfrac{\partial^2 \psi}{\partial^2 x^2} = 0 \quad \Rightarrow \quad \nabla^2 \psi = 0
\end{equation}
得到流函数,即可求解速度分量;一系列的流函数等值线称为\dy[流线]{LX}。
\vspace*{1em}

\sssection[叠加原理]

拉普拉斯方程是线性方程,可以用\dy[叠加原理]{DJYL}求复合解。

如果多个位函数分别满足拉普拉斯方程,即
\begin{equation}
	\nabla^2 \phi_i = 0
\end{equation}
则这些位函数的线性组合也必定满足别满足拉普拉斯方程,即
\begin{equation}
	\phi = \sum_{i=1}^{n} a_i \phi_i \quad \Rightarrow \quad \nabla^2 \phi = \sum_{i=1}^{n} a_i \nabla^2 \phi_i
\end{equation}
由于速度分量和位函数也是线性关系,则速度分量也满足叠加原理,即
\begin{equation}
	\begin{cases}
		\, \displaystyle u = \dfrac{\partial \phi}{\partial x} = \sum_{i = 1}^{n}a_i \dfrac{\partial \phi_i}{\partial x} = \sum_{i=1}^n a_iu_i\\[1em]
		\, \displaystyle v = \dfrac{\partial \phi}{\partial y} = \sum_{i = 1}^{n}a_i \dfrac{\partial \phi_i}{\partial y} = \sum_{i=1}^n a_iv_i
	\end{cases}
\end{equation}

\sssection[边界条件]
\begin{equation*}
	\mbox{\dy[边界条件]{BJTJ}}
	\, 
	\begin{cases}
		\, \mbox{\blue[外边界条件](足够远处)}\\
		\, \mbox{\blue[内边界条件](物体表面)}
	\end{cases}
\end{equation*}

\noindent 对于空气动力学问题,已知速度位$\phi$,则
\begin{itemize}
	\item 外边界条件(无穷远处):
	\begin{equation}
		\dfrac{\partial \phi}{\partial x} = V_\infty \quad \dfrac{\partial \phi}{\partial y} = \dfrac{\partial \phi}{\partial z} = 0
	\end{equation}
	\item 内边界条件:(物体表面法向速度分量为零)
	\begin{equation}
		\dfrac{\partial \phi}{\partial \bm{n}} = 0
	\end{equation}
\end{itemize}
在边界条件下,拉普拉斯方程的解唯一。

\subsection{位函数和流函数的性质及其相互联系}

\sssection[位函数的性质]
\vspace*{-1em}
\begin{enumerate}[\hspace*{1.5em} (1) ]
	\item 有无旋流动定义得到;位函数值可以相差任意常数;\vspace*{-0.5em}
	
	\item 对于理想不可压无旋流动,位函数满足拉普拉斯方程和叠加原理;\vspace*{-0.5em}
	
	\item 位函数沿某一方向的偏导数等于该方向的速度分量;($\bm{s}$为速度切向)
	\begin{equation}
		\begin{aligned}
			v_s &= u \cos(\bm{x},\bm{s}) + v \cos (\bm{y}, \bm{s}) \\
			& = \dfrac{\partial \phi}{\partial x} \cdot \dfrac{\d x}{\d s} + \dfrac{\partial \phi}{\partial y}\cdot \dfrac{\d y}{\d s} \\
			&= \dfrac{\partial \phi}{\partial \bm{s}}
		\end{aligned}
	\quad \Rightarrow \quad v_s = \dfrac{\partial \phi}{\partial \bm{s}}
	\end{equation}

	\item 速度位函数沿着流线方向增加
	\begin{equation}
			\begin{aligned}
			\d \phi &= \dfrac{\partial \phi}{\partial x}\, \d x + \dfrac{\partial \phi}{\partial y}\d y\\
			& = u\, \d x + v \, \d y\\
			& = \bm{V}\cdot \d \bm{s}
		\end{aligned} 
	\quad \Rightarrow \quad \d \phi = \bm{V}\cdot \bm{s}
	\end{equation}

	\item 速度位函数值相等的点连成的线称为\dy[等位(势)线]{DWX},与速度方向垂直;
	\begin{equation}
		\begin{cases}
			\, \d \phi = 0\\
			\, \d \phi = \bm{V} \cdot \bm{s}
		\end{cases}
		\quad \Rightarrow \quad 
		\bm{V}  \d \bm{s}
	\end{equation}
	
	\item 任意两点的速度线积分等于这两点的速度位函数之差;速度线积分与路径无关,仅决定于两点的位置;对封闭曲线,速度环量为零。
	\begin{align}
		\int_A^B \bm{V}\cdot \d \bm{s} & = \int_A^B (u\,d x + v\,d y + w \, \d z) \notag \\[0.5em]
		& = \int_A^B \, \d \phi = \phi_B - \phi_A
	\end{align}
\end{enumerate}

\sssection[流函数的性质]
\vspace*{-1em}
\begin{enumerate}[\hspace*{1.5em}(1) ]
	\item 
\end{enumerate}






















%第四章

\chapter{粘性流体动力学基础}
\thispagestyle{empty}
\section{流体的粘性及其对流动的影响}
\subsection{流体的粘性}
由于流体粘性影响,均匀流经平板时,贴着平板表面的流体速度降为零,称为\red{流体与板面间 “无滑移” 边界条件}。由于收到内层流体的摩擦力、外层流体的速度有变慢趋势,反过来,由于收到外层流体的摩擦力,内层流体的速度有变快趋势。流层简单“互相牵扯”作用一层层向外传递,在距离板面一定距离后,这种作用逐步消失,速度分布变为均匀。

\defination[流体粘性]
{流层之间“阻碍”流体相对变形趋势的能力称为\dy[流体粘性]{LTNX},相对错动(剪切)流层间的一对摩擦力即\dy[粘性剪切力]{NXJQL}。}

\begin{figure}[!htb]
	\centering
	\begin{minipage}{0.45 \linewidth}
		\centering
		\includegraphics[width=\linewidth]{pic/流体粘性实验.pdf}
		\caption{流体剪切实验}
		\label{流体剪切实验2}
	\end{minipage}
	\begin{minipage}{0.45 \linewidth}
		\centering
		\includegraphics[width=0.7\linewidth]{pic/粘性几何关系.pdf}
		\vspace*{0.1em}
		\caption{流体剪切几何关系}
		\label{流体剪切几何关系2}
	\end{minipage}
\end{figure}
\vspace*{-1em}

如图\ref{流体剪切实验2}所示,可以得到剪切力和粘性剪切应力的表达式
\begin{equation}
	F = \mu \dfrac{U}{h}A, \qquad \tau = \dfrac{F}{A} = \mu \dfrac{U}{h}
\end{equation}

\theorem[牛顿粘性应力公式]
{
	\quad \vspace*{-1em}
	\begin{equation}
		\tau = \mu \dfrac{\d u}{\d y}
		\label{牛顿粘性应力公式2}
	\end{equation}
	其中,$\mu$是流体的粘性系数,$u$是流体的运动速度。\dy[牛顿粘性应力公式]{NDNXYLGS}表明粘性剪切应力不仅与速度梯度有关,而且与物性有关。
}
\noindent 从牛顿粘性应力公式可以看出:\vspace*{-0.5em}
\begin{itemize}
	\item 流体的剪应力与压强$p$无关。\vspace*{-0.5em}
	\item 当$\tau \neq 0$时,$\dfrac{\d u}{\d y} \neq 0$,即无论剪应力多小,只要存在剪应力,流体就会发生变形运动,呈现速度梯度。\vspace*{-0.5em}
	\item $\dfrac{\d u}{\d y}=0$时,$\tau = 0$,即只要流体静止或无变形,就不存在剪应力,流体不存在静摩擦力。
\end{itemize}
因此,\textbf{牛顿粘性应力公式可看成流体易流性的数学表达}。

如图\ref{流体剪切几何关系2}所示,可以找到几何关系
\begin{equation}
	\d \theta \d y = \d u \d t , \quad \dfrac{\d \theta }{\d t} = \dfrac{\d u}{\d y}
\end{equation}
即微团垂直线在单位时间内顺时针的转角$=$速度梯度,速度梯度也表示流体微团的\dy[剪切变形速度]{JQBXSD}或\dy[角变形率]{JBXL}。

一般地,流体剪切应力与速度梯度的关系表示为:
\begin{equation}
	\tau = A + B \left(\dfrac{\d u}{\d y}\right)^n
\end{equation}
但是我们一般只研究\dy[牛顿流体]{NDLT}(如水、空气、汽油、酒精等),即满足牛顿粘性应力公式\eqref{牛顿粘性应力公式2}的流体。

\warn[\textbf{液体和气体产生粘性的物理原因不同}\\
\hspace*{2em}液体的粘性主要来自于液体分子间的{\blue[内聚力]},气体的粘性主要来自于气体分子的{\blue[热运动]}。因此液体与气体{\red[动力粘性系数随温度变化的趋势相反]}(气体粘性系数随温度的升高而升高,液体粘性系数随温度的升高而降低),但动力粘性系数与压强基本无关。]

\defination[动力学粘性系数和运动学粘性系数]
{
	在许多空气动力学问题里,粘性力和惯性力同时存在,在式子中$\mu$和$\rho$往往以$\dfrac{\mu}{\rho}$的组合形式出现,用符号$\nu$表示:(注意$\mu$和$\nu$的物理区别)
	{
		\begin{itemize}
			\item \dy[动力学粘性系数]{DLXNXXS}:$\mu \quad \left[\dfrac{\text{N}\cdot \text{s}}{\text{m}^2}\right]\qquad \mu_{\text{a}} = 1.7894\times 10^{-5} \, \text{kg/m/s}, \quad \mu_{\text{w}} = 1.139 \times 10^{-3}\, \text{kg/m/s}$  
			
			\item \dy[运动学粘性系数]{YDXNXXS}:$\nu = \dfrac{\mu}{\rho} \quad \left[\dfrac{\text{m}^2}{\text{s}}\right]\qquad \nu_\text{a} = 1.461\times 10^{-5} \, \text{m}^2/\text{s} \quad \nu_\text{w} = 1.139\times 10^{-6} \, \text{m}^2/\text{s}$
		\end{itemize}
	}
}
可以看出:空气动力粘性不大,初步近似研究时可忽略其粘性作用,忽略粘性的流体称为\dy[理想流体]{LXLT}。
\vspace*{0.5em}

\noindent 流体粘性的特点总结如下:
\vspace*{-0.5em}
\begin{itemize}
	\item 流体的 剪切变形 是指流体质点之间出现相对运动 例如流体层间的相对运动、产生速度梯度;\vspace*{-0.5em}
	\item 流体的粘性是指流体抵抗剪切变形或质点之间的相对运动的能力,是流体的物理属性;\vspace*{-0.5em}
	\item 流体的粘性力是抵抗流体质点之间相对运动,例如流体层间的相对运动 的剪应力或内摩擦力;\vspace*{-0.5em}
	\item 在静止状态下流体不能承受剪切力;但是在运动状态下流体可以承受剪力,剪切力大小与流体速度梯度有关 而且与流体种类有关;\vspace*{-0.5em}
	\item 粘性流体在流动过程中必然要克服内摩擦力做功,因此流体粘性是流体发生机械能损失的根源 。
\end{itemize}

\subsection{粘性对流动的影响}
\noindent \textbf{1. 绕平板的直匀流}






































%打印索引—————————————
\newpage
\addcontentsline{toc}{chapter}{附录}
\addcontentsline{toc}{section}{索引}
\color{titlepurplec}
\appendix
\printindex
%———————————————

\end{document}

%表格模板
%\begin{table}[h]
%\begin{center}
%	\begin{tabular}{|c|c|c|}
%		\hline
%		母线&旋转轴  &旋转曲面方程  \\
%		\hline
%		\multirow{2}{*}{\hspace*{1cm}$yOz$面上的曲线$C:f(y,z)=0$} \hspace*{1cm}& $z$轴 & \hspace*{1cm}$S:f(\pm  \sqrt{x^2+y^2},z)=0$  \hspace*{1cm}\\
%		\cline{2-3}
%		&$y$轴   & \hspace*{1cm}$S:f(y,\pm  \sqrt{x^2+z^2})=0$ \hspace*{1cm}\\
%		\hline
%		\multirow{2}{*}{\hspace*{1cm}$xOz$面上的曲线$C:f(x,z)=0$\hspace*{1cm}} & $z$轴 &  \hspace*{1cm} $S:f(\pm  \sqrt{x^2+y^2},z)=0$ \hspace*{1cm}\\
%		\cline{2-3}
%		& $x$轴 &\hspace*{1cm} $S:f(x,\pm  \sqrt{y^2+z^2})=0$  \hspace*{1cm} \\
%		\hline
%		\multirow{2}{*}{\hspace*{1cm}$xOy$面上的曲线$C:f(x,y)=0$\hspace*{1cm}} & $y$轴 &  \hspace*{1cm} $S:f(\pm  \sqrt{x^2+z^2},y)=0$ \hspace*{1cm}\\
%		\cline{2-3}
%		& $x$轴 & \hspace*{1cm} $S:f(x,\pm  \sqrt{y^2+z^2})=0$  \hspace*{1cm} \\
%		\hline
%	\end{tabular}
%\end{center}
%\caption{母线在坐标平面内、旋转轴为坐标轴的旋转曲面方程}
%\label{特殊情况的旋转曲面方程}
%\end{table}
