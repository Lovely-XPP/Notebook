\chapter{参考内容}
\label{参考内容}
\thispagestyle{empty}
\section{旋转四元数参数的化简}
\label{旋转四元数参数的化简}
旋转四元数原始表达式为
\begin{align}
	b &= b_\parallel + b_\perp \notag \\
	&= a_\parallel + q a_\perp \qquad \mbox{(其中}\,\, q = \big[ \cos \varPhi, \sin \varPhi \bm{e}\big]\mbox{)}
\end{align}

在进一步化简前,我们需要证明几个引理:

\lemma[]
	{
		如果$q = \big[ \cos \varPhi, \sin \varPhi \bm{e} \big]$,而且$\bm{e}$为单位向量,那么$q^2 = qq = \big[ \cos (2 \varPhi) , \sin (2 \varPhi)\bm{a} \big].$
	}
	\label{lemma:1}

\proof 这个引理证明只需要用到Grassmann 积的定义和三角函数的公式。
\begin{align*}
	q^2 &= \big[ \cos \varPhi, \sin \varPhi \bm{e} \big] \cdot \big[ \cos \varPhi, \sin \varPhi \bm{e} \big] \\
	&= \big[ \cos^2 \varPhi - (\sin \varPhi \bm{e} \cdot \sin \varPhi \bm{e}), (\cos \varPhi \sin \varPhi + \sin \varPhi \cos \varPhi) \bm{e} + (\sin \varPhi \bm{e} \times \sin \varPhi \bm{e}) \big] \\
	&= \big[ \cos^2 \varPhi - \sin^2 \varPhi \norm[a]^2, 2\sin \varPhi \cos \varPhi \bm{e} + 0 \big] \\
	&= \big[ \cos^2 \varPhi - \sin^2 \varPhi, 2 \sin \varPhi \cos \varPhi \bm{e} \big] \\
	&= \big[ \cos (2 \varPhi), \sin (2\varPhi)\bm{e}\big]
\end{align*}
\hfill $\square$
\vspace*{0.5em}

这个引理的几何意义就是,如果绕着同一个轴$\bm{e}$连续旋转$\varPhi$度2次,那么所做出的的变换等同于直接绕着$\bm{a}$旋转$2 \varPhi$度。

\lemma[]
{
		假设$a_{\parallel}=\big[ 0, \bm{a}_{\parallel}\big]$是一个纯四元数,而$q = \big[ \alpha, \beta \bm{e} \big]$,其中$\bm{e}$是一个单位向量且$\alpha, \beta \in \mathbb{R}$.在这种条件下,如果$\bm{a}_\parallel$ 平行于$\bm{e}$,那么$qa_\parallel= a_\parallel q.$
}
\label{lemma:2}


\proof 这个引理的证明同样用到了Grassmann 积,分别计算等式的左右两边。等式左边:
\begin{flalign*}
	\text{LHS} &= q \bm{a}_\parallel &\\
	&= \big[ \alpha, \beta \bm{e} \big] \cdot \big[ 0, \bm{a}_\parallel \big] &\\
	&= \big[ 0- \beta \bm{e} \cdot \bm{a}_\parallel, \alpha \bm{a}_\parallel + \bm{0} + \beta \bm{e}\times \bm{a}_\parallel \big] &\\
	&= \big[ - \beta \bm{e} \cdot \bm{a}_\parallel, \alpha \bm{a}_\parallel \big]  
	&&\mbox{(}\bm{a}_\parallel \mbox{平行于} \bm{e} \mbox{,所以} \beta \bm{e} \times \bm{a}_\parallel = \bm{0} \mbox{)}
\end{flalign*}
等式右边:
\vspace*{-0.5em}
\begin{flalign*}
	\text{RHS} &= a_\parallel q &\\
	&= \big[0, \bm{a}_\parallel \big] \cdot \big[ \alpha, \beta \bm{e} \big] &\\
	&= \big[ 0 -\bm{a}_\parallel \cdot \beta \bm{e}, \bm{0} + \alpha + \alpha \bm{a}_\parallel + \bm{a}_\parallel \times \beta \bm{e}\big] &\\
	&= \big[ - \bm{a}_\parallel \cdot \beta \bm{e}, \alpha \bm{a}_\parallel \big] 
	& \mbox{(}\bm{a}_\parallel \mbox{平行于} \bm{e} \mbox{,所以} \beta \bm{a}_\parallel \times \beta \bm{e}l = \bm{0} \mbox{)}\\
	&= \big[ -\beta \bm{e} \cdot \bm{a}_\parallel, \alpha \bm{a}_\parallel \big] 
	& \mbox{(点乘遵守交换律)} \\
	& = \text{LHS}&
\end{flalign*}
\hfill $\square$
\vspace*{1em}

\lemma[]
{
	假设$a_\perp = \big[ 0, \bm{a}_\perp \big]$是一个纯四元数,而$q = \big[ \alpha, \beta \bm{e} \big]$,其中$\bm{e}$是一个单位向量且$\alpha, \beta \in \mathbb{R}$.在这种条件下,如果$\bm{a}_\parallel$ 正交于$\bm{e}$,那么$qa_\parallel= a_\parallel q^*.$
}
\label{lemma:3}


\proof 这个引理的证明和\textbf{\ref{lemma:2}} 类似。
\begin{flalign*}
	\text{LHS} &= e a_\perp &\\
	&= \big[ \alpha, \beta \bm{e} \big] \cdot \big[ 0, \bm{a}_\perp \big] &\\
	&= \big[ 0 - \beta \bm{e} \cdot \bm{a}_\perp, \alpha \bm{a}_\perp + \bm{0} + \beta \bm{e} \times \bm{a}_\perp \big] &\\
	&= \big[ 0, \alpha \bm{a}_\perp + \beta \bm{e} \times \bm{a}_\perp \big] 
	&& \mbox{(}\bm{a}_\perp \mbox{正交于} \bm{e} \mbox{,所以} \beta \bm{e} \times \bm{a}_\perp = \bm{0} \mbox{)}
\end{flalign*}
\vspace*{-2.5em}

\begin{flalign*}
	\text{RHS} &= a_\perp q^* &\\
	&= \big[0, \bm{a}_\perp \big] \cdot \big[ \alpha, - \beta \bm{e} \big] &\\
	&= \big[0 + \bm{a}_\perp \cdot \beta\bm{e}, \bm{0} + \alpha \bm{a}_\perp + \bm{a}_\perp \times(-\beta \bm{e}) \big] &\\
	&= \big[ 0, \alpha \bm{a}_\perp + \bm{a}_\perp \times (-\beta \bm{e}) \big]
	&& \mbox{(}\bm{a}_\perp \mbox{正交于} \bm{e} \mbox{,所以} \beta \bm{e} \times \bm{a}_\perp = \bm{0} \mbox{)} \\
	&= \big[ 0, \alpha \bm{a}_\perp - (- \beta \bm{e} ) \times \bm{a}_\perp \big]
	&& \hspace*{6.2em}\mbox{(} \bm{a} \times \bm{b} = - \bm{b} \times \bm{a}\mbox{)} \\
	& = \big[0, \alpha \bm{a}_\perp + \beta \bm{e} \times \bm{a}_\perp \big] &\\
	& = \text{LHS} &
\end{flalign*}
\hfill $\square$

现在,我们利用引理对公式进行变形。首先,利用四元数逆的定义$qq^{-1} = 1$,可得
\begin{align}
	b &= a_\parallel + q a_\perp \notag \\
	&= 1 \cdot a_\parallel + q a_\perp \notag \\
	&= pp^{-1} a_\parallel + ppa_\perp
	\label{旋转几何意义}
\end{align}
其中,引入了新的四元数$p$且
\begin{align*}
	q &= p^2 = \big[ \cos \varPhi, \sin \varPhi \bm{e} \big] \\
	p &= \left[ \cos \dfrac{\varPhi}{2}, \sin \dfrac{\varPhi}{2} \bm{e} \right]
\end{align*}

根据\textbf{\ref{lemma:1}} 可以验证其正确性:
\begin{align*}
	pp &= p^2 \\
	& = \left[ \cos \left( 2 \cdot \dfrac{\varPhi}{2} \right), \sin \left( 2 \cdot \dfrac{\varPhi}{2} \right) \bm{e} \right] \\
	& = \big[ \cos \varPhi, \sin \varPhi \bm{e} \big] = q
\end{align*}
同样的,容易得到$p$也是一个单位四元数,即
\begin{equation*}
	p^{-1} = p^*
\end{equation*}
那么
\begin{align*}
	b &= pp^{-1} a_\parallel + ppa_\perp \\
	& = pp^* a_\parallel + pp a_\perp
\end{align*}

结合\textbf{\ref{lemma:2}} 和\textbf{\ref{lemma:3}} ,对公式再次变形
\begin{align}
	b & = pp^* a_\parallel + pp a_\perp \notag\\
	&= p a_\parallel p^* + p a_\perp p^* \\
	&= p(a_\parallel + a_\perp)p^*
\end{align}
由因为$a_\parallel, a_\perp$是$a$的分量,所以$a_\parallel + a_\perp = a$,即
\begin{equation}
	b = p a p^*
\end{equation}
其中,
\begin{align}
	p = \left[ \cos \dfrac{\varPhi}{2}, \sin \dfrac{\varPhi}{2} \bm{e} \right]
\end{align}
\vspace*{-1em}



\section{旋转的欧拉轴 / 角参数表达和四元数表达的等价性}
\label{旋转的欧拉轴 / 角参数表达和四元数表达的等价性}
若定义这些旋转参数和向量为四元数
\begin{align*}
	a &= \big[ 0, \bm{a} \big]  && b = \big[ 0, \bm{b} \big] \\
	e &= \big[0, \bm{e} \big] && q = \left[ \cos \dfrac{\varPhi}{2}, \sin \dfrac{\varPhi}{2} \bm{e} \right]
\end{align*}
其中,$\bm{e}$是旋转轴单位向量,$\varPhi$是旋转的角度。则
\begin{equation}
	b = q a q^* = \bm{b} = \cos \varPhi \bm{a} + ( 1 - \cos \varPhi ) (\bm{e} \cdot \bm{a})\bm{e} + \sin \varPhi(\bm{e} \times \bm{a})
\end{equation}
\proof 首先,引入一个引理,即\dy[向量二重叉乘公式]{XLECCCGS}.

\lemma[向量二重叉乘公式]
{
	向量$\bm{a}, \bm{b}, \bm{c}$满足
	\begin{align}
		(\bm{a} \times \bm{b}) \times \bm{c} = (\bm{c} \cdot \bm{a}) \bm{b} - (\bm{c} \cdot \bm{b}) \bm{a}\\
		\bm{a} \times (\bm{b} \times \bm{c}) = (\bm{a} \cdot \bm{c}) \bm{b} - (\bm{a} \cdot \bm{b}) \bm{c}
	\end{align}
\vspace*{-2em}
	\label{lemma:4}
}
这个引理的证明略。
\clearpage
下面我们开始推导旋转的欧拉轴 / 角参数表达和四元数表达的等价性,由Grassmann积,
\begin{flalign*}
	b &= qaq^* \\
	&= \left[ \cos \dfrac{\varPhi}{2}, \sin \dfrac{\varPhi}{2} \bm{e} \right] \cdot  \big[ 0, \bm{a} \big] \cdot \left[ \cos \dfrac{\varPhi}{2}, -\sin \dfrac{\varPhi}{2} \bm{e} \right] & \\[0.5em]
	&= \left[ 0 - \sin \dfrac{\varPhi}{2} \bm{e} \cdot \bm{a} , \cos \dfrac{\varPhi}{2} \bm{a} + \bm{0} + \sin \dfrac{\varPhi}{2} \bm{e} \times \bm{a} \right] \cdot \left[ \cos \dfrac{\varPhi}{2}, -\sin \dfrac{\varPhi}{2} \bm{e} \right] & \\[0.5em]
	&= \left[- \sin \dfrac{\varPhi}{2} \bm{e} \cdot \bm{a} , \cos \dfrac{\varPhi}{2} \bm{a} +  \sin \dfrac{\varPhi}{2} \bm{e} \times \bm{a} \right] \cdot \left[ \cos \dfrac{\varPhi}{2}, -\sin \dfrac{\varPhi}{2} \bm{e} \right]& \\[0.5em]
	& = \bigg[  - \sin\dfrac{\varPhi}{2} \cos \dfrac{\varPhi}{2} \bm{e} \cdot \bm{a} - \left(- \sin \dfrac{\varPhi}{2} \bm{e} \cdot \left( \cos \dfrac{\varPhi}{2} \bm{a} +  \sin \dfrac{\varPhi}{2} \bm{e} \times \bm{a}  \right) \right),& \\[0.5em]
	& \hspace*{1.6em} - \sin \dfrac{\varPhi}{2} \bm{e} \cdot \bm{a} \cdot \left(- \sin \dfrac{\varPhi}{2} \bm{e}\right) + \cos \dfrac{\varPhi}{2} \left( \cos \dfrac{\varPhi}{2} \bm{a} +  \sin \dfrac{\varPhi}{2} \bm{e} \times \bm{a} \right) +  \left( \cos \dfrac{\varPhi}{2} \bm{a} + \sin \dfrac{\varPhi}{2} \bm{e} \times \bm{a} \right)  \times \left(- \sin \dfrac{\varPhi}{2} \bm{e}\right) \bigg] &\\[0.5em]
	&= \bigg[  - \sin \dfrac{\varPhi}{2}\cos \dfrac{\varPhi}{2} \bm{e} \cdot \bm{a} + \sin \dfrac{\varPhi}{2} \cos \dfrac{\varPhi}{2}\bm{e}  \cdot  \bm{a}  +  \sin^2 \dfrac{\varPhi}{2} \bm{e} \cdot (\bm{e} \times \bm{a}) , & \\[0.5em]
	& \hspace*{1.6em} \sin^2 \dfrac{\varPhi}{2} (\bm{e} \cdot \bm{a}) \cdot \bm{e} + \cos^2 \dfrac{\varPhi}{2} \bm{a} +  \sin \dfrac{\varPhi}{2}\cos \dfrac{\varPhi}{2}  \bm{e} \times \bm{a} - \sin \dfrac{\varPhi}{2}\cos \dfrac{\varPhi}{2}  \bm{a} \times \bm{e} - \sin^2 \dfrac{\varPhi}{2} (\bm{e} \times \bm{a})  \times \bm{e} \bigg]  &\\[0.5em]
	&= \bigg[  - \sin \dfrac{\varPhi}{2}\cos \dfrac{\varPhi}{2} \bm{e} \cdot \bm{a} + \sin \dfrac{\varPhi}{2} \cos \dfrac{\varPhi}{2}\bm{e}  \cdot  \bm{a}  +  \sin^2 \dfrac{\varPhi}{2} \bm{e} \cdot (\bm{e} \times \bm{a}) , & \\[0.5em]
	& \hspace*{1.6em} \sin^2 \dfrac{\varPhi}{2} (\bm{e} \cdot \bm{a}) \cdot \bm{e} + \cos^2 \dfrac{\varPhi}{2} \bm{a} +  2\sin \dfrac{\varPhi}{2}\cos \dfrac{\varPhi}{2}  \bm{e} \times \bm{a} - \sin^2 \dfrac{\varPhi}{2} (\bm{e} \times \bm{a})  \times \bm{e} \bigg]  &
\end{flalign*}
由于$\bm{e} \times \bm{a}$正交于$\bm{e}$,即
\begin{equation*}
	\bm{e} \cdot (\bm{e} \times \bm{a}) = 0
\end{equation*}
进一步化简得
\begin{flalign*}
	b &= \bigg[  - \sin \dfrac{\varPhi}{2}\cos \dfrac{\varPhi}{2} \bm{e} \cdot \bm{a} + \sin \dfrac{\varPhi}{2} \cos \dfrac{\varPhi}{2}\bm{e}  \cdot  \bm{a}  +  \sin^2 \dfrac{\varPhi}{2} \bm{e} \cdot (\bm{e} \times \bm{a}) , & \\[0.5em]
	& \hspace*{1.6em} \sin^2 \dfrac{\varPhi}{2} (\bm{e} \cdot \bm{a}) \cdot \bm{e} + \cos^2 \dfrac{\varPhi}{2} \bm{a} + 2 \sin \dfrac{\varPhi}{2} \cos \dfrac{\varPhi}{2}  \bm{e} \times \bm{a} - \sin^2 \dfrac{\varPhi}{2} (\bm{e} \times \bm{a}) \times \bm{e} \bigg]& \\[0.5em]
	&= \bigg[  - \sin \dfrac{\varPhi}{2}\cos \dfrac{\varPhi}{2} \bm{e} \cdot \bm{a} + \dfrac{\varPhi}{2} \cos \dfrac{\varPhi}{2}\bm{e}  \cdot  \bm{a} , \sin^2 \dfrac{\varPhi}{2} (\bm{e} \cdot \bm{a}) \cdot \bm{e} + \cos^2 \dfrac{\varPhi}{2} \bm{a} +  2\sin \dfrac{\varPhi}{2}\cos \dfrac{\varPhi}{2}  \bm{e} \times \bm{a} - \sin^2 \dfrac{\varPhi}{2} (\bm{e} \times \bm{a})  \times \bm{e} \bigg] &\\[0.5em]
	& = \bigg[ 0, \sin^2 \dfrac{\varPhi}{2} (\bm{e} \cdot \bm{a} )\cdot \bm{e} + \cos^2 \dfrac{\varPhi}{2} \bm{a} +  2\sin \dfrac{\varPhi}{2}\cos \dfrac{\varPhi}{2}  \bm{e} \times \bm{a} - \sin^2 \dfrac{\varPhi}{2} (\bm{e} \times \bm{a})  \times \bm{e} \bigg] &
\end{flalign*}
利用倍角公式化简和\ref{lemma:4} 进一步化简得
\begin{flalign*}
	b &=\bigg[ 0, \sin^2 \dfrac{\varPhi}{2} (\bm{e} \cdot \bm{a}) \cdot \bm{e} + \cos^2 \dfrac{\varPhi}{2} \bm{a} +  2\sin \dfrac{\varPhi}{2}\cos \dfrac{\varPhi}{2}  \bm{e} \times \bm{a} - \sin^2 \dfrac{\varPhi}{2} (\bm{e} \times \bm{a})  \times \bm{e} \bigg] & \\[0.5em]
	&= \bigg[ 0, \dfrac{1 - \cos \varPhi}{2}  \bm{e} \cdot \bm{a} \cdot \bm{e} + \dfrac{1 + \cos \varPhi}{2} \bm{a} +  \sin \varPhi (\bm{e} \times \bm{a}) - \dfrac{1 - \cos \varPhi}{2} (\bm{e} \times \bm{a})  \times \bm{e} \bigg] &\\[0.5em]
	& = \bigg[ 0, \dfrac{1 - \cos \varPhi}{2}  (\bm{e} \cdot \bm{a}) \cdot \bm{e} + \dfrac{1 + \cos \varPhi}{2} \bm{a} +  \sin \varPhi ( \bm{e} \times \bm{a}) - \dfrac{1 - \cos \varPhi}{2} \big((\bm{e} \cdot \bm{e})\bm{a} - (\bm{e} \cdot \bm{a})\bm{e} \big) \bigg]  & \\[0.5em]
	& = \bigg[ 0, \dfrac{1 - \cos \varPhi}{2}  (\bm{e} \cdot \bm{a}) \cdot \bm{e} + \dfrac{1 + \cos \varPhi}{2} \bm{a} +  \sin \varPhi  (\bm{e} \times \bm{a}) - \dfrac{1 - \cos \varPhi}{2} \big(\bm{a} - (\bm{e} \cdot \bm{a})\bm{e} \big) \bigg]  & \\[0.5em]
	& =\bigg[ 0, (1 - \cos \varPhi)(\bm{e} \cdot \bm{a}) \cdot \bm{e} + \cos \varPhi \bm{a} +  \sin \varPhi( \bm{e} \times \bm{a}) \bigg] = \bm{b}
\end{flalign*}
\hfill $\square$



\section{向量运算的矩阵表示}
\label{向量运算的矩阵表示}
\subsection{向量叉乘的矩阵表示}
已知向量在三维坐标系下的表达为$\bm{u} = u_1\bm{i} +u_2\bm{j} +u_3\bm{k}, \, \bm{v} = v_1\bm{i} + v_2\bm{j} + v_3 \bm{k}$,其中,$\bm{i}, \bm{j}, \bm{k}$是单位标准正交基,且其运算满足表 \ref{向量叉乘} ,即满足右手系。
\begin{table}[!htb]
	\centering
	\setlength{\tabcolsep}{2.5em}{
		\begin{tabular}{|c|c|c|c|}
			\hline
			\rowcolor{Azure2} $\times$  & $ \bm{i} $ & $ \bm{j} $ & $ \bm{k} $ \\
			\hline
			\cellcolor{Azure2}  $\bm{i}$ & $0$ &\cellcolor{MistyRose} $\bm{k}$ & \cellcolor{DarkSlateGray2} $-\bm{j}$ \\
			\hline
			\cellcolor{Azure2}  $\bm{j}$ & \cellcolor{MistyRose} $-\bm{k}$ & $0$ & \cellcolor{LightGoldenrod1} $\bm{i}$ \\
			\hline
			\cellcolor{Azure2}  $\bm{k}$ & \cellcolor{DarkSlateGray2} $\bm{j}$ &  \cellcolor{LightGoldenrod1} $-\bm{i}$ & $0$ \\
			\hline
		\end{tabular}
	}
	\caption{单位标准正交向量的叉乘计算表(左$\times$上)}
	\label{向量叉乘}
\end{table}

那么,由向量叉乘的分配律及单位标准正交基的运算法则,
\vspace*{-0.5em}
\begin{align*}
	\bm{u} \times \bm{v} &= (u_1\bm{i} +u_2\bm{j} +u_3\bm{k}) \times (v_1\bm{i} + v_2\bm{j} + v_3 \bm{k}) \\
	&= u_1v_1( \bm{i} \times \bm{i}) + u_1v_2 (\bm{i} \times \bm{j}) + u_1v_3 (\bm{i} \times \bm{k} ) 
	+ u_2v_1 (\bm{j} \times \bm{i}) + u_2v_2 (\bm{j} \times \bm{j}) + u_2v_3 (\bm{j} \times \bm{k}) \\
	&+ u_3v_1 (\bm{k} \times \bm{i}) + u_3v_2 (\bm{k} \times \bm{j}) + u_3v_3 (\bm{k} \times \bm{k}) \\
	&= u_1v_2 \bm{k} + u_1v_3 (-\bm{j} )+ u_2v_1 (- \bm{k}) + u_2v_3 \bm{i} + u_3v_1 \bm{j} + u_3v_2 (-\bm{i}) \\
	&= (u_2v_3 - u_3v_2)\bm{i} + (u_3v_1 - u_1v_3) \bm{j} + (u_1v_2 - u_2v_1)\bm{k} =
	\begin{vmatrix}
		\bm{i} & \bm{j} & \bm{k} \\
		u_1 & u_2 & u_3 \\
		v_1& v_2 & v_3
	\end{vmatrix}
\end{align*}
将向量用矩阵表达,即
\begin{align*}
	\bm{u} \times \bm{v} &= (u_2v_3 - u_3v_2)\bm{i} + (u_3v_1 - u_1v_3) \bm{j} + (u_1v_2 - u_2v_1)\bm{k} \\
	& = 
	\big[ \bm{i} \quad \bm{j} \quad \bm{k} \big] \,
	\begin{bmatrix}
		u_2v_3 - u_3v_2 \\
		u_3v_1 - u_1v_3 \\
		u_1v_2 - u_2v_1
	\end{bmatrix} 
	=
	\big[ \bm{i} \quad \bm{j} \quad \bm{k} \big] \,
	\begin{bmatrix}
		0 & -u_3 & u_2 \\
		u_3 & 0 & -u_1 \\
		-u_2  &  u_1 & 0
	\end{bmatrix}
	\, 
	\begin{bmatrix}
		v_1 \\
		v_2 \\
		v_3 
	\end{bmatrix}
\end{align*}
所以,我们可以得到\dy[向量叉乘矩阵]{XLCCJZ}的定义

\defination[向量叉乘矩阵]
{
	已知向量$\bm{u} = u_1\bm{i} +u_2\bm{j} +u_3\bm{k}, \, \bm{v} = v_1\bm{i} + v_2\bm{j} + v_3 \bm{k}$,其在同一个坐标系下的叉乘运算可以用矩阵表达
	\begin{equation}
		\bm{u} \times \bm{v} = 	
		\big[ \bm{i} \quad \bm{j} \quad \bm{k} \big] \,
		\begin{bmatrix}
			0 & -u_3 & u_2 \\
			u_3 & 0 & -u_1 \\
			-u_2  &  u_1 & 0
		\end{bmatrix}
		\, 
		\begin{bmatrix}
			v_1 \\
			v_2 \\
			v_3 
		\end{bmatrix}
	=\ubm{e}^\T \ubm{u}^\times \ubm{v} = \ubm{v}^\T (\ubm{u}^\times)^\T \ubm{e}
	\end{equation}
	同时,叉乘矩阵有一个很重要的性质,从其定义很容易可以发现
	\begin{equation}
		(\ubm{u}^\times )^\T = 
		\begin{bmatrix}
			0 & u_3 & -u_2 \\
			-u_3 & 0 & u_1 \\
			u_2  &  -u_1 & 0
		\end{bmatrix}
		= -
		\begin{bmatrix}
			0 & -u_3 & u_2 \\
			u_3 & 0 & -u_1 \\
			-u_2  &  u_1 & 0
		\end{bmatrix}
		= - \ubm{u}^\times
	\end{equation}
}

\clearpage
进一步,我们将叉乘运算进一步写为分量形式后可以定义\dy[广义向量叉乘运算]{GYXLCCYS}如下。

\defination[广义向量叉乘运算]
{
	已知向量$\bm{u}, \bm{v}$在坐标系$S$下的分量形式为$\ubm{u}^\T \ubm{e}, \, \ubm{e}^\T \ubm{v}$,则
	\begin{equation}
		\ubm{u}^\T \ubm{e} \times \ubm{e}^\T \ubm{b} = \big[ \bm{i} \quad \bm{j} \quad \bm{k} \big] \,
		\begin{bmatrix}
			0 & -u_3 & u_2 \\
			u_3 & 0 & -u_1 \\
			-u_2  &  u_1 & 0
		\end{bmatrix}
		\, 
		\begin{bmatrix}
			v_1 \\
			v_2 \\
			v_3 
		\end{bmatrix}
		=\ubm{e}^\T \ubm{u}^\times \ubm{v}
	\end{equation}
	定义广义向量叉乘运算为
	\begin{equation}
		\ubm{u}^\T \ubm{e} \times \ubm{e}^\T = \ubm{e}^\T \ubm{u}^\times
	\end{equation}
}
\label{广义向量叉乘运算}



\subsection{叉乘矩阵的坐标变换}
\label{叉乘矩阵的坐标变换}
对于向量叉乘运算$\bm{w} = \bm{u} \times \bm{v}$,在坐标系$S_a,\, S_b$下的矩阵形式分别为
\begin{equation*}
	\ubm{w}_a = \ubm{u}_a^\times \ubm{v}_a, \qquad \ubm{w}_b = \ubm{u}_b^\times \ubm{v}_b
\end{equation*}
由$\ubm{w}_a = \ubm{C}_{ab} \ubm{w}_b$,
\begin{equation*}
	\ubm{C}_{ab} \ubm{w}_b = \ubm{a}^\times \ubm{C}_{ab} \ubm{v}_b \quad \Rightarrow \quad \ubm{w}_b = \ubm{C}_{ba} \ubm{u}_a^\times \ubm{C}_{ab} \ubm{v}_b
\end{equation*}
通过对比前后两个式子可得

\theorem[叉乘矩阵的坐标变换]
{
	$\bm{u}$在坐标系$S_a, \, S_b$下的叉乘矩阵的坐标变换关系为
	\begin{equation}
		\ubm{u}_b^\times = \left(\ubm{C}_{ba} \ubm{u}_a \right) = \ubm{C}_{ba} \ubm{u}_a^\times \ubm{C}_{ab}
		= \ubm{C}_{ba} \ubm{u}_a^\times \ubm{C}_{ba}^\T
	\end{equation}
	进一步,可以扩展得到\dy[叉乘矩阵恒等式]{CCJZHDS}
	\begin{equation}
		\left( \ubm{C} \ubm{u} \right)^\times = \ubm{C} \ubm{u}^\times \ubm{C}^\T
	\end{equation}
}


\subsection{向量两边乘同一个向量运算的矩阵表示}
已知向量在三维坐标系下的表达为$\bm{u} = u_1\bm{i} +u_2\bm{j} +u_3\bm{k}, \, \bm{v} = v_1\bm{i} + v_2\bm{j} + v_3 \bm{k}$,则
\begin{align*}
	\bm{u} \cdot \bm{v} \cdot \bm{u}  &= (u_1\bm{i} +u_2\bm{j} +u_3\bm{k}) \cdot (v_1\bm{i} + v_2\bm{j} + v_3 \bm{k}) \bm{u} \\
	& = (u_1v_1 + u_2v_2 + u_3v_3) \bm{u} \\
	& =  (u_1v_1 + u_2v_2 + u_3v_3) (u_1\bm{i} +u_2\bm{j} +u_3\bm{k}) \\
	& = (u_1^2 v_1 + u_1u_2v_2 + u_1u_3 v_3) \bm{i} + (u_1u_2v_1 + u_2^2 v_2 + u_2u_3v_3)\bm{j} + (u_1u_3v_1 + u_2u_3v_2 + u_3^2v_3) \bm{k}
\end{align*}
用矩阵表达为
\begin{align*}
	\bm{u} \cdot \bm{v} \cdot \bm{u} &= (u_1^2 v_1 + u_1u_2v_2 + u_1u_3 v_3) \bm{i} + (u_1u_2v_1 + u_2^2 v_2 + u_2u_3v_3)\bm{j} + (u_1u_3v_1 + u_2u_3v_2 + u_3^2v_3) \bm{k} \\
	& =
	\big[ \bm{i} \quad \bm{j} \quad \bm{k} \big] \,
	\begin{bmatrix}
		u_1^2v_1 + u_1u_2 v_2 + u_1u_3 v_3 \\
		u_1u_2v_1 + u_2^2 v_2 + u_2u_3v_3 \\
		u_1u_3v_1 + u_2u_3v_2 + u_3^2v_3
	\end{bmatrix}\\[0.5em]
	& = 
	\big[ \bm{i} \quad \bm{j} \quad \bm{k} \big] \,
	\begin{bmatrix}
		u_1^2 & u_1u_2 & u_1u_3 \\
		u_1u_2 & u_2^2 & u_2u_3 \\
		u_1u_3 & u_2u_3 & u_3^2
	\end{bmatrix}
	\,
	\begin{bmatrix}
		v_1 \\
		v_2 \\
		v_3
	\end{bmatrix} \\[0.5em]
& = 
\big[ \bm{i} \quad \bm{j} \quad \bm{k} \big] \,
\begin{bmatrix}
	u_1 \\
	u_2 \\
	u_3
\end{bmatrix}
\,
\big[ u_1 \quad u_2 \quad u_3 \big]
\,
\begin{bmatrix}
	v_1 \\
	v_2 \\
	v_3
\end{bmatrix}
\end{align*}
所以,我们可以得到

\theorem[向量两边乘同一个向量运算的矩阵表示]
{
	已知向量$\bm{u} = u_1\bm{i} +u_2\bm{j} +u_3\bm{k}, \, \bm{v} = v_1\bm{i} + v_2\bm{j} + v_3 \bm{k}$,则
	\begin{equation}
		\ubm{u} \cdot \ubm{v} \cdot \ubm{u} = 
		\big[ \bm{i} \quad \bm{j} \quad \bm{k} \big] \,
		\begin{bmatrix}
			u_1 \\
			u_2 \\
			u_3
		\end{bmatrix}
		\,
		\big[ u_1 \quad u_2 \quad u_3 \big]
		\,
		\begin{bmatrix}
			v_1 \\
			v_2 \\
			v_3
		\end{bmatrix}
		= \ubm{e}^\T \ubm{u} \ubm{u}^{\text{T}} \ubm{v} =\ubm{v}^\T \ubm{u} \ubm{u}^\T \ubm{e}
	\end{equation}
}



\section{向量旋转矩阵和坐标系旋转矩阵的关系}
\label{向量旋转矩阵和坐标系旋转矩阵的关系}
假设向量$\bm{a}$在坐标系$S_a, S_b$下的分量为
\begin{equation*}
	\bm{a} = \ubm{u} \ubm{e}_a = \ubm{v} \ubm{e}_b
\end{equation*}
为了方便计算,这里向量均表示为分量阵列,即$\ubm{u}$和$\ubm{v}$都是$3 \times 3$的对角矩阵
\begin{equation*}
	\bm{a} = 
	\begin{bmatrix}
		u_x & 0 & 0 \\
		0 & u_y & 0 \\
		0 & 0 & u_z
	\end{bmatrix}
	\begin{bmatrix}
		\bm{i}_a \\
		\bm{j}_a \\
		\bm{k}_a
	\end{bmatrix}
	=
	\begin{bmatrix}
		v_x & 0 & 0 \\
		0 & v_y & 0 \\
		0 & 0 & v_z
	\end{bmatrix}
	\begin{bmatrix}
		\bm{i}_b \\
		\bm{j}_b \\
		\bm{k}_b
	\end{bmatrix}
\end{equation*}
考虑向量$\bm{a}$在绕轴$\bm{e}$旋转$\varPhi$后得到的向量$\bm{b}$在坐标系$S_b$的分量为$\ubm{u}$,即
\begin{equation*}
	\bm{b} = 
	\begin{bmatrix}
		u_x & 0 & 0 \\
		0 & u_y & 0 \\
		0 & 0 & u_z
	\end{bmatrix}
	\begin{bmatrix}
		\bm{i}_a \\
		\bm{j}_a \\
		\bm{k}_a
	\end{bmatrix}
	= \ubm{u} \ubm{e}_b
\end{equation*}
且其中的向量旋转矩阵记为$\ubm{R}_{ba}$,坐标系旋转矩阵记为$\ubm{C}_{ba}$,那么
\begin{equation}
	\begin{cases}
		\, \bm{b} = \ubm{u} \ubm{e}_b \\
		\, \bm{b} = \ubm{R}_{ba} \bm{a} = \ubm{R}_{ba} \ubm{u} \ubm{e}_a \\
	\end{cases}
	\quad
	\xrightarrow{\textstyle \quad \mbox{消去}\,\, \bm{b} \quad }
	\quad
	\ubm{u} \ubm{e}_b = \ubm{R}_{ba} \ubm{u} \ubm{e}_a
	\quad
	\xrightarrow{\textstyle \quad \ubm{e}_b  = \ubm{C}_{ba} \ubm{e}_a \quad }
	\quad
	\ubm{u} \ubm{C}_{ba} \ubm{e}_a = \ubm{R}_{ba} \ubm{u} \ubm{e}_a
\end{equation}
等式两边右乘$(\ubm{e}_a)^{-1}$,消去$\ubm{e}_a$,可得
\begin{equation}
	\ubm{u} \ubm{C}_{ba} = \ubm{R}_{ba} \ubm{u}
\end{equation}




\section{欧拉轴角向量恒等式的证明}
\label{欧拉轴角向量恒等式的证明}
\sssection[$\ubm{e}^\T \ubm{e} = 1$]
\vspace*{-0.8em}

\proof 由于欧拉轴向量$\bm{e}$是单位向量,所以
\begin{equation}
	\bm{e} \cdot \bm{e} = \ubm{e}^\T \ubm{e} = 1
\end{equation}
\hfill $\square$


\sssection[$\ubm{e}^\times \ubm{e} = 0$]
\vspace*{-0.8em}

\proof 由于平行向量叉乘为0,所以
\begin{equation}
	\bm{e} \times \bm{e} = \ubm{e}^\times \ubm{e} = 0
	\label{eq:eq2}
\end{equation}
\hfill $\square$


\sssection[$\ubm{e}^\times \ubm{e} = 0$]
\vspace*{-0.8em}

\proof 对公式 \eqref{eq:eq2} 两边同时求导,有
\begin{equation}
	\big( \ubm{e}^\times \ubm{e} \big)' = 0 \quad \Rightarrow \quad \dot{\ubm{e}}^\times \ubm{e} + \ubm{e}^\times \dot{\ubm{e}} = 0 \quad \Rightarrow \quad \dot{\ubm{e}}^\times \ubm{e} =  -\ubm{e}^\times \dot{\ubm{e}}
\end{equation}
\hfill $\square$


\sssection[$\dot{\ubm{e}^\times} \ubm{e}^\times = \ubm{e}\dot{\ubm{e}}^\T$]
\vspace*{-0.8em}

\proof 设$\ubm{e} = \big[ e_x \quad e_y \quad e_z \big]^\T$,则
\begin{equation*}
	\dot{\ubm{e}} =
	\begin{bmatrix}
		\dot{e}_x \\
		\dot{e}_y \\
		\dot{e}_z
	\end{bmatrix},
	\qquad 
	\dot{\ubm{e}}^\times =
	\begin{bmatrix}
		0 & - \dot{e}_x & \dot{e}_y \\
		\dot{e}_z & 0 & - \dot{e}_x \\
		-\dot{e}_y & \dot{e}_x & 0
	\end{bmatrix}
\end{equation*}
注意:$\dot{\ubm{e}}^\times$同样是反对称矩阵,即$\big( \dot{\ubm{e}}^\times \big)^\T = - \dot{\ubm{e}}^\times$,所以有
\begin{equation}
	\dot{\ubm{e}}^\times \ubm{e}^\times = 
	\begin{bmatrix}
		0 & - \dot{e}_z & \dot{e}_y \\
		\dot{e}_z & 0 & - \dot{e}_x \\
		-\dot{e}_y & \dot{e}_x & 0
	\end{bmatrix}
	\begin{bmatrix}
		0 & e_z & e_y \\
		e_z & 0 & -e_x \\
		e_y & e_x & 0
	\end{bmatrix}
	=
	\begin{bmatrix}
		-\dot{e}_y e_y - \dot{e}_z e_z & \dot{e}_y e_x & \dot{e}_z e_x \\
		\dot{e}_x e_y & - \dot{e}_x x - \dot{e}_z e_z & \dot{e}_z e_y \\
		\dot{e}_x e_z & \dot{e}_y e_z & - \dot{e}_x x - \dot{e}_y y
	\end{bmatrix}
	\label{eq:eq3-1}
\end{equation}
由于向量的导数与向量正交($\bm{u} = \bm{\omega} \times \bm{u}$),所以
\begin{equation}
	\dot{\bm{e}} \cdot \bm{e} = \dot{\ubm{e}}^\T \ubm{e} = \big[ \dot{e}_x \quad \dot{e}_y \quad \dot{e}_z \big] 
	\begin{bmatrix}
		e_x \\
		e_y \\
		e_z
	\end{bmatrix}
	= \dot{e}_x x + \dot{e}_y e_y + \dot{e}_z e_z = 0
\end{equation}
替换掉公式 \eqref{eq:eq3-1} 中的对角线元素,可以得到
\begin{equation}
	\dot{\ubm{e}}^\times \ubm{e}^\times = 
	\begin{bmatrix}
		-\dot{e}_y e_y - \dot{e}_z e_z & \dot{e}_y e_x & \dot{e}_z e_x \\
		\dot{e}_x e_y & - \dot{e}_x x - \dot{e}_z e_z & \dot{e}_z e_y \\
		\dot{e}_x e_z & \dot{e}_y e_z & - \dot{e}_x x - \dot{e}_y y
	\end{bmatrix}
	=
	\begin{bmatrix}
		\dot{e}_x e_x & \dot{e}_y e_x & \dot{e}_z e_x \\
		\dot{e}_x e_y & \dot{e}_y e_y & \dot{e}_z e_y \\
		\dot{e}_x e_z & \dot{e}_y e_z & \dot{e}_z z
	\end{bmatrix}
	=\dot{\ubm{e}} \ubm{e}^\T
\end{equation}
\hfill $\square$































