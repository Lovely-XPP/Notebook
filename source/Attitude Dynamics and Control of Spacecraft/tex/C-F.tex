\chapter{参考内容}
\label{参考内容}
\thispagestyle{empty}
\section{旋转四元数参数的化简}
\label{旋转四元数参数的化简}
旋转四元数原始表达式为
\begin{align}
	b &= b_\parallel + b_\perp \notag \\
	&= a_\parallel + q a_\perp \qquad \mbox{(其中}\,\, q = \big[ \cos \varPhi, \sin \varPhi \bm{e}\big]\mbox{)}
\end{align}

在进一步化简前,我们需要证明几个引理:

\lemma
[
	{
		如果$q = \big[ \cos \varPhi, \sin \varPhi \bm{e} \big]$,而且$\bm{e}$为单位向量,那么$q^2 = qq = \big[ \cos (2 \varPhi) , \sin (2 \varPhi)\bm{a} \big].$
	}
	\label{lemma:1}
]

\proof 这个引理证明只需要用到Grassmann 积的定义和三角函数的公式。
\begin{align*}
	q^2 &= \big[ \cos \varPhi, \sin \varPhi \bm{e} \big] \cdot \big[ \cos \varPhi, \sin \varPhi \bm{e} \big] \\
	&= \big[ \cos^2 \varPhi - (\sin \varPhi \bm{e} \cdot \sin \varPhi \bm{e}), (\cos \varPhi \sin \varPhi + \sin \varPhi \cos \varPhi) \bm{e} + (\sin \varPhi \bm{e} \times \sin \varPhi \bm{e}) \big] \\
	&= \big[ \cos^2 \varPhi - \sin^2 \varPhi \norm[a]^2, 2\sin \varPhi \cos \varPhi \bm{e} + 0 \big] \\
	&= \big[ \cos^2 \varPhi - \sin^2 \varPhi, 2 \sin \varPhi \cos \varPhi \bm{e} \big] \\
	&= \big[ \cos (2 \varPhi), \sin (2\varPhi)\bm{e}\big]
\end{align*}
\hfill $\square$
\vspace*{0.5em}

这个引理的几何意义就是,如果绕着同一个轴$\bm{e}$连续旋转$\varPhi$度2次,那么所做出的的变换等同于直接绕着$\bm{a}$旋转$2 \varPhi$度。

\lemma
[
	{
		假设$a_{\parallel}=\big[ 0, \bm{a}_{\parallel}\big]$是一个纯四元数,而$q = \big[ \alpha, \beta \bm{e} \big]$,其中$\bm{e}$是一个单位向量且$\alpha, \beta \in \mathbb{R}$.在这种条件下,如果$\bm{a}_\parallel$ 平行于$\bm{e}$,那么$qa_\parallel= a_\parallel q.$
	}
	\label{lemma:2}
]

\proof 这个引理的证明同样用到了Grassmann 积,分别计算等式的左右两边。等式左边:
\begin{align*}
	\text{LHS} &= q \bm{a}_\parallel \\
	&= \big[ \alpha, \beta \bm{e} \big] \cdot \big[ 0, \bm{a}_\parallel \big] \\
	&= \big[ 0- \beta \bm{e} \cdot \bm{a}_\parallel, \alpha \bm{a}_\parallel + \bm{0} + \beta \bm{e}\times \bm{a}_\parallel \big] \\
	&= \big[ - \beta \bm{e} \cdot \bm{a}_\parallel, \alpha \bm{a}_\parallel \big]  
	&&\mbox{(}\bm{a}_\parallel \mbox{平行于} \bm{e} \mbox{,所以} \beta \bm{e} \times \bm{a}_\parallel = \bm{0} \mbox{)}
\end{align*}

等式右边:
\begin{align*}
	\text{RHS} &= a_\parallel q \\
	&= \big[0, \bm{a}_\parallel \big] \cdot \big[ \alpha, \beta \bm{e} \big] \\
	&= \big[ 0 -\bm{a}_\parallel \cdot \beta \bm{e}, \bm{0} + \alpha + \alpha \bm{a}_\parallel + \bm{a}_\parallel \times \beta \bm{e}\big] \\
	&= \big[ - \bm{a}_\parallel \cdot \beta \bm{e}, \alpha \bm{a}_\parallel \big] 
	&& \mbox{(}\bm{a}_\parallel \mbox{平行于} \bm{e} \mbox{,所以} \beta \bm{a}_\parallel \times \beta \bm{e}l = \bm{0} \mbox{)}\\
	&= \big[ -\beta \bm{e} \cdot \bm{a}_\parallel, \alpha \bm{a}_\parallel \big] 
	&& \hspace*{6.1em}\mbox{(点乘遵守交换律)}\\
	& = \text{LHS}
\end{align*}
\hfill $\square$


\lemma
[
{
	假设$a_\perp = \big[ 0, \bm{a}_\perp \big]$是一个纯四元数,而$q = \big[ \alpha, \beta \bm{e} \big]$,其中$\bm{e}$是一个单位向量且$\alpha, \beta \in \mathbb{R}$.在这种条件下,如果$\bm{a}_\parallel$ 正交于$\bm{e}$,那么$qa_\parallel= a_\parallel q^*.$
}
	\label{lemma:3}
]

\proof 这个引理的证明和\textbf{\ref{lemma:2}} 类似。
\begin{align*}
	\text{LHS} &= e a_\perp \\
	&= \big[ \alpha, \beta \bm{e} \big] \cdot \big[ 0, \bm{a}_\perp \big] \\
	&= \big[ 0 - \beta \bm{e} \cdot \bm{a}_\perp, \alpha \bm{a}_\perp + \bm{0} + \beta \bm{e} \times \bm{a}_\perp \big] \\
	&= \big[ 0, \alpha \bm{a}_\perp + \beta \bm{e} \times \bm{a}_\perp \big] 
	&& \mbox{(}\bm{a}_\perp \mbox{正交于} \bm{e} \mbox{,所以} \beta \bm{e} \times \bm{a}_\perp = \bm{0} \mbox{)}
\end{align*}

\begin{align*}
	\text{RHS} &= a_\perp q^* \\
	&= \big[0, \bm{a}_\perp \big] \cdot \big[ \alpha, - \beta \bm{e} \big] \\
	&= \big[0 + \bm{a}_\perp \cdot \beta\bm{e}, \bm{0} + \alpha \bm{a}_\perp + \bm{a}_\perp \times(-\beta \bm{e}) \big] \\
	&= \big[ 0, \alpha \bm{a}_\perp + \bm{a}_\perp \times (-\beta \bm{e}) \big]
	&& \mbox{(}\bm{a}_\perp \mbox{正交于} \bm{e} \mbox{,所以} \beta \bm{e} \times \bm{a}_\perp = \bm{0} \mbox{)} \\
	&= \big[ 0, \alpha \bm{a}_\perp - (- \beta \bm{e} ) \times \bm{a}_\perp \big]
	&& \hspace*{6.2em}\mbox{(} \bm{a} \times \bm{b} = - \bm{b} \times \bm{a}\mbox{)} \\
	& = \big[0, \alpha \bm{a}_\perp + \beta \bm{e} \times \bm{a}_\perp \big] \\
	& = \text{LHS}
\end{align*}
\hfill $\square$

现在,我们利用引理对公式进行变形。首先,利用四元数逆的定义$qq^{-1} = 1$,可得
\begin{align}
	b &= a_\parallel + q a_\perp \notag \\
	&= 1 \cdot a_\parallel + q a_\perp \notag \\
	&= pp^{-1} a_\parallel + ppa_\perp
	\label{旋转几何意义}
\end{align}
其中,引入了新的四元数$p$且
\begin{align*}
	q &= p^2 = \big[ \cos \varPhi, \sin \varPhi \bm{e} \big] \\
	p &= \left[ \cos \left( \dfrac{1}{2} \varPhi \right), \sin \left( \dfrac{1}{2} \varPhi \right) \bm{e} \right]
\end{align*}

根据\textbf{\ref{lemma:1}} 可以验证其正确性:
\begin{align*}
	pp &= p^2 \\
	& = \left[ \cos \left( 2 \cdot \dfrac{1}{2} \varPhi \right), \sin \left( 2 \cdot \dfrac{1}{2} \varPhi \right) \bm{e} \right] \\
	& = \big[ \cos \varPhi, \sin \varPhi \bm{e} \big] = q
\end{align*}
同样的,容易得到$p$也是一个单位四元数,即
\begin{equation*}
	p^{-1} = p^*
\end{equation*}
那么
\begin{align*}
	b &= pp^{-1} a_\parallel + ppa_\perp \\
	& = pp^* a_\parallel + pp a_\perp
\end{align*}

结合\textbf{\ref{lemma:2}} 和\textbf{\ref{lemma:3}} ,对公式再次变形
\begin{align}
	b & = pp^* a_\parallel + pp a_\perp \notag\\
	&= p a_\parallel p^* + p a_\perp p^* \\
	&= p(a_\parallel + a_\perp)p^*
\end{align}
由因为$a_\parallel, a_\perp$是$a$的分量,所以$a_\parallel + a_\perp = a$,即
\begin{equation}
	b = p a p^*
\end{equation}
其中,
\begin{align}
	p = \left[ \cos \left( \dfrac{1}{2} \varPhi \right), \sin \left( \dfrac{1}{2} \varPhi \right) \bm{e} \right]
\end{align}

















