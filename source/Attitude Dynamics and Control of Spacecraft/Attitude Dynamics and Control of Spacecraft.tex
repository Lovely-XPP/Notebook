%模板
\documentclass[10pt,a4paper,twoside]{book}

%文章标题设置
%中文名
\newcommand{\titleCN}{航天器姿态动力学与控制}
%英文名
\newcommand{\titleEN}{Attitude Dynamics and Control of Spacecraft}
%版本号
\newcommand{\version}{V1.12.028$\,\,$(内测版)}

% ---------------------------- 宏包导入 ----------------------------  %
\usepackage{ctex}
\usepackage{xeCJK}
\usepackage{cite}
\usepackage{makecell}
\usepackage{yhmath}
\usepackage{verbatim}
\usepackage{enumerate}%罗列专用宏包
\usepackage{graphicx}%插入图片的宏包
\usepackage{subfigure}
\let\widering\relax
\usepackage{newtxtext}
\usepackage{newtxmath}
\usepackage{bm}
\DeclareMathSizes{10}{10}{5.5}{4}
\usepackage{makeidx}%索引专用
\makeindex  %添加索引
\usepackage{fancyhdr}
%\usepackage{textcomp}%树叶图案在这个包里
%\usepackage{bbding}%很多漂亮的图案
\usepackage[dvipsnames, svgnames, x11names]{xcolor}%导入了所有颜色配置文件的宏包
\usepackage{geometry}%页边距调整
\geometry{left=2cm,right=2cm,bottom=2cm,top=2cm}
\usepackage{titletoc}%目录页的宏包
\usepackage{titlesec}%改变章节或标题的样式的宏包
\usepackage[bookmarks=true,colorlinks,linkcolor=black]{hyperref}
\usepackage{enumerate}%使用改宏包优化罗列环境
\usepackage{tcolorbox}%box宏包
\tcbuselibrary{most}
\usepackage{xcolor}
\usepackage{colortbl,booktabs}%第二个包定义了几个*rule  
\usepackage{multicol}
\usepackage{multirow}
\usepackage{tikz}
\usepackage{capt-of}
\usepackage{nomencl}%符号说明包
\makenomenclature%制作符号说明
%\usepackage{longtable}
%\usepackage{polynom}% 除法竖式
\usetikzlibrary{shapes.geometric}
\usetikzlibrary{arrows,arrows.meta}
%\usetikzlibrary{circuits.ee.IEC}
% ------------------------------------------------------------- %

% ---------------------------- 基本设置 ----------------------------  %
%字体设置
\setCJKmainfont[BoldFont={PingFangSC-Medium}]{PingFangSC-Regular}

%调整间距(倍数)
\linespread{1.5}

%定义颜色
%定义某个颜色,对应颜色代号查表
\definecolor{titlepurple}{HTML}{5758BB}%一级标题(目前:蓝紫色)
\definecolor{titlepurpleb}{HTML}{3A006F}%二级标题(目前:深紫色)
\definecolor{titlepurplec}{HTML}{006266}%三级标题(目前:墨绿色)
\definecolor{tab1}{HTML}{9698ED}%表格1
\definecolor{tab2}{HTML}{DBDCFF}%表格2
\definecolor{dy0}{HTML}{EA7500}%小标题定义专用(目前:橙黄色)
\definecolor{dl}{HTML}{007500}%小标题定理专用(目前:深绿色)
\definecolor{inference}{HTML}{343300}%小标题推论专用(目前:墨绿色)
\definecolor{ex}{HTML}{7158e2}%小标题例专用(目前:紫色)
\definecolor{dy}{HTML}{BF0060}%夹杂在文本中的定义词的颜色1(目前:深红色)
\definecolor{dy2}{HTML}{FF0000}%夹杂在文本中的定义词的颜色2(目前:红紫色)
\definecolor{dya}{HTML}{FFFFFF}
\definecolor{超链接}{HTML}{0000C6}%含超链接的文字专用色(目前:蓝紫色)
\definecolor{文字底色}{HTML}{F8FF00}%强调的文字底色(目前:黄色)
\definecolor{eq}{HTML}{F0F0F0}
\definecolor{tl}{HTML}{D94600}

%章节或标题的样式
\titleformat{\chapter}{\bfseries\Huge\color{titlepurple}}{第\ \thechapter\ 章\ \quad}{0pt}{}
\titleformat{\section}{\Large\color{titlepurpleb}}{\bfseries{\thesection}\quad  }{0pt}{}
\titleformat{\subsection}{\large\color{titlepurplec}}{\bfseries{\thesubsection}\quad  }{0pt}{}
%\titlespacing{\subsection}{1.5em}{0.1em}{1em}[1em]
%格式如下:\titlespacing*{章节名称}{左间距}{(前)行间距}{(后)行间距}[右间距(一般都没用,填0.1em即可,但不能不填)]


%目录调整
\newcounter{mycontents}
\newcommand{\thecontents}{\refstepcounter{mycontents} \alph{mycontents}.}
%\titlecontents{标题名}[左间距]{标题格式}{标题标志}{无序号标题}{指引线与页码}[下间距]
\titlecontents{chapter}
[0cm]
{\bf \large \vspace{0.8em} }{\contentspush{第 \thecontentslabel\ 章 \hspace*{0.8em}}}{}{\titlerule*[0.5pc]{$\cdot$}\contentspage}
\titlecontents{section}[1.7cm]{\bf  \vspace{0.5em} }{\contentslabel{2.4em}}{\hspace*{-2.5em} \thecontents \hspace*{0.8em}}{\titlerule*[0.5pc]{$\cdot$}\contentspage}
\titlecontents{subsection}[2.5cm]{\small \vspace{0.2em} }{\contentslabel{3em}}{}{\titlerule*[0.5pc]{$\cdot$}\contentspage}

%自定义页眉页脚---------------
\pagestyle{fancy}
\renewcommand{\chaptermark}[1]{\markboth{\;第\ \thechapter\ 章\quad#1\;}{}}
\renewcommand{\sectionmark}[1]{\markright{\;\thesection\ #1\;}}
\fancyhf{}
%\fancyfoot[C]{\bfseries\thepage}
\fancyhead[LO]{\small\CJKfamily{song}\rightmark}
\fancyhead[RE]{\small\CJKfamily{song}\leftmark}
\fancyhead[RO,LE]{\;\thepage\;}
\fancyfoot[RO,LE]{\footnotesize\CJKfamily{heilight}{\titleCN}}
\fancyfoot[RE,LO]{\footnotesize\CJKfamily{heilight}\titleEN}
\renewcommand{\headrulewidth}{0.4pt} % 注意不用\setlength
%\renewcommand{\footrulewidth}{0pt}
\fancyheadoffset[LE,RO]{0cm}
\fancyfootoffset[LE,RO]{0cm}
% 注意不用\setlength
%\renewcommand{\footrulewidth}{0pt}
% ------------------------------------------------------------- %


% ---------------------------- 定义环境 ----------------------------  %

%定义计数器
\newcounter{theorem}[chapter]
\newcounter{defination}[chapter]
\newcounter{example}[chapter]
\newcounter{inference}[chapter]
\newcounter{examples}[chapter]
\newcounter{tl}[chapter]
\newcounter{lemma}[chapter]
\newcounter{F}[section]
\newcounter{G}[section]
\newcounter{H}[section]
\renewcommand{\thetheorem}{{ 定理} \textbf{\thechapter.\arabic{theorem}}}
\renewcommand{\thedefination}{{ 定义} \textbf{\thechapter.\arabic{defination}}}
\renewcommand{\theexample}{{ 题型} \textbf{\thechapter.\arabic{example}}}
\renewcommand{\theinference}{{ 方法} \textbf{\thechapter.\arabic{inference}}}
\renewcommand{\theexamples}{{ 例}  \textbf{\thechapter.\arabic{examples}}}
\renewcommand{\thelemma}{{ 引理}  \textbf{\thechapter.\arabic{lemma}}}
\renewcommand{\thetl}{{ 推论}  \textbf{\thechapter.\arabic{tl}}}
\newcommand{\s}{\hspace*{-2.7em}}


\newcommand{\mybox}[2][]{
	\begin{tcolorbox}[on line,
		arc=0pt,outer arc=0pt,colback=#1!10!white,colframe=#1,
		boxsep=0pt,left=3pt,right=3pt,top=6pt,bottom=6pt,
		boxrule=0pt,leftrule=1.5pt]#2
\end{tcolorbox}}

%定理类
\newcommand{\theorem}[2][]{\vspace{1em}\s \refstepcounter{theorem} \hspace*{0.15em} \mybox[dl]{{\color{dl}\thetheorem\hspace{1em}#1}\\[0.1em] \hspace*{2em}#2}\vspace{0.5em}  \par}

%推论类
\newcommand{\inference}[2][]{\vspace{1em}\s \refstepcounter{inference} \hspace*{0.15em} \mybox[inference]{{\color{inference}\theinference\hspace{1em}#1}\\ \hspace*{1.5em}#2}\vspace{0.5em}   \par}

%定义类
\newcommand{\defination}[2][]{\vspace{1em}\s \refstepcounter{defination} \hspace*{0.15em} \mybox[dy0]{{\color{dy0}\thedefination\hspace{1em}#1}\\[0.1em] \hspace*{2em}#2}\vspace{0.5em} \par}

%引理类
\newcommand{\lemma}[2][]{\vspace{1em}\s \refstepcounter{lemma} \hspace*{0.15em} \mybox[inference]{{\color{inference}\thelemma\hspace{1em}#1}\\ \hspace*{1.5em}#2}\vspace{0.5em}   \par}

%题型类(无标签)
\newcommand{\example}[1][]{\vspace{1em} \s \refstepcounter{example} \mybox[ex]{\color{ex}\theexample\hspace{1em}#1}\vspace{0.5em} \par }
% ------------------------------------------------------------- %



% ---------------------------- 定义命令 ----------------------------  %
%% 文本设置类
\newcommand{\link}[1][]{\hyperref[#1]{#1},Page \pageref{#1}}
\newcommand{\ds}[1][]{\colorbox{文字底色}{#1}}
\newcommand{\red}[1][]{\textcolor{red}{#1}}
\newcommand{\blue}[1][]{\textcolor{blue}{#1}}
\newcommand{\highlights}[1][]{\tcbox[colframe =Chocolate , colback =Coral,boxrule=0.5mm,size=small,on line]{\color{white}{#1}}}%文本高亮
\newcounter{sssection}[subsection]
\newcommand{\sssection}[1][]{\noindent \refstepcounter{sssection} \textbf{\thesssection. #1} \vspace*{0.5em}}
\newcommand{\noa}[1][]{\par (#1) \hspace*{0.3em}}

%% 公式字符类
\renewcommand{\d}{{\rm{d}}}
\newcommand{\e}{{\rm{e}}}
\newcommand{\n}{{\rm{n}}}
\renewcommand{\t}{\text{t}}
\renewcommand{\j}{\text{j}}
\newcommand{\T}{\text{T}}
\def\degree{{}^{\circ}}
\newcommand{\hvdots}{\hspace*{2mm}\vdots\hspace*{2mm}}
\newcommand{\RMn}[1][]{\romannumeral#1}
\newcommand{\RMN}[1][]{\uppercase\expandafter{\romannumeral#1}}
\newcommand{\vi}{\bm{i}}
\newcommand{\vj}{\bm{j}}
\newcommand{\vk}{\bm{k}}
\newcommand{\norm}[1][]{\Vert #1 \Vert}
\newcommand{\ubm}[1]{\underline{\bm{#1}}}
\newcommand{\eqrefp}[1][]{第 \pageref{#1} 页的公式 \eqref{#1} }

%% 定义索引类
\newcommand{\dy}[2][]{{\color{dy}#1}\index{#2@#1}}
\newcommand{\dya}[2][]{\vspace*{0.7em} \noindent \tcbox[colframe =Chocolate , colback =Coral,boxrule=0.5mm,size=small,on line]{\color{dya}{\textbf{#1}}}  \index{#2@#1} \hspace*{1em}}
\newcommand{\dyb}[1][]{\vspace*{0.7em} \noindent \tcbox[colframe =Chocolate, colback =Coral,boxrule=0.5mm,size=small,on line]{\color{dya}{\textbf{#1}}} \hspace*{1em} }
% ------------------------------------------------------------- %

%------------------------------- 图框定义 ------------------------------%
%证明和解
\newcommand{\proof}{\vspace*{1em} \noindent  \hspace*{0.2em}  \tcbox[colframe =black, colback =black!10!white,boxrule=0.5mm,size=small,on line]{\color{black}{{ 证}}\hspace*{0.25em}}\hspace{1.5em}}
\newcommand{\solve}{\vspace*{1em} \noindent  \hspace*{0.2em}  \tcbox[colframe =black, colback =black!10!white,boxrule=0.5mm,size=small,on line]{\color{black}{{ 解}}\hspace*{0.25em}}\hspace{1.5em}}
\newcommand{\solveother}{\vspace*{1em} \noindent  \hspace*{0.2em}  \tcbox[colframe =black, colback =black!10!white,boxrule=0.5mm,size=small,on line]{\color{black}{{ 另解}}\hspace*{0.25em}}\hspace{1.5em}}
\newcommand{\errsolve}{\vspace*{1em} \noindent  \hspace*{0.2em}  \tcbox[colframe =red, colback =red!10!white,boxrule=0.5mm,size=small,on line]{\color{red}{{ 错解}}\hspace*{0.25em}}\hspace{1.5em}}
\newcommand{\errreason}{\vspace*{1em} \noindent  \hspace*{0.2em}  \tcbox[colframe =red, colback =red!10!white,boxrule=0.5mm,size=small,on line]{\color{red}{{ 错因}}\hspace*{0.25em}}\hspace{1.5em}}
\newcommand{\solvereason}{\vspace*{1em} \noindent  \hspace*{0.2em}  \tcbox[colframe =ForestGreen
	, colback =ForestGreen!15!white,boxrule=0.5mm,size=small,on line]{\color{ForestGreen}{{ 解析}}\hspace*{0.25em}}\hspace{1.5em}}

%例
\newcommand{\examples}{\vspace*{1em}\noindent  \refstepcounter{examples} \tcbox[colframe =ex, colback =ex!10!white,boxrule=0.5mm,size=small,on line]{\color{ex}{\theexamples}\hspace*{0.3em}}\hspace{1.5em}}
\newcommand{\simpleexamples}{ \noindent  \tcbox[colframe =ex, colback =ex!10!white,boxrule=0.5mm,size=small,on line]{  \color{ex}{例}}\hspace{1.5em}}

%推论
\newcommand{\tl}{\vspace*{1em}\noindent \refstepcounter{tl} \tcbox[colframe =tl, colback =tl!10!white,boxrule=0.5mm,size=small,on line]{\color{tl}{\thetl}\hspace*{0.3em}}\hspace{1.5em}}

%注意
\newcommand{\warn}[1][]{
	\vspace*{0.5em}
	\begin{tcolorbox}[colframe=red!75!black, colback=yellow!10!white,title=注意,fonttitle = ]
		#1
\end{tcolorbox}}

%评注/总结
\newcommand{\summarize}[1][]{
	\vspace*{0.5em}
	\begin{tcolorbox}[colframe=white!20!black, colback=white!98!black,title=评注,fonttitle = ]
		#1
\end{tcolorbox}}

% ------------------------------------------------------------- %
%文章标题
	\title{
		\Huge{\textbf{\titleCN}}
		\vspace*{18em}
	}
	\author{
		{  \large {易鹏}}\\
		{  \large 中山大学}\vspace*{0.5em}\\
		内部版本号:\version \\
	}



%文档开始
\begin{document}
	%标题及目录
	\pagenumbering{Roman}
	\maketitle
	\clearpage \phantom{s} \thispagestyle{empty} \clearpage
	\setcounter{page}{1}
	\tableofcontents
	\cleardoublepage
	
	%符号说明
	\addcontentsline{toc}{chapter}{符号说明}
	\renewcommand{\nomname}{符号说明}
	\renewcommand{\pagedeclaration}[1]{, #1}
	\twocolumn%双栏
	\setcounter{page}{1}
	\printnomenclature
	\onecolumn
	
	%正文部分
	\cleardoublepage
	\setcounter{page}{1}
	\pagenumbering{arabic}
	
	% 第一章:航天器姿态运动学
	\chapter{火箭发动机推进原理}
\thispagestyle{empty}
\section{概述}
\noindent \textbf{1. 火箭发动机工作过程与能量转化}
{
	\begin{center}
		\begin{tikzpicture}[node distance=1.2cm]
			%定义流程图具体形状
			\node(O) [minimum height=0cm,draw, xshift = -9cm,,inner sep=8pt] {燃烧室};
			\node (B) [minimum height=0cm,draw, xshift = -4.25cm,node distance=3.5cm, inner sep=8pt] {喷管};
			\node (C) [minimum height=0cm,draw, xshift = 0cm,node distance=2cm, inner sep=8pt] {推力传递系统};
			
			%连接具体形状
			\draw[arrows={-Stealth}](-12cm,0cm) -- (O)  node[midway,above=0cm]{推进剂} node[midway, below = 0cm]{化学能};
			\draw[arrows={-Stealth}](O) -- (B) node[midway, above = 0cm]{高温高压燃气} node[midway, below = 0cm]{热能};
			\draw[arrows={-Stealth}](B) -- (C) node[midway, above = 0cm]{高速燃气} node[midway, below = 0cm]{动能};
			\draw[arrows={-Stealth}](C) -- +(3.4cm,0) node[midway,above=0cm]{飞行器} node[midway, below = 0cm]{动能};
		\end{tikzpicture}
		\captionof{figure}{火箭发动机工作过程与能量转化}
		\label{火箭工作过程与能量转化}
	\end{center}
}
\vspace*{-0.5em}
火箭发动机的工作过程和能量转化如图\ref{火箭工作过程与能量转化}所示,实际的误差如表\ref*{火箭实际误差}所示。

\begin{table}[!htb]
	\centering
	\setlength{\tabcolsep}{10mm}{
	\begin{tabular}{|c|c|c|c|}
		\hline
		主要过程 & 分析方法 & 实际过程效率 & 产生原因 \\
		\hline
		燃烧放热 & $Q = \dot{m} \Delta t$ & 燃烧不完全 & 液滴,不均匀等\\
		\hline
		燃气加热 & $\Delta T = UI\dot{m}c_\gamma$ & 热损失 & 壁面传热等\\
		\hline
		膨胀加速 & $\Delta E_k  = \Delta H$ &不完全膨胀 & 分离、摩擦等\\
		\hline
		反作用推进 & $F = \dot{m}I_J$ & 分离,非对称等 & \\
		\hline
	\end{tabular}
	}
	\caption{火箭发动机的实际过程与主要误差}
	\label{火箭实际误差}
\end{table}

\noindent 在分析时,做出以下假设以简化模型\vspace*{-0.5em}
\begin{itemize}
	\item 燃烧室内:完全燃烧,化学能全部转化为热能;\vspace*{-0.5em}
	\item 燃烧室内:忽略热损失,热能全部用于燃气升温;\vspace*{-0.5em}
	\item 喷管内:燃气绝热等熵流动,热能转化为动能。
\end{itemize}
\vspace*{-0.5em}

\section{推进系统的推力}
\subsection{牛顿三大定律}
\vspace*{-1em}
\theorem[牛顿三大定律]
{
	第一运动定律
	\begin{equation}
		\sum \bm{F}_i = \dfrac{\d \bm{v}}{\d t} = 0
	\end{equation}
	\hspace*{1.8em} 第二运动定律
	\begin{equation}
		\bm{F} = \dfrac{\d \bm{p}}{\d t} \qquad \bm{F} = \dfrac{\d m}{\d t}\bm{v} + m \dfrac{\d \bm{v}}{\d t}
	\end{equation}
	\hspace*{1.8em} 第三运动定律
	\begin{equation}
		\bm{F}_{12} = \bm{F}_{21}
	\end{equation}
}

其中,$\bm{F}$为力,$\bm{v}$为速度,$m$为质量,$t$为时间,$\bm{p} = m \bm{v}$为动量。
\vspace*{1em}

\subsection{推力公式}
\noindent \textbf{1. 假设条件(理想情况)}\vspace*{-0.5em}
\begin{enumerate}[\hspace*{1.5em} (1) ]
	\item 一维定常流动;\vspace*{-0.5em}
	\item 外界大气压均匀;\vspace*{-0.5em}
	\item 忽略推进剂入口造成的动量。
\end{enumerate}
\vspace*{1em}

\noindent \textbf{2. 推力公式的推导}
\begin{figure}[!htb]
	\centering
	\includegraphics[width=0.8\linewidth]{pic/推力推导.png}
	\vspace*{-1em}
	\caption{火箭喷射示意图}
	\label{推力推导}
\end{figure}

如图\ref{推力推导}所示,火箭发动机在工作时不仅在内表面会受到燃气压强的作用,其外表面还会受到环境气压的作用,即
\begin{equation}
	\bm{F} = \bm{F}_{\text{in}} + \bm{F}_{\text{out}}
\end{equation}

取$x$方向为正方向,由于火箭是沿轴向对称的,垂直于轴线方向的力相互抵消,则只需考虑喷管的出口部分,由气体受到喷管出口向前的力(和速度方向相反),设$\bm{F}_{\text{in}}$方向为$x$轴正向,则由动量定理
\begin{equation}
		-\bm{F}_{\text{in}} - p_{\text{e}} \bm{n} A_{\text{e}} = m (\bm{u}_{\text{e}} - \bm{u}_{\text{in}})
\end{equation}
即
\begin{equation}
	\bm{F}_{\text{in}} = - m (\bm{u}_{\text{e}} - \bm{u}_{\text{in}}) - p_\e \bm{n} A_\e
\end{equation}
由气压外力与$x$轴正向方向一致,则
\begin{equation}
	\bm{F}_{\text{out}} = p_{\e}
\end{equation}
\vspace*{1em}

\noindent \textbf{3. 推力公式}

\theorem[推力公式]
{
	\quad \vspace*{-1em}
	\begin{equation}
		F = \mathop{\underbrace{\dot{m}u_\e}}_{\scriptsize \mbox{\blue[动量推力]}} + \,\,\, \mathop{\underbrace{(p_\e - p_\a)A_\e}}_{\scriptsize \mbox{\blue[压力推力]}}
	\end{equation}
	其中,\vspace*{-0.5em}
	{
		\begin{enumerate}[\hspace*{1.5em}]
			\item $\bm{F}$ \quad 推力,N \vspace*{-0.5em}
			\item $\dot{m}$ \quad 流量,kg/s \vspace*{-0.5em}
			\item $\bm{u}_\e$ \quad 火箭喷气出口速度, m/s \vspace*{-0.5em}
			\item $\bm{p}_\e$ \quad 火箭喷气出口压强,kPa \vspace*{-0.5em}
			\item $\bm{p}_\a$ \quad 当前高度的大气压强,kPa \vspace*{-0.5em}
			\item $A_\e$ \quad 火箭喷气出口的截面积,$\text{m}^2$ 
		\end{enumerate}
	}
}
\vspace*{1em}

\noindent \textbf{4. 推力公式的讨论}

(1) \hspace*{0.5em}$\dot{m}u_\e$为\dy[动量推力]{DLTL},占推力总值的90\%以上,$(p_\e - p_\a)A_\a$为\dy[压力推力]{YLTL}。

(2) \hspace*{0.5em}进入推力室燃料有初始速度?固体发动机和液体发动机。

(3) \hspace*{0.5em}火箭发动机的推力与飞行器的飞行速度无关,与高度(大气压强)相关。

(4) \hspace*{0.5em}实际飞行中,有外阻力? \quad \blue[有]

(5) \hspace*{0.5em}排气速度不均匀?非沿轴向?\quad 

(6) \hspace*{0.5em}是杏需要假设:完全气体假设?绝热等熵?无热损失和摩擦损失?\quad \blue[不需要]

\vspace*{1em}

\noindent \textbf{5. 推力相关概念}

\defination[特征推力]
{
	\dy[特征推力]{TZTL}(\dy[额定推力]{EDDL}、\dy[发动机设计状态推力]{FDJSJZTTL}):当$p_\e = p_\a$时的发动机推力。对应的高度为火箭发动机的设计高度。即$F^0 = \dot{m}u_\e$
}

\defination[真空推力]
{
	发动机在真空环境下工作时的推力,$P_\a = 0$,即$F_\text{V} = \dot{m}u_\e + P_\e A_\e$
}

\defination[地面推力(海平面推力)]
{
	\quad \vspace*{-1em}
	\begin{equation}
		F_0 = \dot{m}u_\e + A_\e (p_\e - p_{\a 0})
	\end{equation}
	 其中,$p_{\a 0}$为地面的大气压强。
}

\section{喷气速度与流量特性}
\subsection{喷气速度}

\noindent \textbf{1. 基本假设}

(1) \hspace*{0.5em}燃烧室内的燃气参数$T,P_\text{c}, \rho$处处相等,忽略燃烧室速度。

(2) \hspace*{0.5em}喷管中的流动是一维定常、等熵流动,且忽略燃气对喷管壁的传热和摩擦。

(3) \hspace*{0.5em}燃气是定压比热为常数的理想气体(量热完全气体)。

(4) \hspace*{0.5em}燃烧室内化学平衡,喷管内组分不变(冻结流)。
\vspace*{1em}

\noindent \textbf{2. 公式推导}

由假设可得燃气流动的能量守恒方程
\begin{equation}
	H + \dfrac{u^2}{2} = H_0
\end{equation}

在截面处,有
\begin{align*}
	&H_\c + \dfrac{u_\c^2}{2} = H_\e + \dfrac{u_\e^2}{2}\\
	\Rightarrow \quad& H_0 = H_\e + \dfrac{u_\e^2}{2}\\
	\Rightarrow \quad& u_\e = \sqrt{2(H_0 - H_\e)}
\end{align*}

将$H_\e = c_pT_\e, \quad H_0 =c_p T_\f$代入,得
\begin{equation}
	u_\e = \sqrt{2c_p(T_\f - T_\e)} = \sqrt{2c_p T_\f \left(1 - \dfrac{T_\e}{T_\f}\right)}
\end{equation}
对于绝热等熵流动,有
\begin{equation*}
	\dfrac{T_\e}{T_\f} = \left(\dfrac{p_\e}{p_\c}\right)^{\textstyle \frac{k-1}{k}}, \qquad c_p = \dfrac{k}{k-1}\dfrac{R_0}{MW}
\end{equation*}
则可以得到喷气速度方程。

\theorem[喷气速度方程]
{
	\vspace*{-1em}
	\begin{equation}
		u_e = \sqrt{\dfrac{2k}{k-1} \dfrac{R_0}{MW}T_\f \left[1 - \left( \dfrac{p_\e}{p_\c} \right)^{\textstyle \frac{k-1}{k}}\right]}
	\end{equation}
	其中,\vspace*{-0.5em}
	{
		\begin{enumerate}[\hspace*{1.5em}]
			\item $R_0$ \quad 通用气体常数,N \vspace*{-0.5em}
			\item $M$ \quad 燃气平均相对分子质量,g / mol \vspace*{-0.5em}
			\item $k$ \quad 比热比 \vspace*{-0.5em}
			\item $T_\f$ \quad 推进剂绝热燃烧温度,K
		\end{enumerate}
	}
}

\noindent \textbf{2. $u_\e$的影响因素}
\begin{enumerate}[\hspace*{1em}(1) \hspace*{0.5em}]
	\item $T_\f$\quad $t_\f \uparrow$,可转换成动能的热能增加$\rightarrow u_\e \uparrow$
	\item $MW$ \quad $MW \downarrow \, \rightarrow \, u_\e \uparrow$
	\item $k$ \quad $k \uparrow \, \rightarrow\, \sqrt{\dfrac{2k}{k-1}}\, \bigg \downarrow$而$\displaystyle \left[1 - \left(\dfrac{p_\e}{p_\c}\right)^{\textstyle \frac{k-1}{k}}\right] \Bigg \uparrow\, \, \rightarrow$综合考虑,$u_\e$随$k$的增大而略有减小
	\item $\dfrac{p_\e}{p_\c}$ \quad 在$MW,k,T_\f$一定的情况下,$u_\e$随$\dfrac{p_\e}{p_\c}$的减小而增大。其物理意义为燃气在喷管中的膨胀程度,膨胀压强比$\dfrac{p_\e}{p_\c}$越小,燃气膨胀得越充分,有更多的热能转换为动能,喷气速度越高。若$p_\e = 0$,则说明此时燃气的全部热能都转换为动能,喷气速度达到极限值,称为\dy[极限喷气速度]{JXPQSD}。
\end{enumerate}

	\defination[极限喷气速度]
	{
		\vspace*{-1em}
		\begin{equation}
			u_{\text{L}} = \sqrt{2h_\c} = \sqrt{\dfrac{2k}{k-1} R T_\f}
		\end{equation}
	}

\section{飞行器飞行性能指标及分析}

\subsection{飞行器加速度特性}
简化飞行过程,垂直升空的起飞加速度为
\begin{equation}
	a = \dfrac{F}{m_0} - g_0 \quad \Rightarrow \quad \dfrac{a}{g_0} = \dfrac{F}{mg_0} - 1
\end{equation}
\dy[起飞推重比]{QFTZB}定义为$\dfrac{F}{m_0g_0}$,大飞行器约为1.2$\, \sim \,$2.2,小导弹约为$50 \, \sim \, 100$.

\subsection{小结}
航天飞行器或火箭飞行器的主要性能参数:最大速度$V_{\max}$(单级火箭),速度增量$\Delta V$(多级火箭),有效载荷,最大射程,最大飞行高度等
\vspace*{0.5em}

\noindent \textbf{1. 性能最佳}
\begin{enumerate}[\hspace*{1.5em} (1)  ]
	\vspace*{-0.5em}
	\item 有效载荷确定时,缩短完成任务的时间\vspace*{-0.5em}
\end{enumerate}

\vspace*{0.5em}
\noindent \textbf{2. 改进推进系统性能提高飞行器性能途径}
\begin{enumerate}[\hspace*{1.5em} (1)  ]
	\vspace*{-0.5em}
	\item 有效出口速度或比冲:直接影响飞行性能。方法:高能推经剂,高室压及大膨胀比喷管(高空上面级)\vspace*{-0.5em}
	\item 质量数增大:对数影响效果。方法:减小结构质量,增大起飞质量,采用多级推进。\vspace*{-0.5em}
	\item 加大起飞推力,缩短动力飞行时间,减少重力损失\vspace*{-0.5em}
	\item 减小阻力\vspace*{-0.5em}
	\item 最优喷管设计\vspace*{-0.5em}
	\item 发射初速度\vspace*{-0.5em}
	\item 有效喷出速度接近飞行器主飞行段的飞行速度
\end{enumerate}

\vspace*{0.5em}
\noindent \textbf{3. 任务速度}

\defination[任务速度]
{
	\dy[任务速度]{RWSU}:
}


\section{本章总结}
\begin{figure}[!htb]
	\centering
	\begin{tikzpicture}
		\node (A) [draw, inner sep = 5pt]{发动机推力};
		\node (A1) [draw, inner sep = 5pt, right of = A, node distance = 9cm]{\makecell[c]{$F=\dot{m}u_e+A_e(P_e - P_a)$:牛顿定律;\\假设条件,推力公式的推导*及其应用}};
		\node (B) [draw, inner sep = 5pt, below of = A, node distance = 2.4cm]{排气速度};
		\node (B1) [draw, inner sep = 5pt, right of = B, node distance = 9cm]{$\displaystyle u_e = \sqrt{\dfrac{2k}{k-1}\rho_e  P_e \left[\left(\dfrac{P}{P_e}\right)^{\textstyle \frac{2}{k}} - \left(\dfrac{P}{P_e}\right)^{\textstyle \frac{k+1}{k}} \right]}$};
		\node (C) [draw, inner sep = 5pt, below of = B, node distance = 2.4cm]{流量特性};
		\node (C1) [draw, inner sep = 5pt, right of = C, node distance = 9cm]{$\dot{m} = A \sqrt{\dfrac{2k}{k-1} \rho_\e P_\e \left[ \left(\dfrac{P}{P_\text{e}}\right)^{\textstyle\frac{2}{k}} - \dfrac{P}{P_\e}^{\textstyle\frac{k+1}{k}}\right]} \Rightarrow \dot{m} = \dfrac{\Gamma}{\sqrt{RT_\f}}P_\e A_\text{t} = C_\text{D}P_\text{c}A_\text{t} = \dfrac{P_\text{c}A_\text{t}}{c^*}$};
		\node (D) [draw, inner sep = 5pt, below of = C, node distance = 2cm]{推力系数};
		\node (D1) [draw, inner sep = 5pt, right of = D, node distance = 9cm]{$F = C_\text{F}P_\text{c}A_\text{t}$,其物理意义及影响因素};
		\node (E) [draw, inner sep = 5pt, below of = D, node distance = 2.4cm]{总冲和比冲};
		\node (E1) [draw, inner sep = 5pt, right of = E, node distance = 9cm]{$\displaystyle I = \int_0^{t_a} F\, \d t, \quad I_s = \dfrac{I}{M_P} = \dfrac{\displaystyle \int_0^{t_a} F \, \d t }{\displaystyle\int_0^{t_a} \dot{m}\, \d t}$,其物理意义、相关概念及影响因素};
		\node (F) [draw, inner sep = 5pt, below of = E, node distance = 2.4cm]{\makecell[c]{推进剂的混合比\\当量比、余氧系数}};
		\node (F1) [draw, inner sep = 5pt, right of = F, node distance = 9cm]{组元确定情况下,氧化剂和燃料之间的比值,决定燃烧室温度。};
		\node (G) [draw, inner sep = 5pt, below of = F, node distance = 2cm]{飞行器运动};
		\node (G1) [draw, inner sep = 5pt, right of = G, node distance = 9cm]{$F = M\dfrac{\d V}{\d t}\quad \red[V_{\max} - V_0 = \ln \mu] \quad M_0 = M_P + M_e + M_s$};
		\node (H) [draw, inner sep = 5pt, below of = G, node distance = 2cm]{\makecell[c]{飞行器性能\\与发动机性能关系}};
		\node (H1) [draw, inner sep = 5pt, right of = H, node distance = 9cm]{$\displaystyle \Delta u_f = \sum_{i = 1}^{n} I_{s_i} \ln \mu_i$};
		
		\draw[arrows={-Stealth[scale=0.8]}] (A) -- (B);
		\draw[arrows={-Stealth[scale=0.8]}] (B) -- (C);
		\draw[arrows={-Stealth[scale=0.8]}] (C) -- (D);
		\draw[arrows={-Stealth[scale=0.8]}] (D) -- (E);
		\draw[arrows={-Stealth[scale=0.8]}] (E) -- (F);
		\draw[arrows={-Stealth[scale=0.8]}] (F) -- (G);
		\draw[arrows={-Stealth[scale=0.8]}] (G) -- (H);
		\draw (A) -- (A1);
		\draw (B) -- (B1);
		\draw (C) -- (C1);
		\draw (D) -- (D1);
		\draw (E) -- (E1);
		\draw (F) -- (F1);
		\draw (G) -- (G1);
		\draw (H) -- (H1);
	\end{tikzpicture}
	\caption{第1章总结图}
	\label{第1章总结图}
\end{figure}

































	
	% 第二章:航天器姿态动力学
	\chapter{飞行动力学的基础知识}
\thispagestyle{empty}
\section{地球的运动及形状}
\subsection{地球的运动}
\vspace*{-1em}
\defination[地球运动]
{\dy[地球运动]{DQYD}分为\dy[质心运动]{ZXYD}(公转)和\dy[绕心运动]{RXYD}(自转)\vspace*{-0.5em}
{
\begin{itemize}
	\item \dy[公转]{GZ}:以近圆轨道绕太阳公转,周期为1年。\vspace*{-0.5em}
	\item \dy[自转]{ZZ}:地球自转轴成为\dy[地轴]{DZ},地球绕地轴自西向东匀速转动。
\end{itemize}
}
}
\noindent 地球运动的基本参数:\vspace*{-0.5em}
\begin{itemize}
	\item 地球公转周期:$T = 365.25636$个平日\vspace*{-0.5em}
	\item 地球自转周期:$t = 86164.099\,$s $=$ 23$\,$h$\,$56$\,$m$\,$4.099$\,$s\vspace*{-0.5em}
	\item 地球自转角速度:$\omega_e = \dfrac{2 \pi}{86164.1} = 7.292115\times 10^{-5} \text{rad/s}$
\end{itemize}
\vspace*{-0.5em}

\theorem[地球运动的假设]
{在导弹飞行时间内(飞行时间短),可认为\textcolor{red}{地轴在惯性空间指向不变},且\textcolor{red}{地球作匀速直线运动}。但实际上地轴的指向是变化的,存在极移(物质变化)、进动(太阳引力)和章动(月球引力),地球本身也存在加速度。}

其中,进动是指在太阳引力作用下,地轴会绕一个轴作周期约为27500年的圆周运动,类似于不平衡的陀螺,如图所示。而章动是指在月球引力的作用下,地轴并不是做完美的圆周运动,会上下浮动,浮动的周期约为18.6年,如图所示。



\subsection{地球的形状}
\vspace*{-1em}
\defination[地球形状]
{实际应用中采用简单形状描述地球:\vspace*{-0.5em}
{
\begin{itemize}
	\item \red[均质圆球]:$R=6371004$m\vspace*{-0.5em}
	\item \red[总地球椭球体]:$a_e = 6378149\text{m},\,\, b_e = 6356775\text{m}$,地球扁率(离心率)$\alpha_e = \dfrac{(a_e - b_e)}{a_e} = \dfrac{1}{298.257}$
\end{itemize}
}
}


\section{地球大气}
\subsection{地球大气分层}
\vspace*{-1em}
\defination[地球大气分层]
{\dy[地球大气分层]{DQDQFC}是按大气温度分层:\vspace*{-0.5em}
{
\begin{itemize}
	\item \dy[对流层]{DLC}:$0\, \sim \, 18\, \text{km} / 8 \, \text{km}$,75\%大气质量,95\%水汽;\vspace*{-0.5em}
	\item \dy[平流层]{PLC}:$\sim 50 \, \text{km}$,同温层$+$臭氧层,温度升高,大气密度和压强降低,只有地表的0.08\%.\vspace*{-0.5em}
	\item \dy[中间层]{ZJC}:$50\, \text{km} \, \sim \, 90\,\text{km}$,温度降低\vspace*{-0.5em}
	\item \dy[热成层]{RCC}:$90 \, \text{km} \, \sim \, 500\, \text{km}$,温度升高\vspace*{-0.5em}
	\item \dy[外逸层]{WYC}:$>500\, \text{km}$.\vspace*{-0.5em}
\end{itemize}
}
对于运载火箭,一般只考虑90km以下的大气影响。
}

\noindent 大气的物理性质分布如下:

\begin{enumerate}
	\item \textbf{温度分布}\\
	\hspace*{2em}温度随高度的变化曲线在$0 \, \sim \, 80 \, \text{km}$内可以由一系统的折线表示:
	\begin{equation}
		T(h) = T_0 + Gh
	\end{equation}
	
	对于不同的层,相应的参数$G$取值不同。
	
	\item \textbf{压强分布}\\
	\hspace*{2em}大气的实际压强与气温一样变化一场复杂,为了得到一般意义的标准分布,常采用“大气垂直平衡”假设,即认为大气在铅锤方向是静止的,处于力的平衡状态。由$p=Rg_0\rho T$得:
	\begin{equation}
		p(h) = p_0 \e^{\textstyle - \frac{1}{R}\int_0^h \frac{\d h}{T}}
	\end{equation}
	\proof 由“大气垂直平衡”假设,可以得到
	\begin{equation*}
		(p + \d p)\d S + \rho g_0 \d S \d h = p \d S
	\end{equation*}
	化简得到
	\begin{equation*}
		\d p + \rho g \d h = 0
	\end{equation*}
	代入$p=Rg_0\rho T$可得
	\begin{equation*}
		\dfrac{\d p}{p} = - \dfrac{g}{Rg_0T}\, \d h
	\end{equation*}
	积分可得
	\begin{equation}
		\frac{\ln p}{\ln p_0} = \int_{h_0}^h - \frac{g}{Rg_0T}\, \d h \quad \Rightarrow \quad \frac{p}{p_0} = \e^{\textstyle - \frac{1}{R}\int_{h_0}^h \frac{g}{g_0 T}\, \d h}
	\end{equation}
	\item \textbf{密度分布}\\
	\hspace*{2em}由气体状态方程,已知温度$T$和气压$p$,可得
	\begin{equation}
		\dfrac{\rho}{\rho_0} = \frac{pT_0}{p_0 T} = \frac{T_0}{T} \e ^{\textstyle -\frac{1}{R}\int_0^H \frac{\d H}{T}},\, H = \dfrac{1}{g_0}\int_0^h g \, \d h
	\end{equation}
	其中,$H$为\dy[地势高度]{DSGD},相当于具有同等势能的均匀重力场中的高度,其总小于几何高度$h$,但在高度不打时二者差别较小。若认为在某一高度范围内为等温过程,则:
	\begin{equation}
		\dfrac{\rho_2}{\rho_1} = \e^{\textstyle - \frac{H_2 - H_1}{H_{M_1}}}, \, H_{M_1} = RT_1
	\end{equation}
	其中,$H_{M_1}$称为\dy[基准高]{JZG}或\dy[标高]{BG}。
	
	\hspace*{2em} 如果假设在$0 \, \sim \, 80\, \text{km}$内为恒温过程,则有
	\begin{equation}
		\frac{p}{p_0} = \dfrac{\rho}{\rho_0} = \e^{-\beta h}, \quad \beta = \frac{1}{H_{\text{MCP}}}=\frac{1}{7.11\text{km}}
	\end{equation}
	这个模型称为\dy[指数大气模型]{ZSDQMX}。
\end{enumerate}

\subsection{标准大气}
导弹飞行状态随与随高度变化的大气参数有密切关系(压强、密度、温度及音速等)。

\section{坐标系间的方向余弦阵及矢量导数的关系}
由于不同坐标系对同一物理量的描述形式或者坐标投影不同,为了在统一坐标系中描述飞行器的运动,存在不同物理量到基准坐标系的转换需求。

\defination[坐标转换]
{
	同一矢量在不同坐标系下的坐标不同,将矢量$S_a$坐标系中的坐标转换到$S_b$坐标系称为\dy[坐标系转换]{ZBXZH}。
}

\subsection{坐标系之间的方向余弦阵}
\vspace*{-1em}
\defination[坐标系]
{
	\dy[笛卡尔坐标系]{DKEZBX}:由原点及过原点的两(三)条具有方向的坐标轴组成,坐标轴上的度量单位通常相等。\\
	\hspace*{2em} \dy[球坐标系]{QZBX}:球坐标系由原点、方位角、仰角和距离构成。\\
	\hspace*{2em} \dy[极坐标系]{JZBX}:极坐标系由极点、极径及极角构成。
}

考虑两个直角坐标系:$P:O_p-\bm{x}_p\bm{y}_p\bm{z}_p,\quad Q:O_q-\bm{x}_q\bm{y}_q\bm{z}_q$,定义$P_Q$为$Q$系中单位矢量$E_q$变换到$P$系中单位矢量$E_p$的转换矩阵,由
\begin{equation}
	E_p = P_QE_q, \qquad E_p =
	\begin{bmatrix}
		\bm{x}_p^0 & \bm{y}_p^0 & \bm{z}_p^0
	\end{bmatrix}^{\text{T}}
\qquad 
	E_q = 
	\begin{bmatrix}
		\bm{x}_q^0 & \bm{y}_q^0 & \bm{z}_q^0
	\end{bmatrix}^{\text{T}}
\end{equation}
由于
\begin{equation*}
	E_q \cdot E_q^{\text{T}} = 
	\begin{bmatrix}
		\bm{x}_q^0\\
		\bm{y}_q^0\\
		\bm{z}_q^0
	\end{bmatrix}
	\begin{bmatrix}
	\bm{x}_q^0 & \bm{y}_q^0 & \bm{z}_q^0
	\end{bmatrix}
	=
	\begin{bmatrix}
		\bm{x}_q^0 \cdot \bm{x}_q^0 & \bm{x}_q^0 \cdot \bm{y}_q^0 & \bm{x}_q^0 \cdot \bm{z}_q^0 \\ 
		\bm{y}_q^0 \cdot \bm{x}_q^0 & \bm{y}_q^0 \cdot \bm{y}_q^0 & \bm{y}_q^0 \cdot \bm{z}_q^0 \\ 
		\bm{z}_q^0 \cdot \bm{x}_q^0 & \bm{z}_q^0 \cdot \bm{y}_q^0 & \bm{z}_q^0 \cdot \bm{z}_q^0 
	\end{bmatrix}
	=
	\begin{bmatrix}
		1 & 0 & 0 \\
		0 & 1 & 0 \\
		0 & 0 & 1
	\end{bmatrix} = E
\end{equation*}
那么
\begin{equation}
	P_Q = E_p \cdot E_q^{\text{T}} = 
	\begin{bmatrix}
		\bm{x}_p^0 \cdot \bm{x}_q^0 & \bm{x}_p^0 \cdot \bm{y}_q^0 & \bm{x}_p^0 \cdot \bm{z}_q^0 \\ 
		\bm{y}_p^0 \cdot \bm{x}_q^0 & \bm{y}_p^0 \cdot \bm{y}_q^0 & \bm{y}_p^0 \cdot \bm{z}_q^0 \\ 
		\bm{z}_p^0 \cdot \bm{x}_q^0 & \bm{z}_p^0 \cdot \bm{y}_q^0 & \bm{z}_p^0 \cdot \bm{z}_q^0 
	\end{bmatrix}
	=
	\begin{bmatrix}
		\cos(\bm{x}_p, \bm{x}_q) & \cos(\bm{x}_p, \bm{y}_q) & \cos(\bm{x}_p, \bm{z}_q)\\
		\cos(\bm{y}_p, \bm{x}_q) & \cos(\bm{y}_p, \bm{y}_q) & \cos(\bm{y}_p, \bm{z}_q)\\
		\cos(\bm{z}_p, \bm{x}_q) & \cos(\bm{z}_p, \bm{y}_q) & \cos(\bm{z}_p, \bm{z}_q)
	\end{bmatrix}
	\triangleq
	\begin{bmatrix}
		a_{ij}
	\end{bmatrix}
	\quad i,j=1,2,3
	\label{方向余弦阵}
\end{equation}

公式\eqref{方向余弦阵}称为\dy[方向余弦阵]{FXYXZ},同理可以得到
\begin{equation}
	Q_p = E_q \cdot E_p^{\text{T}}
	\begin{bmatrix}
		\bm{x}_q^0 \cdot \bm{x}_p^0 & \bm{x}_q^0 \cdot \bm{y}_p^0 & \bm{x}_q^0 \cdot \bm{z}_p^0 \\ 
		\bm{y}_q^0 \cdot \bm{x}_p^0 & \bm{y}_q^0 \cdot \bm{y}_p^0 & \bm{y}_q^0 \cdot \bm{z}_p^0 \\ 
		\bm{z}_q^0 \cdot \bm{x}_p^0 & \bm{z}_q^0 \cdot \bm{y}_p^0 & \bm{z}_q^0 \cdot \bm{z}_p^0 
	\end{bmatrix}
	=
	\begin{bmatrix}
		\cos(\bm{x}_q, \bm{x}_p) & \cos(\bm{x}_q, \bm{y}_p) & \cos(\bm{x}_q, \bm{z}_p)\\
		\cos(\bm{y}_q, \bm{x}_p) & \cos(\bm{y}_q, \bm{y}_p) & \cos(\bm{y}_q, \bm{z}_p)\\
		\cos(\bm{z}_q, \bm{x}_p) & \cos(\bm{z}_q, \bm{y}_p) & \cos(\bm{z}_q, \bm{z}_p)
	\end{bmatrix}
\end{equation}
又
\begin{equation*}
	P_q^{-1} = Q_p = E_q \cdot E_p^{\text{T}} = \left(E_p \cdot E_q^{\text{T}} \right)^{\text{T}} = P_q^{\text{T}}
\end{equation*}
这说明:\red[方向余弦阵是正交矩阵],那么方向余弦阵只有\blue[三个独立变量]。

\defination[初等转换矩阵]
{
	当两个坐标系之间存在平行轴的时候,此时的方向余弦矩阵称为\dy[初等转换矩阵]{CDZHJZ}。分别绕$x,y,z$轴旋转的初等转换矩阵为
	\begin{equation}
		M_x(\theta) = 
		\begin{bmatrix}
			1 & 0 & 0\\
			0 & \cos \theta & \sin \theta \\
			0 & -\sin \theta & \cos \theta
		\end{bmatrix}
		\qquad
		M_y(\theta) =
		\begin{bmatrix}
			\cos \theta & 0 & -\sin\theta\\
			0 & 1 & 0\\
			\sin \theta & 0 & \cos \theta
		\end{bmatrix}
		\qquad
		M_z (\theta)= 
		\begin{bmatrix}
			\cos \theta & \sin \theta & 0 \\
			-\sin \theta & \cos \theta & 0 \\
			0 & 0 & 1
		\end{bmatrix}
	\end{equation}
}

\theorem[转换矩阵的传递性]
{
	对于直角坐标系$P,Q,S$,它们相互之间的转换矩阵为$P_Q,S_Q,S_P$,则\vspace*{-1em}
	\begin{equation}
		S_Q = S_P \cdot P_Q, \quad P_Q = P_S \cdot S_Q, \quad P_S = P_Q\cdot Q_S
	\end{equation}
}

\subsection{坐标系转换阵的欧拉角表示方法}
\vspace*{-1em}
\defination[转换矩阵的欧拉角表示]
{
	将坐标系视作刚体,则经过三次旋转后可以与另一个坐标系重合,因此可以用这三个旋转角(\dy[欧拉角]{OLJ})作为独立变量,来描述方向余弦阵。
}

\begin{figure}[!htb]
	\centering
	\includegraphics[width=0.3\linewidth]{pic/欧拉角.jpg}
	\caption{坐标系旋转转换}
	\label{欧拉角}
\end{figure}

例如,如图\ref{欧拉角}所示,$o-\bm{x}_q\bm{y}_q\bm{z}_q$分别绕$z,y,x$轴旋转三次,得到
\begin{equation*}
	\begin{split}
		o-\bm{x}_q\bm{y}_q\bm{z}_q \, \xrightarrow{\quad \textstyle M_z[\xi] \quad } o - x_1 y_1 \bm{z}_q \, \xrightarrow{\quad \textstyle M_y[\eta] \quad } o - \bm{x}_p y_1 \bm{z}_1\ \, \xrightarrow{\quad \textstyle M_x[\zeta] \quad } \,o - \bm{x}_p \bm{y}_p \bm{z}_p
	\end{split}
\end{equation*}
即
\begin{equation}
	P_Q = M_x[\xi] \cdot  M_y[\eta]\cdot  M_z[\zeta]
\end{equation}
进一步代入,得
\begin{equation}
	P_Q = 
	\begin{bmatrix}
		\cos \xi \cos \eta & \sin \xi \cos \eta & - \sin \eta \\
		\cos \xi \sin \eta \sin \zeta - \sin \xi \cos \zeta & \sin \xi \sin \eta \sin \zeta + \cos \xi \cos \zeta & \cos \eta \sin \zeta \\
		\cos \xi \sin \eta \cos \zeta + \sin \xi \sin \zeta & \sin \zeta \sin \eta \cos \zeta - \cos \xi \sin \zeta & \cos \eta \cos \zeta
	\end{bmatrix}
\end{equation}

\subsection{坐标间矢量导数的关系}
\vspace*{-1em}
\defination[矢量导数]
{
	\dy[矢量导数]{SLDS}:同一矢量在不同坐标系有不同的投影,其导数的数值不同。
}

设坐标系$O:o-xyz$相对于坐标系$P:o_p-x_py_pz_p$有角速率$\omega.$ 矢量$\bm{a}$在坐标系$O$的投影为
\begin{equation}
	\bm{a} = a_x \bm{x}^0 + a_y \bm{y}^0 + a_z \bm{z}^0
\end{equation}
取微分,得
\begin{equation}
	\dfrac{\d \bm{a}}{\d t} = \dfrac{\d a_x}{\d t}\bm{x}^0 + \dfrac{\d a_y}{\d t} \bm{y}^0 + \dfrac{\d a_z}{\d t} \bm{z}^0 + a_x \dfrac{\bm{x}^0}{\d t} + \dfrac{}{分母}
\end{equation}

\section{常用坐标系及其相互转换}
\subsection{常用坐标系}
常用的坐标系分类
\begin{itemize}
	\item 取地心为原点:地心惯性坐标系,地心固连坐标系\vspace*{-0.5em}
	\item 取发射点为原点:发射坐标系,发射惯性坐标系\vspace*{-0.5em}
	\item 取对象质心为原点:体坐标系,速度坐标系,半速度坐标系
\end{itemize}

\begin{enumerate}
	\item \dy[地心惯性坐标系$I$]{DXGXZBX}:$O_E - X_I Y_I Z_I$(静系)
	\vspace*{1em}
	
	\begin{minipage}{0.6\linewidth}
		\centering
		\setlength{\tabcolsep}{12mm}{
		\begin{tabular}{cl}
			\hline
			原点 & 地心$O_E$\\
			\hline
			$X$轴 & $X_I$:平春分点\\
			\hline
			$Y$轴 & $Y_I$:右手法则\\
			\hline
			$Z$轴 & $Z_I$:地球自转轴\\
			\hline
		\end{tabular}
	}
	\end{minipage}

	\vspace*{1.5em}
	\item \dy[地心坐标系$E$]{DXZBX}:$O_E - X_E Y_E Z_E$(动系)
	\vspace*{1em}
	
	\begin{minipage}{0.6\linewidth}
		\centering
		\setlength{\tabcolsep}{12mm}{
		\begin{tabular}{cl}
			\hline
			原点 & 地心$O_E$\\
			\hline
			$X$轴 & $X_E$:给定子午线\\
			\hline
			$Y$轴 & $Y_E$:右手法则\\
			\hline
			$Z$轴 & $Z_E$:地球自转轴\\
			\hline
		\end{tabular}
	}
	\end{minipage}
	
	\vspace*{1.5em}
	\item \dy[发射坐标系$G$]{FSZBX}:$O - xyz$(动系)
	\vspace*{1em}
	
	\begin{minipage}{0.6\linewidth}
		\centering
		\setlength{\tabcolsep}{8mm}{
			\begin{tabular}{cl}
				\hline
				原点 & 发射点$O$\\
				\hline
				$X$轴 & $x$:发射水平面内指向瞄准方向\\
				\hline
				$Y$轴 & $y$:发射水平面指向上方\\
				\hline
				$Z$轴 & $z$:右手法则\\
				\hline
			\end{tabular}
		}
	\end{minipage}
	\vspace*{0.5em}
	
	对于球模型:$\varphi_0$:地心纬度,$\alpha_0$:发射方位角。\\
	对于椭球模型:$\varphi_0$:地心纬度,$\alpha_0$:发射方位角。
	
	\item \dy[发射惯性坐标系$A$]{FSGXZBX}:$O_A - x_A y_A z_A$(静系)
	\vspace*{1em}
	
	\begin{minipage}{0.6\linewidth}
		\centering
		\setlength{\tabcolsep}{8mm}{
			\begin{tabular}{cl}
				\hline
				原点 & 发射点$O_A$,起飞瞬间与发射点$O$重合\\
				\hline
				$X$轴 & $x_A$:起飞瞬间的发射水平面内指向瞄准方向\\
				\hline
				$Y$轴 & $y_A$:起飞瞬间的发射水平面指向上方\\
				\hline
				$Z$轴 & $z_A$:右手法则\\
				\hline
			\end{tabular}
		}
	\end{minipage}
	
	\vspace*{1.5em}
	\item \dy[平移坐标系$A$]{PYZBX}:$o_T - x_T y_T z_T$(动系)
	\vspace*{1em}
	
	\begin{minipage}{0.6\linewidth}
		\centering
		\setlength{\tabcolsep}{8mm}{
			\begin{tabular}{cl}
				\hline
				原点 & 发射点$O_A$,起飞瞬间与发射点$O$重合\\
				\hline
				$X$轴 & $x_A$:起飞瞬间的发射水平面内指向瞄准方向\\
				\hline
				$Y$轴 & $y_A$:起飞瞬间的发射水平面指向上方\\
				\hline
				$Z$轴 & $z_A$:右手法则\\
				\hline
			\end{tabular}
		}
	\end{minipage}
	
	\vspace*{1.5em}
	\item \dy[弹体坐标系$B$]{DTZBX}:$o_1 - x_1 y_1 z_1$(动系)
	\vspace*{1em}
	
	\begin{minipage}{0.6\linewidth}
		\centering
		\setlength{\tabcolsep}{8mm}{
			\begin{tabular}{cl}
				\hline
				原点 & 弹体质心$O_1$\\
				\hline
				$X$轴 & $x_1$:沿弹体对称轴指向头部\\
				\hline
				$Y$轴 & $y_1$:位于主对称面内,垂直于$X$轴\\
				\hline
				$Z$轴 & $z_1$:右手法则,顺着发射方向看向右为正\\
				\hline
			\end{tabular}
		}
	\end{minipage}
	
	\vspace*{1.5em}
	\item \dy[速度坐标系$V$]{SDZBX}:$o_1 - x_v y_v z_v$(动系)
	\vspace*{1em}
	
	\begin{minipage}{0.6\linewidth}
		\centering
		\setlength{\tabcolsep}{8mm}{
			\begin{tabular}{cl}
				\hline
				原点 & 弹体质心$O_1$\\
				\hline
				$X$轴 & $x_v$:弹体的速度方向\\
				\hline
				$Y$轴 & $y_v$:位于主对称面内,垂直于$X$轴\\
				\hline
				$Z$轴 & $z_v$:右手法则\\
				\hline
			\end{tabular}
		}
	\end{minipage}

	\vspace*{1.5em}
	\item \dy[半速度坐标系$H$]{BSDZBX}:$o_1 - x_h y_h z_h$(动系)
	\vspace*{1em}
	
	\begin{minipage}{0.6\linewidth}
		\centering
		\setlength{\tabcolsep}{5mm}{
			\begin{tabular}{cl}
				\hline
				原点 & 弹体质心$O_1$\\
				\hline
				$X$轴 & $x_h$:弹体的速度方向\\
				\hline
				$Y$轴 & $y_h$:包含速度矢量的铅锤面内垂直于$x_h$,向上为正\\
				\hline
				$Z$轴 & $z_h$:右手法则\\
				\hline
			\end{tabular}
		}
	\end{minipage}
\end{enumerate}

\subsection{各坐标系间的转换关系}
\begin{enumerate}
	\item $I \to E$:$Z$轴重合,$X$轴处于赤道面内相差角$\varOmega_G = \omega_e \cdot t$(时角),则转换矩阵为
	\begin{equation}
		E_I = M_z [\varOmega_G]
	\end{equation}
	
	\item $E \to G$:设地球为圆球,发射点可用经纬度$(\lambda_0, \varphi_0)$来描述,即
	\begin{equation}
		G_E = M_y\big[-(90\degree + \alpha_0)\big]\cdot M_x[\varphi_0] \cdot M_z \big[-(90 \degree - \lambda_0)\big]
	\end{equation}

	\item $G \to B$:设$G \to B$转序为$321$,旋转角为$\varphi, \psi ,\gamma$,则
	\begin{equation}
		B_G = M_x[\gamma]\cdot M_y[\psi] \cdot M_z[\varphi]
	\end{equation}
	\hspace*{1em}\defination[新的欧拉角{\RMN[1]}]
	{
		\dy[俯仰角$\varphi$]{FYJ}:轴$ox_1$在发射面$xoy$上的投影与$x$的夹角,投影在$x$的上方为正。\\
		\hspace*{2.2em}\dy[偏航角$\psi$]{PHJ}:轴$ox_1$与发射面$xoy$的夹角,$ox_1$在发射面左边为正。\\
		\hspace*{2.2em}\dy[滚动角$\gamma$]{GDJ}:旋转角速度矢量与$ox_1$轴方向一致时为正。
	}

	\item $G \to V$:设$G \to V$转序为$321$,旋转角为$\theta, \sigma ,\nu$,则
	\begin{equation}
		V_G = M_x[\nu]\cdot M_y[\sigma] \cdot M_z[\theta]
	\end{equation}
	\hspace*{1em}\defination[新的欧拉角{\RMN[2]}]
	{
		\dy[速度倾角$\theta$]{SDQJ}:轴$ox_v$在发射面$xoy$上的投影与$x$的夹角,投影在$x$的上方为正。\\
		\hspace*{2.2em}\dy[航迹偏航角$\sigma$]{HJPHJ}:轴$ox_v$与发射面$xoy$的夹角,$ox_v$在发射面左边为正。\\
		\hspace*{2.2em}\dy[倾侧角$\nu$]{QCJ}:旋转角速度矢量与$ox_v$轴方向一致时为正。
	}

	
	\item $V \to B$:由于速度系$y_v$轴位于主对称面内,因此$V \to B$只有两个欧拉角,设为$\alpha,\beta$,设定$V \to B$的转序为23,则
	\begin{equation}
		B_V = M_z[\alpha] \cdot M_y [\beta]
	\end{equation}
	
	\hspace*{1em}\defination[新的欧拉角{\RMN[3]}]
	{
		\dy[测滑角$\beta$]{CHJ}:速度轴$x_v$与弹体主对称面的夹角,右方为正。\\
		\hspace*{2.2em}\dy[攻角$\alpha$]{GJ}:速度轴$x_v$与在主对称面投影与弹体纵轴的夹角,下方为正。
	}
	
	
	\item $A \to G$:由于在发射时刻$A,G$坐标系重合,因此其转换角与飞行时间$t$相关,发射系绕地轴旋转角为$\omega_e t$,则
	\begin{equation}
		G_A = M_y[-\alpha_0] \cdot M_z[-\phi_0] \cdot M_x[\omega_e t] \cdot M_z[\phi_0] \cdot M_y[\alpha_0]
	\end{equation}
	如果火箭飞行时间较短,认为$\omega_e t$为小量,在转换矩阵中取其一次项,则
	\begin{equation}
		G_A = 
		\begin{bmatrix}
			1 & \omega_{ez}t & -\omega_{ey} t\\
			-\omega_{ez}t & 1 &\omega_{ex} t \\
			\omega_{ey} t & -\omega_{ex}t & 1 
		\end{bmatrix}
	\end{equation}
	其中,
	\begin{equation}
		\begin{cases}
			\,\omega_{ex} = \omega_e \cos \phi_0 \cos \alpha_0 \\
			\,\omega_{ey} = \omega_e \sin \phi_0\\
			\,\omega_{ez} = - \omega_e \cos \phi_0 \sin \alpha_0
		\end{cases}
	\end{equation}
\end{enumerate}

\subsection{常用欧拉角的联系方程}
由于各坐标系之间定义有欧拉角,必然存在一定的联系。

\begin{enumerate}
	\item \textbf{$B,G,V$之间的联系}
	\begin{equation}
		V_G[\theta, \sigma, \nu] = V_B [\alpha, \beta] \cdot B_G [\varphi, \psi, \gamma]
	\end{equation}
	利用姿态角$\varphi, \psi ,\gamma$和攻角侧滑角$\alpha, beta$来确定速度角$\theta, \sigma , \nu$.如果侧向角为小量,则
	\begin{equation}
		\begin{cases}
			\, \sigma= \psi \cos \alpha + \gamma \sin \alpha - \beta \\
			\, \nu = - \psi \sin \alpha + \gamma \cos \alpha \\
			\, \theta = \varphi - \alpha
		\end{cases}
	\end{equation}
	若功角$\alpha$也为小量,则
	\begin{equation}
		\begin{cases}
			\, \theta= \varphi - \alpha \\
			\, \sigma = \psi - \beta \\
			\, \nu = \gamma
		\end{cases}
	\end{equation}

\item \textbf{$B,G,V$之间的联系}\\
	\hspace*{2em}弹体相对发射系姿态角为$\varphi, \psi, \gamma$,相对平移系姿态角为$\varphi_T, \psi_T, \gamma_T$,相对系与发射惯性系矩阵$G_T$,有
	\begin{equation}
		B_T[\varphi_T, \psi_T, \gamma_T] = B_G [\varphi, \psi, \gamma] \cdot G_T[\alpha_0, \phi_0, \omega_e t]
	\end{equation}
	由此可利用姿态角$\varphi,\psi, \gamma $和飞行时间$t$来确定惯性姿态角$\varphi_T, \psi_T, \gamma_T$。如果认为侧向角及时间为小量,则
	\begin{equation}
		\begin{cases}
			\, \varphi_T = \varphi + \omega_{ez}t\\
			\, \psi_T = \psi + (\omega_{ey}\cos \varphi - \omega_{ex}\sin \varphi)\cdot t\\
			\, \gamma_t = \gamma + (\omega_{ey} \sin \varphi + \omega_{ex}\cos \varphi)\cdot t
		\end{cases}
	\end{equation}

\end{enumerate}

\section{变质量力学基本原理}
\subsection{变质量质点基本方程}
	\textbf{1. 火箭质量}
	
	由于发动机工作,飞行中有大量质点从发动机中喷出,因此必须规定一个表面,以此表面内质量作为火箭的总质量。通常此表面取为火箭外表面和发动机喷口断面。因此火箭是一个存在质点流动的变质量物体。
	
	\vspace*{1em}
	\textbf{2. 变质量质点基本方程}
	
	设当前质量为$m(t)$,当前绝对速度为$\bm{V}$,则其动量为
	\begin{equation}
		\bm{Q}(t) = m(t) \cdot \bm{V}
	\end{equation}
	质点在$\d t$时间内,有外界作用力$\bm{F}$,且向外以相对速度$\bm{V}_r$喷射质量元$-\d m$。设质点速度变化为$\d \bm{V}$,则有
	\begin{equation}
		\begin{split}
			\bm{Q}(t+\d t) &= \big(m - (-\d m)\big)\cdot \big(\bm{V} + \d \bm{V}\big)+(-\d m) \cdot (\bm{V} + \bm{V}_r)\\
			& = m(t)(\bm{V} + \d \bm{V}) - \d m \bm{V}_r
		\end{split}
	\end{equation}
	则
	\begin{equation}
		\begin{split}
			\d \bm{Q} &= \bm{Q}(t + \d t) - \bm{Q}(t) \\
			&= m(\bm{V} + \d \bm{V}) - \d m \bm{V}_r - m \cdot \bm{V} \\
			& = m \d \bm{V} -\d m \bm{V}_r
		\end{split}
	\end{equation}
	对常质量质点有动量定律
	\begin{equation}
		\d \bm{Q} = \bm{F} \d t
	\end{equation}
	所以
	\begin{equation}
		\bm{F} \d t = m \d \bm{V} -\d m \bm{V}_r
	\end{equation}
	由此可以得到
	
	\theorem[变质量质点基本方程(密歇尔斯基方程)\index{BZLZDJBFC@变质量质点基本方程}\index{MXESJFC@密歇尔斯基方程}]
	{
		\vspace*{-1em}
		\begin{equation}
			m \dfrac{\d \bm{V}}{\d t} = \bm{F}+\dfrac{\d m}{\d t} \bm{V}_r = \bm{F} + \textcolor{red}{\bm{P}_r}
		\end{equation}
		其中,$\bm{F}$为牛顿第二定律的外力;$\textcolor{red}{\bm{P}_r}$为喷射反作用力,加速力。
	}
	假设质点不受外力作用,且假设有$\bm{V}_r$与$\bm{V}$反向,则
	\begin{equation}
		m \dfrac{\d v}{\d t} = - \dfrac{\d m}{\d t} v_r \quad \Rightarrow \quad \d v = - \d v_r \dfrac{\d m}{m}
	\end{equation}
	再假设质量元喷射速度为常值,则有质点速度为
	\begin{equation}
		v = v_0 + v_r \ln \dfrac{m_0}{m}
	\end{equation}
	设初始速度为0,则可以得到
	
	\theorem[齐奥尔科夫斯基公式\index{QAEKFSJGS@齐奥尔科夫斯基公式}]
	{
		\quad \vspace*{-1em}
		\begin{equation}
			v_k = v_r \ln \dfrac{m_0}{m_k}
		\end{equation}
		设\dy[结构比]{JGB}为$\mu_k = \dfrac{m_k}{m_0}$,则
		\begin{equation}
			v_k = -v_r \ln \mu_k
		\end{equation}
		其中,$v_k$为理想速度。
	}

\subsection{变质量质点系的运动方程}
	对于变质量质点系,除了质点随物体作牵连运动外,在物体内部还有相对运动,会对物体运动有影响,应用密歇尔斯基方程存在近似性。
	
	在惯性参考系内,质点系总外力为$\bm{F}_s$,总力矩为$\bm{M}_s$,则运动方程为
	\begin{align}
		\bm{F}_s &= \sum_{i=1}^N m_i \dfrac{\d^2 \bm{r}_i}{\d t^2} \\[0.5em]
		\bm{M}_s &= \sum_{i=1}^N m_i\bm{r}_i \times \dfrac{\d^2 \bm{r}_i}{\d t^2} 
	\end{align}
	对于连续质点系(物体)的运动方程,则有
	\begin{align}
		\bm{F}_s &= \int_m \dfrac{\d^2 \bm{r}}{\d t^2} \, \d m \label{相对牛二}\\[0.5em]
		\bm{M}_s &= \int_m \bm{r} \times \dfrac{\d^2 \bm{r}}{\d t^2} \, \d m
	\end{align}

	\textbf{1. 质心运动方程}
	
	任一质点在惯性系中的矢径为
	\begin{equation}
		\bm{r} = \bm{r}_{c.m} + \bm{\rho}
	\end{equation}
	其中$\bm{r}_{c.m}$为质心的矢径,$\bm{\rho}$是相对位置矢径,则有绝对加速度为
	\begin{equation}
		\dfrac{\d^2 \bm{r}}{\d t^2} = \dfrac{\d^2 \bm{r}_{c.m}}{\d t^2} + 2 \bm{\omega}_T \times \dfrac{\delta \bm{\rho}}{\delta t} + \dfrac{\delta^2 \bm{\rho}}{\delta t^2} + \dfrac{\bm{\omega}_T}{\d t}\times \bm{\rho} + \bm{\omega}_T \times (\bm{\omega}_T \times \bm{\rho}) 
		\label{绝对加速度}
	\end{equation}
	将\eqref{绝对加速度}代入\eqref{相对牛二},可以得到
	\begin{align*}
		\bm{F}_s & = \int_m \Bigg[\dfrac{\d \bm{r}_{c.m}}{\d t^2} + 2 \bm{\omega}_T \times \dfrac{\delta \bm{\rho}}{\delta t} + \dfrac{\delta^2 \bm{\rho}}{\delta t^2} + \dfrac{\bm{\omega}_T}{\d t}\times \bm{\rho} + \bm{\omega}_T \times (\bm{\omega}_T \times \bm{\rho}) \Bigg]\,\d m \\[0.5em]
		& = m \dfrac{\d^2 \bm{r}_{c.m}}{\d t^2} + 2 \bm{\omega}_T \times \int_m \dfrac{\delta \bm{\rho}}{\delta t}\, \d m + \int_m \dfrac{\delta^2 \bm{\rho}}{\delta t^2}\, \d m + \bm{\omega}_T \times \left(\bm{\omega}_T \times \int_m \bm{\rho}\, \d m \right)
	\end{align*}
由质心的定义,有$\displaystyle \int_m \bm{\rho} \, \d m = 0$,则可以得到

\theorem[任意变质量物体的一般运动方程]
{
	 \vspace*{-1em}
\begin{equation}
	\bm{F}_S = m \dfrac{\d^2 \bm{r}_{c.m}}{\d t^2} + 2 \bm{\omega}_T \times \int_m \dfrac{\delta \bm{\rho}}{\delta t}\, \d m + \int_m \dfrac{\delta^2 \bm{\rho}}{\delta t^2}\, \d m 
\end{equation}
}

\noindent 相应地可以得到

\theorem[任意变质量物体的质心运动方程]
{
	\vspace*{-1em}
	\begin{equation}
		m \dfrac{\d^2 \bm{r}_{c.m}}{\d t^2}  = \bm{F}_s + \bm{F}'_{k} + \bm{F}'_{rel} 
		\label{运动方程}
	\end{equation}
	其中,
	{
		\begin{enumerate}[\hspace*{2em}]
			\item \dy[附加哥氏力]{FJGSL}\quad $\displaystyle \bm{F}'_k = - 2 \bm{\omega}_T \times \int_m \dfrac{\delta \bm{\rho}}{\delta t}\, \d m$
			\item \dy[附加相对力]{FJXDL} \quad $\displaystyle \bm{F}'_{rel} = - \int_m \dfrac{\delta^2 \bm{\rho}}{\delta t^2}\, \d m$
		\end{enumerate}
	}
}

\textbf{2. 绕质心运动方程}

系统$S$绕质心的力矩方程为
\begin{equation}
	\bm{M}_{c.m} = \int_m \bm{\rho} \times \dfrac{\d^2 \bm{r}}{\d t^2} \, \d m
\end{equation}
将加速度的表达式\eqref{绝对加速度}代入,可以得到
\begin{align*}
	\bm{M}_{c.m} = \int_m \bm{\rho} \times \dfrac{\d^2 \bm{r}_{c.m}}{\d t^2}\, \d m 
	+ 2 \int_m \bm{\rho} \times \left(\bm{\omega}_T \times \dfrac{\delta \bm{\rho}}{\delta t}\right)\, \d m 
	+ \int_m \bm{\rho} \times \dfrac{\delta^2 \bm{\rho}}{\delta t^2}\, \d m 
	+ \int_m \bm{\rho}\times \left(\dfrac{\d \bm{\omega}_T}{\d t} \times \bm{\rho}\right)\, \d m 
	+ \int_m \bm{\rho}\times \big[\bm{\rho} \times (\bm{\omega}_T \times \bm{\rho})\big]\, \d m
\end{align*}

由于质心的定义,$\displaystyle \int_m \bm{\rho} \, \d m = 0$且$\dfrac{\d^2 \bm{r}_{c.m}}{\d t^2}$与质量无关,所以$\displaystyle \int_m \bm{\rho} \times \dfrac{\d^2 \bm{r}_{c.m}}{\d t^2}\, \d m = 0$,即化简为
\begin{align}
	\bm{M}_{c.m} = 2 \int_m \bm{\rho} \times \left(\bm{\omega}_T \times \dfrac{\delta \bm{\rho}}{\delta t}\right)\, \d m 
	+ \int_m \bm{\rho} \times \dfrac{\delta^2 \bm{\rho}}{\delta t^2}\, \d m 
	+ \int_m \bm{\rho}\times \left(\dfrac{\d \bm{\omega}_T}{\d t} \times \bm{\rho}\right)\, \d m 
	+ \int_m \bm{\rho}\times \big[\bm{\rho} \times (\bm{\omega}_T \times \bm{\rho})\big]\, \d m
	\label{转动方程}
\end{align}
记
\begin{align*}
	\bm{M}'_k &= - 2 \int_m \bm{\rho} \times \left(\bm{\omega}_T \times \dfrac{\delta \bm{\rho}}{\delta t}\right)\, \d m \\[0.5em]
	\bm{M}'_{rel} &= - \int_m \bm{\rho} \times \dfrac{\delta^2 \bm{\rho}}{\delta t^2}\, \d m
\end{align*}
则\eqref{转动方程}可以写为
\begin{equation}
	\int_m \bm{\rho}\times \left(\dfrac{\d \bm{\omega}_T}{\d t} \times \bm{\rho}\right)\, \d m 
	+ \int_m \bm{\rho}\times \big[\bm{\rho} \times (\bm{\omega}_T \times \bm{\rho})\big]\, \d m 
	= \bm{M}_{c.m} + \bm{M}'_k + \bm{M}'_{rel}
	\label{转动方程2}
\end{equation}
公式\eqref{转动方程2}左端第二项可以处理为
\begin{equation}
	\int_m \bm{\rho} \times \big[\bm{\omega}_T \times (\bm{\omega}_T \times \bm{\rho})\big]\, \d m = \bm{\omega}_T \times \int_m \bm{\rho} \times (\bm{\omega}_T \times \bm{\rho}) \, \d m \xlongequal[]{ \textstyle \hspace*{0.5em} \Delta \hspace*{0.5em}} \bm{\omega}_T \times \bm{H}_{c.m}
\end{equation}
其中$\bm{H}_{c.m}$为将系统视为刚体后,刚体对质心的角动量。

建立与物体固连的坐标系$o_1 - xyz$,有
\begin{equation*}
	\bm{\omega}_T = 
	\begin{bmatrix}
		\omega_{Tx} & \omega_{Ty} & \omega_{Tz}
	\end{bmatrix}^{\text{T}} \qquad \bm{\rho} = 
\begin{bmatrix}
	x & y & z
\end{bmatrix}^{\text{T}}
\end{equation*}
则角动量为
\begin{align*}
	\bm{H}_{c.m} & = \int_m \big[\bm{\rho} \times (\bm{\omega}_T \times \bm{\rho})\big]\, \d m 
	= \int_m \big[(\bm{\rho \cdot \bm{\rho}}) \bm{\omega}_T - (\bm{\rho \cdot \bm{\omega_T}})\bm{\rho }\big]\, \d m\\[0.5em]
	& = \int_m \big[(\bm{\rho}\cdot \bm{\rho})\bm{\omega}_T - (\bm{\rho} \cdot \bm{\rho}^{\text{T}})\bm{\omega}_T\big]\, \d m\\[0.5em]
	& = \int_m 
	\begin{bmatrix}
		y^2 + z^2 & -xy & -xz \\
		-xy & z^2 + x^2 & -yz \\
		-zx & -zy & x^2 + y^2 
	\end{bmatrix}
	\,
	\begin{bmatrix}
		\omega_{Tx}\\
		\omega_{Ty}\\
		\omega_{Tz}
	\end{bmatrix}
	\, \d m\\
	& =  \bm{I}\cdot \bm{\omega}_T
\end{align*}

其中,$\bm{I}$为\dy[惯性张量]{GXZL}
\begin{equation}
	\bm{I} = 
	\begin{bmatrix}
		I_{xx} & - I_{xy} & -I_{xz} \\
		-I_{xy} & I_{yy} & -I_{yz} \\
		-I_{zx} & -I_{zy} & I_{zz}
	\end{bmatrix}
\end{equation}

定义\dy[转动惯量]{ZDGL}$I_{xx},I_{yy},I_{zz}$和\dy[惯量积]{GLJ}
\begin{equation}
	\begin{cases}
		\, \displaystyle I_{xx} = \int_m(y^2 + z^2)\, \d m \\[0.8em]
		\, \displaystyle I_{yy} = \int_m(x^2 + z^2)\, \d m \\[0.8em]
		\,  \displaystyle I_{zz} = \int_m(y^2 + x^2)\, \d m \\[0.8em]
		\,  \displaystyle I_{xy} = I_{yx} = \int_m xy \, \d m\\[0.8em]
		\,  \displaystyle I_{xz} = I_{zx} = \int_m xz \, \d m\\[0.8em]
		\,  \displaystyle I_{xy} = I_{yx} = \int_m yz \, \d m
	\end{cases}
\end{equation}
则
\begin{equation}
	\bm{H}_{c.m} = \int_m \bm{\rho} \times (\bm{\omega}_T \times \bm{\rho})\, \d m =  \bm{I}\cdot \bm{\omega}_T 
	\label{角动量}
\end{equation}

同理可以对方程\eqref{转动方程2}左边第一项处理得到
\begin{equation}
	\int_m \bm{\rho} \times \left(\dfrac{\d \bm{\omega}_T}{\d t} \times \bm{\rho}\right) = \int_m \begin{bmatrix}
		y^2 + z^2 & -xy & -xz \\
		-xy & z^2 + x^2 & -yz \\
		-zx & -zy & x^2 + y^2 
	\end{bmatrix}
	\,
	\begin{bmatrix}
		\dfrac{\d \omega_{Tx}}{\d t}\\[0.8em]
		\dfrac{\d \omega_{Ty}}{\d t}\\[0.8em]
		\dfrac{\d \omega_{Tz}}{\d t}
	\end{bmatrix}
	\, \d m = \bm{I}\cdot \dfrac{\d \bm{\omega}_T}{\d t}
	\label{角动量导}
\end{equation}

则最终的转动方程为

\theorem[任意变质量物体的绕质心运动方程]
{
	\vspace*{-1em}
	\begin{equation}
		\bm{I}\cdot \dfrac{\d \bm{\omega}_T}{\d t} + \bm{\omega}_T\times(\bm{I}\cdot \bm{\omega}_T) = \bm{M}_{c.m}+\bm{M}'_k+\bm{M}'_{rel}
		\label{绕质心的运动方程}
	\end{equation}
	其中,
	{
		\begin{enumerate}[\hspace*{2em}]
			\item \dy[附加哥氏力矩]{FJGSLJ}\quad $\displaystyle \bm{M}'_k = -2 \int_m \bm{\rho}\times \left(\bm{\omega}_T \times \dfrac{\delta \bm{\rho}}{\delta t}\right)\, \d m$
			\item \dy[附加相对力矩]{FJXDLJ} \quad $\displaystyle \bm{F}'_{rel} = - \int_m \bm{\rho} \times \dfrac{\delta^2 \bm{\rho}}{\delta t^2} \, \d m$
			\item \dy[惯量张量]{GXZL} \quad $\displaystyle \bm{I} = 
			\int_m \begin{bmatrix}
				y^2 + z^2 & -xy & -xz \\
				-xy & z^2 + x^2 & -yz \\
				-zx & -zy & x^2 + y^2 
			\end{bmatrix} =
		\begin{bmatrix}
			I_{xx} & - I_{xy} & -I_{xz} \\
			-I_{xy} & I_{yy} & -I_{yz} \\
			-I_{zx} & -I_{zy} & I_{zz}
		\end{bmatrix}$
		\end{enumerate}
	}
}
\vspace*{1em}

\textbf{3. 钢化原理}

\theorem[钢化原理]
{
	一般情况下,任意变质量系统的运动方程,可用一个刚体的运动方程表示。这个刚体的质量等于系统瞬时质量,其受力除真实的外力与外力矩外,还要加上两个附加力和两个附加力矩。
}






























	
	% 附章:参考内容
	% 附章专用计数器
%\newcounter{CFsection}[chapter]
%\renewcommand{\theCFsection}{\stepcounter{CFsection} \textbf{F.\arabic{CFsection}}}
%\newcounter{CFsubsection}[section]
%\renewcommand{\theCFsubsection}{\stepcounter{CFsubsection} \textbf{F.\arabic{CFsection}.\arabic{CFsubsection}}}
%\renewcommand{\theequation}{F.\arabic{equation}}


% 附章专用标题格式
%\titleformat{\chapter}{\bfseries\Huge\color{titlepurple}}{附章 \quad}{0pt}{}
%\titleformat{\section}{\Large\color{titlepurpleb}}{\bfseries{\theCFsection}\quad  }{0pt}{}
%\titleformat{\subsection}{\large\color{titlepurplec}}{\bfseries{\theCFsubsection}\quad  }{0pt}{}


\chapter{补充公式及基本化简方法}
\thispagestyle{empty}
\section{三角函数公式}
\subsection{和差化积}
\begin{equation}
	\boxed{\sin \alpha + \sin \beta = 2 \sin \dfrac{\alpha + \beta}{2}  \cos  \dfrac{\alpha - \beta}{2} }
	\vspace*{0.5em}
\end{equation}
\renewcommand{\arraystretch}{1.6}
\begin{tabular}{l}
\proof \quad $\displaystyle \sin \alpha + \sin \beta = \sin  \left( \dfrac{\alpha + \beta}{2} + \dfrac{\alpha - \beta}{2}\right) + \sin \left( \dfrac{\alpha + \beta}{2} - \dfrac{\alpha - \beta}{2} \right)$\\[-1em]
$\displaystyle = \sin  \dfrac{\alpha + \beta}{2} \cos  \dfrac{\alpha - \beta}{2}  + \cos  \dfrac{\alpha + \beta}{2}\sin  \dfrac{\alpha - \beta}{2} + \sin \dfrac{\alpha + \beta}{2}\cos \dfrac{\alpha - \beta}{2} - \cos \dfrac{\alpha + \beta}{2} \sin \dfrac{\alpha - \beta}{2}$\\
$= 2 \sin \dfrac{\alpha + \beta}{2}  \cos  \dfrac{\alpha - \beta}{2} $
\end{tabular}

\vspace*{1.2em}

\begin{equation}
	\boxed{\sin \alpha - \sin \beta = 2 \cos \dfrac{\alpha + \beta}{2}\sin \dfrac{\alpha- \beta}{2}}
	\vspace*{0.5em}
\end{equation}
\begin{tabular}{l}
	\proof \quad $\displaystyle \sin \alpha - \sin \beta = \sin \left( \dfrac{\alpha + \beta }{2} + \dfrac{\alpha - \beta}{2}\right) - \sin \left( \dfrac{\alpha - \beta}{2} - \dfrac{\alpha - \beta}{2} \right)$ \\[-1em]
	$=\sin \dfrac{\alpha + \beta}{2} \cos \dfrac{\alpha - \beta}{2} + \cos \dfrac{\alpha + \beta}{2}\sin \dfrac{\alpha - \beta}{2} - \sin \dfrac{\alpha + \beta}{2}\cos\dfrac{\alpha - \beta}{2} + \cos \dfrac{\alpha + \beta}{2} \sin \dfrac{\alpha - \beta}{2}$\\
	$= 2 \cos \dfrac{\alpha + \beta}{2} \sin \dfrac{\alpha - \beta}{2}$
\end{tabular}

\vspace*{1.2em}

\begin{equation}
	\boxed{\cos \alpha + \cos \beta = 2 \cos \dfrac{\alpha + \beta}{2}\cos \dfrac{\alpha- \beta}{2}}
	\vspace*{0.5em}
\end{equation}
\begin{tabular}{l}
	\proof \quad $\displaystyle \cos \alpha + \cos \beta = \cos \left( \dfrac{\alpha + \beta }{2} + \dfrac{\alpha - \beta}{2}\right) + \cos \left( \dfrac{\alpha - \beta}{2} - \dfrac{\alpha - \beta}{2} \right)$ \\[-1em]
	$=\cos \dfrac{\alpha + \beta}{2} \cos \dfrac{\alpha - \beta}{2} - \sin \dfrac{\alpha + \beta}{2}\sin \dfrac{\alpha - \beta}{2} + \cos \dfrac{\alpha + \beta}{2} \cos\dfrac{\alpha - \beta}{2} - \sin \dfrac{\alpha + \beta}{2} \sin \dfrac{\alpha - \beta}{2}$\\
	$= 2 \cos \dfrac{\alpha + \beta}{2} \cos \dfrac{\alpha - \beta}{2}$
\end{tabular}

\vspace*{2em}

\begin{equation}
	\boxed{\cos \alpha - \cos \beta = - 2 \sin \dfrac{\alpha + \beta}{2}\sin \dfrac{\alpha- \beta}{2}}
	\vspace*{0.5em}
\end{equation}
\begin{tabular}{l}
	\proof \quad $\displaystyle \cos \alpha - \cos \beta = \cos \left( \dfrac{\alpha + \beta }{2} + \dfrac{\alpha - \beta}{2}\right) - \cos \left( \dfrac{\alpha - \beta}{2} - \dfrac{\alpha - \beta}{2} \right)$ \\[-1em]
	$=\cos \dfrac{\alpha + \beta}{2} \cos \dfrac{\alpha - \beta}{2} - \sin \dfrac{\alpha + \beta}{2}\sin \dfrac{\alpha - \beta}{2} - \cos \dfrac{\alpha + \beta}{2} \cos\dfrac{\alpha - \beta}{2} + \sin \dfrac{\alpha + \beta}{2} \sin \dfrac{\alpha - \beta}{2}$\\
	$= - 2 \sin \dfrac{\alpha + \beta}{2} \sin \dfrac{\alpha - \beta}{2}$
\end{tabular}

\begin{equation}
	\boxed{\tan \alpha + \tan \beta = \dfrac{\sin \big(\alpha + \beta \big)}{\cos \alpha \cos \beta} }
	\vspace*{0.5em}
\end{equation}
\quad \proof \quad $\displaystyle \tan \alpha + \tan \beta = \dfrac{\sin \alpha}{\cos \alpha} + \dfrac{\sin \beta}{\cos \beta} = \dfrac{\sin \alpha \cos \beta + \cos \alpha + \sin \beta}{\cos \alpha \cos \beta} = \dfrac{\sin \big( \alpha + \beta)}{\cos \alpha \cos \beta}$


\subsection{积化和差}
\vspace*{-1em}
\begin{align}
	\boxed{\sin \alpha \cos \beta = \dfrac 1 2 \Big[\sin \big( \alpha + \beta \big) + \sin \big( \alpha - \beta \big) \Big]} \\[0.5em]
	\boxed{\sin \alpha \sin \beta = -\dfrac 1 2 \Big[\cos \big(\alpha + \beta \big) - \sin \big( \alpha - \beta \big) \Big]} \\[0.5em]
	\boxed{\cos \alpha \sin \beta = \dfrac 1 2 \Big[\sin \big( \alpha + \beta \big) - \sin \big( \alpha - \beta \big) \Big]} \\[0.5em]
	\boxed{\cos \alpha \cos \beta = \dfrac 1 2 \Big[\cos \big(\alpha + \beta \big) + \cos \big( \alpha - \beta \big) \Big]}
\end{align}

\subsection{降幂公式(升幂公式)}
由二倍角公式,得
\begin{equation*}
	\cos 2 \alpha = \cos^2 \alpha - \sin^2 \alpha = 1 - 2\sin^2 \alpha = 2 \cos^2 \alpha - 1
\end{equation*}

\noindent 从而推导出以下公式
\begin{equation}
	\boxed{\sin^2 \alpha = \dfrac{1 - \cos 2 \alpha }{2}} \qquad \qquad \boxed{\cos^2 \alpha = \dfrac{1 + \cos 2 \alpha }{2}} \qquad \qquad \boxed{\tan^2 \alpha = \dfrac{1 - \cos 2 \alpha }{1 + \cos 2 \alpha}}
	\vspace*{0.5em}
\end{equation}

\subsection{万能公式}
\begin{equation}
	\boxed{\sin 2 \theta = \dfrac{2 \tan \theta}{1 + \tan^2 \theta}} \qquad \qquad \boxed{\cos 2 \theta = \dfrac{1 - \tan^2 \theta}{1 + \tan^2 \theta}} \qquad \qquad \boxed{\tan 2 \theta = \dfrac{2 \tan \theta}{1 - \tan^2 \theta}}
	\vspace*{0.5em}
\end{equation}
\proof 由二倍角公式,得
\vspace*{-1em}
\begin{align*}
	& \sin 2 \theta = 2 \sin \theta \cos \theta = \dfrac{2 \sin \theta \cos \theta }{\sin^2 \theta + \cos^2 \theta} = \dfrac{2 \tan \theta}{1 + \tan^2 \theta}. \\[0.5em]
	&\cos 2 \theta = \cos^2 \theta - \sin^2 \theta = \dfrac{\cos^2 \theta - \sin^2 \theta}{\cos^2 \theta + \sin^2 \theta} = \dfrac{1 - \tan^2 \theta}{1 + \tan^2 \theta}.\\[0.5em]
	&\tan 2 \theta = \tan (\theta + \theta) = \dfrac{2 \tan\theta}{1 - \tan^2 \theta}.
\end{align*}


\subsection{其他公式}
\vspace*{-1em}
\begin{align}
	\dfrac{1 + \sin \theta}{1 - \sin \theta} = \dfrac{\big(1 + \sin \theta \big)^2}{1 - \sin^2 \theta} = \left( \dfrac{1 + \sin \theta}{\cos \theta} \right)^2 = \big(\csc \theta + \tan \theta \big)^2 \\[0.5em]
	\dfrac{1 + \cos \theta}{1 - \cos \theta} = \dfrac{\big( 1 + \cos \theta \big)^2}{1 - \cos^2 \theta} = \left( \dfrac{1 + \cos \theta}{\sin \theta} \right)^2 = \big( \sec \theta + \cot \theta \big)^2
\end{align}

\subsection{双曲函数公式}
双曲函数有着特别的性质(与三角函数类似)。具体如下:【注:以下性质用双曲函数的定义可以直接证明,故证明省略】
\begin{align}
	\boxed{\ch^2 x - \sh^2 x = 1} \qquad \quad\\[0.5em]
	\boxed{\sh (x + y) = \sh x \,\ch y + \ch x \,\sh y} \\[0.5em]
	\boxed{\sh (x - y) = \sh x \,\ch y - \ch x \,\sh y} \\[0.5em]
	\boxed{\ch (x + y) = \ch x \,\ch y + \sh x \,\sh y} \\[0.5em]
	\boxed{\ch (x - y) = \ch x \,\ch y - \sh x \,\sh y}  
\end{align}

















	
	
	%附录输出—————————————
	% 注意:参考文献和索引首页无页码需要在相应生成的文件中手动插入命令\thispagestyle{empty}
	% 参考文献:*.bbl
	% 索引:*.ind
	\cleardoublepage
	\addcontentsline{toc}{chapter}{附录}
	\color{titlepurplec}
	\addcontentsline{toc}{section}{参考文献}
	\bibliographystyle{IEEEtran}
	\bibliography{ref}
	\addcontentsline{toc}{section}{插图目录}
	\renewcommand{\listfigurename}{插图目录}
	\thispagestyle{empty}
	\listoffigures
	\addcontentsline{toc}{section}{表格目录}
	\renewcommand{\listtablename}{表格目录}
	\thispagestyle{empty}
	\listoftables
	\thispagestyle{empty}
	\cleardoublepage
	\addcontentsline{toc}{section}{索引}
	\appendix
	\printindex
	%———————————————
	
\end{document}