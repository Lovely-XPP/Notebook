\chapter{傅立叶变换与拉普拉斯变换}
\thispagestyle{empty}

\section{傅立叶变换}
\subsection{傅立叶级数}
\tdefination[傅立叶级数]
周期为$T$的任一周期函数$f(t)$,若满足下列Dirichlet条件:\index{FLYJS@傅立叶级数}
\begin{enumerate}[\hspace*{3em} 1.]
	\item 在一个周期内只有有限个不连续点;
	\item 在一个周期内只有有限个极小和极大值;
	\item 积分$\displaystyle \int_{-T/2}^{T/2}\big|f(t)\big|\, \d t$存在。
\end{enumerate}
则$f(t)$可展开为如下的Fourier级数
\begin{equation}
	f(t) = \dfrac{1}{2}a_0 + \sum_{n = 1}^{\infty}\big(a_n \cos n\omega t + b_n \sin n \omega t\big)
\end{equation}
其中,
\begin{align}
	a_n &= \dfrac{2}{T}\int_{-T/2}^{T/2} f(t)\cos n \omega t \, \d t, \quad n = 0,1,\cdots,\infty\\[0.5em]
	b_n &= \dfrac{2}{T}\int_{-T/2}^{T/2} f(t)\sin n \omega t \, \d t, \quad n = 1,\cdots,\infty
\end{align}

周期函数$f(t)$的傅氏级数还可以写为复数形式(指数形式)
\begin{equation}
	f(t) = \sum_{n = -\infty}^{+\infty}\alpha_n\e^{\j n \omega t}
\end{equation}
式中系数
\begin{equation}
	\alpha_n = \dfrac{1}{T}\int_{-T/2}^{T/2} f(t)\e^{-\j n \omega t}\, \d t
\end{equation}
其中, $\omega = \dfrac{T}{2\pi}$称为\dy[角频率]{JPL}。
\vspace*{0.5em}

\subsection{傅立叶变换}
\tdefination[傅立叶变换]
对非周期函数$f(t)$,可以视为周期$T$为无穷大的周期函数,此时
\[
\omega_0 = \dfrac{2\pi}{T} \to 0,\quad \Delta \omega = (n+1)\omega_0 - n \omega_0 = \omega_0 \to 0
\]
即
\begin{align}
	f(t) &= \sum_{n = -\infty}^{+\infty} \alpha_n \e^{\j n \omega_0 t} \equiv  \sum_{\omega = -\infty}^{+\infty} \alpha_\omega \e^{\j \omega t}\\[0.5em]
	\alpha_\omega &= \dfrac{1}{T} \int_{-T/2}^{T/2} f(t)\e^{-\j n \omega_0 t}\, \d t = \dfrac{\Delta \omega}{2 \pi}\int_{-T/2}^{T/2}f(t)\e^{-\j\omega t}\, \d t
\end{align}
所以
\begin{align*}
	f(t) & =  \sum_{\omega = -\infty}^{+\infty} \alpha_\omega \e^{\j \omega t} =  \sum_{\omega = -\infty}^{+\infty}\Bigg[ \dfrac{\Delta \omega}{2 \pi}\int_{-T/2}^{T/2}f(t)\e^{-\j\omega t}\, \d t\Bigg]\e^{\j \omega t}\\[0.5em]
	& = \dfrac{1}{2\pi}\sum_{\omega = -\infty}^{+\infty}\Bigg[\int_{-T/2}^{T/2}f(t)\e^{-\j\omega t}\, \d t\Bigg]\e^{\j \omega t}\Delta \omega
\end{align*}
当$T\to \infty,\,\Delta \omega \to 0$,上式求和变积分
\begin{align}
	f(t) &= \dfrac{1}{2\pi}\int_{-\infty}^{+\infty} F(\omega)\e^{\j\omega t}\, \d \omega\\[0.5em]
	F(\omega) & = \int_{-\infty}^{+\infty}f(t)\e^{-\j \omega t}\, \d t
\end{align}
$F(\omega)$称为$f(t)$的\dy[Fourier变换]{FourierBH}。


\section{拉普拉斯变换}
\subsection{拉普拉斯变换的定义}
\tdefination[拉普拉斯变换]
设函数$f(t)$在$t \ge 0$时有定义,而且积分
\begin{equation*}
	\int_{0}^{+\infty} f(t)\e^{-st}\,\d t \quad\quad s = \sigma+\j \omega
\end{equation*}
在复变量$s$的某个域内收敛,则由此积分所确定的函数可写为
\begin{equation}
	F(s) = \int_{0}^{\infty}f(t)\e^{st}\,\d t
\end{equation}
称为函数$f(t)$的\dy[拉普拉斯变换式]{LPLSBHS},简称\dy[拉氏变换式]{LSBHS},并记作
\begin{equation*}
	F(s) = \mathcal{L}\big[f(t)\big]
\end{equation*}
$F(s)$称为$f(t)$的\dy[象函数]{XHS},而$f(t)$称为$F(s)$的原函数。有象函数求原函数的运算称为\dy[拉氏逆变换]{LSNBH},记作
\begin{equation*}
	f(t) = \mathcal{L}^{-1}\big[F(s)\big]
\end{equation*}

\subsection{拉普拉斯变换的性质}

\ttheorem[线性性质]
若$\alpha,\beta$是任意常实数,且$\mathcal{L}\big[f_1(t)\big] = F_1(s), \mathcal{L}\big[f_2(t)\big] = F_2(s)$,则有
\begin{equation}
	\mathcal{L}\big[\alpha f_1(t)\pm \beta f_2(t)\big] = \alpha F_1(s) \pm \beta F_2(s)
\end{equation}

\theorem[微分性质]
若$\mathcal{L}\big[f(t)\big] = F(s)$,则有
\begin{equation}
	\mathcal{L}\big[f^{(n)}(t)\big] = s^nF(s) -s^{n-1}f(0)-s^{n-2}f'(0)- \cdots - f^{(n-1)}(0)
\end{equation}

\theorem[积分性质]
若$\mathcal{L}\big[f(t)\big] = F(s)$,则有
\begin{equation}
	\mathcal{L}\left[\underbrace{\int\cdots\int}_{n} f(t) \,\d t^n\right]=\frac{1}{s^n}F(s)+\frac{1}{s^n}f^{-1}(0) +\frac{1}{s^{n-1}}f^{(-2)}(0)+\cdots +\frac{1}{s}f^{(-n)}(0)
\end{equation}
其中,$f^{(-1)}(0),f^{(-2)}(0),f^{(-n)}(0)$分别为$f(t)$的各重积分在$t=0$处的值。\\

\theorem[位移性质]
若$\mathcal{L}\big[f(t)\big] = F(s)$,则有
\begin{equation}
	\mathcal{L}\big[\e^{at} f(t)\big] = F(s - a)
\end{equation}

\theorem[延迟性质]
若$\mathcal{L}\big[f(t)\big] = F(s)$,则对于任意实数$\tau > 0$有
\begin{equation}
	\mathcal{L}\big[f(t - \tau ) \cdot 1(t - \tau)\big] = \e^{-\tau s}F(s)
\end{equation}

\theorem[初值定理]
若$\mathcal{L}\big[f(t)\big] = F(s)$,且$\lim_{s \to \infty}s F(s)$存在,则
\begin{equation}
	\lim_{t \to 0}f(t) = \lim_{s \to \infty} s F(s)
\end{equation}

\theorem[终值定理]
若$\mathcal{L}\big[f(t)\big] = F(s)$,且$\lim_{t \to \infty}f(t), \lim_{s \to 0}s F(s)$均存在,则
\begin{equation}
	\lim_{t \to \infty}f(t) = \lim_{s \to 0} s F(s)
\end{equation}

\theorem[卷积定理]
若$\mathcal{L}\big[f_1(t)\big] = F_1(s), \mathcal{L}\big[f_2(t)\big] = F_2(s)$,则
\begin{equation}
	\mathcal{L}\big[f_1(t) * f_2(t)\big] = F_1(s) \cdot F_2(s)
\end{equation}
\clearpage

\subsection{几种典型函数的拉氏变换}
{
	\centering
	\setlength{\tabcolsep}{12mm}{
		\begin{longtable}{cc}
			
			\toprule
			原函数$f(t)$ & 象函数$F(s)$\\
			\midrule
			\endfirsthead
			
			\multicolumn{2}{r}{续表}\\
			\toprule
			原函数$f(t)$ & 象函数$F(s)$\\
			\midrule
			\endhead
			
			% 表格“尾页前”,表格最后显示内容
			\bottomrule
			\endfoot
			
			% 表格“尾页”,表格最后显示内容
			\bottomrule
			\endlastfoot
			
			单位跃阶函数 $\,1 (t) = \begin{cases}
				1, &t \ge 0\\0, & t<0
			\end{cases}$
			& $\dfrac{1}{s}$\\
			\hline
			单位脉冲函数 $\,\delta (t) =\begin{cases}
				0, & t \neq 0\\
				\infty, & t = 0
			\end{cases}$ & 1\\
			
			\hline
			单位斜坡函数 $ \, f(t) = \begin{cases}
				t, &t \ge 0\\
				0, &t <0
			\end{cases}$& $\dfrac{1}{s^2}$\\
			\hline
			单位加速度函数 $ \, f(t) = \begin{cases}
				\dfrac{1}{2}t^2, &t \ge 0\\
				0, &t <0
			\end{cases}$& $\dfrac{1}{s^3}$\\
			\hline
			\multirow{2}*{指数函数$\displaystyle \e^{-at}$}  & \multirow{2}*{$\dfrac{1}{s+a}$} \\
			&\\
			\hline
			\multirow{2}*{正弦函数$\sin \omega t$}  & \multirow{2}*{$\dfrac{\omega}{s^2+\omega^2}$} \\
			&\\
			\hline
			\multirow{2}*{余弦函数 $\cos \omega t$
			} & \multirow{2}*{$\dfrac{s}{s^2+\omega^2}$} \\
			&\\
		\end{longtable}
	}
	\vspace*{-2em}
	\captionof{table}{几种典型函数的拉氏变换表}
}
\vspace*{1em}

\subsection{拉普拉斯逆变换的求解}

设$s_1,s_2,\cdots,s_3$是函数$F(s)$的所有孤立点,适当选取$\beta$使得这些奇点全在$\text{Re}(s)<\beta $的范围内,且
\begin{equation*}
	\lim_{s \to \infty} F(s) = 0
\end{equation*}
则
\begin{equation}
	f(t)=\mathcal{L}\big[F(s)\big]=\frac{1}{2 \pi \j}\int_{\beta + \j \infty}^{\beta- \j \infty} F(s)\e^{st}\,\d s=\sum_{k}\text{Res}\big[F(s)\e^{st}, s_k\big]
\end{equation}
\vspace*{0.5em}

\subsection{拉普拉斯的应用——求解常微分方程(组)}
\texample[求解拉普拉斯求解常微分方程(组)]\vspace*{1em}
\noindent \vspace*{1.5em} \noindent  \hspace*{0.2em}  \tcbox[colframe =black, colback =black!10!white,boxrule=0.5mm,size=small,on line]{\color{black}{{ 解题步骤}}\hspace*{0.25em}}\hspace{1.5em}
\vspace*{-1em}
\begin{enumerate}
	\item 对方程两边同时做拉普拉斯变换。
	\item 利用拉普拉斯变换的线性性质、微分性质代替各阶的的导数,解出$Y(s)$。
	\item 做$Y(s)$的拉普拉斯逆变换,得到$y(t)$。
\end{enumerate}








