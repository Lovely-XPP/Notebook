\chapter{振动} 
\thispagestyle{empty}
\section{简谐运动}
\dy[简谐运动]{JXYD} \jg
\par 1. 运动学定义:运动函数为
\margin{\\[1em] \kg  $\omega t+\varphi$是时刻$t$的相,$\varphi$是初相.}
\begin{equation}
\eq[x=A\cos(\omega t+\varphi)]
\end{equation}
\par 其中$A$叫做简谐运动的振幅,
\margin
{
	$\omega,T,\nu$的关系如下
	\scriptsize{$$\disp \omega = \frac{2\pi }{T}$$}
	\vspace*{-1.5em}
	\scriptsize{$$\omega = 2\pi \nu$$}
}
它表示质点可能离开原点(平衡位置)的最大距离;$\omega$叫简谐运动的角频率,$T$是简谐运动的周期,$\nu$ 是简谐运动的频率。\jg
\par 加速度特征:$a=-\omega^2 x$.\jg
\par 三个特征量:$A,\omega ,\varphi$,知道这三个特征量就可以确定简谐运动方程.\jg
\par  \dy[相量图法]{XLT} 质点作匀速圆周运动时,它在直径上投影的运动就是简谐运动,因此可以用一个长度等于振幅$A$的旋转径矢表示一个简谐运动。这样旋转矢量图叫相量图,旋转的角速度为$\omega$,矢量的初角位置为初相$\varphi$.
\jg \jg
\par 2. 动力学定义:质点受的合力$F$与质点对平衡位置的位移$x$成反比而反向,即
\margin{\\[1.5em] \kg  $F$也称为回复力.}
\begin{equation}
\eq[F=-kx]
\end{equation}
\par 时,就是简谐运动。由牛顿第二定律可得
\margin{\\[1.5em] \kg  这个微分方程称为简谐运动的动力学方程.}
\begin{equation}
	\eq[\frac{\d^2 x}{\d t^2}+\frac{k}{m}x = 0]
\end{equation}
\par \tkg 任一物理量$x$随时间的变化遵守这一形式的微分方程时,它的变化就是简谐变化。\jg\jg

\par \dy[固定角频率]{GDJPL}
\margin{对应固定角频率的周期称为固定周期,表达式如下:
\scriptsize{$$\disp T = 2\pi \sqrt{\frac{m}{k}}$$}}
\begin{equation}
\eq[\omega=\sqrt{\frac{k}{m}}]
\end{equation}
\par 知道初始条件$t=0$时的位移$x_0$和速度$v_0$,可以求得
\begin{equation*}
\begin{split}
&A=\sqrt{x_0^2+\frac{v_0^2}{\omega^2}}\\
&\varphi = \arctan(-\frac{v_0}{\omega x_0} )
\end{split}
\end{equation*}

\section{简谐运动实例}
\dy[弹簧振子]{THZZ}
\margin{\\[1.5em] \kg $k$ 为弹簧的劲度系数.}
\begin{equation}
\eq[\frac{\d ^2 x}{\d t^2} = -\frac{k}{m}x],\,\,\eq[\omega = \sqrt{\frac{k}{m}}],\,\,\eq[T = 2\pi \sqrt{\frac{m}{k}}]
\end{equation}


\par \dy[单摆]{DB}
\margin{\\[1.5em] \kg $l$ 为摆长.}
\begin{equation}
\eq[\frac{\d ^2 \theta }{\d t^2} = -\frac{g}{l}\theta],\,\,\eq[\omega = \sqrt{\frac{g}{l}}],\,\,\eq[T = 2\pi \sqrt{\frac{l}{g}}]
\end{equation}

\par \dy[微小振动]{WXZD}
\margin{\\[1.5em] \kg $E_p$ 为系统势能函数.}
\begin{equation}
\omega = \sqrt{\frac{k}{m}} = \left[ \,\frac{1}{m}\left( \frac{\d^2 E_p}{\d x^2}\right)_{x=0} \,\,\right]^{1/2}
\end{equation}

\section{简谐运动的能量}
\par \dy[简谐运动的能量]{WXZD}
\margin{\\[4.5em] \kg 利用$\omega^2=k/m$}
\margin{\\[1em] \kg $E$ 为系统总机械能.\\\kg $\overline{E}_k$ 为系统动能平均值.\\\kg $\overline{E}_p$ 为系统势能平均值.}
\begin{equation}
\begin{split}
E_p&=\frac{1}{2}kx^2=\frac{1}{2}kA^2\cos^2(\omega t +\varphi)\\
E_k&=\frac{1}{2}mv^2=\frac{1}{2}m\omega^2 A^2\sin^2(\omega t +\varphi)=\frac{1}{2}kA^2\sin^2(\omega t +\varphi)\\
E&=E_k+E_p=\frac{1}{2}kA^2\\
\overline{E}_k&=\overline{E}_p=\frac{1}{2}E=\frac{1}{4}kA^2
\end{split}
\end{equation}

\section{阻尼振动}
\par 1. 欠阻尼(阻力较小的情况下)
\margin{记阻力$f_r = - \gamma v$,\\ $A_0$为初始振幅.\\ $\omega_0$为初始固有角频率.\\$\beta =\frac{\gamma}{2m}$为阻尼系数}
\begin{equation}
\eq[A=A_0\e^{-\beta t}],\,\,\eq[\omega=\sqrt{\omega_0^2-\beta^2}]
\end{equation}

\par \dy[时间常量]{SJCS} 能量减小到起始能量的$1/\e$所经过的时间
\margin{当$\beta \le \omega_0$时(弱阻尼),有$E \approx E_0\,\e^{-2\beta t}$,其中$E_0$为起始能量.时间常量又称为鸣响时间.}
\begin{equation}
\tau = \frac{1}{2\beta}
\end{equation}

\par \dy[品质因数$\,\,\bm{Q}$]{PZYS} 鸣响时间内振动次数的$2\pi$倍
\margin{在阻尼不严重的情况下,$T$和$\omega$可以分别用振动系统的固有周期和固有角频率替代.}
\begin{equation}
\eq[Q=2\pi \, \frac{\tau}{T}=\omega \tau]
\end{equation}

\par 2. 过阻尼(阻尼较大)的情况下,质点慢慢回到平衡位置,不再振动;
\jg 
\par 3. 临界阻尼(阻尼适当)的情况下,质点以最短的时间回到平衡位置,不再振动.

\section{受迫振动}
\dy[受迫振动]{SPZD} 在周期性驱动力作用下的振动。稳态时的振动频率等于驱动力的频率;当驱动力的频率等于振动系统的固有频率时发生共振现象,这时系统最大限度地从外界吸收能量。

\section{两个简谐运动的合成}
1. 同一直线上的两个同频率的振动合成时,其合振决定于两个振动的相差:
\par \tkg 二者同相时合振幅为二分振幅之和,即$A=A_1+A_2$;
\par \tkg 反相时为二分振幅之差,即$A=|A_1-A_2|$.
\par 2. 同一直线上的两个不同频率的振动合成时,如果者频率差较小,就会产生拍的现象。拍频等于二分振动的频率之差。
\par 3. 相互垂直的两个同频率振动合成时,合运动轨迹一般为椭圆。

\section{简谐运动的研究方法}
1. 解析法
\begin{equation*}
\begin{split}
x&=A\cos(\omega t + \varphi)\\
v&=- \omega A\cos(\omega t +\varphi)\\
a&=- \omega^2 A\cos(\omega t +\varphi)
\end{split}
\end{equation*}
\par 2. 曲线法\quad 通过给出的曲线图像找特殊点求出简谐运动方程.
\par 3. 相量图法
\par 注:求振动方程主要是求出三相关量$A,T,\varphi$,初相可以用特殊点代入解析法求出,也可以用相量图法求解。

\section{简谐运动的合成}
1. 同方向、同频率的简谐运动的合成
\begin{equation*}
\begin{cases}
x_1 = A_1\cos(\omega t + \varphi_1)\\
x_2 = A_2\cos(\omega t + \varphi_2)
\end{cases}
\,\,
\Longrightarrow
\,\,
x = A\cos(\omega t +\varphi)
\end{equation*}
其中,
\begin{equation*}
A=\sqrt{A_1^2+A_2^2+2A_1A_2\cos(\varphi_2-\varphi_1)},\,\,\tan \varphi =\frac{A_1\sin \varphi_1+A_2\sin \varphi_2}{A_1\cos \varphi_1+A_2\cos \varphi_2}
\end{equation*}
\par 注:求合振动方程除了利用辅助角公式,还需要利用和差化积进行化简。
\par 2. 相互垂直、同频率的简谐运动的合成——平面运动.
\par 3. 同方向、不同频率的简谐运动的合成——拍.

\section{弹簧振子}
\dy[弹簧串并联]{THCBL}\jg
\par 当两根弹簧(劲度系数分别为$k_1,k_2$)按同一条直线拼接(串联)后形成的新弹簧系统的劲度系数为$k$,则
\begin{equation}
\eq[k=\frac{1}{\frac{1}{k_1}+\frac{1}{k_2}}]
\end{equation}
特别地,当$k_1=k_2=k_0$,有
\margin{\\[-3em] \kg $k_1=k_2=k_0$可以认为是弹簧的两半,由此可以看出,将弹簧减成两半后,剩下的部分的劲度系数为原弹簧的$2$倍.}
\begin{equation}
\eq[k=\frac{k_0}{2}]
\end{equation}
\par 当两根弹簧(劲度系数分别为$k_1,k_2$)并排放置(并联)后形成的新弹簧系统的劲度系数为$k$,则
\begin{equation}
\eq[k=k_1+k_2]
\end{equation}

