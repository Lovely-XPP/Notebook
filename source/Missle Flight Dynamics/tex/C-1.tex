\chapter{课程简介}
\section{研究对象与内容}
\subsection{研究对象}
火箭和弹道导弹\vspace*{1em}

\subsection{研究内容}
以\textcolor{red}{运载火箭}、\textcolor{red}{弹道导弹}为背景,在\textcolor{red}{理论力学}、\textcolor{red}{空气动力学}、\textcolor{red}{自动控制原理}等基础上,以\textcolor{red}{计算机}为仿真建模工具,研究\textcolor{red}{飞行器质心运动}和\textcolor{red}{绕质心运动}特性及其伴随现象的一门课程。

\section{课程组织及要求}
\begin{itemize}
	\item 总学时数:46学时$+$18学时\vspace*{-0.5em}
	\item 教学与考核:讲授$+$实验,闭卷考试(60\%)$+$实验报告评价(40\%)\vspace*{-0.5em}
	\item 参考书:陈克俊等编著,远程火箭飞行动力学与制导,2014.\vspace*{-0.5em}
	\item 课程要求:掌握飞行动力学建模、弹道设计、弹道计算、弹道分析的理论和方法。
\end{itemize}

\section{导弹概述}
\subsection{导弹概念}
\vspace*{-1em}
\defination[导弹]
{
\dy[导弹]{DD}是一种包含制导控制系统,将战斗部导向目标导飞行器,以实现既定导作战目的。
}

\defination[弹道导弹]
{
\dy[弹道导弹]{DDDD}是关机后按惯性椭圆弹道飞向目标导导弹,分为固—液;战术—战役—战略;地地—潜地—空地。
}

\defination[助推滑翔导弹]
{
\dy[助推滑翔导弹]{ZTHXDD}是主要在临近空间\footnote[1]{临近空间:距离地面$20\sim100$公里的空域}飞行,助推关机后采用无动力滑翔命中目标的导弹。
}

\subsection{战技性能}
\begin{itemize}
	\item 火力性能:战斗部数量、战斗部总威力、最大最小射程、精确度或CEP等\vspace*{-0.5em}
	\item 使用性能:部署方式、可靠性、反应时间、保质期、维护操作等
\end{itemize}

















