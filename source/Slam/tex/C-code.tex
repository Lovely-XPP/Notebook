\chapter{部分示例代码}
\label{code}
\thispagestyle{empty}

\section{Eigen}
\label{Eigen}
\subsection{Eigen的基本使用}
\label{Eigen1}
\begin{lstlisting}[style=C++]
#include <iostream>

using namespace std;

// Eigen 核心部分
#include <Eigen/Core>
// 稠密矩阵的代数运算(逆,特征值等)
#include <Eigen/Dense>

using namespace Eigen;

#define MATRIX_SIZE 50

void solve_eqn(MatrixXd mtr, MatrixXd vt);

int main(int argc, char **argv)
{
    // Eigen 中所有向量和矩阵都是Eigen::Matrix,它是一个模板类。前三个参数为:数据类型,行,列
    // 声明一个2*3的float矩阵
    Matrix<float, 2, 3> matrix_23;

    // 同时,Eigen 通过 typedef 提供了许多内置类型,不过底层仍是Eigen::Matrix
    // 例如 Vector3d 实质上是 Eigen::Matrix<double, 3, 1>,即三维向量
    Vector3d v_3d;
    // 这是一样的
    Matrix<float, 3, 1> vd_3d;

    // Matrix3d 实质上是 Eigen::Matrix<double, 3, 3>
    Matrix3d matrix_33 = Matrix3d::Zero(); //初始化为零
    // 如果不确定矩阵大小,可以使用动态大小的矩阵
    Matrix<double, Dynamic, Dynamic> matrix_dynamic;
    // 更简单的
    MatrixXd matrix_x;
    // 这种类型还有很多,我们不一一列举

    // 下面是对Eigen阵的操作
    // 输入数据
    matrix_23 << 1, 2, 3, 4, 5, 6;
    // 输出
    cout << "matrix 2x3 from 1 to 6: \n"
         << matrix_23 << endl;

    // 用()访问矩阵中的元素,循环遍历输出各个元素
    cout << "print matrix 2x3: " << endl;
    for (int i = 0; i < 2; i++)
    {
        for (int j = 0; j < 3; j++)
            cout << matrix_23(i, j) << "\t";
        cout << endl;
    }

    // 矩阵和向量相乘(实际上仍是矩阵和矩阵)
    v_3d << 3, 2, 1;
    vd_3d << 4, 5, 6;

    // 但是在Eigen里你不能混合两种不同类型的矩阵,像这样是错的
    // Matrix<double, 2, 1> result_wrong_type = matrix_23 * v_3d;
    // 原因是 matrix_23 为float型,而 v_3d 为 double 型
    // 应该显式转换
    auto result = matrix_23.cast<double>() * v_3d;
    cout << "[1,2,3;4,5,6]*[3;2;1]=" << result.transpose() << endl;

    auto result2 = matrix_23 * vd_3d;
    cout << "[1,2,3;4,5,6]*[4;5;6]: " << result2.transpose() << endl;

    // 同样你不能搞错矩阵的维度
    // 如果不想知道矩阵的大小,矩阵乘法的结果可以直接用 auto 类型

    // 一些矩阵运算
    // 四则运算就不演示了,直接用+-*/即可。
    matrix_33 = Matrix3d::Random(); // 随机数矩阵
    cout << "random matrix: \n"
         << matrix_33 << endl;
    cout << "transpose: \n"
         << matrix_33.transpose() << endl;          // 转置
    cout << "sum: " << matrix_33.sum() << endl;     // 各元素和
    cout << "trace: " << matrix_33.trace() << endl; // 迹
    cout << "times 10: \n"
         << 10 * matrix_33 << endl; // 数乘
    cout << "inverse: \n"
         << matrix_33.inverse() << endl;                // 逆
    cout << "det: " << matrix_33.determinant() << endl; // 行列式

    // 求解特征值
    // 实对称矩阵可以保证对角化成功
    SelfAdjointEigenSolver<Matrix3d> eigen_solver(matrix_33.transpose() * matrix_33);
    cout << "Eigen values = \n"
         << eigen_solver.eigenvalues() << endl; // 特征值
    cout << "Eigen vectors = \n"
         << eigen_solver.eigenvectors() << endl; // 特征向量

    // 解方程
    // 我们求解 matrix_NN * x = v_Nd 这个方程
    // N的大小在前边的宏里定义,它由随机数生成
    // 直接求逆自然是最直接的,但是求逆运算量大

    Matrix<double, MATRIX_SIZE, MATRIX_SIZE> matrix_NN = MatrixXd::Random(MATRIX_SIZE, MATRIX_SIZE);
    matrix_NN = matrix_NN * matrix_NN.transpose(); // 保证半正定
    Matrix<double, MATRIX_SIZE, 1> v_Nd = MatrixXd::Random(MATRIX_SIZE, 1);

    solve_eqn(matrix_NN, v_Nd);

    return 0;
}

void solve_eqn(MatrixXd mtr, MatrixXd vt)
{
    // 直接求逆
    MatrixXd x = mtr.inverse() * vt;
    cout << "time of normal inverse is "
         << 1000 * (clock() - time_stt) / (double)CLOCKS_PER_SEC << "ms" << endl;
    cout << "x = " << x.transpose() << endl;

    // 通常用矩阵分解来求,例如QR分解,速度会快很多
    time_stt = clock();
    x = mtr.colPivHouseholderQr().solve(vt);
    cout << "time of Qr decomposition is "
         << 1000 * (clock() - time_stt) / (double)CLOCKS_PER_SEC << "ms" << endl;
    cout << "x = " << x.transpose() << endl;

    // 对于正定矩阵,还可以用cholesky分解来解方程
    time_stt = clock();
    x = mtr.ldlt().solve(vt);
    cout << "time of ldlt decomposition is "
         << 1000 * (clock() - time_stt) / (double)CLOCKS_PER_SEC << "ms" << endl;
    cout << "x = " << x.transpose() << endl;
}
\end{lstlisting}